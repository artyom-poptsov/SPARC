\documentclass[../sparc.tex]{subfiles}
\graphicspath{{\subfix{../images/}}}
\begin{document}

%%%%%%%%%%%%%%%%%%%%%%%%%%%%%%%%%%%%%%%%%%%%%%%%%%%%%%%%%%%%%%%%%%%%%%%%%%%%%%%%
\section{Последовательный порт}
\label{section:serial-port}
\index{Электроника!Последовательный порт}

%%%%%%%%%%%%%%%%%%%%%%%%%%%%%%%%%%%%%%%%%%%%%%%%%%%%%%%%%%%%%%%%%%%%%%%%%%%%%%%%
\subsection{Общие сведения}

\emph{Последовательный порт}, или по-английски ``serial port'' -- это один из
простейших способов коммуникации между цифровыми устройствами.

Как следует из названия, последовательный порт передаёт биты последовательно, по
одному за единицу времени.  Существуют другие интерфейсы передачи данных,
которые также используют последовательную передачу данных -- например, USB и
Ethernet.  Однако последовательным портом обычно называется аппаратное
обеспечение, совместимое со стандартом RS-232 или подобными стандартами
(например, RS-485, RS-422.)

Последовательный порт является \emph{полнодуплексным} средством коммуникации,
так как позволяет передавать данные в обе стороны.  Для этого используются две
независимые линии: одна -- для передачи данных (``Transmit'', или ``Tx'') и
другая -- для приёма (``Receive'', ``Rx''.)

Недостатком последовательного порта является более низкая скорость передачи
данных по сравнению с \emph{параллельным портом}, где сразу можно передать 8
бит.  Однако для единовременной передачи 8 бит параллельному порту требуется 8
проводов между передачиком и приёмником, тогда как для последовательного порта
нужно всего два провода для передачи данных.

\end{document}
