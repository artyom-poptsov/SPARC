\documentclass[../sparc.tex]{subfiles}
\graphicspath{{\subfix{../images/}}}
\begin{document}

%%%%%%%%%%%%%%%%%%%%%%%%%%%%%%%%%%%%%%%%%%%%%%%%%%%%%%%%%%%%%%%%%%%%%%%%%%%%%%%%
\section{Последовательный порт}
\label{section:communication-serial-port}
\index{Электроника!Последовательный порт}

%%%%%%%%%%%%%%%%%%%%%%%%%%%%%%%%%%%%%%%%%%%%%%%%%%%%%%%%%%%%%%%%%%%%%%%%%%%%%%%%
\subsection{Общие сведения}

\emph{Последовательный порт}, или по-английски ``serial port'' -- это один из
простейших способов коммуникации между цифровыми устройствами.

Как следует из названия, последовательный порт передаёт биты последовательно, по
одному за единицу времени.  Существуют другие интерфейсы передачи данных,
которые также используют последовательную передачу данных -- например, USB и
Ethernet.  Однако последовательным портом обычно называется аппаратное
обеспечение, совместимое со стандартом RS-232 или подобными стандартами
(например, RS-485, RS-422.)

Последовательные порты разделяют на \emph{синхронные} (использующие специальный
тактирующий импульс для синхронизации приёмопередачи на устройствах) и
\emph{асинхронные} (где взаимодействующие устройства должны быть до начала
обмена данными быть настроены на одну и ту же скорость приёмопередачи.)

Последовательный порт как правило является \emph{полнодуплексным} средством
коммуникации, так как позволяет передавать данные в обе стороны.  Для этого
используются две независимые линии: одна -- для передачи данных (``Transmit'',
или ``Tx'') и другая -- для приёма (``Receive'', ``Rx''.)

Недостатком последовательного порта является более низкая скорость передачи
данных по сравнению с \emph{параллельным портом}, где сразу можно передать 8
бит.  Однако для единовременной передачи 8 бит параллельному порту требуется 8
проводов между передачиком и приёмником, тогда как для последовательного порта
нужно всего два провода для передачи данных.  Таким образом, последовательная
передача данных является более дешёвой.

%%%%%%%%%%%%%%%%%%%%%%%%%%%%%%%%%%%%%%%%%%%%%%%%%%%%%%%%%%%%%%%%%%%%%%%%%%%%%%%%
\subsection{Универсальный асинхронный приёмопередатчик (УАПП)}
\newglossaryentry{UART}{name=UART, description={Universal Asynchronous
    Receiver-Transmitter}}
\newglossaryentry{УАПП}{name=УАПП, description={Универсальный асинхронный
    приёмопередатчик -- тоже самое, что \gls{UART}}}

Универсальный асинхронный приёмопередатчик (\gls{УАПП}), более широко известный
под англоязычным названием \gls{UART} -- это специальное устройство или
логическая цепь для асинхронной последовательной передачи данных.  По сути UART
является одной из реализаций последовательного порта, которая широко
используется для обмена данными между устройствами.

Для осуществления передачи UART берёт байты данных и последовательно пересылает
их в виде отдельных бит по шине, через равные промежутки времени.  На приёмнике
второй UART собирает отдельные биты в исходные байты.  На передатчике и на
приёмнике в UART имеется специальная микросхема, называемая \emph{сдвиговый
регистр}, которая являеется фундаментальным способом преобразования между
последовательным и параллельным представлением данных.

Скорость пересылки данных определяется параметром, называемым битрейт
(англ. \emph{bit rate}) или бодрейт (англ. \emph{baud rate}), который измеряется
в бодах (битах в секунду.)  Скорость $S$ (в бодах) и длительность одного бита
$T$ (в секундах) связаны отношением \ref{equation:uart-speed}.

\begin{equation}
  T = \frac{1}{S}
  \label{equation:uart-speed}
\end{equation}

Общепринятными скоростями считаются: 300, 1200, 2400, 4800, 9600, 19200, 38400,
57600, 115200, 230400, 460800 и 921600 бод.

\example{ Если подставить в формулу \ref{equation:uart-speed} скорость 9600 бод,
  то получаем:
  \begin{equation}
    T = \frac{1}{9600} \approx 0.00010416 \mbox{с} \approx 104.16 \mbox{мкс}
  \end{equation}
}

Иногда бывает удобнее оперировать не битами в секунду, а байтами в секунду,
поскольку в большинстве случаев мы измеряем объём информации именно в байтах.
Чтобы пересчитать боды в байты в секуду, необходимо использовать формулу
\ref{equation:uart-bauds-to-bytes-per-sec} (предполагается, что в 1 байте 8
бит.)

\begin{equation}
  S_{\mbox{байты/с}} = \frac{S_{\mbox{биты/с}}}{8}
  \label{equation:uart-bauds-to-bytes-per-sec}
\end{equation}

\example{ Популярной скоростью передачи текстовых данных при использовании
  Arduino является 9600 бод.  По формуле
  \ref{equation:uart-bauds-to-bytes-per-sec} мы можем узнать, сколько байт за
  секунду можно передать между компьютером и Arduino:
  \begin{equation}
    S_{\mbox{байты/с}} = \frac{9600_{\mbox{биты/с}}}{8} = 1200 \mbox{байт/с} = 1.2 \mbox{КБ/с}
  \end{equation}

  Допустим, в среднем длина одной музыкальной композиции составляет 5 Мбайт.
  Если мы заходим передать музыкальную композицию с компьютера на Arduino со
  скоростью 9600 бод, то мы можем посчитать, сколько времени на это потребуется:
  \begin{align*}
    5 \mbox{МБ} = 5242880 \mbox{байт}\\
    T = \frac{5242880 \mbox{байт}}{1200 байт} \approx 4369,06 \mbox{с} \approx 72.8 \mbox{мин.}
  \end{align*}

  Таким образом получаем, что для передачи 5 МБ данных потребуется 72 минуты
  времени, то есть, больше часа.
}

%%%%%%%%%%%%%%%%%%%%%%%%%%%%%%%%%%%%%%%%%%%%%%%%%%%%%%%%%%%%%%%%%%%%%%%%%%%%%%%%
\subsection{Функции и классы стандартной библиотеки}

\subsubsection{Класс \texttt{Serial}}

В Arduino для работы с последовательным портом есть класс \texttt{Serial},
который позволяет настраивать параметры передачи данных через последовательный
порт, передавать и принимать данные.  Мы уже работали с последовательным портом
в главе \ref{section:serial-port}, сейчас мы посмотрим на него более пристально.

На всех платах Arduino есть как минимум один последовательный порт, на некоторых
моделях (вроде Arduino Mega 2560) есть несколько последовательных портов.
Таблица последовательных портов и ассоциированных с ними цифровых портов можно
найти в таблице \ref{table:serial-ports--pins}. \footnote{Источник таблицы:
\url{https://www.arduino.cc/reference/en/language/functions/communication/serial/}}

\begin{table}[h]
\centering
\begin{tabular}{|m{9em}|m{3.5em}|m{3.5em}|m{3.5em}|m{3.5em}|m{3.5em}|}
  \hline
  \multirow{2}{2em}{\textbf{Плата}} & \multicolumn{5}{c|}{\textbf{Цифровые порты}} \\
  \cline{2-6}
   & \textbf{Serial}  & \textbf{Serial1}     & \textbf{Serial2}     & \textbf{Serial3}     & \textbf{Serial4}     \\
\hline
UNO R3, UNO R3 SMD Mini    & 0(RX), 1(TX) &                  &                  &                  &                  \\
\hline
Nano (classic)             & 0(RX), 1(TX) &                  &                  &                  &                  \\
\hline
UNO R4 Minima, UNO R4 WiFi &              & 0(RX0), 1(TX0)   &                  &                  &                  \\
\hline
Leonardo, Micro, Yún Rev2  &              & 0(RX), 1(TX)     &                  &                  &                  \\
\hline
Uno WiFi Rev.2             &              & 0(RX), 1(TX)     &                  &                  &                  \\
\hline
MKR boards                 &              & 13(RX), 14(TX)   &                  &                  &                  \\
\hline
Zero                       &              & 0(RX), 1(TX)     &                  &                  &                  \\
\hline
GIGA R1 WiFi               &              & 0(RX), 1(TX)     & 19(RX1), 18(TX1) & 17(RX2), 16(TX2) & 15(RX3), 14(TX3) \\
\hline
Due                        & 0(RX), 1(TX) & 19(RX1), 18(TX1) & 17(RX2), 16(TX2) & 15(RX3), 14(TX3) &                  \\
\hline
Mega 2560 Rev3             & 0(RX), 1(TX) & 19(RX1), 18(TX1) & 17(RX2), 16(TX2) & 15(RX3), 14(TX3) &                  \\
\hline
Nano 33 IoT                &              & 0(RX0), 1(TX0)   &                  &                  &                  \\
\hline
Nano RP2040 Connect        &              & 0(RX0), 1(TX0)   &                  &                  &                  \\
\hline
Nano BLE / BLE Sense       &              & 0(RX0), 1(TX0)   &                  &                  & \\
\hline
\end{tabular}
\caption{Последовательные порты на Arduino.}
\label{table:serial-ports--pins}
\end{table}

У каждого последовательного порта есть ассоциированные с ним цифровые порты
Arduino.  На большинстве плат Arduino для последовательного порта по-умолчанию
используются порты 0 (RX) и 1 (TX.)  При передачи данных между компьютером и
Arduino используется именно этот \texttt{Serial}; таким образом передаваемые
значения дублируются на цифровых портах, связанных с данным последовательном
портом.



\subsubsection{Класс \texttt{SoftwareSerial}}

Встроенная библиотека
\texttt{SoftwareSerial}\footnote{\url{https://docs.arduino.cc/learn/built-in-libraries/software-serial/}}
позволяет программно реализовать протокол последовательного порта на цифровых
портах, отличных от тех, что указаны в таблице \ref{table:serial-ports--pins}.

С помощью данной библиотеки можно реализовать несколько последовательных портов
со скоростями до 115200 бод.

Тем не менее, \texttt{SoftwareSerial} имеет ряд ограничений:
\begin{itemize}
\item Библиотека позволет в каждый момент времени либо передавать, либо
  принимать данные -- одновременный приём и передача не возможны. Таким образом
  можно сказать, что коммуникация выполняется в \emph{полу-дуплексном режиме}.
\item Если используется несколько \texttt{SoftwareSerial}-портов, то только один
  из них может принимать данные в каждый момент времени.
\item Для \texttt{SoftwareSerial} подходят только те порты, которые поддерживают
  прерывания.
\item На платах Arduino и Genuino 101 максимальная скорость передачи данных
  составляет 57600 бод.
\item На платах Arduino и Genuino 101 порт номер 13 не может быть использован в
  роли RX-порта.
\end{itemize}

Пример создания программного последовательно порта показан ниже.

\begin{listing}[H]
  \begin{minted}{cpp}
    #include <SoftwareSerial.h>

    const byte RX_PIN = 2;
    const byte TX_PIN = 3;

    SoftwareSerial my_serial(RX_PIN, TX_PIN);

    void setup() {
      my_serial.println("Hello, World!");
    }

    void loop() {

    }
  \end{minted}
  \label{listing:communication-serial-software}
  \caption{Пример использования программного последовательного порта
    (\texttt{SoftwareSerial}.)}
\end{listing}

\end{document}
