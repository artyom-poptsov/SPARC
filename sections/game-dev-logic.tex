\documentclass[../sparc.tex]{subfiles}
\graphicspath{{\subfix{../images/}}}
\begin{document}

%%%%%%%%%%%%%%%%%%%%%%%%%%%%%%%%%%%%%%%%%%%%%%%%%%%%%%%%%%%%%%%%%%%%%%%%%%%%%%%%
\section{Игровая логика}

В большинстве игр обычно есть определённые условия выигрыша и проигрыша, по
которым программа определяет исход игрового процесса.  В данном разделе мы
научимся реализовывать игровую логику, делать подсчёт очков, добавлять
дополнительные условия для выигрыша и проигрыша -- в общем, изучим всё то, что
сделает нашу игру интересной.

%%%%%%%%%%%%%%%%%%%%%%%%%%%%%%%%%%%%%%%%%%%%%%%%%%%%%%%%%%%%%%%%%%%%%%%%%%%%%%%%
\subsection{Условие выигрыша}

В нашем игровом процессе выигрыш будет засчитываться тогда, когда игрок войдёт в
некоторую дверь.  Дверь, как и всё остальное в игре, будет представлена
специальным символом -- это может быть встроенный символ, или же нарисованный
нами.  Пусть для начала дверь будет обозначаться символом ``D'':

\begin{minted}{cpp}
  const char SPACE = ' ';    // Пустое пространство.
  const char WALL  = '#';    // Стена.
  const char HP    = '*';    // Аптечка.
  const char DOOR  = 'D';    // Дверь.
\end{minted}

Создадим функцию \texttt{is\_door} для проверки наличия двери в клетке:

\begin{minted}{cpp}
  // Функция, возвращающая 1 (true) в случае, если
  // на клетке карты по координатам x,y находится дверь.
  // В противном случае функция возвращает 0 (false).
  bool is_door(int x, int y) {
    return game_map[y][x] == DOOR;
  }
\end{minted}

Используя данную функцию, мы можем реализовать проверку, вошёл ли игрок в дверь.
Когда игрок входит в дверь, то вызывается функция \texttt{win}:

\begin{minted}{cpp}
  void loop() {
    if (digitalRead(BUTTON_R) == LOW) {
      // ...
    }

    if (digitalRead(BUTTON_L) == LOW) {
      // ...
    }

    if (is_hp(player_x, player_y)) {
      // ...
    }

    if (is_door(player_x, player_y)) {
      win();
    }

    // Вызов функции отрисовки карты на экране дисплея.
    map_show();

    // ...
  }
\end{minted}

Функция \texttt{win} может быть реализована следующим образом:

\begin{minted}{cpp}
  void win() {
    lcd.setCursor(0, 0);
    lcd.print("Y O U  W I N!");
    player_x = 0;
    player_y = 0;
    generate_map();
    delay(1000);
  }
\end{minted}

Как можно видеть, при выигрыше мы отображаем надпись ``Y O U  W I N'', затем
происходит ``сброс'' игры в исходное состояние путём зануления координат
персонажа и повторной генерации карты.

%%%%%%%%%%%%%%%%%%%%%%%%%%%%%%%%%%%%%%%%%%%%%%%%%%%%%%%%%%%%%%%%%%%%%%%%%%%%%%%%
\subsection{Условие проигрыша}

Проигрыш в игре можно сделать, например, при столкновении игрока с монстром.
Обозначим монстра в виде константы:

\begin{minted}{cpp}
  const char SPACE   = ' ';    // Пустое пространство.
  const char WALL    = '#';    // Стена.
  const char HP      = '*';    // Аптечка.
  const char DOOR    = 'D';    // Дверь.
  const char MONSTER = 'M';  // Монстр.
\end{minted}

Далее добавим функцию проверки наличия монстра в клетке карты:

\begin{minted}{cpp}
  // Функция, возвращающая 1 (true) в случае, если
  // на клетке карты по координатам x,y находится дверь.
  // В противном случае функция возвращает 0 (false).
  bool is_monster(int x, int y) {
    return game_map[y][x] == MONSTER;
  }
\end{minted}

В функцию \texttt{loop} добавим проверку столкновения игрока с монстром:

\begin{minted}{cpp}
  void loop() {
    if (digitalRead(BUTTON_R) == LOW) {
      // ...
    }

    if (digitalRead(BUTTON_L) == LOW) {
      // ...
    }

    if (is_hp(player_x, player_y)) {
      // ...
    }

    if (is_door(player_x, player_y)) {
      win();
    }

    if (is_monster(player_x, player_y)) {
      game_over();
    }

    // Вызов функции отрисовки карты на экране дисплея.
    map_show();

    // ...
  }
\end{minted}

Простейшая реализация функциии проигрыша \texttt{game\_over} не сильно
отличается от функции выигрыша \texttt{win}:

\begin{minted}{cpp}
  void game_over() {
    lcd.setCursor(0, 0);
    lcd.print("G A M E  O V E R");
    player_x = 0;
    player_y = 0;
    generate_map();
    delay(1000);
  }
\end{minted}

%%%%%%%%%%%%%%%%%%%%%%%%%%%%%%%%%%%%%%%%%%%%%%%%%%%%%%%%%%%%%%%%%%%%%%%%%%%%%%%%
\subsection{Дополнительные квесты}

%%%%%%%%%%%%%%%%%%%%%%%%%%%%%%%%%%%%%%%%%%%%%%%%%%%%%%%%%%%%%%%%%%%%%%%%%%%%%%%%
\subsection{Подсчёт отчков}

%%%%%%%%%%%%%%%%%%%%%%%%%%%%%%%%%%%%%%%%%%%%%%%%%%%%%%%%%%%%%%%%%%%%%%%%%%%%%%%%
\subsection{Отображение очков с помощью светодиодов}
