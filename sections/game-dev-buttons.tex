\documentclass[../sparc.tex]{subfiles}
\graphicspath{{\subfix{../images/}}}
\begin{document}

%%%%%%%%%%%%%%%%%%%%%%%%%%%%%%%%%%%%%%%%%%%%%%%%%%%%%%%%%%%%%%%%%%%%%%%%%%%%%%%%
\section{Реализация управления}

Для управления игровым персонажем нам потребуется некоторое устройство ввода
информации в наш компьютер (микроконтроллер).  Мы сделаем своё ``устройство
ввода'', состоящее из нескольких \emph{тактовых кнопок}, расположенных на
макетной плате.

``Тактовыми'' называются кнопки, которые не ``запоминают'' своё состояние, и
сразу же после нажатия возвращаются в исходное состояние (как правило,
размкнутое.)

\subsection{Подключение кнопки}

\subsection{Обработка нажатий}

Самым простым для нас способом обработки нажатий является поочерёдный опрос
кнопок.

\end{document}
