\documentclass[../sparc.tex]{subfiles}
\graphicspath{{\subfix{../images/}}}
\begin{document}

%%%%%%%%%%%%%%%%%%%%%%%%%%%%%%%%%%%%%%%%%%%%%%%%%%%%%%%%%%%%%%%%%%%%%%%%%%%%%%%%
\section{Реализация управления}

Для управления игровым персонажем нам потребуется некоторое устройство ввода
информации в наш компьютер (микроконтроллер).  Мы сделаем своё ``устройство
ввода'', состоящее из нескольких кнопок, расположенных на макетной плате.

\subsection{Подключение кнопки}

Попробуем подключить кнопку к Arduino.  Кнопка по своей сути не более чем
замыкатель двух контактов.  Самым простым аналогом кнопки являются два
разъединённых провода (разомкнутая электрическая цепь), которые можно соединить
между собой, или опять разъединить.  Само собой, такой способ управления
устройством неудобен (и даже небезопасен, при больших напряжениях и токах в
электрической цепи), поэтому кнопки обычно представлены некими закрытыми
устройствами, которые имеют некий способ замыкать цепь без необходимости брать в
руки концы проводов.

Кнопки по принципу работы бывают разные.  Самый простой их вид -- \emph{тактовые
кнопки}.

``Тактовыми'' называются кнопки, которые не ``запоминают'' своё состояние, и
сразу же после нажатия возвращаются в исходное состояние (как правило,
разомкнутое.)

Другой вид кнопок, с которыми мы сталкиваемся в быту -- это те, которые
``запоминают'' своё состояние; такие кнопки обычно называются
``переключателями''.

%% \begin{figure}[ht]
%%   \centering
%%   \begin{circuitikz}
%%     \draw (3.5, 0) node[
%%       dipchip,
%%       num pins=2,
%%       external pins width=0.0,
%%       no topmark,
%%       hide numbers,
%%       xscale = 2.5,
%%       yscale = 2.5](C1){Arduino};
%%     \node [above left, font=\small] at (C1.bpin 1) {5V};
%%     \node [above right, font=\small] at (C1.bpin 2) {GND};
%%     \draw
%%     (C1.bpin 1) to[short]
%%     (0, 0) to[short]
%%     (0, 4) to[resistor, l=$R_1$] (3, 4)
%%     (3, 4) to[full led, l=Светодиод] (7, 4)
%%     (7, 4) to[short]
%%     (7, 0) to[short]
%%     (C1.bpin 2);
%%   \end{circuitikz}
%%   \caption{Схема подключения светодиода к Arduino.}
%%   \label{fig:electronics-arduino-circuit-00}
%% \end{figure}

\subsection{Обработка нажатий}

Самым простым для нас способом обработки нажатий является поочерёдный опрос
кнопок.

\end{document}
