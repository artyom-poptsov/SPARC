\documentclass[../sparc.tex]{subfiles}
\graphicspath{{\subfix{../images/}}}
\begin{document}

%%%%%%%%%%%%%%%%%%%%%%%%%%%%%%%%%%%%%%%%%%%%%%%%%%%%%%%%%%%%%%%%%%%%%%%%%%%%%%%%
\section{Переменные}
\index{Программирование!Переменная}

\emph{Переменная} -- это ключевое понятие в программировании. Любая программа
работает с данными. Возьмём для наглядности некую программу- калькулятор,
умеющую складывать два числа. Чтобы микроконтроллер мог работать с этими числами
их нужно где-то хранить. Где? В оперативной памяти. Все данные, которые
используются микроконтроллером во время работы, хранятся именно там. Для работы
нашего калькулятора нужно загрузить в ячейки оперативной памяти два
числа-операнда, которые нужно сложить, например 15 и 3:

\begin{tabular}{p{4cm}|p{6cm}}
  Адрес ячейки & Значение ячейки \\
  \hline \hline
  0000 & 15 \\
  \hline
  0001 & 3 \\
  \hline
  0003 & 0 \\
  ... & ... \\
\end{tabular}

\emph{Переменная} -- это ячейка данных в оперативной памяти (ОЗУ). Объявить
переменную -- значит сказать компьютеру выделить какую-нибудь ячейку памяти для
наших нужд.

Переменная в языке С++ имеет определённый тип и уникальное имя. Объявление
(\emph{инициализация}) переменной выглядит следующим образом:

\begin{listing}[ht]
  \begin{verbatim}
    тип имя = значение;
  \end{verbatim}
  \label{listing:dialogues-with-computer-variable-definition-structure}
  \caption{Общая структура объявления переменной.}
\end{listing}

То есть, чтобы загрузить в оперативную память два числа 15 и 3, мы должны написать
следующее:

\begin{listing}[ht]
  \begin{minted}{cpp}
    int a = 15;
    int b = 3;
  \end{minted}
  \label{listing:dialogues-with-computer-variable-definition-example}
  \caption{Пример объявления переменных.}
\end{listing}

Слово \texttt{int} это тип переменной, означает, что эта переменная является
числом.

Также следует объявить переменную для хранения результата сложения:

Дальше -- складываем значения двух переменных \texttt{a} и \texttt{b}:

\begin{listing}[ht]
  \begin{minted}{cpp}
    int result = a + b;
  \end{minted}
  \label{listing:dialogues-with-computer-variable-operations-example}
  \caption{Пример операции над переменными.}
\end{listing}

Здесь мы присвоили переменной \texttt{result} результат операции сложения двух
переменных.

Заметьте, что имя переменной должно отражать её предназначение, чтобы код
программы был легко читаем для человека.

\emph{ВАЖНО!} Имя переменной может состоять только из букв, цифр и нижнего
подчёркивания, причём имя не может начинаться с цифры.  Мы между словами в
названии переменных в данной книге будем всегда использовать символ нижнего
подчёркивания.  Например, если мы хотим задать время задержки между включением и
выключением светодиодов, то логичным названием для переменной будет ``значение
задержки'', или ``delay value'' по-английски.  Тогда название переменной будет
следующим:

\begin{listing}[ht]
  \begin{minted}{cpp}
    int delay_value = 500;
  \end{minted}
  \label{listing:dialogues-with-computer-variable-names}
  \caption{Разделение слов в названии переменной символом нижнего
    подчёркивания.}
\end{listing}

Далее значение переменной \texttt{delay\_value} мы подставим в программе мигания
светодиодом, заменив ей явно указанное значение задержки в функции
\texttt{delay}:

\begin{listing}[ht]
  \begin{minted}{cpp}
    void loop() {
      int delay_value = 500;
      digitalWrite(2, HIGH);
      delay(delay_value);
      digitalWrite(2, LOW);
      delay(delay_value);
    }
  \end{minted}
  \label{listing:dialogues-with-computer-variable-usage}
  \caption{Пример подстановки значения переменной в программе.}
\end{listing}

Таким образом, мы сможем поменять значения всех задержек одной заменой значения
\texttt{delay\_val}:

\begin{listing}[ht]
  \begin{minted}{cpp}
    void loop() {

      int delay_value = 600; // <--

      digitalWrite(2, HIGH);
      delay(delay_value);
      digitalWrite(2, LOW);
      delay(delay_value);
    }
  \end{minted}
  \label{listing:dialogues-with-computer-variable-set-value}
  \caption{Если поменять значение переменной, то везде, где она используется,
    значение также поменяется..}
\end{listing}

Можно, например, увеличивать \texttt{delay\_value} на 100 при каждом выполнении
\texttt{loop}:

\begin{listing}[ht]
  \begin{minted}{cpp}
    void loop() {
      int delay_value = 100;
      digitalWrite(2, HIGH);
      delay(delay_value);
      digitalWrite(2, LOW);
      delay_value = delay_value + 100; // <--
      delay(delay_value);
    }
  \end{minted}
  \label{listing:dialogues-with-computer-variable-change}
  \caption{Если поменять значение переменной, то везде, где она используется,
    значение также поменяется.}
\end{listing}

В примере \ref{listing:dialogues-with-computer-variable-change} в первом
\texttt{delay} используется значение переменной 100, но во втором \texttt{delay}
значение уже будет 200.

Кстати, строчку \texttt{delay\_value = delay\_value + 100} можно заменить на
\texttt{delay\_value += 100} и результат будет тем же, но запись короче.

\texttt{+=} -- оператор присваивания, совмещённый со сложением.

Существуют также другие операторы подобного рода -- например, ``-='' (читается
``минус-равно''.)

%%%%%%%%%%%%%%%%%%%%%%%%%%%%%%%%%%%%%%%%%%%%%%%%%%%%%%%%%%%%%%%%%%%%%%%%%%%%%%%%
\subsection{Область видимости}
\index{Программирование!Переменная!Область видимости}

Обратите внимание, что при повторении функции \texttt{loop} значение переменной
всегда сбрасывается к исходному -- то есть, к 100.  Это происходит потому, что
после завершения выполнения \texttt{loop} переменная \texttt{delay\_value}
удаляется, а при перезапуске \texttt{loop} она создаётся заново.

У переменных, как и у других элементов программы, есть так называемая
\emph{область видимости}, которая определяет, где данная переменная ``видна'' --
иными словами, из какой части программы к ней можно обратиться.

В примере \ref{listing:dialogues-with-computer-variable-change} переменная
\texttt{delay\_value} является \emph{локальной переменной} по отношению к
функции \texttt{loop}.  Локальные переменные существуют, пока функция, где они
были объявлены, выполняется.

Если же мы хотим, чтобы значение переменной сохранялось при перезапусках функции
\texttt{loop}, мы должны сделать её \emph{глобальной переменной} -- то есть,
вынести за пределы функции, поместив её где-нибудь выше функции \texttt{loop}
(но не в \texttt{setup}, так если мы поместим объявление переменной в
\texttt{setup}, её область видимости будет ограничена этой функцией -- иными
слвоами, мы не сможем к ней обратиться из функции \texttt{loop}!)

\begin{listing}[ht]
  \begin{minted}{cpp}
    int delay_value = 100;   // <--

    void loop() {
      digitalWrite(2, HIGH);
      delay(delay_value);
      digitalWrite(2, LOW);
      delay(delay_value);

      delay_value += 100;
    }
  \end{minted}
  \label{listing:dialogues-with-computer-global-variable}
  \caption{Пример объявления глобальной переменной.}
\end{listing}

Теперь всё будет работать. Но так задержка будет бесконтрольно расти. Решением
будет сделать так, чтобы \texttt{delay\_value} увеличивалась до какого-то
порогового значения, например, до 600. Для этого нужно добавить условие
(\texttt{if}):

\begin{listing}[ht]
  \begin{minted}{cpp}
    int delay_value = 100;

    void loop() {
      digitalWrite(2, HIGH);
      delay(delay_value);
      digitalWrite(2, LOW);
      delay(delay_value);

      if (delay_value < 600){
        delay_value += 100;
      }
    }
  \end{minted}
  \label{listing:dialogues-with-computer-global-variable-2}
  \caption{Пример работы с глобальной переменной.}
\end{listing}

Об условиях и других управляющих конструкциях будет подробнее рассказано в главе
``Управляющие конструкции''.

%%%%%%%%%%%%%%%%%%%%%%%%%%%%%%%%%%%%%%%%%%%%%%%%%%%%%%%%%%%%%%%%%%%%%%%%%%%%%%%%
\section{Константы}
\index{Программирование!Константа}

\emph{Константа} -- это некоторое фиксированное, неизменяемое значение.
Константы бывают двух видов: безымянные и именованные.  Безымянные константы,
как следует из их названия (вот вам и каламбур!) не имеют имени в программе, и
представляют собой явно записанные значения -- например, значение ``500'' в коде
ниже является безымянной константой:

\begin{listing}[ht]
  \begin{minted}{cpp}
    delay(500);
  \end{minted}
  \label{listing:dialogues-with-computer-unnamed-const}
  \caption{Безымянная константа 500.}
\end{listing}

\emph{Именованные константы} же имеют явно заданное имя.  Например, в
приведённом ниже примере кода \texttt{D} является именованной константой, равной
числу 500:

\begin{listing}[ht]
  \begin{minted}{cpp}
    void loop() {
      const int D = 500;
      digitalWrite(2, HIGH);
      delay(D);
      digitalWrite(2, LOW);
      delay(D);
    }
  \end{minted}
  \label{listing:dialogues-with-computer-const}
  \caption{Пример объявления константы с именем ``D''.}
\end{listing}

Обратите внимание, что структура объявления константы практически повторяет
структуру объявления переменных, добавляется только слово ``const'' перед типом.

\begin{listing}[ht]
  \begin{verbatim}
    const тип имя = значение;
  \end{verbatim}
  \label{listing:dialogues-with-computer-const-definition-structure}
  \caption{Общая структура объявления константы.}
\end{listing}

Как и в случае с переменными, мы можем ссылаться на константу по её имени;
основное различие между переменными и константами в том, что после объявления
константу нельзя поменять.  Попытка изменить константу после объявления является
ошибкой в языке программирования C++.

\end{document}
