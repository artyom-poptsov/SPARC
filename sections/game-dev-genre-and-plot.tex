\documentclass[../sparc.tex]{subfiles}
\graphicspath{{\subfix{../images/}}}
\begin{document}

%%%%%%%%%%%%%%%%%%%%%%%%%%%%%%%%%%%%%%%%%%%%%%%%%%%%%%%%%%%%%%%%%%%%%%%%%%%%%%%%
\section{Выбор жанра и разработка сюжета}
\index{Разработка игр!Жанр и сюжет}

Разработка игры, как и любого другого серьёзного проекта, как правило,
начинается с проектирования -- выбора жанра, разработки сюжета.

В плане жанра у нас очень широкий выбор опций: здесь и ролевые игры
(Role-Playing Game, ``RPG''), и игры-''стрелялки'' (First-Person Shooter,
``FPS''), платформеры, аркады и т.п.

Предпочтение одних жанров игр другим -- очень субъективная вещь, так что сразу
оговоримся, что наши решения в рамках данной книги ни в коем случае не должны
убеждать вас, что они являются единственно верными.  Вы можете использовать
навыки, полученные в рамках данной практики, для разработки игр других жанров.

В качестве жанра мы выберем игру-платформер, поскольку это перекликается со
многими старыми играми.

Одна из игр, которую мы будем брать, как источник вдохновения -- ``Dangerous
Dave in the Haunted Mansion'' 1991-го года, разработанную компанией ``id
Software''.

Сюжет нашей игры будет разворачивается в некотором имении, которое кишит
различными агрессивными монстрами.  Наша задача будет пройти через ряд уровней,
чтобы добраться до заветной двери, которая должна вывести нас из здания.

\end{document}
