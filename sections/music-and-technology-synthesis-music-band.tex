\documentclass[../sparc.tex]{subfiles}
\graphicspath{{\subfix{../images/}}}
\begin{document}

%%%%%%%%%%%%%%%%%%%%%%%%%%%%%%%%%%%%%%%%%%%%%%%%%%%%%%%%%%%%%%%%%%%%%%%%%%%%%%%%
\newpage
\section{Создание музыкальной группы}

Теперь, когда мы понимаем, как устроен ритм, и умеем читать ноты, то можно
приступить к формированию музыкальной ``группы'' из двух или более Arduino.

Многие музыкальные композиции, которые мы могли видеть, включают в себя две или
более музыкальные дорожки.  Часто это партия для второй руки при игре на
пианино, либо же несколько разных музыкальных инструментов (например, гитара и
вокал.)

В этой ситуации самый простой для нас способ сыграть композицию в целом -- это
взять несколько микроконтроллеров Arduino и научить их играть совместно.

%%%%%%%%%%%%%%%%%%%%%%%%%%%%%%%%%%%%%%%%%%%%%%%%%%%%%%%%%%%%%%%%%%%%%%%%%%%%%%%%
\subsection{Выбор названия}

В первую очередь, для создания музыкальной группы нам надо придумать ей хорошее
название.  Пусть в нашем случае названием будет ``Second Hand Of Darkness''
(сокращённо ``SHOD'') -- в переводе с английского ``Вторая рука тьмы''.  Такое
название отлично подходит для рок-группы, да и просто звучит круто.

Далее это название надо правильно стилизовать, чтобы нас совсем приняли за
``своего'' в рядах профессиональных музыкантов. \footnote{Для названия
используется шрифт ``Guttural'', распространяемый под лицензией Creative Commons
Attribution: \url{https://www.fontspace.com/guttural-font-f8257}.}

\begin{figure}[ht]
\centering
\includesvg[width=10.00cm]{shod}
\caption{Логотип нашей музыкальной группы.}
\label{fig:shod-band-logo}
\end{figure}

Подготовительные этапы выполнены. Теперь надо выбрать музыкальную композицию и
сыграть её.

%%%%%%%%%%%%%%%%%%%%%%%%%%%%%%%%%%%%%%%%%%%%%%%%%%%%%%%%%%%%%%%%%%%%%%%%%%%%%%%%
\subsection{Выбор дебютной композиции}

Многие музыкальные группы начинают свой путь в ``большую музыку'' с исполнения
каверов на известные композиции.  Пусть и нашей первой композицией будет
Arduino-кавер на композицию ``Sweet Dreams'' под авторством Marilyn Manson
(см. рис.  \ref{fig:lilypond-melody-sweet-dreams}.)\footnote{За основу взята
партитура \url{https://musescore.com/user/13484026/scores/5329623}, выполненная
пользователем ``Instrumental Rock'', лицензия CC-BY-NC
(\url{https://creativecommons.org/licenses/by-nc/4.0/}.)}

Данная композиция имеет две основных музыкальных ``дорожки'' --- первая является
вокалом, вторая представляет собой гитарную партию.

\begin{figure}[ht]
  \begin{lilypond}
    <<
    \new Staff \with { instrumentName = "Вокал" } {
      \relative c' {
        \tempo 4 = 93
        \key es \major
        \numericTimeSignature
        \time 4/4
        %% 01
        \bar ".|:" r1 |
        %% 02
        r1 \bar ":|."
        %% 03
        r4 es'4 es4 c8 es8~ |
        %% 04
        es8 es4. d2 |
        %% 05
        es4 c8 es4. c4 |
        %% 06
        es8 f4. d4. c8 |
        %% 07
        es8 es8 c8 es4. c8 c8 |
        %% 08
        es8 es4. d2 |
        %% 09
        es4 c8 es4 c4. |
        %% 10
        es8 es8 f8 es8 d8 d4. |
        %% 11
        es8 es8 c4 es4 c8 es8~ |
        %% 12
        es8 c4. r2 |
      }
    }
    \new Staff  \with { instrumentName = "Гитара" } {
      \relative c' {
        \key es \major
        \numericTimeSignature
        \time 4/4
        %% 01
        \bar ".|:" c8 c g'8 c,8 as'8 c,8 g'8 c,8 |
        %% 02
        as8 as es'8 f8 g,8 g8 d'8 es8 \bar ":|."
        %% 03
        c8 c g'8 c,8 as'8 c,8 g'8 c,8 |
        %% 04
        as8 as es'8 f8 g,8 g8 d'8 es8 |
        %% 05
        c8 c g'8 c,8 as'8 c,8 g'8 c,8 |
        %% 06
        as8 as es'8 f8 g,8 g8 d'8 es8 |
        %% 07
        c8 c g'8 c,8 as'8 c,8 g'8 c,8 |
        %% 08
        as8 as es'8 f8 g,8 g8 d'8 es8 |
        %% 09
        c8 c g'8 c,8 as'8 c,8 g'8 c,8 |
        %% 10
        as8 as es'8 f8 g,8 g8 d'8 es8 |
        %% 11
        c8 c g'8 c,8 as'8 c,8 g'8 c,8 |
        %% 12
        as8 as es'8 f8 g,8 g8 d'8 es8 |
      }
    }
    >>
    \layout {
      indent = 0\mm
      line-width = 120\mm
      ragged-last = ##t
    }
  \end{lilypond}
  \caption{Marilyn Manson, Trent Reznor, ``Sweet Dreams''.}
  \label{fig:lilypond-melody-sweet-dreams}
\end{figure}

%%%%%%%%%%%%%%%%%%%%%%%%%%%%%%%%%%%%%%%%%%%%%%%%%%%%%%%%%%%%%%%%%%%%%%%%%%%%%%%%
\subsection{Реализация мелодии}

Для того, чтобы реализовать первую ``дорожку'' нашей мелодии, нам следует
создать музыкальные проект для Arduino, как мы делали до этого.  Воспроизведение
звуков по-прежнему будем делать через нашу функцию \texttt{play\_tone}.

Обе части композиции будут заданы в виде двумерных массивов.  Первый массив
будет играть басовую часть.  Обратите внимание, что основная тема композиции
повторяется.

\begin{minted}{cpp}
  float sweet_dreams_guitar[28][2] = {
    /* 01 */
    {c4,  8}, {c4,  8}, {g4,  8}, {с4,  8},
    {a4b, 8}, {c4,  8}, {g4,  8}, {с4,  8},
    /* 02 */
    {a3b, 8}, {a3b, 8}, {e4b, 8}, {f4,  8},
    {g3,  8}, {g3,  8}, {d4,  8}, {e4b, 8}
    /* 03 */
    {a3b, 8}, {a3b, 8}, {e4b, 8}, {f4,  8},
    {g3,  8}, {g3,  8}, {d4,  8}, {e4b, 8}
    /* 04 */
    {a3b, 8}, {a3b, 8}, {e4b, 8}, {f4,  8},
    {g3,  8}, {g3,  8}, {d4,  8}, {e4b, 8}
    /* 05 */
    {a3b, 8}, {a3b, 8}, {e4b, 8}, {f4,  8},
    {g3,  8}, {g3,  8}, {d4,  8}, {e4b, 8}
    /* 06 */
    {a3b, 8}, {a3b, 8}, {e4b, 8}, {f4,  8},
    {g3,  8}, {g3,  8}, {d4,  8}, {e4b, 8}
    /* 07 */
    {c4,  8}, {c4,  8}, {g4,  8}, {с4,  8},
    {a4b, 8}, {c4,  8}, {g4,  8}, {с4,  8},
    /* 08 */
    {a3b, 8}, {a3b, 8}, {e4b, 8}, {f4,  8},
    {g3,  8}, {g3,  8}, {d4,  8}, {e4b, 8}
    /* 09 */
    {a3b, 8}, {a3b, 8}, {e4b, 8}, {f4,  8},
    {g3,  8}, {g3,  8}, {d4,  8}, {e4b, 8}
    /* 10 */
    {a3b, 8}, {a3b, 8}, {e4b, 8}, {f4,  8},
    {g3,  8}, {g3,  8}, {d4,  8}, {e4b, 8}
    /* 11 */
    {a3b, 8}, {a3b, 8}, {e4b, 8}, {f4,  8},
    {g3,  8}, {g3,  8}, {d4,  8}, {e4b, 8}
    /* 12 */
    {a3b, 8}, {a3b, 8}, {e4b, 8}, {f4,  8},
    {g3,  8}, {g3,  8}, {d4,  8}, {e4b, 8}
  };
\end{minted}

Вторая часть мелодии (вокал) будет реализована аналогичным образом.

\begin{minted}{cpp}
  float sweet_dreams_vocals[28][2] = {
    /* 01 */
    {R, 1},
    /* 02 */
    {R, 1},
    /* 03 */
    {R, 4}, {e5b, 4}, {e5b, 4}, {c5, 8}, {e5b, 4},
    /* 04 */
    {e5b, 8.0 / 3.0}, {d5, 2},
    /* 05 */
    {e5b, 4}, {c5, 8}, {e5b, 8.0 / 3.0}, {c5, 4},
    /* 06 */
    {e5b, 8}, {f5, 8.0 / 3.0}, {d5, 8.0 / 3.0}, {c5, 8},
    /* 07 */
    {e5b, 8}, {e5b, 8}, {c5, 8}, {e5b, 8.0 / 3.0},
    /* 08 */
    {e5b, 8}, {e5b, 8.0 / 3.0}, {d5, 2},
    /* 09 */
    {e5b, 4}, {c5, 8}, {e5b, 4}, {c5, 8.0 / 3.0},
    /* 10 */
    {e5b, 8}, {e5b, 8}, {f5, 8}, {e5b, 8}, {d5, 8}, {d5, 8.0 / 3.0},
    /* 11 */
    {e5b, 8}, {e5b, 8}, {c5, 4}, {e5b, 4}, {c5, 8}, {e5b, 4},
    /* 12 */
    {c5, 8.0 / 3.0}, {R, 2}
  };
\end{minted}

При работе с двумя Arduino возникает проблема -- как загрузить в каждую из них
свою мелодию.  Самый очевидный, однако не самый удобный вариант -- это сделать
отдельный проект для каждой из них, и загружать по-очереди.  Неудобство тут в
том, что придётся каждый раз следить, чтобы в каждую из Arduino был загружен
свой проект.

Мы можем предложить более удобный способ -- использовать один проект для обеих
Arduino, при этом по значению на одном из цифровых портов определять, какая из
мелодий должна воспроизводиться.  Ставя перемычку (провод), замыкающую порт
\texttt{SWITCH\_PORT} на землю, мы говорим Arduino воспроизводить гитарную
партию.  Если же перемычки нет, то тогда воспроизводится локальная партия.
Таким образом, мы при любом раскладе получим воспроизведение разных дорожек
мелодии на разных Arduino.

\begin{minted}{cpp}
  // Порт, куда подключен динамик.
  const int SPEAKER     = 2;
  // Порт-переключатель мелодии.
  const int SWITCH_PORT = 10;

  void setup() {
    pinMode(SPEAKER, OUTPUT);
    pinMode(SWITCH_PORT, INPUT_PULLUP);
  }

  void loop() {
    const int BPM = 93;
    const long MINUTE = 60000000;
    const long T = (MINUTE / BPM) * 4;

    // Исходя из состояния порта SWITCH_PORT решаем,
    // какую из мелодий играть.
    if (digitalRead(SWITCH_PORT) == LOW) {
      int count = sizeof(sweet_dreams_guitar)
                      / sizeof(sweet_dreams_guitar[0]);
      for (int i = 0; i < count; i++) {
        play_tone(SPEAKER,
                  sweet_dreams_guitar[i][0],
                  T / sweet_dreams_guitar[i][1]);
        delay(10);
      }
    } else {
      int count = sizeof(sweet_dreams_vocals)
                      / sizeof(sweet_dreams_vocals[0]);
      for (int i = 0; i < count; i++) {
        play_tone(SPEAKER,
                  sweet_dreams_vocals[i][0],
                  T / sweet_dreams_vocals[i][1]);
        delay(10);
      }
    }
  }
\end{minted}

%%%%%%%%%%%%%%%%%%%%%%%%%%%%%%%%%%%%%%%%%%%%%%%%%%%%%%%%%%%%%%%%%%%%%%%%%%%%%%%%
\subsection{Синхронизация мелодий}

Для запуска мелодий одновременно можно сразу на двух Arduino нажать кнопки
сброса (``Reset'') и одновременно их отпустить.

Несмотря на то, что мы получили одновременное воспроизведение обеих частей
мелодии, со временем вы можете заметить, что они рассинхронизируются -- ноты
одной части начинают играть невпопад с другой частью.

Этому неприятному эффекту есть несколько причин.  Начнём их разбирать по мере
значимости их влияния на мелодию.

Первая причина в рассинхронизации кроется вот в этих строчках:

\begin{minted}{cpp}
for (int i = 0; i < count; i++) {
  play_tone(SPEAKER,
            sweet_dreams_vocals[i][0],
            T / sweet_dreams_vocals[i][1]);

  delay(10); // <----- Проблема здесь.
}
\end{minted}

Представим, что у нас воспроизводятся первые два такта мелодии
(см. \ref{fig:lilypond-melody-sweet-dreams-part}.)

\begin{figure}[h!]
  \begin{lilypond}
    <<
    \new Staff \with { instrumentName = "Вокал" } {
      \relative c' {
        \tempo 4 = 93
        \key es \major
        \numericTimeSignature
        \time 4/4
        %% 01
        \bar ".|:" r1 |
        %% 02
        r1 \bar ":|."
      }
    }
    \new Staff  \with { instrumentName = "Гитара" } {
      \relative c' {
        \key es \major
        \numericTimeSignature
        \time 4/4
        %% 01
        \bar ".|:" c8 c g'8 c,8 as'8 c,8 g'8 c,8 |
        %% 02
        as8 as es'8 f8 g,8 g8 d'8 es8 \bar ":|."
      }
    }
    >>
    \layout {
      indent = 0\mm
      line-width = 120\mm
      ragged-last = ##t
    }
  \end{lilypond}
  \caption{Marilyn Manson, Trent Reznor, ``Sweet Dreams'' (отрывок.)}
  \label{fig:lilypond-melody-sweet-dreams-part}
\end{figure}

Здесь видно, что в вокальной партии у нас две целых паузы -- по сути, две глухие
ноты, тишина.  Во второй же у нас суммарно за эти два такта воспроизводится $8 +
8 = 16 \mbox{нот}$.  После каждого воспроизведения ноты (т.е. после каждого
выполнения \texttt{play\_tone}) у нас явно прописана задержка в некоторое
количество миллисекунд.  Таким образом, в первом такте у нас суммарная задержка
будет:

\begin{equation}
  2 * 10 \mbox{мс} = 20 \mbox{мс}
\end{equation}

Во гитарной же партии общая задержка будет составлять:

\begin{equation}
  16 * 10 \mbox{мс} = 160 \mbox{мс}
\end{equation}

Получается разница в 140 мс между мелодиями, и данная погрешность будет
накапливаться со временем -- чем длиннее мелодия и больше различаются
длительности нот в двух её частях, тем более заметна будет рассинхронизация.

Чтобы решить данную проблему, мы должны избавиться от вызова фиксированного
\texttt{delay} и перейти к добавлению задержки между нотами, вычисляемой на
основе длины отдельно взятой ноты.  Удобнее всего это сделать внутри функции
\texttt{play\_tone}.

\begin{minted}{cpp}
  // Функция воспроизведения звука указанной частоты.
  void play_tone(int port, float f, long t) {
    if (f > 0) {
      const int T = 1000000 / f;
      int d = T / 2;
      int count = t / T;
      for (int i = 0; i < count; i++) {
        digitalWrite(port, HIGH);
        delayMicroseconds(d);
        digitalWrite(port, LOW);
        delayMicroseconds(d);
      }
    } else {
      delay(t / 1000); // Пауза
    }
    // Задержка после ноты составляет примерно 10%
    // от её длины:
    delayMicroseconds(t * 0.10);
  }
\end{minted}

После этих изменений в \texttt{play\_tone} надо поменять и код воспроизведения
мелодии, убрав \texttt{delay}:

\begin{minted}{cpp}
  // ...

  void loop() {
    const int BPM = 93;
    const long MINUTE = 60000000;
    const long T = (MINUTE / BPM) * 4;

    if (digitalRead(SWITCH_PORT) == LOW) {
      int count = sizeof(sweet_dreams_guitar)
                      / sizeof(sweet_dreams_guitar[0]);
      for (int i = 0; i < count; i++) {
        play_tone(SPEAKER,
                  sweet_dreams_guitar[i][0],
                  T / sweet_dreams_guitar[i][1]);
      }
    } else {
      int count = sizeof(sweet_dreams_vocals)
                      / sizeof(sweet_dreams_vocals[0]);
      for (int i = 0; i < count; i++) {
        play_tone(SPEAKER,
                  sweet_dreams_vocals[i][0],
                  T / sweet_dreams_vocals[i][1]);
      }
    }
  }
\end{minted}

Второй источник накопления ошибки -- это погрешность, возникающая при вычислении
длины периода на основе частоты в \texttt{play\_tone}.  Возьмём к примеру ноту
С4 -- её частота составляет 261.63 Гц.  Код, расчитывающий период колебания из
частоты у нас в \texttt{play\_tone} выглядит так:

\begin{minted}{cpp}
  const int T = 1000000 / f;
\end{minted}

Если мы подставим в эту формулу частоту ноты C4 и посчитаем период колебания, то
получим:

\begin{equation}
  1000000 \mbox{мкс} / 261.63 \mbox{Гц} = 3822.191644689 \mbox{мкс}
\end{equation}

Как видно из формулы, длина одного колебания для C4 не является целым числом,
при этом дробная часть будет отброшена (поскольку мы сохраням период в тип
данных \texttt{int}) и мы останемся со значением 3822 мкс.

Чтобы узнать количество повторений цикла, создающего колебания мембраны
динамика, мы делим длительность звука \texttt{t} на период \texttt{T}:

\begin{minted}{cpp}
  int count = t / T;
\end{minted}

Пусть длина нашей ноты будет 1с (1000000 мкс.)  Подставим в данную формулу
значение 3822 мкс и посмотрим, что получится:

\begin{equation}
  1000000 \mbox{мкс} / 3822 \mbox{Гц} = 261.643118786
\end{equation}

Опять же, дробная часть отбрасывается, и мы получаем значение 261 -- именно это
количество раз повториться наш цикл.  Каждое повторение цикла имеет длительность
примерно в 3822 мкс, если не учитывать накладные расходы на вызов функций.

Таким образом, повторив цикл колебания мембраны динамика 261 раз, мы получим
следующее общее время работы цикла:

\begin{equation}
  261 \mbox{мкс} * 3822 \mbox{мкс} = 997542 \mbox{мкс}
\end{equation}

Ошибка составляет:

\begin{equation}
  1000000 \mbox{мкс} - 997542 \mbox{мкс} = 2458 \mbox{мкс}
\end{equation}

Или примерно 2.5 мс.  Это довольно небольшое значение ошибки, и роль его может
быть незначительна.  К тому же, сам вызов функций \texttt{digitalWrite} и
\texttt{delayMicroseconds} тоже занимает некоторое время.  Если же мы всё-таки
хотим скомпенсировать данную ошибку, то после цикла в \texttt{play\_tone} нам
необходимо добавить дополнительную задержку, рассчитанную по формуле:

\begin{equation}
  \mbox{t} - (\mbox{T} * \mbox{count})
\end{equation}

Код с компенсацией ошибки будет выглядеть так:

\begin{minted}{cpp}
  // Функция воспроизведения звука указанной частоты.
  void play_tone(int port, float f, long t) {
    if (f > 0) {
      const int T = 1000000 / f;
      int d = T / 2;
      int count = t / T;
      for (int i = 0; i < count; i++) {
        digitalWrite(port, HIGH);
        delayMicroseconds(d);
        digitalWrite(port, LOW);
        delayMicroseconds(d);
      }

      // Компенсация ошибки:
      delayMicroseconds(t - (T * count));

    } else {
      delay(t / 1000); // Пауза
    }
    delayMicroseconds(t * 0.10);
  }
\end{minted}

\end{document}
