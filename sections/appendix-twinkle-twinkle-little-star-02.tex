\documentclass[../sparc.tex]{subfiles}
\graphicspath{{\subfix{../images/}}}
\begin{document}

%%%%%%%%%%%%%%%%%%%%%%%%%%%%%%%%%%%%%%%%%%%%%%%%%%%%%%%%%%%%%%%%%%%%%%%%%%%%%%%%
\newpage
\chapter{Мелодия ``Twinkle Twinkle Little Star'' (версия 2.0)}
\label{app:twinkle-twinkle-little-star-02}

В данном примере мы видим полный вариант программирования мелодии ``Twinkle
Twinkle, Little Star'', написанный с использованием одномерных массивов. Если
сравнить с кодом в приложении \ref{app:twinkle-twinkle-little-star-01}, то
видно, что программа стала занчительно короче; однако её можно сократить и даже
упростить в некотором роде ещё -- если использовать двумерные массивы.

\begin{minted}{cpp}
  // Ноты:
  const float c4 = 261.630;
  const float d4 = 293.660;
  const float e4 = 329.630;
  const float f4 = 349.230;
  const float g4 = 392.000;
  const float a4 = 440.000;
  const float b4 = 493.880;

  // Номер порта, куда подключен динамик:
  const int SPEAKER_PIN = 8;

  // Функция воспроизвдения звука на порту PORT
  // с частотой F и длительностью T.
  void play_tone(int port, float f, long t) {
    const long T = 1000000 / f;
    long d = T / 2;
    int count = t / T;
    for (int i = 0; i < count; i++) {
      digitalWrite(port, HIGH);
      delayMicroseconds(d);
      digitalWrite(port, LOW);
      delayMicroseconds(d);
    }
  }

  void setup() {
    pinMode(SPEAKER_PIN, OUTPUT);
  }

  // Массив с нотами (их частотами.)
  float melody[28] = {
    c4, c4, g4, g4,
    a4, a4, g4,
    f4, f4, e4, e4,
    d4, d4, c4,
    g4, g4, f4, f4,
    e4, e4, d4,
    g4, g4, f4, f4,
    e4, e4, d4,
  };

  // Массив с длительностями нот.
  float melody_t[28] = {
    4,  4,  4,  4,
    4,  4,  2,
    4,  4,  4,  4,
    4,  4,  2,
    4,  4,  4,  4,
    4,  4,  2,
    4,  4,  4,  4,
    4,  4,  2,
  };

  void loop() {
    const long BPM = 120; // Удары в минуту.
    const long MINUTE = 60000000; // 1 минута в микросекундах.
    const long T = (MINUTE / BPM) * 4; // Длина целой ноты.
    const byte D = 100; // Задержка между нотами.

    // Цикл воспроизведения мелодии из массива.
    for (int note_idx = 0; note_idx < 28; note_idx++) {
      play_tone(SPEAKER_PIN,
                melody[note_idx],
                T / melody_t[note_idx]);
      delay(D);
    }
  }
\end{minted}

\end{document}
