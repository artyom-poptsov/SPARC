\documentclass[../main.tex]{subfiles}
\graphicspath{{\subfix{../images/}}}
\begin{document}

%%%%%%%%%%%%%%%%%%%%%%%%%%%%%%%%%%%%%%%%%%%%%%%%%%%%%%%%%%%%%%%%%%%%%%%%%%%%%%%%
\section{Основы работы с мультиметром}
\index{Электроника!Мультиметр}

Мультиметр -- незаменимый прибор, с его помощью можно узнать сопротивление
резистора, измерить напряжение, произвести проверку на проводимость
(``прозвонка''), узнать цвет и полярность светодиода и многое другое.

На рис. \ref{fig:multimeter-example} показан один из вариантов мультиметра.

\begin{figure}[ht]
  \centering
  \caption{Пример мультиметра.}
  \includegraphics[width=12cm]{Digital_Multimeter}
  \label{fig:multimeter-example}
\end{figure}

Далее приведена таблица на которой отражены основные символы, встречающиеся на
корпусе прибора, необходимые для работы с мультиметром:

%% TODO: Добавить описание функций мультиметра.

\experiment{0} { Попробуйте померять с помощью мультиметра сопротивление
  резисторов, соединяя их последовательно и параллельно через макетную плату.
  Как меняется сопротивление собранной цепи? }

\experiment{1} { Померяйте сопротивление других проводников -- например, проводов
  для макетной платы Arduino, металлических предметов. Какое у них
  сопротивление?}

\experiment{2} { Известно, что графит является проводником. Померяйте
  сопротивление графитового стержня в карандаше.  Также попробуйте на листе
  бумаги нарисовать жирную графитовую линию и померяйте её сопротивление.  Какое
  сопротивление вы видите? }

\end{document}
