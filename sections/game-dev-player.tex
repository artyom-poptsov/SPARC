\documentclass[../sparc.tex]{subfiles}
\graphicspath{{\subfix{../images/}}}
\begin{document}

%%%%%%%%%%%%%%%%%%%%%%%%%%%%%%%%%%%%%%%%%%%%%%%%%%%%%%%%%%%%%%%%%%%%%%%%%%%%%%%%
\section{Отображение игрового персонажа}

В большинстве жанров компьтерных игр существует некотороый персонаж, который
представляет нас в игре и которым мы можем управлять.

Для начала мы можем в качестве изображения персонажа взять какой-нибудь символ
-- допустим, символ собачки ``@''.

Зададим ``изображение'' в виде символьной константы где-то в начале программы:

\begin{minted}{cpp}
const char PLAYER = '@';
\end{minted}

Далее нам необходимо задать позицию пресонажа в двумерном пространстве нашей
игровой ``карты'':

\begin{minted}{cpp}
int player_x = 0;
int player_y = 0;
\end{minted}

После этого мы можем отобразить персонажа на экране.  Полный код программы будет
выглядеть примерно так, как показано ниже.

\begin{minted}{cpp}
  #include <LiquidCrystal_I2C.h>

  LiquidCrystal_I2C lcd(0x27,  16, 2);

  int player_x = 0;
  int player_y = 0;

  void setup() {
    lcd.init();
    lcd.backlight();
  }

  void loop() {
    lcd.setCursor(player_x, player_y);
    lcd.print(PLAYER);
  }
\end{minted}

После загрузки данной программы в Arduino, на экране на нулевой строке в первом
столбце должен появиться символ ``@''.

Поскольку мы задали позицию персонажа в виде переменных, то меняя эти переменные
мы можем менять позицию персонажа на экране.  Для изменения позиции в играх
обычно используются кнопки и/или манипулятор вида ``джойстик''.  В следующем
разделе мы как раз посмотрим, как подключить кнопки и сделать обработку их
нажатий.

\end{document}
