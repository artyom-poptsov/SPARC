\documentclass[../sparc.tex]{subfiles}
\graphicspath{{\subfix{../images/}}}
\begin{document}

\section{Разность потенциалов}
\index{Электроника!Разность потенциалов}

Посмотрим ещё раз на рис. \ref{fig:electronics-circuits-1}.  Вода перетекает из
ёмкости А в ёмкость Б по трубе.  Как только в уровни воды в ёмкостях сравняются,
то ток воды по трубе остановится (см. рис. \ref{fig:electronics-circuits-2}.)

\begin{figure}[ht]
  \centering
  \begin{tikzpicture}[
      declare function={f1(\x) = 0.15 * sin(8.0 * deg(\x));
    }]

    \draw[thick] (0, 0) -- (0, 4);
    \draw[thick] (2, 0.5) -- (2, 4);
    \draw[thick] (0, 0) -- (2, 0);

    \draw[thick] (3, 0.5) -- (3, 4);
    \draw[thick] (5, 0) -- (5, 4);
    \draw[thick] (3, 0) -- (5, 0);

    \draw[thick] (2, 0) -- (3, 0);
    \draw[thick] (2, 0.5) -- (3, 0.5);


    \begin{scope}[yshift=1.5cm, color=blue]
      \draw (0, 0) plot[domain=0:2, variable=\x, samples=200, smooth] ({\x}, {f1(\x)});
    \end{scope}

    \begin{scope}[yshift=1.5cm, xshift=3cm, color=blue]
      \draw (0, 0) plot[domain=0:2, variable=\x, samples=200, smooth] ({\x}, {f1(\x)});
    \end{scope}

    \draw (1, 0) node[below] {А};
    \draw (4, 0) node[below] {Б};

  \end{tikzpicture}
  \caption{Пример двух ёмкостей с равным уровнем воды.}
  \label{fig:electronics-circuits-2}
\end{figure}

Аналогом этой ситуации в электронике является постепенный разряд батарейки: в
процессе её эксплуатации химические реакции, дающие разность уровней -- или, как
говорят в электронике, \emph{разность потенциалов} -- постепенно замедляются,
вплоть до момента, когда батарейка больше не может давать нужный ток.

Таким образом, вторым условием для электрического тока является наличие разности
потенциалов.

Интересным фактом является то, что для протекания электрического тока (как и
воды) не обязательно иметь разность потенциалов между каким-то положительным
значением и нулём. Проводя опять же аналогию с водой, если мы возьмём две
соединённые ёмкости, стоящие на одном уровне, но с разными уровнями жидкости, то
ток воды будет, пока уровни жидкости не сравняются в обоих ёмкостях (см. рис.
\ref{fig:electronics-circuits-3}.)

\begin{figure}[ht]
  \centering
  \begin{tikzpicture}[
      declare function={f1(\x) = 0.15 * sin(8.0 * deg(\x));
    }]

    \draw[thick] (0, 0) -- (0, 4);
    \draw[thick] (2, 0.5) -- (2, 4);
    \draw[thick] (0, 0) -- (2, 0);

    \draw[thick] (3, 0.5) -- (3, 4);
    \draw[thick] (5, 0) -- (5, 4);
    \draw[thick] (3, 0) -- (5, 0);

    \draw[thick] (2, 0) -- (3, 0);
    \draw[thick] (2, 0.5) -- (3, 0.5);

    \draw[thick, color=blue, ->] (1, 0.25) -- (4, 0.25);


    \begin{scope}[yshift=3cm, color=blue]
      \draw (0, 0) plot[domain=0:2, variable=\x, samples=200, smooth] ({\x}, {f1(\x)});
    \end{scope}

    \begin{scope}[yshift=1.5cm, xshift=3cm, color=blue]
      \draw (0, 0) plot[domain=0:2, variable=\x, samples=200, smooth] ({\x}, {f1(\x)});
    \end{scope}

    \draw (1, 0) node[below] {А};
    \draw (4, 0) node[below] {Б};

  \end{tikzpicture}
  \caption{Пример двух ёмкостей с разными уровнями воды, соединённые трубкой.}
  \label{fig:electronics-circuits-3}
\end{figure}

То же самое относится к ёмкостям, стоящим на разных уровнях.  В электронике для
протекания тока достаточная любая разность потенциалов; примерами разных
потенциалов могут быть 5В и 10В (где разность будет в 5В), или же -5В и +5В (что
даёт разницу в 10В.)

\end{document}
