\documentclass[../sparc.tex]{subfiles}
\graphicspath{{\subfix{../images/}}}
\begin{document}

%%%%%%%%%%%%%%%%%%%%%%%%%%%%%%%%%%%%%%%%%%%%%%%%%%%%%%%%%%%%%%%%%%%%%%%%%%%%%%%%
\section{Паузы в музыке}

Есть ещё один момент, на котором мы до текущего момента не заостряли внимание ---
паузы в произведении. Правильные паузы также важны, как и сами ноты.

В нотной записи паузы отмечаются специальными значками (см. рисунок
\ref{fig:lilypond-rest-example-1}.) \footnote{В музыке существуют паузы,
занммающие несколько тактов, либо очень короткие паузы -- тридцать вторые,
шестьдесят четвёртые и т.п. Используются они редко, поэтому мы их не будем
разбирать здесь.}

\begin{figure}[ht]
  \caption{Паузы в музыке.}
  \centering
  \begin{lilypond}
    \relative c' {
      \numericTimeSignature
      \time 4/4
      r1-"Целая."
    }
  \end{lilypond}
  \begin{lilypond}
    \relative c' {
      \numericTimeSignature
      \time 4/4
      r2-"Половинная." r2
    }
  \end{lilypond}
  \begin{lilypond}
    \relative c' {
      \numericTimeSignature
      \time 4/4
      r4-"Четвертная." r4 r4 r4
    }
  \end{lilypond}
  \begin{lilypond}
    \relative c' {
      \numericTimeSignature
      \time 4/4
      r8-"Восьмая." r8 r8 r8 r8 r8 r8 r8
    }
  \end{lilypond}
  \begin{lilypond}
    \relative c' {
      \numericTimeSignature
      \time 4/4
      r16-"Шестнадцатая." r16 r16 r16
      r16 r16 r16 r16
      r16 r16 r16 r16
      r16 r16 r16 r16
    }
  \end{lilypond}
  \label{fig:lilypond-rest-example-1}
\end{figure}

Целая пауза равна по длине целой ноте, половинная --- половине целой ноты и т.д.
Иными словами, мы можем использовать подход, примененный нами для высчитывания
длительности нот, для расчета длительности пауз в произведении.

Обозначения пауз с их длительностями предтсавлены в таблице
\ref{table:music-rest-legths}.

\begin{table}[ht]
  \caption{Некоторые возможные длительности пауз.}
  \begin{tabular}{p{3cm}|p{4cm}|p{3.5cm}}
    Начертание & Длительность & Название \\
    \hline \hline
    \wholeNoteRest     & $\frac{1}{1}$ & ``Целая'' \\
    \hline
    \halfNoteRest      & $\frac{1}{2}$ & ``Половина'' \\
    \hline
    \crotchetRest        & $\frac{1}{4}$ & ``Четверть'' \\
    \hline
    \quaverRest    & $\frac{1}{8}$ & ``Восьмая'' \\
    \hline
    \semiquaverRest & $\frac{1}{16}$ & ``Шестнадцатая'' \\
    \hline
  \end{tabular}
  \label{table:music-rest-legths}
\end{table}

Для реализации паузы в программе необходимо во-первых создать специальную ноту с
нулевой частотой. Пауза в музыке называется ``Покой'' (или ``Rest''
по-английски), поэтому для обозначения паузы в программе мы будем использовать
заглавную букву ``R''.

\begin{verbatim}
const float R = 0; // Пауза ("Rest")
\end{verbatim}

Далее нам необходимо изменить функцию \texttt{play\_tone} таким образом, чтобы
она могла корректно ``воспроизводить'' паузы. Для этого нам необходимо добавить
такое условие, чтобы, если частота ноты больше нуля, то функция выполняла тот
код, который мы использовали раньше; иначе -- чтобы выполнялась просто задержка.

\begin{minted}{cpp}
// Функция воспроизведения звука указанной частоты.
void play_tone(int port, float f, long t) {
  if (f > 0) {
    const int T = 1000000 / f;
    int d = T / 2;
    int count = t / T;
    for (int i = 0; i < count; i++) {
      digitalWrite(port, HIGH);
      delayMicroseconds(d);
      digitalWrite(port, LOW);
      delayMicroseconds(d);
    }
  } else {
    delay(t / 1000); // Пауза
  }
}
\end{minted}

Обратите внимание, что для создания задержки мы используем код \texttt{delay(t /
  1000)} -- делить \texttt{t} на 1000 необходимо по той причине, что время
проигрывания ноты (\texttt{t}) задаётся в микросекундах, а функция
\texttt{delay} принимает время ожидания в миллисекундах. Чтобы преобразовать
микросекунды в миллисекунды, достаточно поделить количество микросекунд на 1000
(так как в каждой микросекунде содержится 1000 миллисекунд.) Почему же мы не
могли использовать функцию \texttt{delayMicroseconds} для организации задержек
(пауз) прямо в микросекундах, без преобразования? Ответ прост --
\texttt{delayMicroseconds} не умеет долго ждать, и значения \texttt{t} для неё
будут слишком большими; если мы попытаемся использовать
\texttt{delayMicroseconds} с большими отрезками времени, то она не будет
корректно их обрабатывать, и задержка получится неправильной.

Для наглядной демонстрации использования пауз возьмём другое музыкальное
произведение -- ``Кабы небыло зимы'' из мультфильма ``Простоквашино''.

\begin{figure}[ht]
  \caption{Часть мелодии ``Кабы небыло зимы'' из мультфильма ``Простоквашино''.}
  \begin{lilypond}
    \relative c' {
      \key g \major
      \numericTimeSignature
      \time 4/4
      b8 b b'8. fis16 a8 g e4 |
      d8 d << b'8. d8. >> << c16 a >> << c8 a >> << b8 g8 >> r4
      d'8 c a fis << a c >> << g b >> << g4 b >>
      b,8 b << g'8. b8. >> << fis16 a >> << fis8 a >> << e8 g8 >> r4
    }
    \layout {
      indent = 0\mm
      line-width = 120\mm
      ragged-last = ##t
    }
  \end{lilypond}
  \label{fig:lilypond-melody-prostokvashino}
\end{figure}

В нотной записи на рисунке \ref{fig:lilypond-melody-prostokvashino} представлена
часть мелодии, которую мы попытаемся переложить на программный код. Полную
версию мелодии можно увидеть на рисунке
\ref{fig:lilypond-melody-prostokvashino-full}.

Можно заметить две четвертные паузы (\crotchetRest), которые необходимо добавить
в массив нот, чтобы произведение звучало, как в оригинале.

Можно также заметить двойные ноты, записанные одна над другой -- это значит, что
данные ноты должны играться одновременно. Для упрощения нашей задачи мы пока
будем брать самую верхнюю ноту из группы.

Попробуем вписать ноты в массив и послушать, как они будут звучать. Чтобы не
запутаться, разобъём ноты по группам, согласно тактам (по одной группе на
строке) -- а рядом с каждой группой в виде комментария напишем номер этого такта
(например, такт номер ноль мы пометили \texttt{/* 0 */}.)

\begin{minted}{cpp}
float prostokvashino[28][2] = {
  /* 0 */ {b3, 8}, {b3, 8}, {b4, 8}, {f4, 16}, {a4, 8}, {g4, 8}, {e4, 4},
  /* 1 */ {d4, 8}, {d4, 8}, {d5, 8}, {c5, 16}, {c5, 8}, {b4, 8}, {R,  4},
  /* 2 */ {d5, 8}, {c5, 8}, {a4, 8}, {f4, 8},  {c5, 8}, {b4, 8}, {b4, 4},
  /* 3 */ {b3, 8}, {b3, 8}, {b4, 8}, {a4, 16}, {a4, 8}, {g4, 8}, {R,  4},
};
\end{minted}

При воспроизведении мелодия будет похожей на оригинал, однако вы можете заметить
некоторые ``несоответствия''. Источников данных несоответствий несколько. Первый
источник проблем в том, мы не учитываем, что длительность некоторых нот, которые
помечены точкой справа (вот так: ``\eighthNoteDotted'') больше стандартной.

\end{document}
