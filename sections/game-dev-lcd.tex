\documentclass[../sparc.tex]{subfiles}
\graphicspath{{\subfix{../images/}}}
\begin{document}

%%%%%%%%%%%%%%%%%%%%%%%%%%%%%%%%%%%%%%%%%%%%%%%%%%%%%%%%%%%%%%%%%%%%%%%%%%%%%%%%
\section{ЖК-дисплей}

Для вывода информации мы будем использовать \emph{жидкокристаллический дисплей}
(ЖК-дисплей), подключаемый к Arduino.  Принцип работы данных дисплеев аналогичен
обычным ЖК-дисплеям, которые выводят информацию на вашем компьютере.  Более
конкретно мы будем использовать \emph{текстовый} ЖК-дисплей, который
предназначен для вывода тестовых символов.

На текстовых ЖК-дисплеях, экран поделён на клетки, внутри каждой из которых
можно отрисовать один символ (букву, знак препинания, цифру, или просто
какую-либо картинку.)  Между клетками обычно находится расстояние в один
пиксель.  Подобные дисплеи плохо подходят для отрисовки произвольных
изображений, тем не менее, некоторый простор для творчества у нас имеется.

Несмотря на свойства дисплея, которые на первый взгляд кажутся слишком
ограничивающими для наших задач, используя наше мастерство и творческий подход,
мы можем добиться достаточно интересных результатов в плане разработки игр.

\end{document}
