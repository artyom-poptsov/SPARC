\documentclass[../sparc.tex]{subfiles}
\graphicspath{{\subfix{../images/}}}
\begin{document}

%%%%%%%%%%%%%%%%%%%%%%%%%%%%%%%%%%%%%%%%%%%%%%%%%%%%%%%%%%%%%%%%%%%%%%%%%%%%%%%%
\section{Последовательный порт}

Последовательный порт в Arduino -- это тот самый USB-B, который мы подключаем
всякий раз, когда желаем включить наш микроконтроллер или загрузить в Arduino
какую-либо программу.  С помощью последовательного порта можно передавать данные
с Arduino на компьютер и наоборот.

\subsection{Основы работы с Arduino через последовательный порт}

Прежде, чем начать работать с последовательным портом, нам необходимо его
настроить; делается это следующим образом: в теле функции setup мы должны
написать:

\begin{minted}{cpp}
void setup() {
  Serial.begin(9600);
}
\end{minted}

В этом случае мы обеспечиваем обмен данными между компьютером и Arduino с
указанной скоростью, где 9600 -- это скорость, с которой мы передаем данные на
персональный компьютер в \emph{бодах} (битах в секунду.) Обычно данный параметр
принимает одно из следующих значений: 300, 600, 1200, 2400, 4800, 9600, 14400,
19200, 28800, 38400, 57600, 115200.

\subsection{Передача данных на компьютер}

Теперь попробуем передать какие-нибудь данные на компьютер. В качестве примера
мы просто отправим строку ``Hello, world!'' по последовательному порту. Для
начала пропишем настройку порта в функции \texttt{setup}:

\begin{minted}{cpp}
void setup() {
  Serial.begin(9600); // устанавливаем скорость порта
}
\end{minted}

А вот так в нашем случае выглядит функция \texttt{loop}:

\begin{minted}{cpp}
void loop() {
  Serial.println("Hello World");
  delay(1000); // ждём 1000 мс перед следующей отправкой
}
\end{minted}

Результат выполнения программы можно увидеть, открыв монитор порта в Arduino IDE
-- это можно сделать из меню ``Инструменты'' (``Tools''), выбрав пункт ``Монитор
порта''.

Кроме того, открыть монитор порта можно, нажав комбинацию клавиш \hotkey{Ctrl +
  Shift + M}.

Передачу данных с Arduino на компьютер можно использовать в множестве разных
задач. Примером одной из таких задач является простейший способ отладки программ
-- с помощью вывода информации о работе программы в Arduino на последовательный
порт. Иными словами, вместо того, чтобы пытаться самим понять, что же пошло не
так и почему что-то не работает, мы просим Arduino саму рассказывать нам, что
она делает.

\end{document}
