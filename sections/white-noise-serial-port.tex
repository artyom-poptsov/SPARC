\documentclass[../sparc.tex]{subfiles}
\graphicspath{{\subfix{../images/}}}
\begin{document}

%%%%%%%%%%%%%%%%%%%%%%%%%%%%%%%%%%%%%%%%%%%%%%%%%%%%%%%%%%%%%%%%%%%%%%%%%%%%%%%%
\section{Последовательный порт}

Последовательный порт в Arduino -- это тот самый USB-B, который мы подключаем
всякий раз, когда желаем включить наш микроконтроллер или загрузить в Arduino
какую-либо программу.  С помощью последовательного порта можно передавать данные
с Arduino на компьютер и наоборот.

\subsection{Основы работы с Arduino через последовательный порт}

Прежде, чем начать работать с последовательным портом, нам необходимо его
настроить; делается это следующим образом: в теле функции setup мы должны
написать:

\begin{minted}{cpp}
void setup() {
  Serial.begin(9600);
}
\end{minted}

В этом случае мы обеспечиваем обмен данными между компьютером и Arduino с
указанной скоростью, где 9600 -- это скорость, с которой мы передаем данные на
персональный компьютер в \emph{бодах} (битах в секунду.) Обычно данный параметр
принимает одно из следующих значений: 300, 600, 1200, 2400, 4800, 9600, 14400,
19200, 28800, 38400, 57600, 115200.

\subsection{Передача данных на компьютер}

Теперь попробуем передать какие-нибудь данные на компьютер. В качестве примера
мы просто отправим строку ``Hello, world!'' по последовательному порту. Для
начала пропишем настройку порта в функции \texttt{setup}:

\begin{minted}{cpp}
void setup() {
  Serial.begin(9600); // устанавливаем скорость порта
}
\end{minted}

А вот так в нашем случае выглядит функция \texttt{loop}:

\begin{minted}{cpp}
void loop() {
  Serial.println("Hello World");
  delay(1000); // ждём 1000 мс перед следующей отправкой
}
\end{minted}

Результат выполнения программы можно увидеть, открыв монитор порта в Arduino IDE
-- это можно сделать из меню ``Инструменты'' (``Tools''), выбрав пункт ``Монитор
порта''.

Кроме того, открыть монитор порта можно, нажав комбинацию клавиш \hotkey{Ctrl +
  Shift + M}.

Внешний вид монитора порта показан на рис. \ref{fig:arduino-ide-serial-monitor}.
В заголовке окна указано название устройства, которым представлена плата Arduino
в системе и через которое компьютер взаимодействует с платой.

Ниже следует строка ввода данных для передачи их на Arduino.

Основную часть экрана занимает область отображения данных, приходящих с Arduino
на компьютер.

В нижней части окна находится панель настроек.

Цифрами на рисунке указаны:
\begin{enumerate}
\item Строка ввода данных для передачи их с компьютера на Arduino.
\item Кнопка ``Отправить'' (англ. ``Send''), по нажатию на которую данные,
  введённые в строку слева, будут отправлены на Arduino.
\item Отметка о времени прихода данных с Arduino на компьютер.
\item Принятые с Arduino данные.
\item Опция ``Автопрокрутка'' (англ. ``Autoscroll'').  Если данная опция
  включена, то основная область окна, отображающая приходящие данные,
  автоматически прокручивается таким образом, чтобы самые последние данные были
  всегда вверху.
\item Опция ``Показать отметку о времени'' (англ. ``Show timestamp'') позволяет
  включить отображение отметок о времени (цифра \textbf{3} на рисунке).
\item Настрока перевода строк, используемых при коммуникации с Arduino.  Здесь
  можно выбрать один из вариантов: ``No line ending'' (Без перевода строк),
  ``Newline'' (Перевод строки), ``Carriage return'' (Символ ``возврат каретки'')
  и ``Both NL & CR'' (режим, в котором в роли конца строки используются сразу
  два символа -- перевод строки и возврат каретки.)  Данная опция влияет на то,
  какую последовательность символов монитор порта считает за конец строки
  данных.
\item Настройка скорость передачи данных в бодах (битах в секунду).
  По-умолчанию выставлена в 9600 бод.  Если значение данной настройки не
  соответствует тому, который был указан в программе, работающей на Arduino
  (через \texttt{Serial.begin}), то тогда на в мониторе порта вместо данных
  будут отображаться ``кракозяблики'' (иными словами, какая-то нечитаемая
  последовательность символов.)
\item Кнопка ``Очистить вывод'' (англ. ``Clear output'') позволяет очистить окно
  монитора порта.
\end{enumerate}

\begin{figure}[ht]
  \centering
  \includegraphics[width=10cm]{arduino-ide-serial-monitor}
  \caption{Монитор порта (``Serial Monitor'') в Arduino IDE 1.8.19.}
  \label{fig:arduino-ide-serial-monitor}
\end{figure}

Передачу данных с Arduino на компьютер можно использовать в множестве разных
задач. Примером одной из таких задач является простейший способ отладки программ
-- с помощью вывода информации о работе программы в Arduino на последовательный
порт. Иными словами, вместо того, чтобы пытаться самим понять, что же пошло не
так и почему что-то не работает, мы просим Arduino саму рассказывать нам, что
она делает.

\subsection{Визуализация данных}

При большом объёме данных неудобно смотреть их в ``Мониторе порта'' -- от
большого количества чисел может буквально ``рябить в глазах''.  В подобных
случаях нам поможет визуализация данных.

Визуализация данных является очень важной составляющей их анализа -- графическое
представление безликих чисел даёт возможность быстро и ёмко охватить их ``одним
взглядом'', и часто именно это помогает выявить закономерности в данных, в ином
случае скрытых от глаз человека.

Есть разные способы визуализировать данные; одним из таких способов, доступных
нам, является ``Плоттер по последовательному соединению'', который позволяет
прямо из Arduino IDE получить вывод потока данных в виде графика.  Доступ к
``Плоттеру'' осуществляется через меню ``Инструменты'', в котором нужно выбрать
пункт ``Плоттер по последовательному соединению'', либо через комбинацию клавиш
\hotkey{Ctrl + Shift + L}.

Внешний вид плоттера по последовательному соединению показан на
рис. \ref{fig:arduino-ide-serial-plotter}.

Как и для монитора порта, в заголовке окна плоттера указано название устройства,
которым представлена плата Arduino в системе и через которое компьютер
взаимодействует с платой.

\begin{figure}[ht]
  \centering
  \includegraphics[width=10cm]{arduino-ide-serial-plotter}
  \caption{Плоттер по последовательному соединению (``Serial Plotter'') в
    Arduino IDE 1.8.19.}
  \label{fig:arduino-ide-serial-plotter}
\end{figure}

Цифрами на рисунке указаны:

\begin{enumerate}
\item Область отображения графика, динамически генерируемого на основе данных,
  приходящих с Arduino.
\item Настройка скорость передачи данных в бодах (битах в секунду).
  По-умолчанию выставлена в 9600 бод.  Если значение данной настройки не
  соответствует тому, который был указан в программе, работающей на Arduino
  (через \texttt{Serial.begin}), то тогда плоттер не сможет корректно отобразить
  данные.
\item Строка ввода данных для передачи их с компьютера на Arduino.
\item Кнопка ``Отправить'' (англ. ``Send''), по нажатию на которую данные,
  введённые в строку слева, будут отправлены на Arduino.
\item Настрока перевода строк, используемых при коммуникации с Arduino.  Здесь
  можно выбрать один из вариантов: ``No line ending'' (Без перевода строк),
  ``Newline'' (Перевод строки), ``Carriage return'' (Символ ``возврат каретки'')
  и ``Both NL & CR'' (режим, в котором в роли конца строки используются сразу
  два символа -- перевод строки и возврат каретки.)  Данная опция влияет на то,
  какую последовательность символов монитор порта считает за конец строки
  данных.
\end{enumerate}

\end{document}
