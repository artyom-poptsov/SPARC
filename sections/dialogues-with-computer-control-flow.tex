\documentclass[../sparc.tex]{subfiles}
\graphicspath{{\subfix{../images/}}}
\begin{document}

%%%%%%%%%%%%%%%%%%%%%%%%%%%%%%%%%%%%%%%%%%%%%%%%%%%%%%%%%%%%%%%%%%%%%%%%%%%%%%%%
\section{Управляющие конструкции}

%%%%%%%%%%%%%%%%%%%%%%%%%%%%%%%%%%%%%%%%%%%%%%%%%%%%%%%%%%%%%%%%%%%%%%%%%%%%%%%%
\subsection{Условия}
\index{Программирование!Условие}

Иногда во время выполнения программы следует принять решение о том, что делать
дальше. Для того, чтобы компьютер мог сделать правильный выбор, по какому пути
пойти, нам, как программистам, следует описать условия в коде программы: если
условие выполняется, делаем одно, иначе -- делаем другое.

Условия в программах описываются при помощи специальных управляющих конструкций.
В языке C++ у нас есть две основные конструкции. Первая из них -- оператор
\texttt{if} (буквально в переводе с английского ``если''). Пример использования:

\begin{minted}{cpp}
if (a > 10) {
    // действие, выполняемое, если значение
    // переменной 'a' больше 10.
}
\end{minted}

Если нужно проверить равно ли значение переменной чему-либо, используют оператор
сравнения ``=='':

\begin{minted}{cpp}
if (a == 10) {
    // действие, выполняемое, если значение
    // переменной 'a' равно 10.
}
\end{minted}

Не путайте оператор сравнения ``=='' с оператором присваивания ``='' - это важно!

Часто необходимо не только делать что-либо при выполнении условия, но и
предоставить альтернативную инструкцию (или набор инструкций), выполняемую
тогда, когда условие не выполняется. В этом случае используют конструкцию
\texttt{if ... else}:

\begin{minted}{cpp}
if (a > 10) {
  // действие, выполняемое, если значение
  // переменной 'a' больше 10.
} else {
  // действие, выполняемое, если значение
  // переменной 'a' меньше или равно 10.
}
\end{minted}

Второй оператор, который нам будет встречаться, это так называемый
\emph{оператор выбора} \texttt{switch}. С ним познакомимся позже. Он удобен,
например, тогда, когда нам нужно выполнять несколько разных действий в
зависимости от значения переменной, и этих действий много.

%%%%%%%%%%%%%%%%%%%%%%%%%%%%%%%%%%%%%%%%%%%%%%%%%%%%%%%%%%%%%%%%%%%%%%%%%%%%%%%%
\subsection{Циклы}
\index{Программирование!Цикл}

Простые программы, вроде ``бегущего огня'', могут быть написаны простым
копированием и вставкой алгоритма мигания светодиода (возможно, с небольшими
модификациями).

А теперь представьте, что вам требуется запрограммировать ``бегущий огонь'' на
100 светодиодов. Утомительная задача, не правда ли? Для того, чтобы не делать
тупую работу по копированию одного и того же кода много раз, программистами
придуманы специальные управляющие конструкции, называемые \emph{циклами}.

Циклы бывают разные. Основные виды циклов, которые вам будут встречаться
практически в любом языке программирования:
\begin{itemize}
\item Цикл со счётчиком (также называемый ``параметрический цикл'').
\item Цикл с предусловием.
\item Цикл с постусловием.
\end{itemize}

Каждый вид циклов имеет собственную реализацию в языке программирования, который
мы используем (C++).

%%%%%%%%%%%%%%%%%%%%%%%%%%%%%%%%%%%%%%%%%%%%%%%%%%%%%%%%%%%%%%%%%%%%%%%%%%%%%%%%
\subsubsection{Цикл со счётчиком}
\index{Программирование!Цикл!Цикл со счётчиком (for)}

Цикл со счётчиком реализуется конструкцией \texttt{for} -- она позволяет нам
создать счётчик, задать его начальное значение, описать условие выполнения цикла
и операцию изменения счётчика:

\begin{minted}{cpp}
//                   5.
//        1.         2.           4.
for (int pin = 0; pin < 10; pin = pin + 1) {
    // 3. (тело цикла)
}
\end{minted}

Выполняется эта конструкция в следующем порядке:
\begin{enumerate}
\item объявляем переменную и присваиваем ей значение 0 (шаг 1);
\item переходим к проверке, где смотрим, выполняется ли условие (шаг 2);
\item после этого, если условие 2 выполняется, мы переходим к телу цикла (шаг 3);
\item после выполнения тела цикла, мы переходим к изменению значения счётчика (шаг 4);
\item после шага 4 мы опять возвращаемся к шагу 2, если условие выполняется, то
  переходим к шагу 3 и т.д.
\end{enumerate}

%%%%%%%%%%%%%%%%%%%%%%%%%%%%%%%%%%%%%%%%%%%%%%%%%%%%%%%%%%%%%%%%%%%%%%%%%%%%%%%%
\subsubsection{Цикл с предусловием}
\index{Программирование!Цикл!Цикл с предусловием (while)}

Другим распространённым видом цикла является цикл с предусловием, реализуемый в
С++ конструкцией \texttt{while} -- данный вид цикла удобен в тех случаях, когда
мы не знаем точного количества раз, сколько нужно повторить тело цикла (не знаем
количество итераций.)

Общий вид цикла \texttt{while} таков:
\begin{minted}{cpp}
int pin = 2;

while (pin < 10) {
    // тело цикла
}
\end{minted}

%%%%%%%%%%%%%%%%%%%%%%%%%%%%%%%%%%%%%%%%%%%%%%%%%%%%%%%%%%%%%%%%%%%%%%%%%%%%%%%%
\subsubsection{Цикл с постусловием}
\index{Программирование!Цикл!Цикл с постусловием (do..while)}

Кроме вышеперечисленных видов циклов, есть ещё цикл с постусловием, где
проверка условия выполнения цикла осуществляется после выполнения тела цикла.
Реализуется данный вид циклов конструкцией \texttt{do..while}.

Он достаточно редкоиспользуется и мы не будем на нём здесь останавливаться.

%%%%%%%%%%%%%%%%%%%%%%%%%%%%%%%%%%%%%%%%%%%%%%%%%%%%%%%%%%%%%%%%%%%%%%%%%%%%%%%%
\subsection{Зачем столько видов циклов?}

Обратите внимание, что один вид цикла может быть реализован через другой, т.к.
данные управляющие конструкции взаимозаменяемы. Возникает вопрос -- зачем же нам
нужно столько видов циклов? Всё дело в удобстве. В одних случаях удобнее
использовать один вид циклов, в других случаях -- другой.

У программистов есть специальный термин для описания подобных конструкций языка
программирования: синтаксический сахар. Синтаксический сахар -- это конструкции
языка, без которых в принципе можно обойтись при разработке программ, но с ними
всё проще ("слаще").

\section{Задачи}
\begin{enumerate}
\item Перепишите ``бегущий огонь'' с использованием цикла.
\item Модифицируйте алгоритм ``бегущего огня'' таким образом, чтобы светодиоды
  начинали загораться с обоих концов гирлянды, и огни ``бежали'' навстречу друг
  другу.
\end{enumerate}

\end{document}
