\documentclass[../sparc.tex]{subfiles}
\graphicspath{{\subfix{../images/}}}
\begin{document}

Вот и пришло время нам поговорить о том, как организовать общение
микроконтроллера со вспомогательными устройствами, и какой язык для этого будет
использоваться.  ``Разумеется!'' -- вы можете воскликнуть, -- ``Языком общения
машин являются нолики и единички, машинный код!''.  И вы будете правы, общение
компьютеров обычно сводится к передаче нулей и единиц.  Однако, кроме
``алфавита'' и ``слов'', используемых в беседе компьютеров, не менее важны ещё
такие понятия, как \emph{среда передачи данных} и \emph{протокол передачи
данных}.

Средой передачи данных часто выступает одна из трёх сред: провод (например, как
в случае с проводом USB), радиоканал (Bluetooth, Wi-Fi) или же свет
(оптоволоконная передача данных.)  Компьютеры для коммуникации между собой могут
даже использовать звуковые волны, но это скорее исключение из правил.

Протокол передачи данных опредеяет собственно формат, в котором данные
передаются -- где начало и конец ``предложения'', что является ``словами'' и
какой ``алфавит'' используется при коммуникации.  Широко известным примером
протокола может служить USB.

%%%%%%%%%%%%%%%%%%%%%%%%%%%%%%%%%%%%%%%%%%%%%%%%%%%%%%%%%%%%%%%%%%%%%%%%%%%%%%%%
\section{Способы передачи данных}

Передача данных между устройствами делится на две большие категории:

\begin{itemize}
\item \textbf{Параллельная передача данных} -- все биты данных передаются
  единовременно (параллельно.)  Для этого устройства должны быть соединены
  линиями передачи данных в количестве, равном количеству единовременно
  передаваемых бит.
\item \textbf{Последовательная передача данных} -- биты данных передаются
  последовательно, используя временное кодирование.  Для этого потенциально
  достаточно одной линии передачи данных (например, одного провода.)  Фактически
  же для большинства подобных способов передачи данных используются две или
  более линий.
\end{itemize}

Примером параллельной передачи данных является \emph{параллельный порт}
(англ. ``parallel port''.)  Обычно параллельный порт в компьютерном мире
ассоциируется с интерфейом Line Print Terminal (LPT.)  В прошлом (до примерно
2000 года) подобные порты использовались для подключения принтеров к
компьютерам, и были крайне популярны.  В современном мире LPT-порты практически
не используются, будучи вытесненными такими интерфейсами, как USB.

Преимуществом параллельного порта, как следует из его названия, является
возможность передавать сразу несколько бит единовременно, что ускоряет передачу
данных.  Например, в LPT-портах для передачи данных использовалось 8 линий,
позволяя таким образом сразу передать 8 бит (1 байт) информации.  Тем не менее,
недостатком последовательного порта является громоздкость интерфейса подключения
устройств из-за большого количества проводов.

\end{document}
