\documentclass[../sparc.tex]{subfiles}
\graphicspath{{\subfix{../images/}}}
\begin{document}

\newglossaryentry{UART}{name=UART, description={Universal Asynchronous
    Receiver-Transmitter}}

Вот и пришло время нам поговорить о том, как организовать общение
микроконтроллера со вспомогательными устройствами, и какой язык для этого будет
использоваться.  ``Разумеется!'' -- вы можете воскликнуть, -- ``Языком общения
машин являются нолики и единички, машинный код!''.  И вы будете правы, общение
компьютеров обычно сводится к передаче нулей и единиц.  Однако, кроме
``алфавита'' и ``слов'', используемых в беседе компьютеров, не менее важны ещё
такие понятия, как \emph{среда передачи данных} и \emph{протокол передачи
данных}.

Средой передачи данных часто выступает одна из трёх сред: провод (например, как
в случае с проводом USB), радиоканал (Bluetooth, Wi-Fi) или же свет
(оптоволоконная передача данных.)  Компьютеры для коммуникации между собой могут
даже использовать звуковые волны, но это скорее исключение из правил.

Протокол передачи данных опредеяет собственно формат, в котором данные
передаются -- где начало и конец ``предложения'', что является ``словами'' и
какой ``алфавит'' используется при коммуникации.  Широко известным примером
протокола может служить USB.

%%%%%%%%%%%%%%%%%%%%%%%%%%%%%%%%%%%%%%%%%%%%%%%%%%%%%%%%%%%%%%%%%%%%%%%%%%%%%%%%
\section{Классификация методов передачи данных}

\newglossaryentry{LPT}{name=LPT, description={Line Print Terminal}}

Передача данных между устройствами делится на две большие категории:

\begin{itemize}
\item \textbf{Параллельная передача данных} -- все биты данных передаются
  единовременно (параллельно.)  Для этого устройства должны быть соединены
  линиями передачи данных в количестве, равном количеству единовременно
  передаваемых бит.
\item \textbf{Последовательная передача данных} -- биты данных передаются
  последовательно, используя временн\'{о}е кодирование.  Для этого потенциально
  достаточно одной линии передачи данных (например, одного провода.) Фактически
  же для большинства подобных способов передачи данных используются две или
  более линий.
\end{itemize}

\begin{figure}[H]
  \centering
  \begin{tikzpicture}[
      level distance=15mm,
      level 1/.style={sibling distance=30mm},
      level 2/.style={sibling distance=10mm},
      level 3/.style={sibling distance=15mm},
      every node/.style={rectangle,draw,inner sep=2pt},
    ]
    \node {Передача данных}
      child {node {Параллельная}
        child {node {LPT}}
      }
      child {node {Последовательная}
        child {node {UART}
          child {node {RS-232}}
          child {node {RS-485}}
        }
        child {node {SPI}}
        child {node {I2C}}
      };
  \end{tikzpicture}
  \caption{Классификация методов передачи данных.}
  \label{fig:communication-data-transfer-categories}
\end{figure}

Примером параллельной передачи данных является \emph{параллельный порт} (англ.
``parallel port''.) Обычно параллельный порт в компьютерном мире ассоциируется с
интерфейом Line Print Terminal (\gls{LPT}.)  В прошлом (c 1970-х по 2000-е года)
подобные порты использовались для подключения принтеров к компьютерам, и были
крайне популярны.  В современном мире LPT-порты практически не используются,
будучи вытесненными такими интерфейсами, как USB.

Преимуществом параллельного порта, как следует из его названия, является
возможность передавать сразу несколько бит единовременно, что ускоряет передачу
данных.  Например, в LPT-портах для передачи данных использовалось 8 линий (не
считая вспомогательных линий, вроде питания, GND и тактового импульса), позволяя
таким образом сразу передать 8 бит (1 байт) информации.  К преимуществам
параллельного порта также можно отнести простоту реализации.  Тем не менее,
важным (и во многих случаях решающим) недостатком последовательного порта
является громоздкость интерфейса подключения устройств из-за большого количества
проводов.

В связи с вышесказанным мы будем подробно рассматривать только последовательные
интерфейсы передачи данных: \gls{UART} и I2C.

%%%%%%%%%%%%%%%%%%%%%%%%%%%%%%%%%%%%%%%%%%%%%%%%%%%%%%%%%%%%%%%%%%%%%%%%%%%%%%%%
\subsection{Направление передачи данных}

Также системы передачи данных можно поделить по направлению передачи данных на
три категории:

\begin{itemize}
\item \textbf{Симплекс} (лат. \emph{simplex}, односторонний) -- Однонаправленная
  коммуникация.  Симплексный канал передаёт данные только в одну сторону.
\item \textbf{Полудуплекс} (\emph{half-duplex}) -- Двунаправленная коммуникация.
  Соединённые устройства могут передавать данные в обе стороны, однако в каждый
  момент времени активным может быть только либо приём, либо передача.  Примером
  подобных устройств является обычная рация.
\item \textbf{Дуплекс} (лат. \emph{duplex}, двусторонний), также иногда
  называемы ``full-duplex'' -- Двунаправленная передача данных.  Соединённые
  устройства могут одновременно и передавать, и принимать данные.  Примером
  является проводная линия телефоннной связи: оба собеседника могут говорить и
  слышать друг друга одновременно.
\end{itemize}

\gls{UART} -- один из последовательных интерфейсов, которые мы будем
рассматривать далее -- позволяет достаточно легко реализовать как сммплексный,
так и дуплексный режим коммуникации.

\end{document}
