\documentclass[../sparc.tex]{subfiles}
\graphicspath{{\subfix{../images/}}}
\begin{document}

Вот и пришло время нам поговорить о том, как организовать общение
микроконтроллера со вспомогательными устройствами, и какой язык для этого будет
использоваться.  ``Разумеется!'' -- вы можете воскликнуть, -- ``Языком общения
машин являеются нолики и единички, машинный код!''.  И вы будете правы, общение
компьютеров обычно сводится к передаче нулей и единиц.  Однако есть кроме
``алфавита'' и ``слов'', используемых в беседе компьютеров, не менее важны ещё
такие понятия, как \emph{среда передачи данных} и \emph{протокол передачи
данных}.

Средой передачи данных часто выступает одна из трёх сред: провод, радиоканал или
же свет.  Компьютеры для коммуникации между собой могут даже использовать
звуковые волны, но это скорее исключение из правил.

\end{document}
