\documentclass[../sparc.tex]{subfiles}
\graphicspath{{\subfix{../images/}}}
\begin{document}

%%%%%%%%%%%%%%%%%%%%%%%%%%%%%%%%%%%%%%%%%%%%%%%%%%%%%%%%%%%%%%%%%%%%%%%%%%%%%%%%
\section{Ноты с точками}

В музыкальной нотации точка, которая ставится справа от ноты, увеличивает её
длительность на половину от базовой.

Например, если у нас точка идёт после восьмушки (``\eighthNoteDotted''), то
следовательно к её длительности будет прибавляться половина от её длительности.
Половинка от восьмушки -- это шестнадцатая. Чтобы сложить простые дроби, которые
у нас получились, необходимо привести их к общему знаменателю. И формула
вычисления длительности будет следующая:

\begin{equation}
  \mbox{\eighthNoteDotted} = \mbox{\eighthNote} + \mbox{\sixteenthNote}
  = \frac{1}{8} + \frac{1}{16} = \frac{2}{16} + \frac{1}{16} = \frac{3}{16}
\end{equation}

Получившееся число $\frac{3}{16}$ для нас неудобно, так как мы в программе
подставляем в числитель дроби длительность одного такта, а тут у нас получается,
что необходимо поставить длину трёх тактов. Чтобы избавиться от этой неудобной
тройки в числителе, мы можем разделить числитель и знаменатель на 3.

\begin{equation}
  \frac{3 / 3}{16 / 3} = \frac{1}{16 / 3}
\end{equation}

Получившееся число $16 / 3$ необходимо подставить в массив с нашими нотами.
Например, третья нота нулевого такта -- ``B4'' -- как раз восьмая с точкой. В
массиве её длительность надо исправить -- вместо \texttt{\{b4, 8\}} написать
\texttt{\{b4, 16.0 / 3.0\}}. Тоже самое необходимо сделать с другими нотами,
возле которых стоит точка.

Для того, чтобы данный код уместился в книгу, нам пришлось разбить каждый такт
на две строки, но по комментариям и отступам должно быть понятно, что происходит
в коде.

\begin{listing}[ht]
  \begin{minted}{cpp}
    float prostokvashino[28][2] = {
      /* 0 */ {b3, 8},  {b3, 8}, {b4, 16.0 / 3.0},
      /*   */ {f4, 16}, {a4, 8}, {g4,          8}, {e4, 4},
      /* 1 */ {d4, 8},  {d4, 8}, {d5, 16.0 / 3.0},
      /*   */ {c5, 16}, {c5, 8}, {b4,          8}, {R,  4},
      /* 2 */ {d5, 8},  {c5, 8}, {a4,          8},
      /*   */ {f4, 8},  {c5, 8}, {b4,          8}, {b4, 4},
      /* 3 */ {b3, 8},  {b3, 8}, {b4, 16.0 / 3.0},
      /*   */ {a4, 16}, {a4, 8}, {g4,          8}, {R,  4},
    };
  \end{minted}
  \label{listing:music-dotted-notes}
  \caption{Пример указания длительности нот с точкой в коде мелодии.}
\end{listing}

Теперь наша мелодия звучит ещё более похоже на оригинал.

Здесь стоит упомянуть, что в музыке встречаются ноты двумя точками справа, что
даёт удлинение ноты на половину её длительности и на половину от половины -- но
подобные ситуации редки из-за неудобства расчёта необходимой длительности при
чтении музыкантом нот с листа. То ли дело нам, программистам -- прочитали не
спеша, запрограммировали, а там пусть компьютер сам пыжится над нашим творением!

\end{document}
