\documentclass[../main.tex]{subfiles}
\graphicspath{{\subfix{../images/}}}
\begin{document}

%%%%%%%%%%%%%%%%%%%%%%%%%%%%%%%%%%%%%%%%%%%%%%%%%%%%%%%%%%%%%%%%%%%%%%%%%%%%%%%%
\section{Для кого предназначена данная книга?}
Дорогой читатель, добро пожаловать в наш уютный кружок технического творчества.
Здесь мы учимся работать со звуком, светом, электричеством, используя наши
знания для создания неожиданных, интересных и практически полезных проектов.  Мы
постараемся сделать ваш путь в мир электроники и программирования как можно
более интересным и лёгким.  Но и на вас лежит определённая отвественность --
во-первых, без вашего активного участия наши усилия могут не дать желаемого
эффекта.  Во-вторых, мы учимся вместе с вами, и ты, уважаемый читатель,
являешься активным участником работы над этой книгой.  Если найдёшь ошибки или
опечатки, не стесняйся писать нам по указанным в книге контактам -- мы
постараемся всё исправить в следующей версии книги.

Надеемся, что данная книга станет на какое-то время вашей настольной (или хотя
бы \emph{около-стольной}, но во всяком случе не \emph{под-стольной}) книгой,
которая поможет постичь искусство программирования и отчасти исследовать и
понять мир вокруг нас немного лучше, чем вы понимали прежде.

\section{Авторы}
В разработке данной книги принимали участие следующие люди:
\begin{itemize}
\item Денис Киселёв -- вклад в разработку отдельных глав книги; вычитка текста,
  участие в разработке и тестирование примеров, приведённых в книге.
\item Сергей Ермейкин -- вычитка текста, исправление ошибок.
\item Илья Маштаков – вычитка и доработка текста.
\end{itemize}

\section{Лицензия}
Copyright © 2016-2022 Артём ``avp'' Попцов.

Права на копирование сторонних изображений и материалов, использованных в данной
работе, принадлежат их владельцам.

Данная работа распространяется на условиях лицензии «Attribution-ShareAlike»
(«Атрибуция-СохранениеУсловий») 4.0 Всемирная (CC BY-SA 4.0)
\url{https://creativecommons.org/licenses/by-sa/4.0/deed.ru}

\end{document}
