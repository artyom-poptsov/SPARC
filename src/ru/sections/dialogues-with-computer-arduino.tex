\documentclass[../sparc.tex]{subfiles}
\graphicspath{{\subfix{../images/}}}
\begin{document}

%%%%%%%%%%%%%%%%%%%%%%%%%%%%%%%%%%%%%%%%%%%%%%%%%%%%%%%%%%%%%%%%%%%%%%%%%%%%%%%%
\section{Платформа Arduino}
\label{section:arduino}

\newglossaryentry{ЧПУ}{name=ЧПУ, description={Числовое Программное Управление}}

\emph{Микроконтроллер} -- это программируемая микросхема, состоящая из процессора
и широкого набора блоков.\cite{alexgyver:mcu}  Микроконтроллеры используются для
управления электронными устройствами: современными бытовыми приборами,
игрушками, электронными музыкальными инструментами, производственными роботами;
на основе микроконтроллеров работают 3D-принтеры и другие станки с числовым
программным управлением (\gls{ЧПУ}.)

Микроконтроллер отличаются высокой степенью интеграции компонентов: на одной
микросхеме как правило соседствуют центральный поцессор, постоянная память
(ПЗУ), оперативная память (ОЗУ.)  Стоимость микроконтроллеров сильно
варьируется, в зависимости от их возможностей и назначения: самые дешёвые модели
стоят несколько рублей\cite{habr:padauk}, тогда как в среднем потребительские
модели стоят от нескольких сотен до нескольких тысяч рублей.

Мы будем работать с платформой Arduino, которая предоставляет удобный интерфейс
для её программирования.

Большинство платформ Arduino построены на базе микроконтроллеров AVR.

Вот некоторые из популярных вариантов платформ Arduino:
\begin{itemize}
\item \textbf{Arduino Nano} -- одна из самых дешёвых и доступных версий Arduino.
  Имеет малые размеры, скромное количество портов ввода/вывода.
\item \textbf{Arduino Uno} -- популярная версия платформы, которая уже позволяет
  подключать напрямую к ней платы расширения (про них речь пойдёт ниже.)
\item \textbf{Arduino Mega 2560} --- большая плата, постороенная на одноимённом
  контроллере ATmega 2560.  Имеет более 50 цифровых портов ввода-вывода плюс 16
  аналоговых портов.  Может быть использована для создания станков с \gls{ЧПУ}
  (например, 3D-принтеров.)
\item \textbf{Arduino Leonardo} -- данная плата интересна тем, что может
  ``притворяться'' USB-устройством -- или, как говорят программисты,
  \emph{эмулировать его}.
\end{itemize}

\subsection{Платы расширения}
\label{subsection:arduino-shields}

Ко многим вариантам платформ Arduino можно подключить как отдельные датчики, так
и платы расширения -- \emph{шилды} (от англ. ``shield'' -- ``щит''), которые
вставляются в них сверху.  Более того, поверх уже подключенной платы расширения
можно вставить ещё плату, потом ещё -- и получается такой ``слоёный пирог'',
составленный из плат, с Arduino в самом низу.  Платы расширения несут различную
функцию, от просто добавления к Arduino возможности общаться по Wi-Fi, до
предоставления возможности управления целым станком с \gls{ЧПУ} (например,
3D-принтером или же лазером.)

\end{document}
