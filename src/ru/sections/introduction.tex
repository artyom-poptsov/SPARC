\documentclass[../sparc.tex]{subfiles}
\graphicspath{{\subfix{../images/}}}
\begin{document}

%%%%%%%%%%%%%%%%%%%%%%%%%%%%%%%%%%%%%%%%%%%%%%%%%%%%%%%%%%%%%%%%%%%%%%%%%%%%%%%%
\section*{Для кого предназначена данная книга?}
\addcontentsline{toc}{section}{Для кого предназначена данная книга?}

Дорогой читатель, добро пожаловать в наш уютный кружок технического творчества.
Здесь мы учимся работать со звуком, светом, электричеством, используя наши
знания для создания неожиданных, интересных и практически полезных проектов. Мы
постараемся сделать ваш путь в мир электроники и программирования как можно
более интересным и лёгким. Но и на вас лежит определённая ответственность --
во-первых, без вашего активного участия наши усилия могут не дать желаемого
эффекта. Во-вторых, мы учимся вместе с вами, и ты, уважаемый читатель, являешься
активным участником работы над этой книгой. Если найдёшь ошибки или опечатки, не
стесняйся писать нам по указанным в книге контактам -- мы постараемся всё
исправить в следующей версии книги.

Надеемся, что данная книга станет на какое-то время вашей настольной (или хотя
бы \emph{около-стольной}, но во всяком случае не \emph{под-стольной}) книгой,
которая поможет постичь искусство программирования, а также отчасти исследовать
и понять мир вокруг нас немного лучше, чем вы понимали прежде.

\section*{Путеводитель по книге}
\addcontentsline{toc}{section}{Путеводитель по книге}

Книга разделена на крупные тематические главы, каждая следующая глава строится
на основе предыдущих.  Тем не менее, если вы уже имеете опыт в работе с
электроникой и достаточно хорошо умеете программировать, то можете сразу
заглянуть в главу ``Синтез музыки и технологии'', где подробно рассказана тема
программирования музыки, которая обычно мало,  кого оставляет равнодушным.

В тексте книги вам могут встретиться следующие обозначения:

\begin{table}[H]
  \centering
  \def\arraystretch{3.0}%
  \begin{tabular}{|m{4em}|m{15em}|}
    \hline
    \includesvg[width=1.25cm]{the-noun-project/request-mirrored}
    & Практический пример того, о чём говорилось раннее. \\
    \hline
    \includesvg[width=1cm]{the-noun-project/flask}
    & Эксперимент, который вы можете попробовать провести самостоятельно. \\
    \hline
    \includesvg[width=1cm]{the-noun-project/note}
    & Важное примечание, на которое стоит обратить внимание. \\
    \hline
  \end{tabular}
\end{table}

В конце некоторых разделов даны задания для самостоятельного выполнения.  Если
вы хотите хорошо понять материал, излагаемый в книге, то решение подобных
заданий являются хорошим способом закрепить ваши знания.  Не останавливайтесь на
решении задач одним способом, попробуйте придумать несколько способов.
Придумайте себе новые задачи, интересные лично вам!  Придумывание задач -- это
один из ключевых навыков для получения навыков.

\section*{Благодарности}
\addcontentsline{toc}{section}{Благодарности}

Основой для книги послужили годы практики в \textbf{Нижегородском
  радиотехническом колледже} (\url{https://nntc.nnov.ru/}), где велись занятия
по программированию микроконтроллеров, а также ведение мастер-классов и занятий
в \textbf{Нижегородском хакерспейсе ``CADR''} (\url{https://cadrspace.ru/}.)
Отмечу, что именно CADR стал для меня местом, где я мог не только делиться своим
опытом с другими, но и активно учиться техническому творчеству, и это сложно
переоценить.  Автор выражает благодарность данным организациям за
предоставленную возможность к развитию и самореализации!

Также хочется выразить благодарность следующим людям, которые приняли активное
участие в разработке книги:
\begin{itemize}
\item Денис Киселёв -- вклад в разработку отдельных глав книги; вычитка текста,
  участие в разработке и тестирование примеров, приведённых в книге.
\item Сергей Ермейкин -- вычитка текста, исправление ошибок.
\item Илья Маштаков – вычитка и доработка текста.
\item Пётр Третьяков -- вычитка текста, масса ценных советов по электронике,
  по низкоуровневой работе с микроконтроллерами, протоколами и шинами данных,
  советы изложению материала.
\item Алина Тараева -- вычитка текста, выявление большого количества ошибок.
\item \href{https://github.com/V4n0M4sk}{Van0Mask} -- выявление и исправление
  пунктуационных и синтаксических ошибок.
\item Антон Шеффер (Agaffer) -- исправление ошибок в главе 5.
\item Всеволод Кожухов -- работа над таблицей компонентов Wokwi.
\end{itemize}

Само существование этой книги стало возможным благодаря вам, за что огромное
спасибо.

\section*{Исходный код}
\addcontentsline{toc}{section}{Исходный код}

Исходный код книги находится по адресу: \\
\url{https://github.com/artyom-poptsov/SPARC}.

\section*{Лицензия}
\addcontentsline{toc}{section}{Лицензия}

Copyright © 2016-2025 Артём ``avp'' Попцов <\href{mailto:poptsov.artyom@gmail.com}{poptsov.artyom@gmail.com}>.

Права на копирование сторонних изображений и материалов, использованных в данной
работе, принадлежат их владельцам.

Данная работа распространяется на условиях лицензии
\\ ``Attribution--ShareAlike'' (``Атрибуция-СохранениеУсловий'') 4.0 Всемирная
(CC BY-SA 4.0) \\ \url{https://creativecommons.org/licenses/by-sa/4.0/deed.ru}

\end{document}
