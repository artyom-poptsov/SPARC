\documentclass[../sparc.tex]{subfiles}
\graphicspath{{\subfix{../images/}}}
\begin{document}

%%%%%%%%%%%%%%%%%%%%%%%%%%%%%%%%%%%%%%%%%%%%%%%%%%%%%%%%%%%%%%%%%%%%%%%%%%%%%%%%
\newpage
\subsection{Двумерные массивы}
\index{Программирование!Массив!Двумерный массив}

Было бы круто разместить наши ноты таким образом, чтобы каждая нота лежала в
ячейке массива вместе со своей длительностью. К счастью, у нас есть такая
возможность -- мы можем использовать \emph{двумерные массивы}.

Схематическое изображение двумерного массива представлено в виде таблицы
\ref{table:array-example-2}.

\begin{table}[ht]
  \centering
  \begin{tabular}{r|l|l|l}
    \multicolumn{1}{l}{№ строки} & \multicolumn{2}{l}{№ столбца}                 &   \\
    \multicolumn{1}{l}{}         & \multicolumn{1}{l}{0} & \multicolumn{1}{l}{1} &   \\ 
    \cline{2-3}
    0                            & с4                    & 4                     &   \\ 
    \cline{2-3}
    1                            & с4                    & 4                     &   \\
    \cline{2-3}
    2                            & g4                    & 4                     &   \\
    \cline{2-3}
    3                            & g4                    & 4                     &   \\
    \cline{2-3}
    4                            & a4                    & 4                     &   \\
    \cline{2-3}
    5                            & a4                    & 4                     &   \\
    \cline{2-3}
    6                            & g4                    & 2                     &   \\
    \cline{2-3}
  \end{tabular}
  \label{table:array-example-2}
\end{table}

Каждая строка нашего массива должна содержать описание одной ноты. Столбец с
номером ноль содержит частоту ноты, а столбец номер один содержит её
длительность в виде знаменателя простой дроби, где в числителе у нас находится
длина такта. Например, нота номер ноль (``C4'') имеет в музыкальном произведении
длительность $\frac{1}{4}$, следовательно её длительность будет записана, как 4.

Записать программно мелодию в виде двумерного массива можно следующим образом:

\begin{minted}{cpp}
float melody[28][2] = {
  {c4, 4}, {c4, 4}, {g4, 4}, {g4, 4},
  {a4, 4}, {a4, 4}, {g4, 2},
  {f4, 4}, {f4, 4}, {e4, 4}, {e4, 4},
  {d4, 4}, {d4, 4}, {c4, 2},
  {g4, 4}, {g4, 4}, {f4, 4}, {f4, 4},
  {e4, 4}, {e4, 4}, {d4, 2},
  {g4, 4}, {g4, 4}, {f4, 4}, {f4, 4},
  {e4, 4}, {e4, 4}, {d4, 2},
};
\end{minted}

Как можно видеть, теперь каждый элемент массива -- это по сути одномерный массив
из двух элементов, записанный в фигурных скобках. Например, элемент номер ноль
нашего массива \texttt{melody} содержит массив \texttt{\{c4, 4\}} -- частота ноты
и её длительность.

Теперь мы можем адаптировать код воспроизведения мелодии под наши задачи:

\begin{minted}{cpp}
// ...

void loop() {
  const long BPM = 120;
  const long MINUTE = 60 * 1000000;
  const long T = (MINUTE / BPM) * 4;

  for (int note_idx = 0; note_idx < 28; note_idx++) {
    play_tone(SPEAKER_PIN,
              melody[note_idx][0],
              T / melody[note_idx][1]);
    delay(100);
  }
}
\end{minted}

Используя двумерные массивы, можно кратко и ёмко описать мелодию, даже намного
более сложную, чем ``Twinkle, Twinkle, Little Star''.

На этом этапе нам необходимо разобрать, как же работает \emph{нотный стан}
(называемый также \emph{нотоносцем}), на котором располагаются ноты -- для того,
чтобы уметь самостоятельно определять, где какая нота (частота) находится.

\end{document}
