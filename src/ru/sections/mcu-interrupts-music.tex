\documentclass[../sparc.tex]{subfiles}
\graphicspath{{\subfix{../images/}}}
\begin{document}

\newpage
%%%%%%%%%%%%%%%%%%%%%%%%%%%%%%%%%%%%%%%%%%%%%%%%%%%%%%%%%%%%%%%%%%%%%%%%%%%%%%%%
\section{Программирование музыки через прерывания}

Имея новые знания о прерываниях и таймерах, пришло время вернуться к теме, уже
рассмотренной нами в главе \ref{chapter:sound} -- а именно к теме
программирования звука и музыки.  Ведь благодаря изученному материалу, у нас
появилась возможность совместить одновременно две мелодии на одном
микроконтроллере.  Но обо всём по-порядку.

%%%%%%%%%%%%%%%%%%%%%%%%%%%%%%%%%%%%%%%%%%%%%%%%%%%%%%%%%%%%%%%%%%%%%%%%%%%%%%%%
\subsection{Генерация звука с помощью таймера}

Как мы уже говорили, прерывания обрабатываются как бы параллельно с работой
процедуры \mintinline{cpp}{loop}.  Используя прерывания, можно добиться как
гораздо более быстрого отклика на внешние события (за счёт внешних прерываний),
так и точной выдержки промежутков времени (за счёт таймеров.)

В главе \ref{chapter:sound} мы использовали самописную процедуру
\mintinline{cpp}{play_tone} для вывода звука, где использовали процедуру
\mintinline{cpp}{delayMicroseconds} для обеспечения времени полупериодов
звуковой волны.  Использование такого подхода имеет как плюсы, так и минусы.  К
плюсам использования простого цикла с задержками можно отнести то, что код
получается достаточно простой и понятный для начинающих (это и послужило
причиной нашего выбора в пользу такого подхода.)  К минусам же можно отнести то,
что процедура \mintinline{cpp}{play_tone} является \emph{блокирующей} -- в том
смысле, что она явно влияет на ход выполнения программы, ведь пока она не
завершиться, компьютер не перейдёт к следующей инструкции.  Кроме того, что для
нас тоже важно -- использование цикла, а также процедур
\mintinline{cpp}{delayMicroseconds} и \mintinline{cpp}{digitalWrite} в нём,
приводило к дополнительным задержкам и неточности (можно даже сказать, к
нестабильности) выходного сигнала.

\end{document}
