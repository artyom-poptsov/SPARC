\documentclass[../sparc.tex]{subfiles}
\graphicspath{{\subfix{../images/}}}
\begin{document}

%%%%%%%%%%%%%%%%%%%%%%%%%%%%%%%%%%%%%%%%%%%%%%%%%%%%%%%%%%%%%%%%%%%%%%%%%%%%%%%%
\section{Сопротивление}
\label{section:electronics-resistance}
\index{Электроника!Сопротивление}

Чтобы понять, с какой скоростью вода будет перетекать из сосуда в сосуд,
необходимо знать параметры трубопровода, соединяющего их -- иными словами,
\emph{проводника воды}.  Основным параметром проводника является
\emph{сопротивление} -- чем выше этот параметр, тем медленнее вода будет течь.

Примером проводника с высоким сопротивлением является трубопровод с малым
сечением (или диаметром) труб, как показано на
рис. \ref{fig:electronics-resistance-0}.

\figureElectronicsResistance{ru}

Уменьшение сечения трубы приводит к увеличению её сопротивления для тока воды.
Данное правило применимо также для электрических цепей, где сечение проводника
(провода) и его сопротивление имеют обратную зависимость: чем больше сечение,
тем меньше сопротивление; чем меньше сопротивление, тем больше ток.

Для наглядности предположим, что ёмкости ``В'' и ``Г'' на
рис. \ref{fig:electronics-resistance-0} соединены трубой с сечением вполовину от
того, что соединяет ёмкости ``А'' и ``Б''.  Тогда ток воды между ёмкостями ``В''
и ``Г'' будет в два раза меньше, чем между ёмкостями ``А'' и ``Б''.

В электронике сопротивление проводника измеряется в \emph{Омах} (англ. ``Ohm'').

Взаимосвязь тока, напряжения и сопротивления описывается \emph{законом Ома}.

Используя закон Ома, можно посчитать ток для электрической цепи по формуле
\ref{equation:elemctronics-ohms-law-0}.

\begin{equation}
  \mbox{I} = \frac{\mbox{U}}{\mbox{R}}
  \label{equation:elemctronics-ohms-law-0}
\end{equation}

Где ``I'' -- это сила тока (в Амперах), ``U'' -- это напряжение (в Вольтах), а
``R'' -- это сопротивление (в Омах).

Как можно видеть из данной формулы, увеличение сопротивления проводника в два
раза уменьшает ток, идущий по электрической цепи, в два раза.

\note{ru}{ Для запоминания закона Ома можно использовать детский стишок: ``Знает
  каждый пионер: сила тока -- U на R!''.  }

В обыденной жизни сопротивление есть у всего: любые провода имеют сопротивление,
равно как и другие окружающие нас вещи.  У некоторых материалов сопротивление
выше, чем у других.  Те материалы, у которых сопротивление низкое и они хорошо
пропускают ток, называют \emph{проводниками}.  Примерами хороших проводников
могут служить медь, серебро и золото.  Материалы, которые плохо пропускают ток,
и часто используются для \emph{изоляции}, называют \emph{диэлектриками}.  К этой
категории можно отнести, например, различные пластики, из которых делают
изоляцию проводов.

\experiment{0}{Посмотрите вокруг -- сколько различных проводников вы можете
  найти?}

При прохождении тока через проводник, имеющий сопротивление (а следовательно,
через любой ``бытовой'' проводник), часть энергии теряется, трансформируясь в
тепло.  Например, если вы включаете электрический чайник в бытовую розетку, то
греется не только нагреватель чайника, который кипятит воду, но и провод
подключения чайника в бытовую сеть (нагрев провода, по сравнению с нагревателем
внутри чайника, будет незначительным, и вряд ли будет ощутим на ощупь.)

В некоторых случаях, нагрев проводника при прохождении тока является желаемым
эффектом, как в случае с нагревателем внутри чайника; в других же случаях нагрев
проводника стараются уменьшить, чтобы сократить тепловые потери энергии.

\experiment{1}{Можете ли вы назвать другие примеры, где нагрев проводника
  является желаемым эффектом?}

Существуют проводники, которые не имеют сопротивления -- они называются
\emph{сверхпроводниками}.  На сегодняшний день, данный вид проводников
встречается только в очень специфических устройствах (например, в научном
оборудовании), где требуется пропускать большой ток без потери энергии в тепло.
Использование в быту таких проводников крайне затруднено из-за их огромной
стоимости и требованиям к особым условиям эксплуатации (например, охлаждение до
сверхнизких температур.)

Сопротивление есть не только у проводников, но и у источников напряжения --
такое сопротивление называется \emph{внутренним сопротивлением источника
напряжения}.

В электронике сопротивление с нужным номиналом (значением) в пределах некоторой
погрешности обычно задаётся специальным элементом -- \emph{резистором}. Слово
``резистор'' произошло от английского слова ``resistance'' -- ``сопротивление''.
Таким образом, если буквально переводить слово ``резистор'', то получится
``сопротивлятор'', хотя конечно так никто обычно не говорит.

Резисторы играют важную роль в электрических схемах -- они позволяют
контролировать ток в цепи.  Так как у каждого электронного компонента (например,
у светодиода) есть определённые условия эксплуатации, то важно пропускать через
них ток в пределах допустимых для данного компонента значений.  Таким образом,
поставив последовательно со светодиодом резистор с номиналом, рассчитанным
исходя из характеристик светодиода, мы защитим его от преждевременного выхода из
строя.  Схема представлена на рис. \ref{fig:electronics-circuit-resistors}.

\figureElectronicsResistorCircuit{en}

\index{Электроника!Сопротивление!Последовательное подключение резисторов}
Будучи подключенными последовательно, значения сопротивлений складываются.
Проводя аналогию с водой, мы можем сказать, что удлинение трубопровода, по
которому течёт вода, повышает сопротивление этого трубопровода.  На рис.
\ref{fig:electronics-resistance-1} показана пара ёмкостей, соединённых
последовательно двумя отрезками труб; сопротивление для тока воды между
ёмкостями ``А'' и ``Б'' равно сумме сопротивлений $R_1$ и $R_2$.  Таким образом,
суммарное сопротивление трубопровода току будет больше, чем $R_1$ или же $R_2$,
взятые по-отдельности.

\begin{figure}[ht]
  \centering
  \def\offset{0}
  \begin{tikzpicture}[
      declare function={f1(\x) = 0.15 * sin(8.0 * deg(\x));
    }]

    \draw[thick] (\offset, 0) -- (\offset, 4);
    \draw[thick] (\offset + 2, 0.5) -- (\offset + 2, 4);

    \draw[thick] (\offset + 4, 0.25) -- (\offset + 4, 4);
    \draw[thick] (\offset + 6, 0) -- (\offset + 6, 4);

    %% "Труба" между ёмкостями "В" и "Г".
    \draw[thick] (\offset, 0) -- (\offset + 6, 0);
    \draw[thick] (\offset + 2, 0.5) -- (\offset + 3, 0.5);
    \draw[thick] (\offset + 3, 0.5) -- (\offset + 3, 0.25);
    \draw[thick] (\offset + 3, 0.25) -- (\offset + 4, 0.25);

    \draw[thick, color=blue, ->] (\offset + 1, 0.125) -- (\offset + 5, 0.125);

    \draw[thick, color=red, <->] (2, 1) -- (3, 1);
    \draw[color=red] (2.5, 1) node[above] {$R_1$};

    \draw[thick, color=red, <->] (\offset + 2, 1) -- (\offset + 3, 1);
    \draw[color=red] (\offset + 2.5, 1) node[above] {$R_1$};
    \draw[thick, dotted, color=red] (\offset + 3, 2) -- (\offset + 3, -1);
    \draw[thick, color=red, <->] (\offset + 3, 1) -- (\offset + 4, 1);
    \draw[color=red] (\offset + 3.5, 1) node[above] {$R_2$};

    \begin{scope}[yshift=1.5cm, color=blue]
      \draw (0, 0) plot[domain=0:2, variable=\x, samples=200, smooth] ({\x}, {f1(\x)});
    \end{scope}

    \draw (1, 0) node[below] {А};
    \draw (5, 0) node[below] {Б};

  \end{tikzpicture}
  \caption{Последовательное подключение сопротивлений на примере воды.}
  \label{fig:electronics-resistance-1}
\end{figure}

Формула \ref{equation:elemctronics-resistance-0} позволяет рассчитать общее
сопротивление проводника при последовательном соединении отдельных
сопротивлений.

\begin{equation}
  \mbox{R}_{\mbox{общее}} = \mbox{R}_{\mbox{1}} + \mbox{R}_{\mbox{2}}
  \label{equation:elemctronics-resistance-0}
\end{equation}

Электрическая схема с последовательным соединением резисторов показана на
рис. \ref{fig:electronics-circuit-resistors-in-series}.

\begin{figure}[ht]
  \centering
  \begin{circuitikz}
    \draw
    (0, 0) to[battery, l=Батарея]
    (0, 4) to[short]
    (1, 4) to[resistor, l=$R_1$] (3, 4)
    (3, 4) to[resistor, l=$R_2$] (5, 4)
    (5, 4) to[full led, l=Светодиод] (7, 4)
    (7, 4) to[short]
    (7, 0) to[short]
    (0, 0);
  \end{circuitikz}
  \caption{Последовательное подключение сопротивлений $R_1$ и $R_2$.}
  \label{fig:electronics-circuit-resistors-in-series}
\end{figure}

\index{Электроника!Сопротивление!Параллельное подключение резисторов}
Другим способом понизить сопротивление является использование нескольких
проводников, соединённых \emph{параллельно}, как показано на
рис. \ref{fig:electronics-resistance-2}.

\begin{figure}[ht]
  \centering
  \def\offset{6}
  \begin{tikzpicture}[
      declare function={f1(\x) = 0.15 * sin(8.0 * deg(\x));
    }]

    \draw[thick] (0, 0) -- (0, 4);
    \draw[thick] (2, 0.5) -- (2, 1);
    \draw[thick] (2, 1.5) -- (2, 4);
    \draw[thick] (0, 0) -- (2, 0);

    \draw[thick] (3, 0.5) -- (3, 1);
    \draw[thick] (3, 1.5) -- (3, 4);
    \draw[thick] (5, 0) -- (5, 4);
    \draw[thick] (3, 0) -- (5, 0);

    \draw[thick] (2, 0) -- (3, 0);
    \draw[thick] (2, 1.5) -- (3, 1.5);
    \draw[thick] (2, 1.0) -- (3, 1.0);
    \draw[thick] (2, 0.5) -- (3, 0.5);

    \draw[thick, color=blue, ->] (1, 1.25) -- (4, 1.25);
    \draw (3.5, 1.25) node[above, color=red] {$R_1$};
    \draw[thick, color=blue, ->] (1, 0.25) -- (4, 0.25);
    \draw (3.5, 0.25) node[above, color=red] {$R_2$};

    \begin{scope}[yshift=3cm, color=blue]
      \draw (0, 0) plot[domain=0:2, variable=\x, samples=200, smooth] ({\x}, {f1(\x)});
    \end{scope}

    \draw (1, 0) node[below] {А};
    \draw (4, 0) node[below] {Б};
  \end{tikzpicture}
  \caption{Пример двух ёмкостей, соединённых двумя трубами, идущими
    параллельно.}
  \label{fig:electronics-resistance-2}
\end{figure}

При этом, общее сопротивление электрической цепи будет меньше, чем наименьшее
сопротивление в группе.

Формула вычисления суммарного сопротивления для двух параллельно соединённых
участков цепи показана ниже (см. \ref{equation:elemctronics-resistance-1}.)

\begin{equation}
  \mbox{R}_{\mbox{общее}} = \frac{\mbox{R}_{\mbox{1}} * \mbox{R}_{\mbox{2}}}{\mbox{R}_{\mbox{1}} + \mbox{R}_{\mbox{2}}}
  \label{equation:elemctronics-resistance-1}
\end{equation}

Данная формула позволяет вычислить общее сопротивление для любого количества
параллельно соединённых сопротивлений в участке цепи.

Электрическая схема с параллельным соединением резисторов представлена на рис.
\ref{fig:electronics-circuit-parallel-resistors}.

\begin{figure}[ht]
  \centering
  \begin{circuitikz}
    \draw
    (0, 0) to[battery, l=Батарея]
    (0, 4) to[short]
    (1, 4) to[short]
    (1, 5) to[resistor, l=$R_1$] (4, 5) -- (4, 4);
    \draw
    (1, 4) to[short]
    (1, 3) to[resistor, l=$R_2$] (4, 3) -- (4, 4);
    \draw
    (4, 4) to[full led, l=Светодиод] (6, 4)
    (6, 4) to[short]
    (6, 0) to[short]
    (0, 0);
  \end{circuitikz}
  \caption{Параллельное подключение резисторов $R_1$ и $R_2$.}
  \label{fig:electronics-circuit-parallel-resistors}
\end{figure}

\end{document}
