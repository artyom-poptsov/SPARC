\documentclass[../sparc.tex]{subfiles}
\graphicspath{{\subfix{../images/}}}
\begin{document}

%%%%%%%%%%%%%%%%%%%%%%%%%%%%%%%%%%%%%%%%%%%%%%%%%%%%%%%%%%%%%%%%%%%%%%%%%%%%%%%%
\newpage
\chapter{Мелодия ``Twinkle, Twinkle, Little Star'' (версия 1.0)}
\label{app:twinkle-twinkle-little-star-01}

Данный пример показывает полный вариант примера мелодии ``Twinkle, Twinkle,
Little Star''.  Как можно видеть, пример получился достаточно длинным, отчасти
потому, что в нём не используются массивы.

\begin{minted}{cpp}
  // Ноты:
  const float c4 = 261.630;
  const float d4 = 293.660;
  const float e4 = 329.630;
  const float f4 = 349.230;
  const float g4 = 392.000;
  const float a4 = 440.000;
  const float b4 = 493.880;

  // Номер порта, куда подключен динамик:
  const int SPEAKER_PIN = 8;

  // Функция воспроизвдения звука на порту PORT
  // с частотой F и длительностью T.
  void play_tone(int port, float f, long t) {
    const long T = 1000000 / f;
    long d = T / 2;
    int count = t / T;
    for (int i = 0; i < count; i++) {
      digitalWrite(port, HIGH);
      delayMicroseconds(d);
      digitalWrite(port, LOW);
      delayMicroseconds(d);
    }
  }

  void setup() {
    pinMode(SPEAKER_PIN, OUTPUT);
  }

  void loop() {
    const long BPM = 120; // Удары в минуту.
    const long MINUTE = 60000000; // 1 минута в микросекундах.
    const long T = (MINUTE / BPM) * 4; // Длина целой ноты.
    const byte D = 100; // Задержка между нотами.

    // 0-й такт.
    play_tone(SPEAKER_PIN, c4, T / 4);
    delay(D);
    play_tone(SPEAKER_PIN, c4, T / 4);
    delay(D);
    play_tone(SPEAKER_PIN, g4, T / 4);
    delay(D);
    play_tone(SPEAKER_PIN, g4, T / 4);
    delay(D);

    // 1-й такт.
    play_tone(SPEAKER_PIN, a4, T / 4);
    delay(D);
    play_tone(SPEAKER_PIN, a4, T / 4);
    delay(D);
    play_tone(SPEAKER_PIN, g4, T / 2);
    delay(D);

    // 2-й такт.
    play_tone(SPEAKER_PIN, f4, T / 4);
    delay(D);
    play_tone(SPEAKER_PIN, f4, T / 4);
    delay(D);
    play_tone(SPEAKER_PIN, e4, T / 4);
    delay(D);
    play_tone(SPEAKER_PIN, e4, T / 4);
    delay(D);

    // 3-й такт.
    play_tone(SPEAKER_PIN, d4, T / 4);
    delay(D);
    play_tone(SPEAKER_PIN, d4, T / 4);
    delay(D);
    play_tone(SPEAKER_PIN, c4, T / 2);
    delay(D);

    // 4-й такт.
    play_tone(SPEAKER_PIN, g4, T / 4);
    delay(D);
    play_tone(SPEAKER_PIN, g4, T / 4);
    delay(D);
    play_tone(SPEAKER_PIN, f4, T / 4);
    delay(D);
    play_tone(SPEAKER_PIN, f4, T / 4);
    delay(D);

    // 5-й такт.
    play_tone(SPEAKER_PIN, e4, T / 4);
    delay(D);
    play_tone(SPEAKER_PIN, e4, T / 4);
    delay(D);
    play_tone(SPEAKER_PIN, d4, T / 2);
    delay(D);

    // 6-й такт.
    play_tone(SPEAKER_PIN, g4, T / 4);
    delay(D);
    play_tone(SPEAKER_PIN, g4, T / 4);
    delay(D);
    play_tone(SPEAKER_PIN, f4, T / 4);
    delay(D);
    play_tone(SPEAKER_PIN, f4, T / 4);
    delay(D);

    // 7-й такт.
    play_tone(SPEAKER_PIN, e4, T / 4);
    delay(D);
    play_tone(SPEAKER_PIN, e4, T / 4);
    delay(D);
    play_tone(SPEAKER_PIN, d4, T / 2);
    delay(D);

    // 8-й такт.
    play_tone(SPEAKER_PIN, c4, T / 4);
    delay(D);
    play_tone(SPEAKER_PIN, c4, T / 4);
    delay(D);
    play_tone(SPEAKER_PIN, g4, T / 4);
    delay(D);
    play_tone(SPEAKER_PIN, g4, T / 4);
    delay(D);

    // 9-й такт.
    play_tone(SPEAKER_PIN, a4, T / 4);
    delay(D);
    play_tone(SPEAKER_PIN, a4, T / 4);
    delay(D);
    play_tone(SPEAKER_PIN, g4, T / 2);
    delay(D);
  }
\end{minted}

\end{document}
