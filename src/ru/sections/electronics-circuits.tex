\documentclass[../sparc.tex]{subfiles}
\graphicspath{{\subfix{../images/}}}
\begin{document}

%%%%%%%%%%%%%%%%%%%%%%%%%%%%%%%%%%%%%%%%%%%%%%%%%%%%%%%%%%%%%%%%%%%%%%%%%%%%%%%%
\section{Электрическая цепь}
\index{Электроника!Электрическая цепь}

Представим теперь две ёмкости, в одном из которой десять литров воды, в другой --
ноль литров (или, как говорят люди, ёмкость \emph{пустая}), как показано на
рисунке \ref{fig:electronics-circuits-0}.

\figureElectronicsWaterVessels{ru}

Если мы соединим ёмкость ``А'' и ``Б'' трубой, то вода потечёт из ``А'' в ``Б'',
по законам физики (см. рисунок \ref{fig:electronics-circuits-1}.)

\figureElectronicsConnectedWaterVessels{ru}

Соединив две ёмкости, мы получили \emph{замкнутую цепь}, по которой возможен ток
воды.  Подобным образом работают электрические цепи, только вместо тока воды в
электрических цепях происходит ток элементарных заряженных частиц.

Итак, второе правило, которое мы должны усвоить: электрический ток возможен в
замкнутой цепи.

Ток в электрической цепи измеряется в \emph{Амперах} (А).

Электрическая схема, аналогичная по своей сути
рис. \ref{fig:electronics-circuits-1}, будет выглядеть, как на
рис. \ref{fig:electronics-simple-circuit}.

\begin{figure}[ht]
  \centering
  \begin{circuitikz}
    \draw (0,0)
    to[battery, l=Батарея, invert] (0,2) % The voltage source
    to[short] (2,2)
    to[full led, l=Светодиод] (2,0) % The lamp
    to[short] (0,0);
  \end{circuitikz}
  \caption{Электрическая цепь с батареей в качестве источника напряжения и светодиодом
    в качестве нагрузки.}
  \label{fig:electronics-simple-circuit}
\end{figure}

Как можно видеть на рис. \ref{fig:electronics-simple-circuit}, в качестве
полезной нагрузки -- источника света -- используется светодиод.  Светодиоды широко
используются в современных устройствах в качестве подсветки, индикаторов или
источников света.

Светодиоды имеют разные формы и размеры, но скорее всего вы рано или поздно
встретите подобные тем, что показаны на рис. \ref{fig:electronics-led}.

\begin{figure}[ht]
  \centering
  \begin{tikzpicture}
    \draw (0, 0) arc [
            start angle=0,
            end angle=180,
            x radius=1cm,
            y radius =1cm
          ];
    \draw (-2, 0) -- (-2, -1.5);
    \draw (0, 0) -- (0, -1.5);
    \draw (-2.25, -1.5) -- (0.25, -1.5) -- (0.25, -2.0) -- (-2.25, -2.0) -- (-2.25, -1.5);
    \draw[line width=0.5mm] (-1.5, -2.0) -- (-1.5, -5) node[left] {Анод};
    \draw[line width=0.5mm] (-0.5, -2.0) -- (-0.5, -4) node[right] {Катод};
  \end{tikzpicture}
  \caption{Схематическое изображение одного из видов светодиода.}
  \label{fig:electronics-led}
\end{figure}

У новых светодиодов такого вида обычна длинная ножка является плюсом, также
называемым \emph{анодом} (сокращённо ``А'').  Короткая же ножка является минусом
-- \emph{катодом} (сокращённо ``К''.)

\end{document}
