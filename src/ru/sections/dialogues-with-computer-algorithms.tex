\documentclass[../sparc.tex]{subfiles}
\graphicspath{{\subfix{../images/}}}
\begin{document}

%%%%%%%%%%%%%%%%%%%%%%%%%%%%%%%%%%%%%%%%%%%%%%%%%%%%%%%%%%%%%%%%%%%%%%%%%%%%%%%%
\section{Алгоритмы}
\index{Программирование!Алгоритм}

Задолго до появления компьютеров люди писали инструкции друг для друга, которые
позволяли понять последовательность действий для достижения какой-либо цели --
например, ``как добыть огонь'', ``как сеять зерно'', ``как запрячь лошадь'' и
т.п.

Последовательность инструкций, позволяющая достичь определённого результата,
называется \emph{алгоритмом}.

Были попытки (достаточно успешные) создания разных механизмов, выполняющих некую
последовательность операций, заменяя тем самым человека.  Но с появлением
компьютеров разработка алгоритмов вышла на ``новый уровень'' -- мы получили
возможность записать алгоритм в память компьютера для того, чтобы он выполнил
его в точности так, как мы задумывали.  Современные компьютеры не понимают язык
человека напрямую, поэтому им необходимо задавать алгоритмы в специальном
\emph{машинном языке}.  Машинный язык представляет собой коды команд обработки
данных, понимаемые процессором (главным вычислителем в компьютере.)

Написание программ на машинном языке является достаточно муторным и сложным
процессом, и вскоре после появления компьютеров были придуманы первые языки
программирования, более близкие человеку.  Первым языком программирования был
\emph{ассемблер}, который по своей сути представлял из себя набор мнемоник --
коротких человекочитаемых имён -- для машинных команды процессора.

Но и ассемблер был по своей сути слишком прост и не позволял кратко и легко
выражать идеи, которые люди хотели заложить в свои алгоритмы.  Поэтому были
придуманы высокоуровневые языки программирования, позволяющие упростить и
ускорить написание программ.  Язык ``C'', основы которого будут рассмотрены в
этой главе, является одним из старейших языков программирования, активно
используемых по сей день.

Чтобы попрактиковаться в составлении алгоритмов, предположим, что у нас есть
светодиод или лампочка, подключенная к компьютеру.  Чтобы объяснить компьютеру,
как сделать эффект мигания лампочки, мы должны сформулировать алгоритм.  Это
процесс называется \emph{формализацией}.  Возможный алгоритм мигания светодиодом
может выглядеть так:

\begin{enumerate}
\item Включить светодиод.
\item Выключить светодиод
\item Повторить алгоритм.
\end{enumerate}

Таким образом, на примере данного алгоритма мы видим не только действия по
включению и выключению условного светодиода, но и некоторую повторяемость,
\emph{цикличность} этих действий.

\experiment{0} { Попробуйте придумать алгоритмы для обыденных операций, которые
  вы выполняете каждый день.  Например, алгоритм заваривания чая, или же
  алгоритм уборки квартиры.  Насколько детально вы сможете объяснить порядок
  выполнениям простых действий?  Насколько сложными получаются алгоритмы?}

\end{document}
