\documentclass[../sparc.tex]{subfiles}
\graphicspath{{\subfix{../images/}}}
\begin{document}

\newpage
%%%%%%%%%%%%%%%%%%%%%%%%%%%%%%%%%%%%%%%%%%%%%%%%%%%%%%%%%%%%%%%%%%%%%%%%%%%%%%%%
\section{Делитель напряжения}

\subsection{Реализация простого делителя напряжения на резисторах}

Теперь, когда мы научились читать сигнал с аналового порта, нам необходимо
подключить что-то осмысленное к Arduino на аналоговый порт -- некое устройство,
сигнал которого мы можем физически регулировать, таким образом влияя на значение
на аналовогом порту.

\note{ru}{ При реализации делителя напряжения необходимо учитывать, какое
  максимальное напряжение можно подать на порт микроконтроллерной платформы.
  Например, для Arduino Mega 2560 и Arduino UNO это напряжение составляет 5В,
  тогда как для некоторых других платформ напряжение может быть ограничено 3.3В.
  При сборке схем всегда следует сверяться с документацией микроконтроллерную
  платформу! }

Начнём с простого эксперимента.  Возьмём два резистора по 5КОм, и подключим их
так, как показано на схеме \ref{fig:electronics-arduino-circuit-01}.

\figureVoltageDivider{ru}

Как видно, в точке ``A'' находится разветвление цепи, где один из проводников
идёт к порту ``A0'' на Arduino.  Поскольку у тока есть два пути, куда он может
течь (либо на ``A0'', либо на ``GND''), то ток делится на две части.  Если
резисторы $R_1$ и $R_2$ имеют одинаковый номинал (допустим, 5КОм), то тогда ток
делится на две равные части.

Схема, которую мы с вами только что собрали, называется \emph{Делитель
напряжения}.  Подобная схема позволяет поделить входное напряжение на некоторый
коэффициент, значение которого зависит от значений резисторов $R_1$ и $R_2$.
Чем меньше значение $R_1$ и больше значение $R_2$, тем ближе напряжение на
выходе делителя к его входному напряжению (в нашем случае, к 5В.)  И наоборот,
чем больше значение $R_1$ и меньше значение $R_2$, тем ближе выходное значение
делителя к 0В.

%%%%%%%%%%%%%%%%%%%%%%%%%%%%%%%%%%%%%%%%%%%%%%%%%%%%%%%%%%%%%%%%%%%%%%%%%%%%%%%%
\subsection{Реализация делителя напряжения на потенциометре}

Таким образом, подбором подходящих номиналов $R_1$ и $R_2$ на схеме
\ref{fig:electronics-arduino-circuit-01} мы можем получить желаемое напряжение в
точке ``A'' и, следовательно, на аналоговом входе ``A0''.  Однако это не удобно,
если мы хотим регулировать напряжение (и значение на аналоговым входе) в
реальном времени.  Для таких задач гораздо лучше подходит \emph{переменный
резистор}, также часто называемый \emph{потенциометром}.

Схема делителя напряжения на базе потенциометра показана на
рис. \ref{fig:electronics-potentiometer}.

\figureVoltageDividerPotentiometer{ru}

Потенциометр обычно имеет три контакта: назовём их левый, центральный и правый --
в соответствии со схематическим изображением потенциометра на
рис. \ref{fig:electronics-potentiometer}.  Между левым и правым контактом
сопротивление всегда фиксировано, как и в обычном резисторе, и зависит от
номинала потенциометра -- например, оно может быть равно 10КОм.  Однако
центральный контакт подвижен (обычно перемещается специальной поворотной ручкой)
-- сопротивление центральным и любым из крайних контактов зависит от положения
ручки потценциометра.  Таким образом, потенциометр как бы объединяет в себе два
резистора, сумма сопротивлений которых задана номиналом потенциометра, при этом
отношение их сопротивлений может быть изменено.

Потенциометр предоставляет удобный способ регулировать значение на аналоговом
входе Arduino.  Если мы загрузим в Arduino программу из листинга
\ref{listing:analog-ports-get-value}, то ``Плоттер по последовательному
соединению'' (вызываемый из меню ``Инструменты'') отобразит изменение
аналогового значения на графике при повороте ручки потенциометра.

\begin{listing}[ht]
  \begin{minted}{cpp}
    void setup() {
      pinMode(2, OUTPUT);
    }

    void loop() {
      int value = analogRead(A0);
      digitalWrite(2, HIGH);
      delay(value);
      digitalWrite(2, LOW);
      delay(value);
    }
  \end{minted}
  \label{listing:analog-ports-001}
  \caption{Регулировка скорости мигания светодиода значением с аналогового
    порта.}
\end{listing}

В листинге \ref{listing:analog-ports-001} показан пример регулировки скорости
мигания светодиода с использованием аналогового значения с порта ``A0''.  Если
собрана схема \ref{fig:electronics-potentiometer}, то поворот ручки
потенциометра будет влиять на скорость мигания светодиода за счёт изменения
задержки между включением и выключением.

%%%%%%%%%%%%%%%%%%%%%%%%%%%%%%%%%%%%%%%%%%%%%%%%%%%%%%%%%%%%%%%%%%%%%%%%%%%%%%%%
\subsection{Задачи}

\begin{enumerate}
\item Модифицируйте код из примера \ref{listing:serial-port-sine-wave-example},
  чтобы на параметры синусоиды можно было влиять, крутя ручку потенциометра.
\item Реализуйте регулировку скорости ``Бегущего огня'' через ручку
  потенциометра.
\item Подключите 5 или более светодиодов к Arduino.  Реализуйте индикацию угла
  поворота ручки потенциометра через зажигание светодиодов: чем ближе поворот
  ручки к одному из крайних положений, тем больше светодиодов должно зажечься; и
  наоборот, чем ближе поворот к другому крайнему значению, тем меньше
  светодиодов должны гореть.
\end{enumerate}

\end{document}
