\documentclass[../sparc.tex]{subfiles}
\graphicspath{{\subfix{../images/}}}
\begin{document}

\newpage
%%%%%%%%%%%%%%%%%%%%%%%%%%%%%%%%%%%%%%%%%%%%%%%%%%%%%%%%%%%%%%%%%%%%%%%%%%%%%%%%
\section{Делитель напряжения}

Теперь, когда мы научились читать сигнал с аналового порта, нам необходимо
подключить что-то осмысленное к Arduino на аналоговый порт -- некое устройство,
сигнал которого мы можем физически регулировать, таким образом влияя на значение
на аналовогом порту.

Начнём с простого эксперимента.  Возьмём два резистора по 200 Ом, и подключим их
так, как показано на схеме \ref{fig:electronics-arduino-circuit-01}.

\figureVoltageDivider{ru}

Как видно, в точке ``A'' находится разветвление цепи, где один из проводников
идёт к порту ``A0'' на Arduino.  Поскольку у тока есть два пути, куда он может
течь (либо на ``A0'', либо на ``GND''), то ток делится на две части.  Если
резисторы $R_1$ и $R_2$ имеют одинаковый номинал (допустим, 200 Ом), то тогда
ток делится на две равные части.

Схема, которую мы с вами только что собрали, называется \emph{Делитель
напряжения}.  Подобная схема позволяет поделить входное напряжение на некоторый
коэффициент, значение которого зависит от значений резисторов $R_1$ и $R_2$.
Чем меньше значение $R_1$ и больше значение $R_2$, тем ближе напряжение на
выходе делителя к его входному напряжению (в нашем случае, к 5В.)  И наоборот,
чем больше значение $R_1$ и меньше значение $R_2$, тем ближе выходное значение
делителя к 0В.

\figureVoltageDividerPotentiometer{ru}

%%%%%%%%%%%%%%%%%%%%%%%%%%%%%%%%%%%%%%%%%%%%%%%%%%%%%%%%%%%%%%%%%%%%%%%%%%%%%%%%
\subsection{Задачи}

\begin{enumerate}
\item Модифицируйте код из примера \ref{listing:serial-port-sine-wave-example},
  чтобы на параметры синусоиды можно было влиять, крутя ручку потенциометра.
\end{enumerate}

\end{document}
