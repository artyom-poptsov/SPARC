\documentclass[../sparc.tex]{subfiles}
\graphicspath{{\subfix{../images/}}}
\begin{document}

%%%%%%%%%%%%%%%%%%%%%%%%%%%%%%%%%%%%%%%%%%%%%%%%%%%%%%%%%%%%%%%%%%%%%%%%%%%%%%%%
\section{Делитель напряжения}

Теперь, когда мы научились читать сигнал с аналового порта, нам необходимо
подключить что-то осмысленное к Arduino на аналоговый порт -- некое устройство,
сигнал которого мы можем физически регулировать, таким образом влияя на значение
на аналовогом порту.

Начнём с простого эксперимента.  Возьмём два резистора по 200 Ом, и подключим их
так, как показано на схеме \ref{fig:electronics-arduino-circuit-01}.

\begin{figure}[ht]
  \centering
  \begin{circuitikz}
    \draw (4, 0) node [
      dipchip,
      num pins=4,
      external pins width=0.0,
      no topmark,
      hide numbers,
      xscale = 2.5,
      yscale = 2.5
    ](C1){Arduino};
    \node [above left, font=\small] at (C1.bpin 1) {A0};
    \node [above right, font=\small] at (C1.bpin 2) {5V};
    \node [above right, font=\small] at (C1.bpin 3) {GND};
    \draw
    (C1.bpin 2) to[short]
    (0, -0.7) to[short]
    (0, 4) to[resistor, l=$R_1$] (4, 4)
    (4, 4) to[resistor, l=$R_2$] (8, 4)
    (8, 4) to[short]
    (8, -0.7) to[short]
    (C1.bpin 3);
    \draw
    (4, 4) to[short]
    (4, 2) to[short]
    (1, 2) to[short]
    (1, 0.7) to[short]
    (C1.bpin 1);
    \node[circle, fill, inner sep=1pt] at (4, 4){};
    \node[above, color=red] at (4, 4) {A};
  \end{circuitikz}
  \caption{Схема с двумя резисторами.}
  \label{fig:electronics-arduino-circuit-01}
\end{figure}

Как видно, в точке ``A'' находится разветвление цепи, где один из проводников
идёт к порту ``A0'' на Arduino.  Поскольку у тока есть два пути, куда он может
течь (либо на ``A0'', либо на ``GND''), то ток делится на две части.  Если
резисторы $R_1$ и $R_2$ имеют одинаковый номинал (допустим, 200 Ом), то тогда
ток делится на две равные части.

Схема, которую мы с вами только что собрали, называется \emph{Делитель
напряжения}.

\end{document}
