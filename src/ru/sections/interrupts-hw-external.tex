\documentclass[../sparc.tex]{subfiles}
\graphicspath{{\subfix{../images/}}}
\begin{document}

%%%%%%%%%%%%%%%%%%%%%%%%%%%%%%%%%%%%%%%%%%%%%%%%%%%%%%%%%%%%%%%%%%%%%%%%%%%%%%%%
\section{Внешние аппаратные прерывания}

В данном разделе мы рассмотрим обработку внешних (по отношению к
микроконтроллеру) прерываний, а именно прерывания от кнопок.  Кнопки
подключаются на цифровые порты Arduino, и используются для изменения уровня
напряжения на цифровом порту.

Внешние аппаратные прерывания являются хорошим примером для начала изучения темы
этой главы, так как Arduino предоставялет удобные библиотечные процедуры для
работы с подобными прерываниями.

%%%%%%%%%%%%%%%%%%%%%%%%%%%%%%%%%%%%%%%%%%%%%%%%%%%%%%%%%%%%%%%%%%%%%%%%%%%%%%%%
\subsection{Обработчик прерывания}

Как уже говорилось ранее, \emph{обработчиком прерывания} (который по-английски
называется ``Interrupt Service Routine'', или сокращённо \gls{ISR}) называется
процедура, которая регистрируется для вызова по прерыванию (то есть, по
некоторому событию.)

По сути, ISR ничем не отличается от обычной процедуры, кроме того, что она не
вызывается нами явно в коде, а вместо этого вызывается процессором по событию.

Допустим, мы можем объявить обработчик прерывания, который меняет значение на
цифровом порту номер 13 с высокого \texttt{HIGH} на низкий \texttt{LOW} и
обратно, таким образом включая или выключая подключенный к данному порту
светодиод при каждом срабатывании.

\begin{listing}[ht]
  \begin{minted}{cpp}
    void toggle_led() {
      digitalWrite(13, ! digitalRead(13));
    }
  \end{minted}
  \caption{Процедура, выполняющая роль обработчика прерывания}
  \label{listing:interrupts-handler}
\end{listing}

В листиге \ref{listing:interrupts-handler} для инверсии значения на цифровом
порту используется трюк, заключающийся в том, что мы читаем с порта текущее
значение (либо \texttt{HIGH}, либо \texttt{LOW}) и инверируем его с помощью
логической операциии ``НЕ'' (записываемой в виде воскликцательного знака ``!''),
а потом выставляем полученное значение на порт.  Если на 13-м порту было
значение \texttt{LOW}, то оно после инверсии становится \texttt{HIGH}, и
наоборот.

Как можно видеть, объявление обработчика \texttt{toggle\_led} ничем не
отличается от объявления обычной процедуры.  Даже название данной процедуры
может быть произвольным, в рамках допустимых наименований процедур в C++.
Однако сама по себе эта процедура не будет зарегистрирована, как обработчик --
нам нужно воспользоваться процедурой \texttt{attachInterrupt} для её
регистрации, которая будет описана в следующем подразделе.

%%%%%%%%%%%%%%%%%%%%%%%%%%%%%%%%%%%%%%%%%%%%%%%%%%%%%%%%%%%%%%%%%%%%%%%%%%%%%%%%
\subsection{Регистрация обработчика прерываний}

Существует четыре вида событий, на которые можно привязать обработчик прерывания
(\gls{ISR}):
\begin{itemize}
\item \texttt{LOW} -- Событие происходит, когда на цифровом порту устанавливается
  низкий уровень напряжения (\texttt{LOW}) на цифровом порту.  Данное событие
  может повторятся, если значение на порту не меняется.
\item \texttt{CHANGE} -- Изменение значения на цифровом порту.  Данное событие
  происходит, когда на цифровом порту значение меняется либо с \texttt{LOW} до
  \texttt{HIGH}, либо наоборот.
\item \texttt{RISING} -- Событие происходит, когда напряжение на цифровом порту
  меняется с \texttt{LOW} на \texttt{HIGH}.
\item \texttt{FALLING} -- Событие происходит, когда напряжение на цифровом порту
  меняется с \texttt{HIGH} на \texttt{LOW}.
\end{itemize}

\end{document}
