\documentclass[../sparc.tex]{subfiles}
\graphicspath{{\subfix{../images/}}}
\begin{document}

%%%%%%%%%%%%%%%%%%%%%%%%%%%%%%%%%%%%%%%%%%%%%%%%%%%%%%%%%%%%%%%%%%%%%%%%%%%%%%%%
\section{Структура программы на Arduino}

Нашим языком для описания алгоритмов будет язык программирования общего
назначения С++.  Это -- один из популярных языков программирования (на момент
написания книги) и имеет массу интересных применений.  Одним из таких применений
является программирования микроконтроллеров.

Язык программирования предоставляет программисту основу для выражения идей, при
этом недостающие части можно дописать, расширив таким образом набор
инструментов.  В нашем случае среда разработки под платформу Arduino (Arduino
IDE) поставляется вместе с набором необходимых библиотек, упрощающих разработку
под данную платформу.

Кроме того, для упрощения начала работы, при создании пустого проекта
программисту сразу предоставляется базовая структура программы.  Программа для
Arduino обычно состоит из двух основных частей, также называемых
\emph{функциями}: \texttt{setup} и \texttt{loop}.  Заготовка программы для
Arduino обычно выглядит так:

\begin{minted}{cpp}
  void setup() {
    // put your setup code here, to run once:
  }

  void loop() {
    // put your main code here, to run repeatedly:
  }
\end{minted}

Строки, начинающиеся с двойного слэша (``//'') считаются комментариями и
игнорируются компьютером.  Обычно эти строки можно добавлять или удалять, и это
не будет влиять на логику работы программы, так как данные комментарии
предназначены для людей.

Функция \texttt{setup} производит инициализацию микроконтроллера при его
включении.  В неё следует помещать все команды, которые должны выполняться
единожды на старте системы.

Функция \texttt{loop} выполняется после завершения \texttt{setup} и
автоматически перезапускается системой, как только она завершается.  Таким
образом, сам по себе \texttt{loop} реализует повторение написанного в нём кода --
до тех пор, пока питание микроконтроллера не будет отключено.

В предыдущем подразделе мы составили алгоритм мигания светодиодом.  Попробуем
реализовать этот алгоритм для Arduino.

Для того, чтобы управлять светодиодом, мы должны подключить его к
\emph{цифровому порту}.  \emph{Цифровой порт} (или, по-другому, \emph{пин})
называется так потому, что оперирует цифровым сигналом -- то есть, работает
только с двумя логическими уровнями напряжения, соответствующие единице и нулю.

Кроме того, цифровой порт на Arduino может быть настроен на один из двух
основных режимов:
\begin{enumerate}
\item В режиме \emph{входа} (\texttt{INPUT}) пин считывает напряжение с
  некоторого внешнего источника.
\item в режиме \emph{выхода} (\texttt{OUTPUT}) – позволяет выдавать на порту
  напраяжение на некоторую внешнюю схему.
\end{enumerate}

Таким образом, для реализации эффекта мигания светодиодом нам необходимо первым
делом настроить режим работы цифрового порта, к которому подключен светодиод, на
режим выхода (\texttt{OUTPUT}.)  Это разумно делать в функции \texttt{setup},
так как настройку достаточно выполнить единожды, при старте системы.

Далее в функции \texttt{loop} мы можем приступить к реализации собственно
алгоритма мигания светодиодом, согласно составленному в предыдущем подразделе
алгоритму.  При этом, нам надо заменить словесное описание шагов алгоритма на
команды, понятные компьютеру:

\begin{enumerate}
\item \texttt{digitalWrite(2, HIGH);} // Включить светодиод.
\item \texttt{delay(500);} // Подождать 500 мс.
\item \texttt{digitalWrite(2, LOW);} // Выключить светодиод
\item \texttt{delay(500);} // Подождать 500 мс.
\item Повторить алгоритм.
\end{enumerate}

Один из возможных вариантов реализации показан ниже:

\begin{minted}{cpp}
void setup() {
  pinMode(2, OUTPUT);
}

void loop() {
  digitalWrite(2, HIGH);
  delay(500);
  digitalWrite(2, LOW);
  delay(500);
}
\end{minted}

Рассмотрим приведённый выше пример.  В \texttt{setup} выполняется функция
\texttt{pinMode}, которая позволяет настроить режим работы указанного пина как
выход (\texttt{OUTPUT}).  Общий синтаксис вызова \texttt{pinMode} выглядит так:

\begin{minted}{cpp}
  pinMode(port, mode);
\end{minted}

где \texttt{port} -- номер порта, а \texttt{mode} -- режим работы (\texttt{INPUT},
\texttt{INPUT\_PULLUP} или \texttt{OUTPUT}.)

В \texttt{loop} мы используем две функции: \texttt{digitalWrite} и
\texttt{delay}.

Функция \texttt{digitalWrite} имеет следующий синаксис вызова:

\begin{minted}{cpp}
  digitalWrite(port, value);
\end{minted}

где \texttt{port} -- номер пина, \texttt{value} -- уровень напряжения: высокий
(\texttt{HIGH}, обычно соответствует значению 5В), либо же низкий (\texttt{LOW},
соответствует 0В.)

Функция \texttt{delay} позволяет нам остановить выполнение программы на
указанный промежуток времени, указанный в миллисекундах (тысячных долях
секунды.)  Общий синтаксис вызова этой функции выглядит так:

\begin{minted}{cpp}
  delay(value);
\end{minted}

Где \texttt{value} -- количество миллисекунд.

\subsection{Задачи}
\begin{itemize}
\item Соберите на макетной плате ``бегущий огонь'': светодиоды должны поочерёдно
  включаться и выключаться, один за другим.  Используйте минимум 5 светодиодов.
\item Модифицируйте ``бегущий огонь'' так, чтобы он бежал сначала в одну
  сторону, затем в другую.
\end{itemize}

\end{document}
