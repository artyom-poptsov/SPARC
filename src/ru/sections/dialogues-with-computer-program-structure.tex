\documentclass[../sparc.tex]{subfiles}
\graphicspath{{\subfix{../images/}}}
\begin{document}

%%%%%%%%%%%%%%%%%%%%%%%%%%%%%%%%%%%%%%%%%%%%%%%%%%%%%%%%%%%%%%%%%%%%%%%%%%%%%%%%
\section{Структура программы на Arduino}

Программа для Arduino обычно состоит из двух основных частей, также называемых
функциями: \texttt{setup} и \texttt{loop}. Пример программы, которая мигает
одним светодиодом:

\begin{minted}{cpp}
void setup() {
  pinMode(2, OUTPUT);
}

void loop() {
  digitalWrite(2, HIGH);
  delay(500);
  digitalWrite(2, LOW);
  delay(500);
}
\end{minted}

Функция \texttt{setup} производит инициализацию микроконтроллера при его
включении. В неё следует помещать все команды, которые должны выполняться
единожды на старте системы.

Цифровой порт (или, по-другому, пин) Arduino может находиться в двух состояниях.
В режиме входа пин считывает напряжение, а в режиме выхода – позволяет выдавать
на пине такое же напряжение.

Рассмотрим приведённый выше пример. В \texttt{setup} выполняется функция
\texttt{pinMode}, которая позволяет настроить режим работы указанного пина как
вход или выход:

\begin{minted}{cpp}
pinMode(pin, mode);
\end{minted}

где \texttt{pin} -- номер пина, \texttt{mode} -- режим работы
(\texttt{INPUT}/\texttt{OUTPUT}).

В \texttt{loop} вызываются две функции: \texttt{digitalWrite} и \texttt{delay}.

Функция
\begin{minted}{cpp}
digitalWrite(pin,value);
\end{minted}

где \texttt{pin} -– номер пина, \texttt{value} -- уровень сигнала
(\texttt{HIGH}/\texttt{LOW}), подаёт на пин высокое или низкое напряжение.

Функция
\begin{minted}{cpp}
delay(value);
\end{minted}

где \texttt{value} -- количество миллисекунд, останавливает выполнение программы
на указанное время.

\subsection{Задачи}
\begin{itemize}
\item Соберите на макетной плате ``бегущий огонь'': светодиоды должны поочерёдно
  включаться и выключаться, один за другим.
\item Модифицируйте ``бегущий огонь'' так, чтобы он бежал сначала в одну
  сторону, затем в другую.
\end{itemize}

\end{document}
