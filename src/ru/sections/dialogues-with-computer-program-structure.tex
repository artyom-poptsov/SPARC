\documentclass[../sparc.tex]{subfiles}
\graphicspath{{\subfix{../images/}}}
\begin{document}

%%%%%%%%%%%%%%%%%%%%%%%%%%%%%%%%%%%%%%%%%%%%%%%%%%%%%%%%%%%%%%%%%%%%%%%%%%%%%%%%
\section{Структура программы на Arduino}

Программа для Arduino обычно состоит из двух основных частей, также называемых
функциями: \texttt{setup} и \texttt{loop}.  Пример программы, которая мигает
одним светодиодом:

\begin{minted}{cpp}
void setup() {
  pinMode(2, OUTPUT);
}

void loop() {
  digitalWrite(2, HIGH);
  delay(500);
  digitalWrite(2, LOW);
  delay(500);
}
\end{minted}

Функция \texttt{setup} производит инициализацию микроконтроллера при его
включении. В неё следует помещать все команды, которые должны выполняться
единожды на старте системы.

\emph{Цифровой порт} (или, по-другому, \emph{пин}) Arduino может находиться в
одном из двух состояний.  В режиме \emph{входа} (\texttt{INPUT}) пин считывает
напряжение с некоторого внешнего источника, а в режиме \emph{выхода}
(\texttt{OUTPUT}) – позволяет выдавать на порту напраяжение на некоторую внешнюю
схему.

Рассмотрим приведённый выше пример. В \texttt{setup} выполняется функция
\texttt{pinMode}, которая позволяет настроить режим работы указанного пина как
выход (\texttt{OUTPUT}).  Общий синтаксис вызова \texttt{pinMode} выглядит так:

\begin{minted}{cpp}
  pinMode(port, mode);
\end{minted}

где \texttt{port} -- номер пина, а \texttt{mode} -- режим работы (\texttt{INPUT},
\texttt{INPUT\_PULLUP} или \texttt{OUTPUT}.)

В \texttt{loop} мы используем две функции: \texttt{digitalWrite} и
\texttt{delay}.

Функция \texttt{digitalWrite} имеет следующий синаксис вызова:

\begin{minted}{cpp}
  digitalWrite(port, value);
\end{minted}

где \texttt{port} -- номер пина, \texttt{value} -- уровень напряжения: высокий
(\texttt{HIGH}, обычно соответствует значению 5В), либо же низкий (\texttt{LOW},
соответствует 0В.)

Функция \texttt{delay} позволяет нам остановить выполнение программы на
указанный промежуток времени, указанный в миллисекундах.  Общий синтаксис вызова
этой функции выглядит так:

\begin{minted}{cpp}
  delay(value);
\end{minted}

Где \texttt{value} -- количество миллисекунд.

\subsection{Задачи}
\begin{itemize}
\item Соберите на макетной плате ``бегущий огонь'': светодиоды должны поочерёдно
  включаться и выключаться, один за другим.  Используйте минимум 5 светодиодов.
\item Модифицируйте ``бегущий огонь'' так, чтобы он бежал сначала в одну
  сторону, затем в другую.
\end{itemize}

\end{document}
