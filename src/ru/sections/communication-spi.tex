\documentclass[../sparc.tex]{subfiles}
\graphicspath{{\subfix{../images/}}}
\begin{document}

\newpage
%%%%%%%%%%%%%%%%%%%%%%%%%%%%%%%%%%%%%%%%%%%%%%%%%%%%%%%%%%%%%%%%%%%%%%%%%%%%%%%%
\section{Шина Serial Peripheral Interface (SPI)}
\label{section:spi}
\index{Электроника!Шина SPI}
\newglossaryentry{SPI}{name=SPI, description={Serial Peripheral Interface}}

%%%%%%%%%%%%%%%%%%%%%%%%%%%%%%%%%%%%%%%%%%%%%%%%%%%%%%%%%%%%%%%%%%%%%%%%%%%%%%%%
\subsection{Общие сведения}

\textit{\gls{SPI}} -- синхронная шина передачи данных, применяемая в устройствах
для связи между компонентами на коротких дистанциях.

Шина использует архитектуру ``ведущий-ведомый'', где ведущее устройство (обычно
называемое ``Master'') управляет коммуникацией с ведомым устройством (обычно
называемым ``Slave''), инициируя передачу данных.  Ведомых устройство может быть
несколько.  Некоторые устройства поддерживают смену роли ``ведущий-ведомый'' на
лету.

Протокол работает в полнодуплексном режиме, и использует четыре логических линии:

\begin{itemize}
\item \textbf{SS} (Slave Select) -- сигнал с ведущего устройства, инициирующий
  коммуникацию с определённым устройством на шине SPI.  Активным значением линии
  является низкий уровень.  На каждое ведомое устройство требуется отдельная
  линия SS, таким образом количество этих линий звисит от количества ведомых
  устройств.
\item \textbf{SCLK} (Serial Clock) -- тактовый сигнал с ведущего устройства.
\item \textbf{MOSI} (Serial Data Output from Master) -- Линия последовательной
  передачи данных с ведущего устройства на ведомое.
\item \textbf{MISO} (Serial Data Output from Slave) -- Линия последовательной
  передачи данных с ведомого устройства на ведущее.
\end{itemize}

Схематически шину SPI можно представить, как показано на рис. \ref{fig:spi-bus}.

\figureSPI{ru}

\end{document}
