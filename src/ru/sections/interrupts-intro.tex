\documentclass[../sparc.tex]{subfiles}
\graphicspath{{\subfix{../images/}}}
\begin{document}

\newglossaryentry{ISR}{
  name=ISR,
  description={Interrupt Service Routine, подпрограмма обработки прерываний.}}

\emph{Прерывание} -- это одна из базовых концепций в вычислительной технике,
заключающаяся в том, что при наступлении какого-либо события процессор
оперативно приостанавливает выполнение программного кода, с которым работал в
данный момент, и переключается на специальную процедуру, называемую
\emph{обработчиком прерываний} (Interrupt Service Routine, ISR.)

При переключении на обработчик прерываний процессор запоминает, с чем работал до
этого, чтобы после завершения обработчика вернуться на прерыванную задачу. После
завершения работы обработчика прерывания управление вновь передаётся прерванной
программе, которая должна начать прерванный процесс в том же состоянии, в
котором она находилась в момент прерывания.\cite[452-456]{tanenbaum2021-ru}

\end{document}
