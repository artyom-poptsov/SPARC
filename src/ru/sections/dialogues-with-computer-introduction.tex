\documentclass[../sparc.tex]{subfiles}
\graphicspath{{\subfix{../images/}}}
\begin{document}

%%%%%%%%%%%%%%%%%%%%%%%%%%%%%%%%%%%%%%%%%%%%%%%%%%%%%%%%%%%%%%%%%%%%%%%%%%%%%%%%
\section{Введение}
По нашему мнению, программирование является двоякой дисциплиной: с одной
стороны, это – один из видов творчества, позволяющий человеку создать что-то
необычное, новое и, возможно, полезное для общества; с другой стороны, это –
инструмент, позволяющий решать практические, прикладные задачи. Как кисти и
краски художника, или инструменты музыканта, инструменты программиста имеют
большое разнообразие в видах и применениях. Чтобы освоить их в полной мере
требуются годы. Тем не менее, долгий путь начинается с первого шага.

Данная глава позволяет людям, желающим освоить программирование, сделать первый
шаг на пути в профессии программиста.  Вам предлагается изучить основы
программирования на C/С++, разрабатывая приложения для платформы Arduino.

\end{document}
