\documentclass[../sparc.tex]{subfiles}
\graphicspath{{\subfix{../images/}}}
\begin{document}

%%%%%%%%%%%%%%%%%%%%%%%%%%%%%%%%%%%%%%%%%%%%%%%%%%%%%%%%%%%%%%%%%%%%%%%%%%%%%%%%
\section{Введение}
Можно сказать, что программирование является двоякой дисциплиной: с одной
стороны, это – один из видов творчества, позволяющий человеку выразить себя,
создать что-то необычное и новое; с другой стороны, это – инструмент,
позволяющий решать практические, прикладные задачи.  Как кисти и краски
художника, или инструменты музыканта, инструменты программиста имеют большое
разнообразие в видах и применениях.  Чтобы освоить их в полной мере, требуются
годы.  Тем не менее, долгий путь начинается с первого шага.

Данная глава позволяет людям, желающим освоить программирование, сделать первый
шаг на пути в профессию программиста.  Вам предлагается изучить основы
программирования на C/С++, разрабатывая приложения для платформы Arduino.

\end{document}
