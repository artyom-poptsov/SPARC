\documentclass[../sparc.tex]{subfiles}
\graphicspath{{\subfix{../images/}}}
\begin{document}

%%%%%%%%%%%%%%%%%%%%%%%%%%%%%%%%%%%%%%%%%%%%%%%%%%%%%%%%%%%%%%%%%%%%%%%%%%%%%%%%
\newpage
\section{Базовые принципы благозвучия}
\index{Музыка!Консонанс}
\index{Музыка!Диссонанс}

Человеческий слух устроен так, что нам может нравиться сочетание одних частот
звуков, и не нравиться сочетание других. Говорят, что те звуки, которые хорошо
сочетаются, образуют \emph{консонанс}, тогда как ``неприятные'' комбинации
образуют \emph{диссонанс}.

Например, частоты 50 Гц и 100 Гц звучат в сочетании довольно неплохо, хотя и не
являются музыкальными -- а всё потому, что одна частота ровно в два раза больше
другой. Понимание такого малозначимого, казалось бы, факта, позволяет нам
строить достаточно мелодичные произведения путём сочетания звуков, кратных по
частоте друг другу.

Воспроизведение одинаковых по частоте звуков образует идеальный консонанс, так
как их волны накладываются друг на друга, при этом усиливаясь.

\experiment{0}{ Запрограммируйте две Arduino таким образом, чтобы они
  воспроизводили одинаковые частоты и включите их одновременно. Сможете ли вы
  услышать, что частоты совпадают?  Что вы чувствуете, когда это происходит?

  Перепрограммируйте Arduino, чтобы частоты различались на 10Гц, 20Гц, 30Гц и
  т.д., увеличивая шаг сначала на 10Гц, потом на 100Гц. Какие комбинации
  получились неприятные, какие терпимые, какие приятные?}

\experiment{1}{ Попробуйте воспроизвести какую-нибудь звуковую частоту на
  Ardu\-ino, и одновременно подстроить звук вашего голоса под этот звук, пропев
  какую-нибудь гласную (например, ``А'') -- возможно у вас произойдёт тот самый
  ``Ага!'' момент, когда вы услышите совпадение частоты вашего голоса и звука
  динамика.}

\end{document}
