\documentclass[../sparc.tex]{subfiles}
\graphicspath{{\subfix{../images/}}}
\begin{document}

%%%%%%%%%%%%%%%%%%%%%%%%%%%%%%%%%%%%%%%%%%%%%%%%%%%%%%%%%%%%%%%%%%%%%%%%%%%%%%%%
\section{ЖК-дисплей}
\index{Разработка игр!Дисплей}

Для вывода информации мы будем использовать \emph{жидкокристаллический дисплей}
(ЖК-дисплей), подключаемый к Arduino.  Принцип работы данных дисплеев аналогичен
обычным ЖК-дисплеям, которые выводят информацию на вашем компьютере.  Более
конкретно мы будем использовать \emph{текстовый} ЖК-дисплей, который
предназначен для вывода текстовых символов.

На текстовых ЖК-дисплеях, экран поделён на клетки, внутри каждой из которых
можно отрисовать один символ (букву, знак препинания, цифру, или просто
какую-либо картинку.)  Между клетками обычно находится расстояние в один
пиксель.  Подобные дисплеи плохо подходят для отрисовки произвольных
изображений, тем не менее, некоторый простор для творчества у нас имеется.

Несмотря на свойства дисплея, которые на первый взгляд кажутся слишком
ограничивающими для наших задач, используя наше мастерство и творческий подход,
мы можем добиться достаточно интересных результатов в плане разработки игр.

Рабочая область дисплея 16x2 (16 столбцов, 2 строки) схематически выглядит, как
показано на рис. \ref{fig:lcd-16x2-schematics}.

\figureSmallLCDShematics{ru}

Внешний вид ЖК-дисплея 16x2 схематически показан на рис. \ref{fig:lcd-16x2}.

\figureSmallLCD{ru}

По возможности рекомендуется использовать для проектов, предложенных в данной
главе, ЖК-дисплей размером 20x4 (20 столбцов, 4 строки.)  Рабочая область
подобного дисплея больше (как показано на рис. \ref{fig:lcd-20x4-schematics}),
что позволяет реализовать более сложные игры.

\figureBigLCDShematics{ru}

Внешний вид ЖК-дисплея 20x4 схематически показан на рис. \ref{fig:lcd-20x4}.

\figureBigLCD{ru}

%%%%%%%%%%%%%%%%%%%%%%%%%%%%%%%%%%%%%%%%%%%%%%%%%%%%%%%%%%%%%%%%%%%%%%%%%%%%%%%%
\subsection{Подключение дисплея}
\index{Электроника!ЖК-дисплей}

Для упрощения нашей работы мы будем использовать ЖК-дисплей с интерфейсом
передачи данных, который называется \gls{I2C} (читается ``ай-ту-си''.)

Подробно об I2C было рассказано в разделе \ref{section:i2c}.

Сейчас же, не вдаваясь в подробности можно сказать, что данный вариант
подключения требует всего 4 провода: питание, земля и две линии передачи данных.

Сам модуль I2C часто уже припаян к дисплею, но может идти и отдельно.  В случае
отдельного модуля, подключение выглядит, как на рис. \ref{fig:lcd-00}.

\figureLCDConnection{ru}

\note{ru}{ Обратите внимание, что схема подключения дисплея может отличатся, в
  зависимости от его производителя.  Во избежание порчи дисплея, крайне
  рекомендуется ознакомиться с документацией на него -- как правило, она есть на
  сайте производителя, либо же может быть найдена по запросу в поисковике вида
  ``<модель-дисплея> datasheet'', где вместо ``<модель-дисплея>'' впишите именно
  модель того дисплея, который у вас. }

%%%%%%%%%%%%%%%%%%%%%%%%%%%%%%%%%%%%%%%%%%%%%%%%%%%%%%%%%%%%%%%%%%%%%%%%%%%%%%%%
\subsection{Вывод текста}
\index{Разработка игр!Вывод текста}

\newglossaryentry{LCD}{name=LCD, description={Liquid Crystal Display --
    Жидкокристаллический дисплей}}

\newglossaryentry{метод}{name=метод, description={Функция, объявленная внутри
    класса (также можно сказать, функция, находящаяся внутри объекта),
    отражающая некоторое действие, которое может выполнить данный объект или же
    действие, которое можно применить к объекту.}}

Согласно традиции, первая программа, которую мы выведем -- это ``Привет, Мир!''.
Однако поскольку вывод русского текста на большинстве дисплеев организован
сложнее, нежели чем латиницы, то мы будем выводить ``Hello, World!''.

Для этого нам необходимо, во-первых, установить библиотеку для работы с
дисплеем; во-вторых, настроить дисплей на нужный режим работы, и только после
этого мы сможем вывести текст.

Начнём со скачивания библиотеки.  Проще всего это сделать через менеджер
библиотек, доступный из меню ``Инструменты'' (``Tools'') $\rightarrow$ ``Менеджер
библиотек'' (``Manage Libraries...''), далее в списке выбираем нужную библиотеку
и нажимаем на кнопку ``Установить'' (``Install''.)  Существуют несколько
библиотек, которые могут обеспечить работу ЖК-дисплея по протоколу \gls{I2C}.
Мы можем взять библиотеку ``Liquid Crystal I2C'' версии 1.1.1 под авторством
Марко Шварца (Marco Schwartz).

Установив библиотеку, мы сможем использовать её функционал.

Первым делом в коде нам необходимо подключить библиотеку
``LiquidCrystal\_I2C.h'' -- это делается через специальную команду
\texttt{\#include}.

\begin{minted}{cpp}
#include <LiquidCrystal_I2C.h>
\end{minted}

В глобальной области кода (до функции \texttt{setup} и \texttt{loop}) необходимо
создать специальную переменную, через которую мы будем работать с дисплеем. Эта
переменная отличается от того, что мы видели раннее -- в данной переменной
хранится не число, а ссылка более сложный \emph{объект}\footnote{Термин
``объект'' относится к методологии программирования, которая называется
\emph{объектно-ориентированное программирование} (сокращённо ``ООП''.)  Мы пока
не будем заострять наше внимание на теме ООП, но любознательный читатель может
получить общее представление о данной методологии, например, из соответствующей
статьи в Wikipedia.}, который располагается где-то в памяти при работе программы
внутри микроконтроллера.

Назовём переменную \texttt{lcd}, по сокращению \gls{LCD} -- ``Liquid Crystal
Display'':

\begin{minted}{cpp}
LiquidCrystal_I2C lcd(0x27,  16, 2);
\end{minted}

В круглых скобках задаются параметры создаваемого объекта.  Посмотрим на них
подробнее.

Число \texttt{0x27} (или в двоичном виде \texttt{0100 111}) -- это адрес
устройства на шине I2C.  Как говорилось в главе про I2C (\ref{section:i2c}),
адрес устройства можно поменять при желании, путём замыкания специальных
перемычек на I2C-модуле.  Старшие четыре бита (\texttt{0100}) идут
предустановленные ``с завода'' и изменены быть не могут, а вот следующие три
бита (\texttt{111}) как раз и управляются с помощью выведенных на плату
I2C-модуля перемычек.  Самый младший бит (восьмой по счёту) здесь не фигурирует,
так как он влияет на направление передачи данных (от устройства или к
устройству), и управляется этот бит библиотекой \texttt{LiquidCrystal\_I2C}.

Числа 16 и 2 являются размером дисплея (ширина и высота соответственно.)  Если у
вас дисплей 20x4, то вам надо будет эти числа поменять на 20 и 4.

С особенностями задания параметров дисплея разобрались, теперь пришло время в
функции \texttt{setup} выполнить настройку дисплея.

\begin{minted}{cpp}
lcd.init();
lcd.backlight();
\end{minted}

Поскольку переменная \texttt{lcd} хранит в себе сложный объект, со своими
свойствами и возможными действиями (\emph{методами}), то работа с ней может показаться
непривычной не неискушённых читателей на первый взгляд.

Можно сказать, что \gls{метод} -- это функция, находящаяся внутри объекта.
Вызов подобных функций не сильно отличается от вызова обычных функций, разве что
мы должны перед именем функции указывать объект, у которого данная функция
(метод) вызывается, разделяя его точкой (без пробелов.)

Метод \texttt{init} производит начальную инициализацию дисплея.  Без неё дисплей
работать корректно не будет, и все другие действия должны быть прописаны так,
чтобы они выполнялсь после вызова \texttt{init}.

Метод \texttt{backlight} включает подсветку дисплея.  Сам по себе ЖК-дисплей, а
точнее его пиксели, не светятся, а подсветка осуществляется специальным
светодиодом, встроенным в дисплей.  Вызов этого метода не обязателен, но обычно
с подсветкой работать интереснее.

Существует также метод \texttt{noBacklight}, который выключает подсветку.

После этого уже в функции \texttt{loop} мы можем вывести текст.  Вывод текста,
как правило, делается в два этапа: во-первых, необходимо установить место, в
которое будет ``впечатан'' текст.  Для этого существует специальный метод
\texttt{setCursor}.

\begin{minted}{cpp}
lcd.setCursor(0, 0);
\end{minted}

Первый параметр \texttt{setCursor} задаёт позицию курсора по оси X, второй
параметр задаёт позицию по оси Y.

После этого наконец-то мы можем вывести текст с помощью метода \texttt{print}:

\begin{minted}{cpp}
lcd.print("Hello, World!");
\end{minted}

Полностью код будет выглядеть примерно так, как показано ниже.

\begin{minted}{cpp}
#include <LiquidCrystal_I2C.h>

LiquidCrystal_I2C lcd(0x27,  16, 2);

void setup() {
  lcd.init();
  lcd.backlight();
}

void loop() {
  lcd.setCursor(0, 0);
  lcd.print("Hello, World!");
}
\end{minted}

При необходимости очистить экран можно ``затереть'' текст, впечатав поверх
пробелы, или же воспользоваться методом \texttt{clear}, который позволяет
очистить экран полностью.

%%%%%%%%%%%%%%%%%%%%%%%%%%%%%%%%%%%%%%%%%%%%%%%%%%%%%%%%%%%%%%%%%%%%%%%%%%%%%%%%
\subsubsection{Задачи}
\begin{enumerate}
\item Выровняйте тест по середине дисплея.
\item Сделайте так, чтобы текст мигал.
\item Разработайте ``Бегущую строку'', где текст будет перемещаться по экрану
  справа на лево.
\item Попробуйте сделать так, чтобы подсветка дисплея мигала.
\end{enumerate}

\end{document}
