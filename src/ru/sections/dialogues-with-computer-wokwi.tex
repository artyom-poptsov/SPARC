\documentclass[../sparc.tex]{subfiles}
\graphicspath{{\subfix{../images/}}}
\begin{document}

%%%%%%%%%%%%%%%%%%%%%%%%%%%%%%%%%%%%%%%%%%%%%%%%%%%%%%%%%%%%%%%%%%%%%%%%%%%%%%%%
\section{Симулятор Wokwi}
Для того, чтобы программировать Arduino, не обязательно иметь реальную плату
Arduino -- вместо этого можно воспользоваться \emph{симулятором}.  Задача
симулятора -- программно эмулировать реальную плату с минимальными отличиями.

Вообще существуют два похожих термина: \emph{симулятор} и \emph{эмулятор}.

Посмотрим на различия\cite{so:simulator-vs-emulator} между ними:
\begin{itemize}
\item Задачей \emph{эмулятора} является имитация внешнего наблюдаемого поведения
  системы.  Внутреннее состояние механизма эмуляции не обязано точно повторять
  внутренее состояние того, что эмулируется.
\item Задачей \emph{симулятора} является моделирование внутреннего состояния
  симулируемой системы.  Конечным результатом хорошей симуляции является то, что
  виртуальная модель максимально близко повторяет целевую платформу.  В
  идеальном случае, вы должны иметь возможность посмотреть внутрь симуляции и
  обнаружить все аспекты работы, которые вы могли бы увидеть, посмотрев внутрь
  симулируемого объекта.
\end{itemize}

Симуляторы можно условно разделить на три категории:
\begin{itemize}
\item \emph{Offline-симуляторы} -- представляют собой отдельную программу,
  которую можно скачать, установить и запустить на компьютере пользователя
\item \emph{Online-симуляторы} -- специальные сайты, которые позволяют эмулировать
  аппаратную платформу прямо в браузере пользователя (либо полностью скачивая
  весь необходимый код в браузер, либо же выполняя часть операций на сервере.)
\item \emph{Комбинированные симуляторы} -- представляют собой отдельную
  программу, которая однако требует доступа в Internet для работы.
\end{itemize}

Online-эмуляторы проще в работе для начинающих пользователей, так как не требуют
установки чего-бы то ни было на компьютер.  Однако многие подобные эмуляторы
требуют регистрации, что не всем людям подходит.

Одним из свободно доступных online-эмуляторов, который не требует регистрации,
является проект Wokwi (\href{https://wokwi.com/}{wokwi.com}.)  Данный эмулятор
позволяет работать не только с платформами Arduino, но и с другими
микроконтроллерными платформами (вроде STM32 и ESP), а также с одноплатными
компьютерами.

Wokwi позволяет достаточно легко собирать проекты на базе Arduino,
программировать их и запускать.

Одним из ключевых параметров эмулятора является скорость работы -- чем быстрее
работает эмулятор, тем ближе к эмулируемой аппаратной платформе будет скорость
работы запускаемых проектов.  На скорость работы online-эмулятора Wokwi влияет
производительность вашего компьютера, вид браузера и сложность собранной схемы.

Самым оптимальным браузером для работы с Wokwi можно назвать Google Chrome или
же Chromium, однако Mozilla Firefox тоже неплохо справляется с задачей.

Wokwi имеет встроенную документацию по различным электронным компонентам,
которые можно использовать в схемах.

К недостаткам Wokwi, присутствующим на момент написания данного раздела, можно
отнести непостоянность скорости работы, что негативно влияет на эмуляцию схем,
требующих предсказуемого отклика (например, схемы, где идёт воспроизведение
звука.)  Другим недостатком для русскоговорящих пользователей является
англоязычность интерфейса, однако в современном мире широко доступных
переводчиков и распространения английского языка данный недостаток по нашему
мнению не является критическим.  Обратите внимание, что не рекомендуется делать
автоматический перевод страницы с редактором схем и кода, так как это может
нарушить работу эмулятора.

%%%%%%%%%%%%%%%%%%%%%%%%%%%%%%%%%%%%%%%%%%%%%%%%%%%%%%%%%%%%%%%%%%%%%%%%%%%%%%%%
\subsection{Начало работы}

Для того, чтобы начать работать с Arduino, на главной странице сайта Wokwi
небходимо в разделе ``Simulate with Wokwi Online'' (``Симулируйте вместе с Wokwi
Online'') выбрать вариант ``Arduino (Uno, Mega, Nano)''.  Далее надо промотать
страницу к разделу ``Start from Scratch'' (``Начать с нуля'') и выбрать там
желаемый вариант Arduino.

После этого откроется страница, где в левой части будет редактор кода (наподобие
Arduino IDE), а в правой части -- область сборки схемы.  Зелёная кнопка со
значком ``>'' в верхнем левом углу области сборки схемы позволяет запустить
проект в симуляторе.  Кнопка ``+'' позволяет выбрать и добавить компонент в
схему, а кнопка с тремя точками позволяет получить доступ к опциям симулятора.

\end{document}
