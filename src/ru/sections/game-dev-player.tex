\documentclass[../sparc.tex]{subfiles}
\graphicspath{{\subfix{../images/}}}
\begin{document}

%%%%%%%%%%%%%%%%%%%%%%%%%%%%%%%%%%%%%%%%%%%%%%%%%%%%%%%%%%%%%%%%%%%%%%%%%%%%%%%%
\section{Отображение игрового персонажа}
\label{section:player}
\index{Разработка игр!Игровой персонаж}

В большинстве жанров компьютерных игр существует некоторый персонаж, который
представляет нас в игре и которым мы можем управлять.

Для начала мы можем в качестве изображения персонажа взять какой-нибудь символ
-- допустим, символ собачки ``@''.

Зададим ``изображение'' в виде символьной константы где-то в начале программы:

\begin{listing}[ht]
  \begin{minted}{cpp}
    const char PLAYER = '@';
  \end{minted}
  \caption{Объявление константы, хранящей отображение игрока.}
  \label{listing:game-dev-player-image}
\end{listing}

Далее нам необходимо задать позицию персонажа в двумерном пространстве нашей
игровой ``карты'':

\begin{listing}[ht]
  \begin{minted}{cpp}
    int player_x = 0;
    int player_y = 0;
  \end{minted}
  \caption{Объявление переменных, задающих позицию игрока на карте.}
  \label{listing:game-dev-player-position}
\end{listing}

После этого мы можем отобразить персонажа на экране.  Полный код программы будет
выглядеть примерно так, как показано ниже.

\begin{listing}[ht]
  \begin{minted}{cpp}
    #include <LiquidCrystal_I2C.h>

    LiquidCrystal_I2C lcd(0x27,  16, 2);

    const char PLAYER = '@';

    int player_x = 0;
    int player_y = 0;

    void setup() {
      lcd.init();
      lcd.backlight();
    }

    void loop() {
      lcd.setCursor(player_x, player_y);
      lcd.print(PLAYER);
    }
  \end{minted}
  \caption{Пример кода, отображающего игрового персонажа на экране ЖК-дисплея.}
  \label{listing:game-dev-player-example}
\end{listing}

После загрузки данной программы в Arduino, на экране на нулевой строке в первом
столбце должен появиться символ ``@''.

Поскольку мы задали позицию персонажа в виде переменных, то меняя эти переменные
мы можем менять позицию персонажа на экране.  Для изменения позиции в играх
обычно используются кнопки и/или манипулятор вида ``джойстик''.  В следующем
разделе мы как раз посмотрим, как подключить кнопки и сделать обработку их
нажатий.

\end{document}
