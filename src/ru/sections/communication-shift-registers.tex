\documentclass[../sparc.tex]{subfiles}
\graphicspath{{\subfix{../images/}}}
\begin{document}

\newpage
%%%%%%%%%%%%%%%%%%%%%%%%%%%%%%%%%%%%%%%%%%%%%%%%%%%%%%%%%%%%%%%%%%%%%%%%%%%%%%%%
\section{Сдвиговые регистры}
\label{section:shift-registers}
\index{Электроника!Сдвиговый регистр}

Теперь, когда мы разобрались с битовыми операциями, пришло время их попробовать
в действии на примере работы со \emph{сдвиговыми регистрами}.  Сдвиговые
регистры, также иногда называемые расширителями портов ввода-вывода, позволяют
нам расширить количество цифровых портов на микроконтроллере.  Часто они
используются для медленной периферии с большим количеством выводов -- например,
для подключения семисегментных индикаторов и массивов кнопок к микроконтроллеру.

%%%%%%%%%%%%%%%%%%%%%%%%%%%%%%%%%%%%%%%%%%%%%%%%%%%%%%%%%%%%%%%%%%%%%%%%%%%%%%%%
\subsection{Последовательно-параллельный регистр}

Последовательно-параллельный сдвиговый регистр преобразует последовательный
набор сигналов (двоичный код) в параллельный.

Примером применения подобного регистра служит управление набором светодиодов или
семисегментным индикатором.

%%%%%%%%%%%%%%%%%%%%%%%%%%%%%%%%%%%%%%%%%%%%%%%%%%%%%%%%%%%%%%%%%%%%%%%%%%%%%%%%
\subsection{Параллельно-последовательный регистр}

Параллельно-последовательный регистр преобразует параллельные сигналы в
последовательный сигнал (двоичный код.)

Примером применения подобного регистра служит получение данных с массива кнопок.

%%%%%%%%%%%%%%%%%%%%%%%%%%%%%%%%%%%%%%%%%%%%%%%%%%%%%%%%%%%%%%%%%%%%%%%%%%%%%%%%
\subsection{Универсальный регистр}

Универсальный регистр может работать в обоих направлениях, в зависимости от
конфигурации.  Имеет параллельные и последовательные порты для загрузки и
выгрузки данных, а также порт для задания направления движения данных.

Примером является микросхема 74LS194.

\end{document}
