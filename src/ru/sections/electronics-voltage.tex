\documentclass[../sparc.tex]{subfiles}
\graphicspath{{\subfix{../images/}}}
\begin{document}

%%%%%%%%%%%%%%%%%%%%%%%%%%%%%%%%%%%%%%%%%%%%%%%%%%%%%%%%%%%%%%%%%%%%%%%%%%%%%%%%
\section{Напряжение}
\index{Электроника!Напряжение}

Представим, что у нас есть некоторая ёмкость с водой (см. рисунок
\ref{fig:electronics-current-0}.)

\figureElectronicsVoltage{ru}

Вода в ёмкости имеет некоторую \emph{потенциальную энергию}, которая может быть
потрачена с какой-либо целью.  Например, если мы внизу ёмкости проделаем
отверстие, то вода из него будет вытекать
(см. рис. \ref{fig:electronics-current-1}); если при этом под струю воды
подставить водяное колесо, то таким образом можно приводить в движение
механизмы.

\figureElectronicsVoltageWithCurrent{ru}

Проводя аналогию с электричеством можно сказать, что ёмкость имеет некоторое
\emph{напряжение} воды.  В электрической батарее как правило запасена
\emph{химическая энергия}, которая может быть высвобождена при определённых
условиях.

Напряжение в электрической цепи измеряется в \emph{Вольтах} (В).

Таким образом, мы можем сделать первый вывод: для протекания тока необходим
некий источник тока, обладающий некоторым напряжением.

\end{document}
