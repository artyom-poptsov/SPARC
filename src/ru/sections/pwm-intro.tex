\documentclass[../sparc.tex]{subfiles}
\graphicspath{{\subfix{../images/}}}
\begin{document}

%%%%%%%%%%%%%%%%%%%%%%%%%%%%%%%%%%%%%%%%%%%%%%%%%%%%%%%%%%%%%%%%%%%%%%%%%%%%%%%%
\section{Общее описание принципов работы}
\newglossaryentry{ШИМ}{name=ШИМ, description={Широтно-Импульсная Модуляция}}

Широтно-импульсная модуляция, или сокращённо \emph{\gls{ШИМ}}, позволяет выдавать на
цифровом порту Arduino напряжение в диапазоне от 0 до 5 вольт, используя при
этом только два сигнала -- \texttt{HIGH} (логическая единица, при которой на
порт подается 5 В) и \texttt{LOW} (логический ноль, при котором на порт подается
0 В.) Меняя быстро данные значения на порту, можно добиться, например,
напряжения в 2.5 В.

\end{document}
