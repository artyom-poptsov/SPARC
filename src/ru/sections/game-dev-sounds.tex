\documentclass[../sparc.tex]{subfiles}
\graphicspath{{\subfix{../images/}}}
\begin{document}

%%%%%%%%%%%%%%%%%%%%%%%%%%%%%%%%%%%%%%%%%%%%%%%%%%%%%%%%%%%%%%%%%%%%%%%%%%%%%%%%
\section{Добавление звуков}
\index{Разработка игр!Звук}

Редкая компьютерная игра обходится без специальных звуковых эффектов.  Добавим
тоже в нашу игру немного звука.

Пусть динамик для вывода звука у нас будет подключен к восьмому порту -- укажем
это в коде с помощью именованной константы до функции \texttt{setup}:

\begin{minted}{cpp}
  const byte SPEAKER_PIN = 8; // Порт динамика
\end{minted}

В \texttt{setup} теперь настроим порт динамика на режим вывода.

\begin{minted}{cpp}
  void setup() {
    lcd.init();
    lcd.backlight();

    // Настройка кнопок управления.
    pinMode(BUTTON_R, INPUT_PULLUP);
    pinMode(BUTTON_L, INPUT_PULLUP);

    // Настройка порта динамика.
    pinMode(SPEAKER_PIN, OUTPUT);

    // Вызов функции генерации карты.
    map_generate();
  }
\end{minted}

Теперь перенесём функцию \texttt{play\_tone} из главы ``Звук'' в исходный код
нашей игры, и добавим константы для нот.

\begin{minted}{cpp}
  // Четвёртая (в научной нотации, считая с нуля) октава.
  const float c4 = 261.630;
  const float d4 = 293.660;
  const float e4 = 329.630;
  const float f4 = 349.230;
  const float g4 = 392.000;
  const float a4 = 440.000;
  const float b4 = 493.880;

  // Функция воспроизведения звука указанной частоты.
  void play_tone(int port, float f, long t) {
    const long T = 1000000 / f;
    long d = T / 2;
    int count = t / T;
    for (int i = 0; i < count; i++) {
      digitalWrite(port, HIGH);
      delayMicroseconds(d);
      digitalWrite(port, LOW);
      delayMicroseconds(d);
    }
  }
\end{minted}

Начнём со звука сбора предметов.  В предыдущих главах мы как раз реализовали
обработку столкновения игрока с предметами на карте.  Теперь мы можем добавить
звук при взятии ``аптечки'' с карты.

Вы можете придумать свой звук ``по вкусу''; мы же возьмём ноты \texttt{e4} и
\texttt{g4} с разными длительностями, что даст (на наш взгляд) интересный
эффект.

\begin{minted}{cpp}
  void loop() {
    if (digitalRead(BUTTON_R) == LOW) {
      // ...
    }

    if (digitalRead(BUTTON_L) == LOW) {
      // ...
    }

    if (is_hp(player_x, player_y)) {
      game_map[player_y][player_x] = SPACE;

      // Воспроизведение звука взятия предмета.
      play_tone(SPEAKER_PIN, e4, 125000);
      play_tone(SPEAKER_PIN, g4, 250000);
    }

    // Вызов функции отрисовки карты на экране дисплея.
    map_show();

    // ...
  }
\end{minted}

Точно так же можно сделать озвучку для других событий игры (например, для
выигрыша и проигрыша) -- но об этом речь пойдёт позже.

%%%%%%%%%%%%%%%%%%%%%%%%%%%%%%%%%%%%%%%%%%%%%%%%%%%%%%%%%%%%%%%%%%%%%%%%%%%%%%%%
\subsection{Задачи}
\begin{enumerate}
\item Попробуйте реализовать звуковой эффект столкновения персонажа со стеной --
  для этого можно взять какой-нибудь звук низкой частоты (не обязательно
  музыкальную ноту) и при попытке игрока ``войти в стену'' воспроизводить его.
\end{enumerate}

\end{document}
