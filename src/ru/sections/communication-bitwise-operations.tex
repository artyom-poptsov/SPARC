\documentclass[../sparc.tex]{subfiles}
\graphicspath{{\subfix{../images/}}}
\begin{document}

%%%%%%%%%%%%%%%%%%%%%%%%%%%%%%%%%%%%%%%%%%%%%%%%%%%%%%%%%%%%%%%%%%%%%%%%%%%%%%%%
\section{Битовые операции}
\index{Программирование!Битовые операции}

Чтобы удобно было оперировать отдельными битами, в языке C существуют так
называемые \emph{битовые операции}.  К ним относятся: побитовое ``И'' (пишется
знаком амперсанда ``\&''), побитовое ``ИЛИ'' (записывается знаком вертикальной
черты ``|''), побитовое отрицание ``НЕ'' (записывается знаком тильды ``$\sim$''),
сдвиги влево (``$<<$'') и вправо (``$>>$''.)

На примере работы с I2C-устройством можно наглядно увидеть работу данных
операций.

Например, для выставления пятого справа бита в байте в единицу, можно
воспользоваться следующим кодом:

\begin{minted}{cpp}
  int a = 0b10000000;
  a = a | 0b00010000;
  // Результат:
  //      0b10010000;
\end{minted}

Чтобы было проще записать ``пятый справа бит'', можно воспользоваться битовым
сдвигом влево на 4 позиции, результат будет тот же -- но запись станет более
краткой:

\begin{minted}{cpp}
  int a = 0b10000000;
  a = a | (1 << 4);
  // Результат:
  //      0b10010000;
\end{minted}

Чтобы узнать, выставлен ли определённый бит в числе в единицу, можно
воспользоваться побитовым ``И'' -- данная операция часто называется ``наложением
битовой маски'':

\begin{minted}{cpp}
  int a = 0b10010000;

  int mask = 0b00010000;
  // Можно также задать битовую маску через битовый сдвиг:
  //   mask = (1 << 4);

  if (a & mask) {
    // Действие, если пятый справа бит выставлен в 1.
  }
\end{minted}

Чтобы ``выключить'' какой-то из битов в числе, мы можем воспользоваться
побитовым ``И'', вместе с инверсией битовой маски:

\begin{minted}{cpp}
  int a = 0b10010000;
  a = a & ~(1 << 4);
  // Результат:
  //      0b10000000;
\end{minted}

%%%%%%%%%%%%%%%%%%%%%%%%%%%%%%%%%%%%%%%%%%%%%%%%%%%%%%%%%%%%%%%%%%%%%%%%%%%%%%%%
\subsection{Задачи}

\begin{enumerate}
\item Реализовать эффект ``Бегущий огонь'', используя I2C-адаптер для ЖК-дисплея
  с подключенными свтодиододами, на базе примера
  \ref{listing:communication-pcf8574}.  Для реализации эффекта можно
  использовать битовые операции.
\end{enumerate}
