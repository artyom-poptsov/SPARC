\documentclass[../sparc.tex]{subfiles}
\graphicspath{{\subfix{../images/}}}
\begin{document}

%%%%%%%%%%%%%%%%%%%%%%%%%%%%%%%%%%%%%%%%%%%%%%%%%%%%%%%%%%%%%%%%%%%%%%%%%%%%%%%%
\section{Программирование простых мелодий}

Для того, чтобы запрограммировать мелодию, нам потребуется узнать ноты, из
которых состоит данная мелодия, и их порядок. Как правило, эта информация
записывается в виде нотной записи -- но если вы ещё не умеете читать нотную
запись, то можно найти мелодии в упрощенной записи, где используется буквенная
(научная) нотация. К примеру, возьмём мелодию ``Twinkle, Twinkle, Little
Star'' \footnote{\url{https://ru.wikipedia.org/wiki/Twinkle,_Twinkle,_Little_Star}}
-- английскую колыбельную:

\figureMusicTwinkleTwinkleLittleStar{ru}

Из таблицы \ref{table:music-notes-legths}, мы уже знаем, как различать нотные
``закорючки'', чтобы понять их длительность, но мы пока не разбирали, как
определить частоты нот по нотной записи. Поэтому ниже приводится полный список
нот композиции в правильном порядке (см.
\ref{table:twinkle-winkle-little-star-notes}), где ноты разбиты по строкам
таким образом, чтобы каждая строка соответствовала одному такту.

\tableMusicTwinkleTwinkleLittleStarNotes{ru}

Используя наши знания про нотные ``закорючки'', попробуем назначить длительность
для каждой из нот, подписав её в скобках
(см. \ref{table:twinkle-twinkle-little-star-notes-lengths}.)

\tableMusicTwinkleTwinkleLittleStar{ru}

Для того, чтобы запрограммировать данную мелодию, удобно в начале программы
объявить каждую ноту в виде константы. Каждая константа будет хранить частоту
звука.\footnote{Имена констант обычно пишутся заглавными буквами, т.е.
правильнее было бы именовать эти константы ``C4'', ``D4'', ``E4'' и т.д. Однако мы
используем здесь буквы в нижнем регистре, чтобы избежать конфликтов имён с
константами, которые уже есть в Arduino (к примеру, ``A4''.)}

Нам пока потребуются только ноты из четвёртой октавы. Константы разумно объявить
до функции \texttt{setup}, в самом верху программы.

\begin{minted}{cpp}
const float c4 = 261.630;
const float d4 = 293.660;
const float e4 = 329.630;
const float f4 = 349.230;
const float g4 = 392.000;
const float a4 = 440.000;
const float b4 = 493.880;
\end{minted}

Не забывайте, что нам нужно ещё объявить константу, хранящую номер порта, куда
подключен динамик -- пусть это будет цифровой порт номер 8:

\begin{minted}{cpp}
  // Номер порта, куда подключен динамик:
  const int SPEAKER_PIN = 8;
\end{minted}

Также нам нужно знать темп мелодии -- то есть, BPM. Для ``Twinkle, Twinkle,
Little Star'' это параметр равен 120 ударов в минуту.

Как только мы объявили все необходимые константы и узнали BPM, то
запрограммировать мелодию не составит труда.  Напишем код нашей функции
\texttt{loop}:

\begin{minted}{cpp}
void loop() {
  const long BPM = 120;
  const long MINUTE = 60 * 1000000;
  const long T = (MINUTE / BPM) * 4;

  // 0-й такт.
  play_tone(SPEAKER_PIN, c4, T / 4);
  delay(100);
  play_tone(SPEAKER_PIN, c4, T / 4);
  delay(100);
  play_tone(SPEAKER_PIN, g4, T / 4);
  delay(100);
  play_tone(SPEAKER_PIN, g4, T / 4);
  delay(100);

  // 1-й такт.
  play_tone(SPEAKER_PIN, a4, T / 4);
  delay(100);
  play_tone(SPEAKER_PIN, a4, T / 4);
  delay(100);
  play_tone(SPEAKER_PIN, g4, T / 2);
  delay(100);

  // и так далее
}
\end{minted}

Допишите необходимые части кода (\texttt{setup}, \texttt{play\_tone} и т.д.) и
загрузите его в Arduino. Если вы всё сделали правильно, то у вас должна заиграть
мелодия. Прекрасно!

Для тех, кому не терпится послушать, как звучит мелодия, мы предлагаем взглянуть
в приложение \ref{app:twinkle-twinkle-little-star-01}, где представлен полный
код программы.

Но данное решение является не совсем оптимальным с точки зрения количества кода,
которое необходимо написать. Решением данной проблемы является использование
\emph{массивов}.



\end{document}
