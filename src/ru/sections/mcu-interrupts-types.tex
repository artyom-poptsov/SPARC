\documentclass[../sparc.tex]{subfiles}
\graphicspath{{\subfix{../images/}}}
\begin{document}

%%%%%%%%%%%%%%%%%%%%%%%%%%%%%%%%%%%%%%%%%%%%%%%%%%%%%%%%%%%%%%%%%%%%%%%%%%%%%%%%
\subsection{Виды прерываний}

\figureInterruptTypes{ru}

Терминология, используемая при описании прерываний, может различаться в разных
источниках.  В соответствии с \cite{toshiba:interrupts}, мы поделим типы
прерываний на две большие группы:
\begin{itemize}
\item Программные прерывания.
\item Аппаратные прерывания.
\end{itemize}

Источником \emph{программных прерываний} служит выполнение определённых
инструкций процессора.  Подобные прерывания могут быть вызваны программой
преднамеренно, или же могут возникнуть вследствие ошибки в программе (например,
в результате деления на ноль.)

Источником \emph{аппаратных прерываний} служат периферийные устройства.
\emph{Внешние} аппаратные прерывания могут быть вызваны внешними (по отношению к
микроконтроллеру) устройствами -- такими, как кнопки.  \emph{Внутренние}
аппаратные прерывания вызываются периферийными цепями внутри самого
микроконтроллера (примером могут служить прерывания по таймеру, если схема
таймера интегрирована в микроконтроллер.)

\end{document}
