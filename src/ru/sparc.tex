\documentclass[a4paper,twoside]{book}

\emergencystretch=1em
\usepackage[utf8]{inputenc}
\usepackage[english,russian]{babel}
\usepackage{fontspec}
\usepackage{graphicx}
\renewcommand{\thefigure}{\thesection.\arabic{figure}}
\usepackage{hyperref}
\usepackage{multirow}
\usepackage[absolute,overlay,showboxes]{textpos}
\usepackage{etoolbox}
\usepackage{longtable}
\usepackage{tikz}
\usepackage{lilyglyphs}
\usepackage{tabularx}
\usepackage{pgfplots}
\usepackage{circuitikz}
\usepackage{minted}
\usepackage{glossaries}
\usepackage{makeidx}
\usepackage{expl3}
\usepackage{chngcntr}
\usepackage{svg}
\usepackage{subfiles}
\usepackage[printonlyused,withpage]{acronym}

\usepackage[backend=bibtex]{biblatex}
\addbibresource{references.bib}

%%%%%%%%%%%%%%%%%%%%%%%%%%%%%%%%%%%%%%%%%%%%%%%%%%%%%%%%%%%%%%%%%%%%%%%%%%%%%%%%
\setmainfont{Liberation Serif}
\setmonofont{Liberation Mono}

\makeindex
\makeglossaries

\urlstyle{same}

\graphicspath{ {images/} {../../../images/} }

%% \documentclass[../main.tex]{subfiles}
%% \usepackage{svg}
%% \graphicspath{{\subfix{../images/}}}
%% \begin{document}

\newcounter{example-counter}
\setcounter{example-counter}{1}

%% This procedure adds the "Example" block to the text.
\newcommand{\example}[1]{
  \vspace{8pt}
  \begin{tabularx}{\textwidth}{m{1cm} m{9cm}}
    \includesvg[width=1.25cm]{the-noun-project/request-mirrored}
    & \textbf{Пример \arabic{example-counter}}: #1 \\
  \end{tabularx}
  \addtocounter{example-counter}{1}
}

\newcounter{experiment-counter}
\setcounter{experiment-counter}{1}

%% This procedure adds the "Experiment" block to the text.
\newcommand{\experiment}[2]{
  \vspace{8pt}
  \begin{tabularx}{\textwidth}{m{.15\textwidth}X m{.85\textwidth-4}X}
    \includesvg[width=1cm]{the-noun-project/flask}
    & \textbf{Эксперимент №\arabic{experiment-counter}:} #2 \\
  \end{tabularx}
  \addtocounter{experiment-counter}{1}
}

\newcommand{\note}[1]{
  \vspace{8pt}
  \begin{tabularx}{\textwidth}{m{1cm} m{9cm}}
    \includesvg[width=1cm]{the-noun-project/note}
    & \textbf{Примечание:} #1 \\
  \end{tabularx}
}

\newcommand{\hotkey}[1]{
  \texttt{#1}
}

%% This procedure allows to insert a music note.
%% Syntax:
%%   \musicnote{<octave>}{<note-name>}{<frequency>}
\newcommand{\musicnote}[3]{
  &
  \ifstrequal{#2}{C}{До   & C#1}{}
  \ifstrequal{#2}{D}{Ре   & D#1}{}
  \ifstrequal{#2}{E}{Ми   & E#1}{}
  \ifstrequal{#2}{F}{Фа   & F#1}{}
  \ifstrequal{#2}{G}{Соль & G#1}{}
  \ifstrequal{#2}{A}{Ля   & A#1}{}
  \ifstrequal{#2}{B}{Си   & B#1 (H#1)}{}
  & #3 \\
}

%% Taken from:
%%   <https://tex.stackexchange.com/questions/184923/how-to-include-a-second-file-only-if-environment-variable-is-set>
\newcommand{\newgetenv}[2][]{%
 \CatchFileEdef{\temp}{"|kpsewhich --var-value #2"}{\endlinechar=-1\relax}%
 \if\relax\detokenize{#1}\relax\temp\else\edef#1{\temp}\fi%
}%

\newcommand\esymbol[1]{%
  \begin{circuitikz}%
    \draw (0,0) to [#1] (1,0);%
  \end{circuitikz}%
}

\newcommand*{\soundWaveIcon}[0]{%
  \begin{tikzpicture}
    \draw[black, fill=black] (0, 0) circle (.25mm);
    \draw (.5mm, 0.7mm) arc (45:-45:1mm);
    \draw (1mm, 1mm) arc (45:-45:1.5mm);
    \draw (1.5mm, 1.3mm) arc (45:-45:2mm);
  \end{tikzpicture}%
}

%% \end{document}


\counterwithin{listing}{section}
\renewcommand\listingscaption{Листинг}

\newgetenv[\REPRODUCIBILITY]{REPRODUCIBILITY}%
\newgetenv[\RANDOMSEED]{RANDOMSEED}

\ifdefstring{\REPRODUCIBILITY}{yes}{%
  \ifthenelse{\equal{\RANDOMSEED}{}}%
  {%
    \typeout{Setting the random seed to a fixed value.}%
    \pgfmathsetseed{\number42}%
  }{%
    \typeout{Setting the random seed to a \RANDOMSEED .}%
    \pgfmathsetseed{\number\RANDOMSEED}%
  }%
}{}

%%%%%%%%%%%%%%%%%%%%%%%%%%%%%%%%%%%%%%%%%%%%%%%%%%%%%%%%%%%%%%%%%%%%%%%%%%%%%%%%
\title{Автомато-программато-компарадио-кружок}
\author{Артём ``avp'' Попцов\\\href{https://memory-heap.org}{memory-heap.org}}
\input{version.tex}

\begin{document}

\maketitle

\tableofcontents

%%%%%%%%%%%%%%%%%%%%%%%%%%%%%%%%%%%%%%%%%%%%%%%%%%%%%%%%%%%%%%%%%%%%%%%%%%%%%%%%
\chapter*{Вступление}
\addcontentsline{toc}{chapter}{Вступление}

\subfile{sections/introduction.tex}

%%%%%%%%%%%%%%%%%%%%%%%%%%%%%%%%%%%%%%%%%%%%%%%%%%%%%%%%%%%%%%%%%%%%%%%%%%%%%%%%
\chapter{Основы электроники}

В первую очередь нам с вами надо рассмотреть базовые принципы того, как работает
электроника, чтобы впоследствии уметь собирать простые схемы.

Начнём с рассмотрения условий, которые необходимо выполнить, чтобы через
электрическую цепь шёл ток.

\subfile{sections/electronics-voltage}
\subfile{sections/electronics-circuits}
\subfile{sections/electronics-potential-difference}
\subfile{sections/electronics-resistance}
\subfile{sections/electronics-building-circuits}

%%%%%%%%%%%%%%%%%%%%%%%%%%%%%%%%%%%%%%%%%%%%%%%%%%%%%%%%%%%%%%%%%%%%%%%%%%%%%%%%
\chapter{Диалоги с компьютером}
\label{chapter:dialogues-with-computer}

\subfile{sections/dialogues-with-computer-title-image}
\subfile{sections/dialogues-with-computer-introduction}
\subfile{sections/dialogues-with-computer-algorithms}
\subfile{sections/dialogues-with-computer-arduino}
\subfile{sections/dialogues-with-computer-breadboard}
\subfile{sections/dialogues-with-computer-multimeter}

\newpage
\subfile{sections/dialogues-with-computer-arduino-ide}
\subfile{sections/dialogues-with-computer-program-structure}
\subfile{sections/dialogues-with-computer-memory}
\subfile{sections/dialogues-with-computer-control-flow}
\subfile{sections/dialogues-with-computer-wokwi}

%%%%%%%%%%%%%%%%%%%%%%%%%%%%%%%%%%%%%%%%%%%%%%%%%%%%%%%%%%%%%%%%%%%%%%%%%%%%%%%%
\chapter{Белый шум}

\subfile{sections/white-noise-introduction}
\subfile{sections/white-noise-signal-types}
\subfile{sections/white-noise-serial-port}
\subfile{sections/white-noise-analog-ports}
\subfile{sections/white-noise-adc}

%%%%%%%%%%%%%%%%%%%%%%%%%%%%%%%%%%%%%%%%%%%%%%%%%%%%%%%%%%%%%%%%%%%%%%%%%%%%%%%%
\chapter{Широтно-импульсная модуляция}

\subfile{sections/pwm-intro}
\subfile{sections/pwm-wavelength}
\subfile{sections/pwm-duty-cycle}
\subfile{sections/pwm-tasks}

%%%%%%%%%%%%%%%%%%%%%%%%%%%%%%%%%%%%%%%%%%%%%%%%%%%%%%%%%%%%%%%%%%%%%%%%%%%%%%%%
\chapter{Синтез музыки и технологии}

\subfile{sections/music-and-technology-synthesis-sound}
\subfile{sections/music-and-technology-synthesis-speaker}
\subfile{sections/music-and-technology-synthesis-rhythm}
\subfile{sections/music-and-technology-synthesis-harmony}
\subfile{sections/music-and-technology-synthesis-octave-system}
\subfile{sections/music-and-technology-synthesis-simple-melodies}
\subfile{sections/music-and-technology-synthesis-arrays}
\subfile{sections/music-and-technology-synthesis-two-dimensional-arrays}
\subfile{sections/music-and-technology-synthesis-staff}
\subfile{sections/music-and-technology-synthesis-rest}
\subfile{sections/music-and-technology-synthesis-dotted-notes}
\subfile{sections/music-and-technology-synthesis-flats-and-sharps}
\subfile{sections/music-and-technology-synthesis-musical-scale}
\subfile{sections/music-and-technology-synthesis-bass-clef}
\subfile{sections/music-and-technology-synthesis-music-band}

%%%%%%%%%%%%%%%%%%%%%%%%%%%%%%%%%%%%%%%%%%%%%%%%%%%%%%%%%%%%%%%%%%%%%%%%%%%%%%%%
\chapter{Язык общения машин}
\subfile{sections/communication-intro}
\subfile{sections/communication-serial-port}
\subfile{sections/communication-i2c}

%%%%%%%%%%%%%%%%%%%%%%%%%%%%%%%%%%%%%%%%%%%%%%%%%%%%%%%%%%%%%%%%%%%%%%%%%%%%%%%%
\chapter{Разработка игр}

\subfile{sections/game-dev-intro}
\subfile{sections/game-dev-lcd}
\subfile{sections/game-dev-genre-and-plot}
\subfile{sections/game-dev-player}
\subfile{sections/game-dev-buttons}
\subfile{sections/game-dev-game-map}
\subfile{sections/game-dev-sounds}
\subfile{sections/game-dev-graphics}
\subfile{sections/game-dev-animation}
\subfile{sections/game-dev-logic}

%%%%%%%%%%%%%%%%%%%%%%%%%%%%%%%%%%%%%%%%%%%%%%%%%%%%%%%%%%%%%%%%%%%%%%%%%%%%%%%%
\addcontentsline{toc}{chapter}{Предметный указатель}
\printindex

%%%%%%%%%%%%%%%%%%%%%%%%%%%%%%%%%%%%%%%%%%%%%%%%%%%%%%%%%%%%%%%%%%%%%%%%%%%%%%%%
\addcontentsline{toc}{chapter}{Словарь терминов}
\printglossaries

%%%%%%%%%%%%%%%%%%%%%%%%%%%%%%%%%%%%%%%%%%%%%%%%%%%%%%%%%%%%%%%%%%%%%%%%%%%%%%%%
\addcontentsline{toc}{chapter}{Список примеров кода}
\renewcommand\listoflistingscaption{Список примеров кода}
\listoflistings

%%%%%%%%%%%%%%%%%%%%%%%%%%%%%%%%%%%%%%%%%%%%%%%%%%%%%%%%%%%%%%%%%%%%%%%%%%%%%%%%
\printbibliography[heading=bibintoc, title={Библиография}]

%%%%%%%%%%%%%%%%%%%%%%%%%%%%%%%%%%%%%%%%%%%%%%%%%%%%%%%%%%%%%%%%%%%%%%%%%%%%%%%%
\appendix

\subfile{sections/appendix-octaves}
\subfile{sections/appendix-prostokvashino-score}
\subfile{sections/appendix-twinkle-twinkle-little-star-01}
\subfile{sections/appendix-twinkle-twinkle-little-star-02}

\end{document}

