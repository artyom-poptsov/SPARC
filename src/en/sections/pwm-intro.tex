\documentclass[../sparc.tex]{subfiles}
\graphicspath{{\subfix{../images/}}}
\begin{document}

%%%%%%%%%%%%%%%%%%%%%%%%%%%%%%%%%%%%%%%%%%%%%%%%%%%%%%%%%%%%%%%%%%%%%%%%%%%%%%%%
\section{General description of operating principles}
\newglossaryentry{PWN}{name=PWM, description={Pulse-Width Modulation}}

Pulse-width modulation (or \emph{\gls{PWM}}), allows us to output voltage in the
range between 0V and 5V on a digital port on Arduino, using only a digital
signal consisting only of two logical levels: \texttt{HIGH} and \texttt{LOW}.
Quickly switching those two values on a digital port allows us to achieve some
intermediate voltage value, such as 2.5V.

\end{document}
