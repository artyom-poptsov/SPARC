\documentclass[../sparc.tex]{subfiles}
\graphicspath{{\subfix{../images/}}}
\begin{document}

%%%%%%%%%%%%%%%%%%%%%%%%%%%%%%%%%%%%%%%%%%%%%%%%%%%%%%%%%%%%%%%%%%%%%%%%%%%%%%%%
\section{Variables}
\index{Programming!Variable}

A \emph{variable} is one of the key terms in the programming.  Any program works
with some data.  For clearness let's take some program as an example -- a
calculator that can add two numbers together.  To allow a micro-controller to
work with those numbers we have to store them somewhere.  But where?  In the
random-access memory (RAM) indeed.  All the data that is used by a
micro-controller during its execution is stored there.  For our calculator to
work we have to upload to the RAM two numbers that we want to add together -- for
example, 15 and 3:

\begin{tabular}{p{4cm}|p{6cm}}
  Memory cell address & Memory cell value \\
  \hline \hline
  0000 & 15 \\
  \hline
  0001 & 3 \\
  \hline
  0003 & 0 \\
  ... & ... \\
\end{tabular}

In other words, a \emph{variable} is a memory cell inside the random-access
memory (RAM.)  To declare a variable is to ask a computer to allocate some
memory for our purposes.

A variable in C++ language has a concrete type and a unique name.  A declaration
(and \emph{initialization}) of a variable looks like follows:

\begin{listing}[ht]
\begin{verbatim}
  type name = value;
\end{verbatim}
\label{listing:dialogues-with-computer-variable-definition-structure}
\caption{The overall structure of a variable declaration.}
\end{listing}

To load numbers 15 and 3 to the RAM we have to write the following:

\begin{listing}[ht]
  \begin{minted}{cpp}
    int a = 15;
    int b = 3;
  \end{minted}
  \label{listing:dialogues-with-computer-variable-definition-example}
  \caption{An example of variable declaration.}
\end{listing}

The \texttt{int} word is the variable type; it means that this variable holds an
integer value.

Below we can see an example of addition of two variables (\texttt{a} and
\texttt{b}) and storing the resulting value into the third variable
\texttt{result} of the same integer type.

\begin{listing}[ht]
  \begin{minted}{cpp}
    int result = a + b;
  \end{minted}
  \label{listing:dialogues-with-computer-variable-operations-example}
  \caption{An example of applying an operation on two variables.}
\end{listing}

It's worth to be noted that the variable name should mirror its purpose, so it
will be easier for us to read and understand the code.

The variable name can consist of only English letters, digits and underscores,
and moreover the name cannot be started with a digit.  We will use underscores
between the words inside variable names throughout the book.  For example, if we
want to specify the time of delay between switching on and switching off an LED
in a ``Chasing lights'' program as a variable, it's logical to call the variable
\texttt{delay\_value}:

\begin{listing}[ht]
  \begin{minted}{cpp}
    int delay_value = 500;
  \end{minted}
  \label{listing:dialogues-with-computer-variable-names}
  \caption{Separating the words inside a variable name with underscores.}
\end{listing}

Next we will substitute the value of the \texttt{delay\_value} variable into a
program in the place of \texttt{delay} value:

\begin{listing}[ht]
  \begin{minted}{cpp}
    void loop() {
      int delay_value = 500;
      digitalWrite(2, HIGH);
      delay(delay_value);
      digitalWrite(2, LOW);
      delay(delay_value);
    }
  \end{minted}
  \label{listing:dialogues-with-computer-variable-usage}
  \caption{An example of variable usage in a program.}
\end{listing}

Thus, we will be able to change all the delay values just by adjusting the
\texttt{delay\_value}:

\begin{listing}[ht]
  \begin{minted}{cpp}
    void loop() {

      int delay_value = 600; // <--

      digitalWrite(2, HIGH);
      delay(delay_value);
      digitalWrite(2, LOW);
      delay(delay_value);
    }
  \end{minted}
  \label{listing:dialogues-with-computer-variable-set-value}
  \caption{If we change the variable value, it changes in all the places where
    the variable is used.}
\end{listing}

If we want to specify the port number for an LED to blink, we can call the
variable \texttt{port} and use it in \texttt{digitalWrite}:

\begin{listing}[ht]
  \begin{minted}{cpp}
    void loop() {
      int delay_value = 600;
      int port = 2;
      digitalWrite(port, HIGH);
      delay(delay_value);
      digitalWrite(port, LOW);
      delay(delay_value);
    }
  \end{minted}
  \label{listing:dialogues-with-computer-variable-set-value}
  \caption{Using a variable to specify an LED to blink.}
\end{listing}

We can, for example, increase the \texttt{port} value by 1 after the first blink
to set it to the next LED:

\begin{listing}[ht]
  \begin{minted}{cpp}
    void loop() {
      int delay_value = 600;

      int port = 2;

      digitalWrite(port, HIGH);
      delay(delay_value);
      digitalWrite(port, LOW);
      delay(delay_value);

      port++;

      digitalWrite(port, HIGH);
      delay(delay_value);
      digitalWrite(port, LOW);
      delay(delay_value);
    }
  \end{minted}
  \label{listing:dialogues-with-computer-variable-increment-value}
  \caption{Incrementing a variable to blink the 2nd LED.}
\end{listing}

%%%%%%%%%%%%%%%%%%%%%%%%%%%%%%%%%%%%%%%%%%%%%%%%%%%%%%%%%%%%%%%%%%%%%%%%%%%%%%%%
\subsection{Scope}
\index{Programming!Variable!Scope}

Note that in the example
\ref{listing:dialogues-with-computer-variable-increment} the value of variable
\texttt{port} is reset to 2 each time \texttt{loop} function repeats.  It is
because after the \texttt{loop} finishes, the variable \texttt{port} is
destroyed, and when \texttt{loop} restarts, the variable is created again.

Variables, as the other elements of a program, have so called \emph{scope} that
limits the area of a program where the variable is available -- in other words,
\emph{scope} specifies from what part of a program we can refer to a variable.

In the example \ref{listing:dialogues-with-computer-variable-increment-value}
the \texttt{port} is a \emph{local variable} in relation to the \texttt{loop}
function.  Local variables exist while the function where they were declared is
executing.

\index{Programming!Variable!Global variables} If we want to keep the variable
value between the \texttt{loop} function reloads, we have to turn the
\texttt{port} into a \emph{global variable} -- that is, we have to move it
outside the boundaries of any function.  We can accomplish this by placing
\texttt{port} before the \texttt{loop} function.  But we have to be careful not
to place the \texttt{port} variable inside other functions (such as
\texttt{setup}), otherwise we won't be able to access it in \texttt{loop}
function.

\begin{listing}[ht]
  \begin{minted}{cpp}
    int port = 2;

    void loop() {
      digitalWrite(port, HIGH);
      delay(delay_value);
      digitalWrite(port, LOW);
      delay(delay_value);

      port++;
    }
  \end{minted}
  \label{listing:dialogues-with-computer-global-variable}
  \caption{An example of a global variable declaration.}
\end{listing}

\end{document}
