\documentclass[../sparc.tex]{subfiles}
\graphicspath{{\subfix{../images/}}}
\begin{document}

%%%%%%%%%%%%%%%%%%%%%%%%%%%%%%%%%%%%%%%%%%%%%%%%%%%%%%%%%%%%%%%%%%%%%%%%%%%%%%%%
\section{Variables}
\index{Programming!Variable}

A \emph{variable} is one of the key terms in the programming.  Any program works
with some data.  For clearness let's take some program as an example -- a
calculator that can add two numbers together.  To allow a micro-controller to
work with those numbers we have to store them somewhere.  But where?  In the
random-access memory (RAM) indeed.  All the data that is used by a
micro-controller during its execution is stored there.  For our calculator to
work we have to upload to the RAM two numbers that we want to add together -- for
example, 15 and 3:

\begin{tabular}{p{4cm}|p{6cm}}
  Memory cell address & Memory cell value \\
  \hline \hline
  0000 & 15 \\
  \hline
  0001 & 3 \\
  \hline
  0003 & 0 \\
  ... & ... \\
\end{tabular}

In other words, a \emph{variable} is a memory cell inside the random-access
memory (RAM.)  To declare a variable is to ask a computer to allocate some
memory for our purposes.

\end{document}
