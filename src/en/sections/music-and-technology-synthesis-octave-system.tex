\documentclass[../sparc.tex]{subfiles}
\graphicspath{{\subfix{../images/}}}
\begin{document}

%%%%%%%%%%%%%%%%%%%%%%%%%%%%%%%%%%%%%%%%%%%%%%%%%%%%%%%%%%%%%%%%%%%%%%%%%%%%%%%%
\section{Octave system}
\index{Music!Octave system}

As you probably know the music is build from the notes -- there are just seven
notes: ``Do'', ``Re'', ``Mi'', ``Fa'', ``Sol'', ``La'' and ``Si''.\footnote{In
English-speaking countries people sometimes use ``Ti'' instead of
``Si''.\cite{happynote}}

Each note has its own frequency.  But if we take a look on a piano (or
synthesizer) keyboard we will see that there are more than seven buttons there,
more than notes.  Why is that?

It turns out that notes are grouped together into groups called \emph{octaves}.
One octave holds seven notes (from ``Do'' up to ``Si''), and there are nine
octaves.

For the convenience we will number all the octaves starting from 0 (the
lowest-frequency octave) up to 8 (the highest-frequency octave.)

This way we get $7 * 9 = 63$ different note frequencies \footnote{In fact there
are more sounds in the octave system -- we will discuss it later.} in the octave
system.

But how to distinguish different notes if they are called the same even in
different octaves?  It turns out there's a convenient \emph{scientific notation}
for naming the notes.  In the table \ref{table:scientific-music-notation}
classical note names (``Do'', ``Re'', ``Mi'', ``Fa'', ``Sol'', ``La'' and
``Si'') are mapped to the scientific notation.

\begin{tabular}{p{4cm}|p{4cm}}
  The syllabic musical notation & Scientific notation \\
  \hline \hline
  Do   & С \\
  \hline
  Re   & D \\
  \hline
  Mi   & E \\
  \hline
  Fa   & F \\
  \hline
  Sol & G \\
  \hline
  La   & A \\
  \hline
  Si   & B (H) \\
  \hline
  \label{table:scientific-music-notation}
\end{tabular}

Note that the ``Si'' note can be named either ``B'' (English variant) or ``H''
(German variant.)  We will be using only the English variant and write the
``Si'' note as ``B''.

The good part about the scientific notation not only in the concise note naming
(which is convenient for the programming purposes), but also in the fact that
after the note name we usually can write its octave number.  For example, ``C0''
is the ``Do'' note of the octave number zero, and ``G5'' is the ``Sol'' of the
fifth octave.

Similar notes in different octaves differ in the frequency -- not in any
arbitrary way, but according to the strict rule: they are multiplies of each
other.  Take, for example, ``C0'' and ``C1'' -- ``C1'' has exactly two times
higher frequency than ``C0''.  But it goes further: if we take ``C2'' it will
have four times higher frequency than of ``C0''.

Here it's a good time to remember that such frequencies that are multiplies of
one another, sound pleasant for us while played together.  Bingo!  Now we get
one piece of the musical theory.

If we want calculate the frequency of notes in octaves that are not adjacent to
each other, we can use special trick from our sleeve.  With each step up on
octave ``ladder'' the note frequency is multiplied by 2, so we can get any note
frequency from the octave number zero just by multiplying its frequency by the
power of 2 (see formula \ref{equation:music-note-frequency-equation}.)

\begin{equation}
  f * 2^n
  \label{equation:music-note-frequency-equation}
\end{equation}

Where ``f'' is the frequency of a note from the zero octave, and ``n'' is the
number of octave, starting from zero.

To test our formula let's use the octave number zero and the octave number four
-- the frequencies of notes are shown in the table \ref{table:fourth-octave}
(full list of octaves can be found in \nameref{section:appendixA}.)

\tableMusicOctaveExample{en}

As can be seen in the table, ``C0'' has the frequency 16.352 Hz.  If we
substitute this value along with the fourth octave number to the formula
\ref{equation:music-note-frequency-equation} we will get the frequency of ``C4''
as shown in \ref{equation:music-note-frequency-equation-example}.

\begin{equation}
  \mbox{C4} = f * 2^n = \mbox{C0} * 2^4 = 16.352\mbox{Hz} * 2^4 = 261.632\mbox{Hz}
  \label{equation:music-note-frequency-equation-example}
\end{equation}

With the new knowledge we can program a simple melody -- and that's what we'll be
doing in the next section.

\end{document}
