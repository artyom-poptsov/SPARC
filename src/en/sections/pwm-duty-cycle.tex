\documentclass[../sparc.tex]{subfiles}
\graphicspath{{\subfix{../images/}}}
\begin{document}

%%%%%%%%%%%%%%%%%%%%%%%%%%%%%%%%%%%%%%%%%%%%%%%%%%%%%%%%%%%%%%%%%%%%%%%%%%%%%%%%
\section{Duty cycle}

How the voltage from the output range is set?  Pretty easy: by changing the
period of logical levels.  The more percentage of time the \texttt{HIGH} value
is supplied to the digital port, the higher the voltage.  In this case, the
length of the period \texttt{T} is kept fixed (e.g. to 1000 microseconds.)  Thus
for the \gls{PWM} to work the percentage relation of one logical level to
another is important; increasing the time of supplying one logical level we have
to decrease the time of another.

\example{
  If we supply the \texttt{HIGH} value for 60\% of the specified period
  \texttt{T}, we have fill the leftover 40\% percent of period with the
  \texttt{LOW} value.
}

The relation of signal period to the length of the impulse is known as the
\emph{duty cycle}.

\example{
  Let's say we want to get 2.5 Volts out of a digital port number 2.  We have
  only two logical values -- \texttt{LOW} (0V) and \texttt{HIGH} (5V.)  To solve
  the task we need to implement a PWM with 50\% duty cycle.  Thus in case of
  1000 $\mu\mbox{s}$ period \texttt{T} we have to set the port to \texttt{HIGH}
  for the half of \texttt{T} (500$\mu\mbox{s}$), and the rest of \texttt{T} we
  have to set the port to \texttt{LOW}.  If we do this right, then on the
  digital port number we will get 2.5 Volts.
}

For the PWM signal generation we have to implement a tool (a \emph{procedure})
that we will be using afterwards.  We have already talked about the importance
of creation and usage our own procedures in programs -- procedures allow us to
create modular programs, make it easier write new code and to maintain the
existing program code.

\figurePWMGraph{en}

Let's think about the \gls{PWM} and how it could be implemented.  On the
fig. \ref{fig:pwm-graph} the PWM signal generation is shown in the form of a
graph.  We will be using this graph as the basis for writing our PWM procedure.

With this graph it is quite easy to lay down the description of the procedure in
natural language.  Let's call the procedure \texttt{pwm}, as the shorthand for
\emph{pulse-width modulation}.

We start from the fact that such procedure have to accept three parameters:

\begin{enumerate}
\item The port number (represented by an integer value) on which we have to
  generate a PWM signal; let's call this parameter just \texttt{port}.
\item The duty cycle (represented by a fraction value) that sets the output
  voltage on the \texttt{port} -- for example, the duty cycle of 50\% will be
  specified as 0.5.  Let's call this parameter \texttt{dc} (as the shorthand for
  ``duty cycle''.)
\item The duration of the PWM signal in microseconds.  Let's call this parameter
  \texttt{t} (as the shorthand of ``time''.)
\end{enumerate}

Let's write the same, but now in C++ language:

\begin{minted}{cpp}
  void pwm(int port, float dc, long t) {
    // Procedure body.
  }
\end{minted}

What is the \texttt{void} we will discuss later, for now we have to accept it as
the fact that it means the start of declaring a procedure.  Likely we also have
some questions about new variable types -- \texttt{float} and \texttt{long}.  Why
can't we use the familiar \texttt{int}?  The reason is that a variable of type
\texttt{int} cannot hold the fractional values, and also on Arduino has the
range from $-2^{15}$ to $2^{15} - 1$.  To store the fractional values we have to
use \texttt{float} (or \texttt{double}, which has twice as much space as
\texttt{float}.)  And to store numbers that have wider range than the
\texttt{int} has we have to use \texttt{long} type, that on Arduino has the
range from $-2^{31}$ to $2^{31} - 1$.  Now it's time to think about the
procedure body.

Let's define the length of one period as the \texttt{T} constant, which is equal
to 1000 $\mu\mbox{s}$

\begin{minted}{cpp}
  const int T = 1000; // microseconds
\end{minted}

It is important to note that we are using the \texttt{const} keyword to make the
\texttt{T} a constant -- we are not going to change it anyways.  Also we added a
commentary saying that the value of the constant is specified in microseconds --
it makes reading the code easier.

The next step is to calculate the values of variables \texttt{d1} and
\texttt{d2}, using the \texttt{dc} value, which is specified as the second
procedure parameter.

The values of the \texttt{d1} and \texttt{d2} delays have to be calculated from
the duty cycle which is specified as a fractional number (e.g. 0.5.)

The formulas to calculate \texttt{d1} and \texttt{d2} are shown below.

\begin{equation}
  \mbox{d1} = \mbox{T} * \mbox{dc}
  \label{A formula to calculate the time of ``HIGH'' value.}
\end{equation}

\begin{equation}
  \mbox{d2} = \mbox{T} - \mbox{d1}
  \label{A formula to calculate the time of ``LOW value.}
\end{equation}

In the program code it should look like this:

\begin{minted}{cpp}
  int d1 = T * dc;
  int d2 = T - d1;
\end{minted}

From this example we can see that it's easy to calculate \texttt{d2} after we
know the \texttt{d1} value.  What is left is to calculate how many times we need
to repeat the wave with period \texttt{T} to generate a signal lasting for
\texttt{t} microseconds:

\begin{minted}{cpp}
  int count = t / T;
\end{minted}

Now we have everything we need to know to generate the needed signal.  Because
we need to repeat the wave \texttt{count} times as we said above, it is
convenient for us to use a \texttt{for} loop:

\begin{listing}[H]
  \begin{minted}{cpp}
    for (int c = 0; c < count; c++) {
      digitalWrite(port, HIGH);
      delayMicroseconds(d1);
      digitalWrite(port, LOW);
      delayMicroseconds(d2);
    }
  \end{minted}
  \label{listing:pwm-cycle}
  \caption{A loop that generates a PWM signal.}
\end{listing}

The whole \texttt{pwm} procedure should look like this:

\begin{listing}[H]
  \begin{minted}{cpp}
    void pwm(int port, float dc, long t) {
      const int T = 1000; // microseconds
      int d1 = T * dc;
      int d2 = T - d1;
      int count = t / T;
      for (int c = 0; c < count; c++) {
        digitalWrite(port, HIGH);
        delayMicroseconds(d1);
        digitalWrite(port, LOW);
        delayMicroseconds(d2);
      }
    }
  \end{minted}
  \label{listing:pwm-procedure}
  \caption{A procedure that generates a PWM signal with the specified
    parameters.}
\end{listing}

And that's it!  Now we need to learn how to use this \texttt{pwm} procedure in
our program.

\end{document}
