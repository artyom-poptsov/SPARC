\documentclass[../sparc.tex]{subfiles}
\graphicspath{{\subfix{../images/}}}
\begin{document}

\newpage
%%%%%%%%%%%%%%%%%%%%%%%%%%%%%%%%%%%%%%%%%%%%%%%%%%%%%%%%%%%%%%%%%%%%%%%%%%%%%%%%

\section{Shift registers}
\label{section:shift-registers}
\index{Electronics!Shift register}

Now that we understood bitwise operations, it's time to try them in action,
using interaction with \emph{shift registers} as an examples.  Shift registers,
also sometimes called I/O port expansions, allow us to expand the number of
digital ports on a micro-controller.  Often they are used for slow periphery
devices that have many pins -- for example, we can use shift registers to connect
seven-segment displays and arrays of buttons to a micro-controller.

%%%%%%%%%%%%%%%%%%%%%%%%%%%%%%%%%%%%%%%%%%%%%%%%%%%%%%%%%%%%%%%%%%%%%%%%%%%%%%%%
\subsection{Serial-in parallel-out (SIPO) shift register}


Serial-in parallel out (SIPO) shift register translates a serial set of signals
(a binary code) to a parallel output.

Examples of tasks where a SIPO shift register can be used include controlling a set of LEDs or controlling a seven-segment indicator.

One of the implementations of SIPO shift register is 74HC595 chip.  The contact
placement of the chip (in other words, its \emph{pinout}) is shown in
fig. \ref{fig:shift-register-sipo}.

\figureShiftRegisterSIPO{en}

The purpose of chip pins is described in the table
\ref{table:shift-register-sipo-pins}.

\begin{table}[H]
  \centering
  \begin{tabular}{ | m{1.5cm} | m{2.5cm} | m{6.0cm} | }
    \hline
    \textbf{Pin}
    & \textbf{Alternative name}
    & \textbf{Description} \\
    \hline
    GND
    & GROUND
    & Ground, must be connected to a ``GND'' pin on Arduino. \\
    \hline
    $V_{cc}$
    &
    & The voltage of power supply, must be connected to a 5V pin
    on Arduino. \\
    \hline
    Q0...Q7
    & QA...QH
    & The outputs of the shift register. \\
    \hline
    DS
    & ``DATA'', ``SER''
    & The input data line. \\
    \hline
    $\overline{OE}$
    & ``Output Enabled''
    & Works as a switch for the output signals on
    Q0...Q8 pins. \\
    \hline
    ST\_CP
    & ``LATCH''
    & ``Latch'', outputs the values from the chip memory
    to Q0..Q7 pins. \\
    \hline
    SH\_CP
    & ``CLOCK'', ``SRCLK'' (Shift Register Clock)
    & Controls the loading the data into a register.  It is expected that
    this line must convey a clock signal.  When its value changes from
    \texttt{LOW} to \texttt{HIGH}, the register reads one bit from the ``DS''
    line into its memory. \\
    \hline
    $\overline{MR}$
    & ``RESET'', ``SRCLR'' (Shift Register Clear)
    & Reset pin.  Can be used to reset all the values in register memory
    to zero.  The reset performed when this line is pulled to \texttt{LOW}. \\
    \hline
    QA
    & Q7', QH'
    & The register serial output.  We can connect a ``DS'' line of the next
    SIPO register.  A shift register ``pushes out'' the most significant bit
    from memory to this pin each time when we push another bit into it through
    ``DS'' pin. \\
    \hline
  \end{tabular}
  \caption{Description of 74HC595 pins.}
  \label{table:shift-register-sipo-pins}
\end{table}

\end{document}
