%%%%%%%%%%%%%%%%%%%%%%%%%%%%%%%%%%%%%%%%%%%%%%%%%%%%%%%%%%%%%%%%%%%%%%%%%%%%%%%%
\newpage
\section{Basic principles of harmony}
\index{Music!Consonance}
\index{Music!Dissonance}

The human hearing is functioning in that way that we like combination of one
frequencies and dislike the combination of others.  We say that the sounds which
go well together produce \emph{consonance}; on the other hand the ``unpleasant''
combinations produce \emph{dissonance}.

For example, 50 Hz and 100 Hz frequencies sound good when combined, although
those frequencies are not ``musical'' -- that's because one frequency is exactly
two times higher than the other.  Understanding this small, on the first glance,
detail, helps us to compose sufficiently melodic pieces using sounds that are
multiplies of each another.

On the other hand, using two exact same sounds produce perfect consonance as the
acoustic waves overlap each other, amplifying the amplitude of waves.

\experiment{0}{Program two Arduino boards is such way so they play the same
  frequencies, and turn them on simultaneously.  Can you hear that the
  frequencies match?  What you feel when it happens?

  Modify one of the programs and increase the frequency of sound by 10Hz, then
  20Hz, 30Hz and so on.  What combinations are turn out to be pleasant, what are
  not?}

\experiment{1}{ Try to generate some reasonable frequency by an Ardu\-ino and
  simultaneously adjust your voice to this frequency, singing some vowel (say,
  ``A''.)  There's a chance that you'll get this ``Aha!'' moment when you
  discover your voice pitch is matched the sound of the Arduino speaker.}

\end{document}
