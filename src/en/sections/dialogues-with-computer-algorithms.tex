\documentclass[../sparc.tex]{subfiles}
\graphicspath{{\subfix{../images/}}}
\begin{document}

%%%%%%%%%%%%%%%%%%%%%%%%%%%%%%%%%%%%%%%%%%%%%%%%%%%%%%%%%%%%%%%%%%%%%%%%%%%%%%%%
\section{Algorithms}
\index{Programming!Algorithm}

Long before computer was made people were already writing instructions to each
other that allow to understand the sequence of steps to achieve some goal -- for
example, ``how to make a fire'', ``how to sow grain'', ``how to harness a
horse'' and so on.

A sequence of instructions that allows to achieve desired results called an
\emph{algorithm}.

There were some (rather successful) attempts to build different mechanisms that
did some sequence of operations to replace human labor.  But after inventing
electronic computers the development of algorithms reach ``another level'' --
now we can write an algorithm into computer memory so the computer will execute
the instructions exactly as we wanted.  Modern computers don't understand human
languages directly so we have to write down our algorithms in the form of
special \emph{machine language}.  A machine language is a set of commands
understood by a processor (the main computation device in a computer.)

Writing programs in machine language is a quite hard and tedious task so soon
after computers were invented people came up with the first programming
languages that are more friendly to humans.  The first programming language was
called \emph{assembler} that is but a set of mnemonics (that is, a set of short
human-understandable names) for machine commands of a processor.

But assembler was too simple and was not provided programmers with tools to easy
express ideas that programmers wanted to implement in their algorithms in
concise manner.  So high-level programming languages were invented that
simplified and speed-up the process of writing programs.  ``C'' language which
basics we will discuss in this chapter is one of the oldest programming language
that are actively used to this day.

To practice with composition of algorithms let's assume that we have an LED that
is connected to a computer.  To explain to a computer how to blink this LED we
have to formulate an algorithm.  This process is called \emph{formalization}.
One of possible algorithms for this task could look as follows:

\begin{enumerate}
\item Turn on the LED.
\item Turn off the LED.
\item Repeat the algorithm.
\end{enumerate}

In this example we see not only the instructions to turn on/turn off the LED but
also some kind of \emph{loop}, the cyclic nature of the process.

\experiment{0} { Try to came up with algorithms for ordinary operations that you
  do daily.  For example, you can try to formulate an algorithm of how to
  prepare a tea, or an algorithm of house cleaning activities.  How precise can
  you describe the sequence of actions for simple tasks?  How complicated is the
  resulting algorithm? }

\end{document}
