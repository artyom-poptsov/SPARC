\documentclass[../sparc.tex]{subfiles}
\graphicspath{{\subfix{../images/}}}
\begin{document}

%%%%%%%%%%%%%%%%%%%%%%%%%%%%%%%%%%%%%%%%%%%%%%%%%%%%%%%%%%%%%%%%%%%%%%%%%%%%%%%%
\section{Wokwi Simulator}

To program an Arduino you don't have to have a real Arduino board -- you can use
a \emph{simulator} instead.  The goal of a simulator is to imitate a real system
with minimal differences from the original.

By the way, there are two similar terms: \emph{simulator} and \emph{emulator}.

Let's look at the differences\cite{so:simulator-vs-emulator} between those two
terms:

\begin{itemize}
\item The task of a \emph{emulator} is to imitate the outer observable behavior
  of a system.  The state of inner workings of an emulation does not have to
  mirror the internal state of an emulated system.
\item The of a \emph{simulator} is to model the internal state of a simulated
  system.  The end result of a good simulation is model that duplicates the
  target system in the smallest details.  In an ideal case we have to be able to
  look inside a simulation and observe all the aspects of inner workings that we
  could observer in the same situation inside a simulated object.
\end{itemize}

Simulators can be divided between three categories:
\begin{itemize}
\item \emph{Offline simulators} are a standalone programs that we can download,
  install and run on a computer.
\item \emph{Online simulators} are special web sites (or rather web
  applications) that allows us to simulate a hardware platform inside a web
  browser (either by downloading all the required code inside a browser or by
  doing some operations on a server.)
\item \emph{Composite simulators} are standalone programs that however requires
  an Internet connection to a server to run.
\end{itemize}

Online simulators are easier for beginners as they don't require installation to
a computer.  However many of such simulators require registration which is not
acceptable for some users.

\end{document}
