\documentclass[../sparc.tex]{subfiles}
\graphicspath{{\subfix{../images/}}}
\begin{document}

%%%%%%%%%%%%%%%%%%%%%%%%%%%%%%%%%%%%%%%%%%%%%%%%%%%%%%%%%%%%%%%%%%%%%%%%%%%%%%%%
\section{Time signature}
\index{Music!Time signature}

As we may notice at the beginning of the musical staff, right near the treble
clef, there's $\frac{4}{4}$ -- what does it mean?

The $\frac{4}{4}$ marking (reads as ``four-four time'') describes the \emph{time
signature} of the composition.  From the point of programming it does not affect
neither the notes frequency, nor the lengths of notes.  But in the same time,
this mark directly affects the sound of the composition, and without it musical
compositions would sound quite boring.

The time signature creates this curious effect by \emph{accents} on certain
notes.

Let's take another look at ``Twinkle, Twinkle, Little Star''
(\ref{fig:lilypond-musical-scale-example-1}.)

\begin{figure}[ht]
  \centering
  \begin{lilypond}
    \relative c' {
      \numericTimeSignature
      \time 4/4
      c4 c g' g
      a a g2
      f4 f e e
      d d c2
      g'4 g f f
      e e d2
      g4 g f f
      e e d2
      c4 c g' g
      a a g2
      f4 f e e
      d d c2
    }
    \layout {
      indent = 0\mm
      line-width = 100\mm
      ragged-last = ##t
    }
  \end{lilypond}
  \caption{``Twinkle, Twinkle, Little Star'' in the time signature of four-four
    time.}
  \label{fig:lilypond-musical-scale-example-1}
\end{figure}

As the composition is written in the ``four-four'' time signature it means that
one bar holds exactly four quarter notes, or one whole note.  The denominator in
the time signature $\frac{4}{4}$ indicates the note value which signature is
counting; the upper value (numerator) in the signature indicates how many such
note values constitute a bar.

When a musical instrument is played the accent usually is placed on the first
note in the bar -- so this note is played stronger.

For clarity we can show the accent using the ``>'' symbol written below or above
a note.  Let's put those markings on the musical staff, as is shown on the
fig. \ref{fig:lilypond-musical-scale-example-2}.

\begin{figure}[ht]
  \centering
  \begin{lilypond}
    \relative c' {
      \numericTimeSignature
      \time 4/4
      c4-> c g'-> g
      a-> a g2->
      f4-> f e->  e
      d-> d c2->
      g'4-> g f-> f
      e-> e d2->
      g4-> g f-> f
      e-> e d2->
      c4-> c g'-> g
      a-> a g2->
      f4-> f e-> e
      d-> d c2->
    }
    \layout {
      indent = 0\mm
      line-width = 100\mm
      ragged-last = ##t
    }
  \end{lilypond}
  \caption{``Twinkle, Twinkle, Little Star'' with accents marked by ``>''
    symbols below the notes.}
  \label{fig:lilypond-musical-scale-example-2}
\end{figure}

Musical time signature ``four-four'' also known as a \emph{complex} time
signature, as it consisting of two \emph{simple} time signatures -- two of
two-fours.

Thus in the four-four time signature aside from the one accent on the 1st note
there's another, weaker, accent on the 3rd note.

As we can see from the fig. \ref{fig:lilypond-musical-scale-example-2}, the fist
(main) accent is placed upon the first note in the bar -- in our case, the first
quarter.  The secondary accent is placed upon the 3rd note in the bar -- or we
can say, it is placed on the 1st note of the 2nd half of the bar.  The secondary
accent is by definition weaker than the main accent.

If we take other time signature -- for example two-four signature ($\frac{2}{4}$)
then the composition will sound differently as the main accent is on the 1st
note of each bar, and the 2nd accent is no more.

\begin{figure}[ht]
  \centering
  \begin{lilypond}
    \relative c' {
      \numericTimeSignature
      \time 2/4
      c4-> c
      g'-> g
      a-> a
      g2->
      f4-> f e->  e
      d-> d c2->
      g'4-> g f-> f
      e-> e d2->
      g4-> g f-> f
      e-> e d2->
      c4-> c g'-> g
      a-> a g2->
      f4-> f e-> e
      d-> d c2->
    }
    \layout {
      indent = 0\mm
      line-width = 100\mm
      ragged-last = ##t
    }
  \end{lilypond}
  \caption{``Twinkle, Twinkle, Little Star'' in the two-four time signature.}
  \label{fig:lilypond-musical-scale-example-3}
\end{figure}

The two-four time signature is used in such musical genres as, for example,
polka.

If we take three-four time signature then there will be three quarter notes in a
bar (as is shown on fig. \ref{fig:lilypond-musical-scale-example-4}.)  Thus the
accent will be placed on the 1st note in each bar.  With that, some half-notes
($\frac{1}{2}$) are sliced by the vertical bar line into two quarters.

\begin{figure}[h]
  \centering
  \begin{lilypond}
    \relative c' {
      \numericTimeSignature
      \time 3/4
      c4->  c g'
      g->   a a
      g2->  f4
      f->   e e
      d->   d c4~4->
      g'4   g
      f->   f e
      e->   d2
      g4->  g f
      f->   e e
      d2    c4
      c->   g' g
      a->   a g4~4->
      f4    f
      e->   e d
      d->   c2
    }
    \layout {
      indent = 0\mm
      line-width = 100\mm
      ragged-last = ##t
    }
  \end{lilypond}
  \caption{``Twinkle, Twinkle, Little Star'' in the three-four time signature.}
  \label{fig:lilypond-musical-scale-example-4}
\end{figure}

Enforcing this time signature on ``Twinkle, Twinkle, Little Star'' feels
unnatural and after such experiments we may expect visits from the musical
inquisition.

Nevertheless if we try to play this composition in this time signature, it will
sound waltzing, as such signature usually is used for waltz.

But how can we bring those musical nuances into our programming code and the
hardware to make our musical compositions more beautiful?  Changes can be done
in several steps.

Firstly the easiest way to make accents on some notes is to connect additional,
less loud speaker, to our Arduino.  The notes that are not accented will be
played by this speaker.  And those notes that are accented will be played by the
loud speaker.

Let's suppose we connected the loud speaker on the digital port number 2, and
the quiet speaker to the port number 3.

\begin{listing}[ht]
  \begin{minted}{cpp}
    const int LOUD_SPEAKER_PIN  = 2;
    const int QUIET_SPEAKER_PIN = 3;

    // ...

    void setup() {
      pinMode(LOUD_SPEAKER_PIN, OUTPUT);
      pinMode(QUIET_SPEAKER_PIN, OUTPUT);
    }
  \end{minted}
  \label{listing:adding-additional-speaker}
  \caption{Adding another speaker to play notes with different sound volumes.}
\end{listing}

Secondly our two-dimensional array for storing the melody now has to have three
columns instead of two -- to store the volume of the sound.  Judging from the 3rd
column we will choose the speaker to play the sound.  Our volume control is
limited to the two levels: quiet (0) and loud (1.)

For the four-four time signature we have to make the first note in each bar
louder than the others.

\begin{listing}[!h]
  \begin{minted}{cpp}
    // ...

    float twinkle_twinkle_little_star[][3] = {
      {c4, 4, 1}, {c4, 4, 0}, {g4, 4, 0}, {g4, 4, 0},  // 0
      {a4, 4, 1}, {a4, 4, 0}, {g4, 4, 0},              // 1
      {f4, 4, 1}, {f4, 4, 0}, {e4, 4, 0}, {e4, 4, 0},  // 2
      {d4, 4, 1}, {d4, 4, 0}, {c4, 4, 0},              // 3

      {g4, 4, 1}, {g4, 4, 0}, {f4, 4, 0}, {f4, 4, 0},  // 4
      {e4, 4, 1}, {e4, 4, 0}, {d4, 4, 0},              // 5
      {g4, 4, 1}, {g4, 4, 0}, {f4, 4, 0}, {f4, 4, 0},  // 6
      {e4, 4, 1}, {e4, 4, 0}, {d4, 4, 0},              // 7

      {c4, 4, 1}, {c4, 4, 0}, {g4, 4, 0}, {g4, 4, 0},  // 8
      {a4, 4, 1}, {a4, 4, 0}, {g4, 4, 0},              // 9
      {f4, 4, 1}, {f4, 4, 0}, {e4, 4, 0}, {e4, 4, 0},  // 10
      {d4, 4, 1}, {d4, 4, 0}, {c4, 4, 0},              // 11
    };

    // ...
  \end{minted}
  \label{listing:adding-musical-scale-to-array}
  \caption{Adding accents to the notes according the time signature.}
\end{listing}

Next in the loop, while playing back our melody we have to choose the right
speaker for each note, based on the 3rd column in the array.

\begin{listing}[h]
  \begin{minted}{cpp}
    // ...

    void loop() {
      const long BPM = 120;
      const long MINUTE = 60 * 1000000;
      const long T = (MINUTE / BPM) * 4;

      for (int note_idx = 0; note_idx < 28; note_idx++) {
        if (melody[note_idx][2] == 1) {
          // Accented note.
          play_tone(LOUD_SPEAKER_PIN,
                    melody[note_idx][0],
                    T / melody[note_idx][1]);
        } else {
          // Regular note.
          play_tone(QUIET_SPEAKER_PIN,
                    melody[note_idx][0],
                    T / melody[note_idx][1]);
        }
        delay(100);
      }
    }
  \end{minted}
  \label{listing:musical-scale-implementation}
  \caption{Implementation of accents based on the time signature.}
\end{listing}

In the code listing \ref{listing:musical-scale-implementation} we added a
condition in the loop body, so if for the current note (referenced by
\texttt{note\_idx}) has the value 1 in the 2nd column, it means that this is the
accented note -- so we have to play it louder than the others.  Such notes we
send to the speaker connected to \texttt{LOUD\_SPEAKER\_PIN} digital port. Other
notes, as described in the \texttt{else} block, are sent to the speaker
connected to the \texttt{QUIET\_SPEAKER\_PIN} port.

\end{document}
