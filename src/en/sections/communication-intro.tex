\documentclass[../sparc.tex]{subfiles}
\graphicspath{{\subfix{../images/}}}
\begin{document}

\newglossaryentry{UART}{name=UART, description={Universal Asynchronous
    Receiver-Transmitter}}
\newglossaryentry{I2C}{name=I2C, description={Inter-Integrated Circuit}}

So now finally time has come for us to talk about the communication between a
micro-controller and external devices, and the language that is used for that.
``Sure!'' -- we can exclaim, -- ``The language of communication between machines
are the ones and zeroes, the machine code!''  And we will be right, the
communication between machines usually boils down to the transferring of ones
and zeroes.  But aside from this ``alphabet'' there are other important terms
such as the \emph{transmission medium} and the \emph{transmission protocol}.

Very often the transmission medium is represented by one of the three mediums: a
wire (like in case of USB cable), radio (Wi-Fi, Bluetooth) or the light (fiber
optic.)  Computers can even use sound waves as the means of communication but
that's more like an exception.

A transmission protocol sets the data transferring format itself -- where the
beginning and the end of a ``sentence'' and what is considered as ``words''
during the communication.  One of the well-known protocol examples is USB.

%%%%%%%%%%%%%%%%%%%%%%%%%%%%%%%%%%%%%%%%%%%%%%%%%%%%%%%%%%%%%%%%%%%%%%%%%%%%%%%%
\section{Classification of data transferring methods}

\newglossaryentry{LPT}{name=LPT, description={Line Print Terminal}}

Data transferring between devices can be divided into two big categories:

\begin{itemize}
\item \textbf{Parallel data transferring} -- all the bits of a message are
  transferred simultaneously (in parallel manner.)  For that the devices must be
  connected with a bus that has the same number of lines as the number of bits
  in the message.
\item \textbf{Serial data transferring} -- bits are transferred one after
  another, using time encoding.  For that even one data line (one wire) is
  enough.  But in fact most of serial data transferring methods use two or more
  lines.
\end{itemize}

\figureProtocolClassification{en}

One of the examples of parallel data transferring is \emph{parallel port}.
Usually in the computer world it is associated with Line Print Terminal
interface (\gls{LPT}.)  Back in the period between 1970-2000 such ports were
used for connecting printers to computers and were very popular.  In the modern
world LPT ports almost pushed out of use as they are replaced with such
interfaces as USB.

One of the advantages of a parallel port (as follows from its name) is the
ability to transfer several bits of information simultaneously, which speeds up
the data transferring.  For example, in LPT ports eight lines are used for the
data transfer (aside from the auxiliary lines such as the power supply, common
ground and the clock signal), thus allowing us to transfer 8 bits (1 byte) of
information in one clock tick.  Also we can consider the simplicity of the
implementation as one of the advantages of the parallel port as well.
Nevertheless it has an important (and usually critical) drawback: the interface
between devices is very clumsy as it requires many wires.  That's why we will
discuss in details only serial data transferring interfaces: \gls{UART} and I2C.

%% TODO: Add SPI.

%%%%%%%%%%%%%%%%%%%%%%%%%%%%%%%%%%%%%%%%%%%%%%%%%%%%%%%%%%%%%%%%%%%%%%%%%%%%%%%%
\subsection{Direction of data transferring}

Data transferring systems can also be divided into three categories according to
the direction of data flow:

\begin{itemize}
\item \textbf{Simplex} -- Uni-directional communication.  A simplex channel
  transfers data only in one direction.
\item \textbf{Half-duplex} -- Bi-directional communication.  Connected devices
  can transfer data in both directions, but only either transmitter or receiver
  can be active at the time.  A walkie-talkie is an example of such devices.
\item \textbf{Duplex} -- Bi-directional communication.  Connected devices can
  transmit and receive data simultaneously.  A telephone connection is a good
  example of such communication: both interlocutors can talk to each other and
  listen to each other at the same time.
\end{itemize}

\gls{UART} -- one of the serial interfaces that we will look at next -- allows us
to implement quite easily either simplex or duplex mode of communication.

%%%%%%%%%%%%%%%%%%%%%%%%%%%%%%%%%%%%%%%%%%%%%%%%%%%%%%%%%%%%%%%%%%%%%%%%%%%%%%%%
\subsection{Synchronous and asynchronous communication}

The last way of protocol classification that we now discuss is
\emph{synchronous} and \emph{asynchronous} communication.

\begin{itemize}
\item \textbf{Synchronous} communication implies the presence of the special
  clock signal that synchronizes two devices.  It is like a metronome in music
  that synchronizes parts of one composition to prevent, for example, a drummer
  from outrunning an electric guitar player.  When members of a musical band
  practicing before a show, often a metronome allows them to achieve performing
  coherency.  In electronics a special line in the data bus is used for the
  clock signal.  Usually it is labeled as ``SCK'', ``CLK'', ``CLOCK'' or simply
  clock signal.
\item \textbf{Asynchronous} communication allows devices to exchange data
  without the clock signal, but they must come to agreement among themselves
  about the speed of the transmitter/receiver.  Drawing an analogy from music we
  can say that if musicians agreed on the overall composition speed and had
  enough practice together then they can maintain the required tempo of the
  composition without using a metronome.
\end{itemize}

In this book we will discuss examples of both protocol types: \gls{UART} is an
asynchronous protocol, while \gls{I2C} is a synchronous protocol.

%% TODO: SPI is synchronous too.

\end{document}
