\documentclass[../sparc.tex]{subfiles}
\graphicspath{{\subfix{../images/}}}
\begin{document}

\newglossaryentry{UART}{name=UART, description={Universal Asynchronous
    Receiver-Transmitter}}

So now finally time has come for us to talk about the communication between a
micro-controller and external devices, and the language that is used for that.
``Sure!'' -- we can exclaim, -- ``The language of communication between machines
are the ones and zeroes, the machine code!''  And we will be right, the
communication between machines usually boils down to the transferring of ones
and zeroes.  But aside from this ``alphabet'' there are other important terms
such as the \emph{transmission medium} and the \emph{transmission protocol}.

Very often the transmission medium is represented by one of the three mediums: a
wire (like in case of USB cable), radio (Wi-Fi, Bluetooth) or the light (fiber
optic.)  Computers can even use sound waves as the means of communication but
that's more like an exception.

A transmission protocol sets the data transferring format itself -- where the
beginning and the end of a ``sentence'' and what is considered as ``words''
during the communication.  One of the well-known protocol examples is USB.

\end{document}
