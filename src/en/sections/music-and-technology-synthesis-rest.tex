\documentclass[../sparc.tex]{subfiles}
\graphicspath{{\subfix{../images/}}}
\begin{document}

%%%%%%%%%%%%%%%%%%%%%%%%%%%%%%%%%%%%%%%%%%%%%%%%%%%%%%%%%%%%%%%%%%%%%%%%%%%%%%%%
\section{Rest in the music}
\index{Music!Rest}

There's another thing what we kept out of focus to this moment -- and that's the
so called \emph{rests} (or pauses) in music.  The right rests are as important
as the regular notes themselves.

In the music staff those rests are marked with special symbols (as is shown on
fig. \ref{fig:lilypond-rest-example-1}.)

\begin{figure}[ht]
  \centering
  \begin{lilypond}
    \relative c' {
      \numericTimeSignature
      \time 4/4
      r1-"Whole rest."
    }
  \end{lilypond}
  \begin{lilypond}
    \relative c' {
      \numericTimeSignature
      \time 4/4
      r2-"Half rest." r2
    }
  \end{lilypond}
  \begin{lilypond}
    \relative c' {
      \numericTimeSignature
      \time 4/4
      r4-"Quarter rest." r4 r4 r4
    }
  \end{lilypond}
  \begin{lilypond}
    \relative c' {
      \numericTimeSignature
      \time 4/4
      r8-"Eighth rest." r8 r8 r8 r8 r8 r8 r8
    }
  \end{lilypond}
  \begin{lilypond}
    \relative c' {
      \numericTimeSignature
      \time 4/4
      r16-"Sixteenth rest." r16 r16 r16
      r16 r16 r16 r16
      r16 r16 r16 r16
      r16 r16 r16 r16
    }
  \end{lilypond}
  \caption{Rests in music.}
  \label{fig:lilypond-rest-example-1}
\end{figure}

The whole rest is equal to two half-rests, one half-rest is equal to two
quarter-rests and so on.  In other words we can use the same approach we use for
calculating the lengths of regular notes.

The symbols for music rests are shown in the table
\ref{table:music-rest-legths}.

\begin{table}[ht]
  \begin{tabular}{p{3cm}|p{4cm}|p{3.5cm}}
    Symbol & Length & Name \\
    \hline \hline
    \wholeNoteRest  & $\frac{1}{1}$   & ``Whole rest'' \\
    \hline
    \halfNoteRest   & $\frac{1}{2}$   & ``Half rest'' \\
    \hline
    \crotchetRest   & $\frac{1}{4}$   & ``Quarter rest'' \\
    \hline
    \quaverRest     & $\frac{1}{8}$ & ``Eighth rest'' \\
    \hline
    \semiquaverRest & $\frac{1}{16}$ & ``Sixteenth rest'' \\
    \hline
  \end{tabular}
  \caption{Some examples of possible music rests.}
  \label{table:music-rest-legths}
\end{table}

For the rests to work in our program we have to create a special note with zero
frequency.  We will be using ``R'' (for ``Rest'') as the name of the constant.

\begin{verbatim}
const float R = 0; // Rest
\end{verbatim}

Next we have to change the \texttt{play\_tone} procedure in such a way that it
can handle ``play back'' rests.  For that we add a condition that checks the
note frequency -- if it's greater than zero, then it will be used to generate a
sound; otherwise the procedure will just wait for the ``note'' length (the
\texttt{t} parameter of the procedure.)

\begin{minted}{cpp}
  // The procedure to generate a sound with the
  // specified parameters.
  void play_tone(int port, float f, long t) {
    if (f > 0) {
      const int T = 1000000 / f;
      int d = T / 2;
      int count = t / T;
      for (int i = 0; i < count; i++) {
        digitalWrite(port, HIGH);
        delayMicroseconds(d);
        digitalWrite(port, LOW);
        delayMicroseconds(d);
      }
    } else {
      delay(t / 1000); // Rest
    }
  }
\end{minted}

We should note that for making the proper rest we use the following code:

\begin{minted}{cpp}
  delay(t / 1000); // Rest
\end{minted}

We cannot use \texttt{delayMicroseconds} here as unfortunately this procedure
cannot wait for long periods of time; if we try to use it for, say, a 1000000
microseconds delay, we will get unexpected results.  We can use plain old
\texttt{delay} here, but it uses milliseconds instead of microseconds.  So we
need to divide our \texttt{t} by 1000 to convert microseconds to milliseconds.

To demonstrate the use of rests we will be using another musical composition --
``If only there will be no winter'' (``Кабы не было зимы'' in Russian), from an
old Soviet cartoon ``Prostokvashino'' (``Простоквашино'' in Russian.)

\begin{figure}[ht]
  \begin{lilypond}
    \relative c' {
      \key g \major
      \numericTimeSignature
      \time 4/4
      b8 b b'8. fis16 a8 g e4 |
      d8 d << b'8. d8. >> << c16 a >> << c8 a >> << b8 g8 >> r4
      d'8 c a fis << a c >> << g b >> << g4 b >>
      b,8 b << g'8. b8. >> << fis16 a >> << fis8 a >> << e8 g8 >> r4
    }
    \layout {
      indent = 0\mm
      line-width = 120\mm
      ragged-last = ##t
    }
  \end{lilypond}
  \caption{Part of the melody ``If only there will be no winter'' from the old
    Soviet cartoon ``Prostokvashino''.}
  \label{fig:lilypond-melody-prostokvashino}
\end{figure}

In the musical score on the fig. \ref{fig:lilypond-melody-prostokvashino} only a
part of the melody, that we will program in this section, is shown.  The full
version of the musical score is provided in
fig. \ref{fig:lilypond-melody-prostokvashino-full}.

As we can see, there are two quarter rests (\crotchetRest), that we have to add
to the array of notes, to make the composition sound as in the original version.

Also there are ``double notes'' that are written one on top of the other -- which
means that we have to play those notes simultaneously.  For simplicity we will
be taking only one note from pair, the higher one.

Let's try to write down notes in the array and listen to the melody.  To make
the code clear, we split the notes into groups according to bars, each bar on
its own line.  In the commentaries above the lines we add a commentary with the
bar number.

\begin{minted}{cpp}
  float prostokvashino[28][2] = {
    // 1st bar.
    {b3, 8}, {b3, 8}, {b4, 8}, {f4, 16}, {a4, 8}, {g4, 8}, {e4, 4},
    // 2nd bar.
    {d4, 8}, {d4, 8}, {d5, 8}, {c5, 16}, {c5, 8}, {b4, 8}, {R,  4},
    // 3rd bar.
    {d5, 8}, {c5, 8}, {a4, 8}, {f4, 8},  {c5, 8}, {b4, 8}, {b4, 4},
    // 4th bar.
    {b3, 8}, {b3, 8}, {b4, 8}, {a4, 16}, {a4, 8}, {g4, 8}, {R,  4},
  };
\end{minted}

When we try to play this melody on Arduino we may note that there are some
``inconsistencies'' when comparing to the original melody.  There are several
sources of such inconsistencies.  For example, we calculate the length of some
notes, marked with small dots on the right (like this: ``\eighthNoteDotted''),
in the wrong way.  The length of such notes are longer than regular notes.

Now it's time to discuss those ``marked'' notes and find out how to properly
calculate their lengths.

\end{document}
