\documentclass[../sparc.tex]{subfiles}
\graphicspath{{\subfix{../images/}}}
\begin{document}

%%%%%%%%%%%%%%%%%%%%%%%%%%%%%%%%%%%%%%%%%%%%%%%%%%%%%%%%%%%%%%%%%%%%%%%%%%%%%%%%
\section{Liquid crystal display (LCD)}
\index{Game development!Display}

\newglossaryentry{LCD}{name=LCD, description={Liquid Crystal Display}}

We will be using a \emph{liquid crystal display} (LCD) connected to the Arduino
for displaying the information.  The working principles of such displays are
similar to the regular LCDs that display information on your computer.
Specifically we will be using a \emph{text} LCD, which designation is to display
text symbols.

On text LCDs the screen is divided into cells, each of which can hold one symbol
(a letter, a punctuation mark, a digit or some other image.)  There are one
pixel-wide gap between the cells.  Such displays are not the best choice for
displaying custom images, although we have some room for creativity here.

Despite the properties of the display that look too restrictive at the first
glance, we can use our mastery and creative approach to achieve quite
interesting results in terms of game development.

The working area of the 16x2 display (16 columns, 2 rows) schematically looks
like is shown on fig. \ref{fig:lcd-16x2-schematics}.

\figureSmallLCDShematics{en}

The overall look of the 16x2 display is shown on fig. \ref{fig:lcd-16x2}.

\figureSmallLCD{en}

It is better to use a bigger 20x4 LCD (20 columns, 4 rows) for the projects
discussed in this chapter.  The working area of this display is bigger (as is
shown on fig. \ref{fig:lcd-20x4-schematics}) that allows us to build more
complicated games.

\figureBigLCDShematics{en}

The overall look of the 20x4 display is shown on fig. \ref{fig:lcd-20x4}.

\figureBigLCD{en}

%%%%%%%%%%%%%%%%%%%%%%%%%%%%%%%%%%%%%%%%%%%%%%%%%%%%%%%%%%%%%%%%%%%%%%%%%%%%%%%%
\subsection{LCD connection}
\index{Electronics!LCD}

To make our work easier we will be using an LCD with the connection interface
called \gls{I2C}. We've already discussed I2C in the section \ref{section:i2c}.

Now then we've discussed I2C we can just say that such connection requires only
4 wires: power, ground and two data transfer lines.

Very often an I2C module is already soldered down to the display, but in other
cases it can be sold as a separate part. In this case the module connection
looks like is shown on fig. \ref{fig:lcd-00}.

\figureLCDConnection{en}

\note{en}{ It should be noted that the I2C module connection schematics can vary
  depending on its vendor and the vendor of the LCD in question.  To avoid
  damaging of the LCD it is strongly recommended to read the documentation --
  usually it can be found on the vendor's website, or by searching for something
  like ``<component> datasheet'' in a web search engine, where instead of
  ``<component>'' we have to substitute the component model. }

%%%%%%%%%%%%%%%%%%%%%%%%%%%%%%%%%%%%%%%%%%%%%%%%%%%%%%%%%%%%%%%%%%%%%%%%%%%%%%%%
\subsection{Printing some text}
\index{Game development!Printing text}

\newglossaryentry{method}{name=method, description={A procedure which is
    declared inside a class (or, we can say, a procedure that is inside an
    object) which represents some action that the object can perform, or an
    action that can be performed upon an object.}}

Following the programmers tradition, the first string that we will print on the
display is ``Hello, World!''.  To do that we have to install an Arduino library
for working with the display first; secondly we have to configure the LCD and
only after that we will be able to print the text.

Let's start with downloading a LCD library.  The easiest way to do that is to
use the Arduino library manager, which can be found in ``Tools'' $\rightarrow$ ``Manage
Libraries...''.  Then we have to select the right library in the list and press
the ``Install'' button.  There are several libraries for working with LCDs using
the \gls{I2C} protocol.  We can use ``Liquid Crystal I2C'' version 1.1.1
developed by Marco Schwartz.

After installing the library we can use its facilities.  Firstly we have to add
the library to our project by including ``LiquidCrystal\_I2C.h'' file.  It can
be done through a special \mintinline{cpp}{#include} directive.

\begin{listing}[ht]
  \begin{minted}{cpp}
    #include <LiquidCrystal_I2C.h>
  \end{minted}
  \caption{Including an LCD library into our project.}
  \label{listing:game-dev-lcd-include}
\end{listing}

In the global scope of our code (before \mintinline{cpp}{setup} and
\mintinline{cpp}{loop} procedures) we have to create a special variable, through
which we will be interacting with the display.  This variable differs from what
we've seen before -- because in this variable we will be storing a reference to
some complex \emph{object} stored somewhere in the memory during program
execution inside a micro-controller.\footnote{The ``object'' term is from the
programming methodology which is called \emph{object-oriented programming}
(``OOP'' for short.)  We won't be focusing our attention on OOP for now, but a
curious reader can get the overall understanding of the methodology from
Wikipedia.}

Let's call the variable \mintinline{cpp}{lcd}.

\begin{listing}[ht]
  \begin{minted}{cpp}
    LiquidCrystal_I2C lcd(0x27,  16, 2);
  \end{minted}
  \caption{Creating an object for working with an LCD.}
  \label{listing:game-dev-lcd-object}
\end{listing}

In the parenthesis we put the parameters for the object being created.  Let's
take a closer look on them.

The \mintinline{cpp}{0x27} number (which in binary code is represented as
\mintinline{cpp}{0100 111} is the device address on the \gls{I2C} bus.  As we
discussed in the section \ref{section:i2c}, the device address can be changed by
shortening special jumper pads on our I2C module.  The higher four address bits
(\mintinline{cpp}{0100}) are pre-configured by the manufacturer and cannot be
changed, but the last three bits the ones that can be configured by the jumper
pads on the module.  The most less-significant eighth bit in the address is not
mentioned here as it affects the data transfer direction, and this bit is
controlled by the \mintinline{cpp}{LiquidCrystal\_I2C} library.

The numbers 16 and 2 set the size of the display (its width and height
respectively.)  If our display is 20x4 then we have to replace those numbers by
20 and 4.

Now then we discussed the nuances of the LCD configuration we can initialize the
display in \mintinline{cpp}{setup} procedure.

\begin{listing}[ht]
  \begin{minted}{cpp}
    lcd.init();
    lcd.backlight();
  \end{minted}
  \caption{LCD initialization code.}
  \label{listing:game-dev-lcd-init}
\end{listing}

As \mintinline{cpp}{lcd} variable stores a reference to some complex object,
with some internal attributes and possible actions (\emph{methods}), the working
with the variable may appear unfamiliar for the inexperienced readers on the
first glance.

We can say that a \gls{method} is a procedure that is stored inside an object.
Calls to such procedures are not very different from the calls to regular
procedures, although we have to write the object name before the procedure name,
separating them with a dot.

The \mintinline{cpp}{init} method performs the initialization of the LCD.
Without it the display will not work correctly, so all other operations with the
display must go after the call to \mintinline{cpp}{init}.

The \mintinline{cpp}{backlight} method enables the LCD backlight.  The display
itself does not produce any light (more precisely, display pixels don't glow),
and the lighting of the display is provided by a special LED embedded into the
display.  The call to \mintinline{cpp}{backlight} is not necessary, but usually
display looks nicer with backlight enabled.

There's also \mintinline{cpp}{noBacklight} method that disables the backlight.

With all preparations done, we can print some text on the display in the
\mintinline{cpp}{loop} procedure.  This is done in two steps: firstly we have to
set our ``cursor'' to the position on the display where we want to print the
text -- there's a special \mintinline{cpp}{setCursor} method that does just that.

\begin{listing}[ht]
  \begin{minted}{cpp}
    lcd.setCursor(0, 0);
  \end{minted}
  \caption{Setting the cursor position on an LCD.}
  \label{listing:game-dev-lcd-set-cursor}
\end{listing}

The first parameter of \mintinline{cpp}{setCursor} sets the cursor position on
the X axis, and the second parameter sets the Y position.

After that we can finally print the text with \mintinline{cpp}{print} method.

\begin{listing}[ht]
  \begin{minted}{cpp}
    lcd.print("Hello, World!");
  \end{minted}
  \caption{Printing some text on an LCD.}
  \label{listing:game-dev-lcd-print}
\end{listing}

The overall code should look similar to the example that is shown below.

\begin{listing}[H]
  \begin{minted}{cpp}
    #include <LiquidCrystal_I2C.h>

    LiquidCrystal_I2C lcd(0x27,  16, 2);

    void setup() {
      lcd.init();
      lcd.backlight();
    }

    void loop() {
      lcd.setCursor(0, 0);
      lcd.print("Hello, World!");
    }
  \end{minted}
  \caption{An simple program example that prints a text to an LCD.}
  \label{listing:game-dev-lcd-example-00}
\end{listing}

If we want, we can ``erase'' parts of the text, by overwritting them with
spaces, or we can call \mintinline{cpp}{clear} method that clears the whole
screen.

%%%%%%%%%%%%%%%%%%%%%%%%%%%%%%%%%%%%%%%%%%%%%%%%%%%%%%%%%%%%%%%%%%%%%%%%%%%%%%%%
\subsubsection{Exercises}

\begin{enumerate}
\item Alight the text to the center of the display.
\item Implement a ``blinking text'' effect.
\item Implement a scrolling text on the display.
\item Try to make the LCD backlight to blink.
\end{enumerate}

\end{document}
