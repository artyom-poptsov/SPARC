\documentclass[../sparc.tex]{subfiles}
\graphicspath{{\subfix{../images/}}}
\begin{document}

%%%%%%%%%%%%%%%%%%%%%%%%%%%%%%%%%%%%%%%%%%%%%%%%%%%%%%%%%%%%%%%%%%%%%%%%%%%%%%%%
\section{Liquid crystal display (LCD)}
\index{Game development!Display}

We will be using a \emph{liquid crystal display} (LCD) connected to the Arduino
for displaying the information.  The working principles of such displays are
similar to the regular LCDs that display information on your computer.
Specifically we will be using a \emph{text} LCD, which designation is to display
text symbols.

On text LCDs the screen is divided into cells, each of which can hold one symbol
(a letter, a punctuation mark, a digit or some other image.)  There are one
pixel-wide gap between the cells.  Such displays are not the best choice for
displaying custom images, although we have some room for creativity here.

Despite the properties of the display that look too restrictive at the first
glance, we can use our mastery and creative approach to achieve quite
interesting results in terms of game development.

The working area of the 16x2 display (16 columns, 2 rows) schematically looks
like is shown on fig. \ref{fig:lcd-16x2-schematics}.

\figureSmallLCDShematics{en}

The overall look of the 16x2 display is shown on fig. \ref{fig:lcd-16x2}.

\figureSmallLCD{en}

It is better to use a bigger 20x4 LCD (20 columns, 4 rows) for the projects
discussed in this chapter.  The working area of this display is bigger (as is
shown on fig. \ref{fig:lcd-20x4-schematics}) that allows us to build more
complicated games.

\figureBigLCDShematics{en}

The overall look of the 20x4 display is shown on fig. \ref{fig:lcd-20x4}.

\figureBigLCD{en}

\end{document}
