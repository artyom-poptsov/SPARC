\documentclass[../sparc.tex]{subfiles}
\graphicspath{{\subfix{../images/}}}
\begin{document}

%%%%%%%%%%%%%%%%%%%%%%%%%%%%%%%%%%%%%%%%%%%%%%%%%%%%%%%%%%%%%%%%%%%%%%%%%%%%%%%%
\section{I2C bus}
\label{section:i2c}
\index{Electronics!I2C bus}

\newglossaryentry{I2C}{
  name=I2C,
  description={Inter-Integrated Circuit}
}
\newglossaryentry{USB}{
  name=USB,
  description={Universal Serial Bus}
}
\newglossaryentry{PCI}{
  name=PCI,
  description={Peripheral component interconnect}
}
\newglossaryentry{I/O}{
  name=I/O,
  description={I/O -- ``Input/Output''}
}

%%%%%%%%%%%%%%%%%%%%%%%%%%%%%%%%%%%%%%%%%%%%%%%%%%%%%%%%%%%%%%%%%%%%%%%%%%%%%%%%
\subsection{General information}

\textit{\gls{I2C}} also sometimes called $I^{2}C$ (should be read as
``I-squared-C'') is a data transfer bus that is used for connection between
integrated circuit inside electronic devices.  Let's discuss first what a ``data
bus'' is.  To put it simply a \textit{bus} in the context of the computer
architecture is some connection that is used for data transferring between
functional parts of a computer.

We can put data transfer buses roughly into two categories -- \textit{external}
buses and \textit{internal} buses.

External buses include \gls{USB} which is already familiar for many of us -- this
bus is used for connection of external devices (such as Arduino, a keyboard, a
computer mouse, a printer and so on) to a computer.  Such buses usually have
some standardized connector and fairly big length of the cable.

Internal buses include \gls{I2C} (already mentioned above) and \gls{PCI}/PCI
Express (used for example for connection of a network card to the computer
system board.)  Internal buses are usually designed for transferring data across
short distances, and don't allow hot-plugging of devices; thus some of the
internal buses don't even have a special connector and bus wires are soldered
down right to the boards.

The maximum length of connection between devices on \gls{I2C} bus should not
exceed 30 centimeters, otherwise data transfer will be unreliable.

\gls{I2C} has quite low data transfer speed (no more than 5MBit/s) between
devices.  Each device has its own \textit{address} on the bus -- in other words,
its serial number.  We can connect up to 127 devices with unique addresses.
With that said, we must note that only one device on the bus is the controller
while the rest of the devices are targets.

On the fig. \ref{fig:i2c-schematics} the \gls{I2C} bus schematic is shown.  As
we can see on the figure, the bus uses only two lines for data transferring --
``Serial Data Line'' (\textbf{SDL}) and ``Serial Clock Line'' (\textbf{SCL}.)
Both lines are pulled up through resistors \textbf{$R_p$} to the voltage of
power supply (\textbf{Vdd}.)  On the schematics, ``$\mu$C\\Controller'' is the
controller device and others (labeled as ``Targets'') are the target devices.

\figureIICSchematics{en}

\end{document}
