\documentclass[../sparc.tex]{subfiles}
\graphicspath{{\subfix{../images/}}}
\begin{document}

%%%%%%%%%%%%%%%%%%%%%%%%%%%%%%%%%%%%%%%%%%%%%%%%%%%%%%%%%%%%%%%%%%%%%%%%%%%%%%%%
\section{Serial Port}
\index{Electronics!Serial Port}
\label{section:serial-port}

Connection of Arduino to a computer is usually is done through a USB connector
placed on the board.  On the Arduino Mega 2560 side USB-B is used for that
purpose.

The USB protocol is quite complicated and the most of the Arduino chips don't
``speak'' USB protocol and need a ``translator'' to be understood by a computer.
A translator is usually a separate chip called USB-UART converter soldered on an
Arduino board.  We will discuss UART in a section
\ref{section:communication-uart}.  Some compact Arduino such as Arduino Pro Mini
does not have its own USB-UART converter and require a separate converter.

An USB-UART converter emulates a simpler data transfer protocol than USB that is
commonly known as ``Serial Port''.  On old computers this port (also known as
COM-port) was represented as a separate interface on the main board and was used
to connect such devices as computer mice.  COM-ports used to be quite widespread
back in the days (for example they were popular on IBM PC AT computers.)  In the
modern world COM-ports were replaced in most of the consumer electronics with
other standards, mostly USB.  But COM-ports are still in use today on
specialized equipment.

One of the examples where serial ports are used to this day is working with
microcontrollers.  USB-UART converter emulates a COM-port for a computer
allowing to ``communicate'' with a microcontroller using a simpler language than
USB.  Thus for a computer an Arduino looks like a COM-port despite that the
physical connection is done through a USB.  Using this line of communication we
can transfer data from a computer to an Arduino and vice versa.

We will discuss serial ports in more detail in the section
\ref{section:communication-serial-port}.  Currently we only need the basic
knowledge on this topic in order to understand the examples in chapter.

%%%%%%%%%%%%%%%%%%%%%%%%%%%%%%%%%%%%%%%%%%%%%%%%%%%%%%%%%%%%%%%%%%%%%%%%%%%%%%%%
\subsection{Basics of Serial Port Usage in Arduino}

Before we start working with a serial port we need to configure it.  It can be
done as follows: first we need to configure the speed of data transfer in
\texttt{setup} function:

\begin{minted}{cpp}
  void setup() {
    Serial.begin(9600);
  }
\end{minted}

In this case we are allowing data transfer between a computer and an Arduino
with the specified speed where 9600 is the speed set in \emph{bauds}.  A
\emph{baud} is a unit of data transfer speed measurement that is equal to a bit
per second.  Usually this parameter is set to one of the standard values: 300,
600, 1200, 2400, 4800, 9600, 14400, 19200, 28800, 38400, 57600, 115200.

%%%%%%%%%%%%%%%%%%%%%%%%%%%%%%%%%%%%%%%%%%%%%%%%%%%%%%%%%%%%%%%%%%%%%%%%%%%%%%%%
\subsection{Transferring Data to a Computer}

Now let's try to transfer some data to a computer.  As the first example we will
send ``Hello, world!'' string through a serial port.  First we need to write
down the serial port configuration in the \texttt{setup} function:

\begin{minted}{cpp}
  void setup() {
    Serial.begin(9600); // Setting the speed of a serial port.
  }
\end{minted}

And that's how our \texttt{loop} function looks like:

\begin{minted}{cpp}
  void loop() {
    Serial.println("Hello World");
    delay(1000); // Waiting 1000 ms before the next transfer
  }
\end{minted}

To see the results of program execution we need to open the ``Serial Monitor''
in the Arduino IDE from the ``Tools'' menu.

Aside from that we can open ``Serial Monitor'' by \hotkey{Ctrl + Shift + M}
hotkey.

\end{document}
