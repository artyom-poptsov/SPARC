\documentclass[../sparc.tex]{subfiles}
\graphicspath{{\subfix{../images/}}}
\begin{document}

%%%%%%%%%%%%%%%%%%%%%%%%%%%%%%%%%%%%%%%%%%%%%%%%%%%%%%%%%%%%%%%%%%%%%%%%%%%%%%%%
\section{Serial Port}
\index{Electronics!Serial Port}
\label{section:serial-port}

Connection of Arduino to a computer is usually is done through a USB connector
placed on the board.  On the Arduino Mega 2560 side USB-B is used for that
purpose.

The USB protocol is quite complicated and the most of the Arduino chips don't
``speak'' USB protocol and need a ``translator'' to be understood by a computer.
A translator is usually a separate chip called USB-UART converter soldered on an
Arduino board.  We will discuss UART in a section
\ref{section:communication-uart}.  Some compact Arduino such as Arduino Pro Mini
does not have its own USB-UART converter and require a separate converter.

An USB-UART converter emulates a simpler data transfer protocol than USB that is
commonly known as ``Serial Port''.  On old computers this port (also known as
COM-port) was represented as a separate interface on the main board and was used
to connect such devices as computer mice.  COM-ports used to be widely used (for
example they were popular on IBM PC AT computers.)  In the modern world
COM-ports were replaced in most of the consumer electronics with other
standards, mostly USB.  But COM-ports are still in use today on specialized
equipment.

\end{document}
