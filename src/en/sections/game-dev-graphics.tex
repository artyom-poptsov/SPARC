\documentclass[../sparc.tex]{subfiles}
\graphicspath{{\subfix{../images/}}}
\begin{document}

\section{Graphics}
\index{Game development!Graphics}
\newglossaryentry{NPC}{name=NPC, description={Non-Playable Character}}

%%%%%%%%%%%%%%%%%%%%%%%%%%%%%%%%%%%%%%%%%%%%%%%%%%%%%%%%%%%%%%%%%%%%%%%%%%%%%%%%
\subsection{Using pre-defined symbols}
\index{Programming!Character table}
\index{Game development!Graphics!Standard symbols}

Computers ``perceive'' symbols as everything else -- that is, as numbers.  For
each symbol there's some code that the computer remembers and processes, when we
ask it to store a symbol into a memory or print it to the screen.

\note{en}{ One of the simplest form of symbol encoding is ASCII (American
  Standard Code for Information Interchange) table that describes symbols with
  codes from 0 to 127.  In modern operating systems Unicode encoding is used
  instead to support different languages.  Unicode table is way bigger than
  ASCII and includes most of the languages of the world.  Nevertheless, Unicode
  keeps the backward compatibility with ASCII, as the first 128 symbols are the
  same as in the ASCII table. }

In text-based LCDs there's an embedded symbol table that controls how the
display prints the symbols.  The set of the symbols can differ depending on the
model of the display in question, that's why it is recommended to refer to a
symbol table for the specific screen.

For example, in many displays the symbol with 255 code (\mintinline{cpp}{0xFF}
in hexadecimal form) represents a filled rectangle.  To use this symbol in a
project, it is convenient to create a constant that represents this game
``object''.

\begin{minted}{cpp}
  const char WALL = 0xFF; // Symbol code.
\end{minted}

It should be noted that the symbol code is specified without quotes, as we are
specifying the number of the symbol in the table instead of its graphical
representation.

After we created the constant for the symbol in the code, we can print it in the
code as follows:

\begin{minted}{cpp}
  void loop() {
    lcd.setCursor(0, 0);
    lcd.print(WALL);
  }
\end{minted}

%%%%%%%%%%%%%%%%%%%%%%%%%%%%%%%%%%%%%%%%%%%%%%%%%%%%%%%%%%%%%%%%%%%%%%%%%%%%%%%%
\subsection{Creating custom symbols}
\index{Game development!Graphics!Custom symbols}

Despite that we're using a text LCD in our project and it does not provide means
for drawing arbitrary images in any place on the screen, we can create custom
symbols for it.  Those custom symbols can be drawn inside cells on the screen.

The size of one screen cell is 8x5 pixels (8 rows, 5 columns.)  Schematically we
can visualize it as is shown in fig. \ref{fig:game-dev-char}.

\figureLCDChar{en}

There are 80 of such cells on a display with 4 rows and 20 columns, as can be
seen in fig. \ref{fig:game-dev-char}.

Each pixel is represented by one bit.  One means that the pixel is filled, and
zero means that the pixel is not filled.  For example, a line which is shown in
fig. \ref{fig:game-dev-char-symbol-encoding} can be written as
\mintinline{cpp}{0b01010}.

\figureLCDCharRow{en}

Each line is encoded as one byte (eight bits) of information, from which
factually only 5 lower bits are used.  Totally there are 8 of such rows, thus
for storing them we need an array of 8 bytes.

An example of game character drawing:

\begin{listing}[H]
  \begin{minted}{cpp}
    byte player_image[8] = {
      0b01110,
      0b01110,
      0b00100,
      0b01110,
      0b10101,
      0b01110,
      0b01010,
      0b01010
    };
  \end{minted}
  \caption{A symbol for text-based LCD, which is represented by an
    one-dimensional array of bytes.}
  \label{listing:game-dev-lcd-custom-char}
\end{listing}

A visual representation of the symbol is shown below:

\figureLCDCharPlayer{en}

We can draw such images on a grid paper and then translate them by the hand into
a program code.  Of course, programmers already automated this process so there
are web pages that allow us to draw such characters and generate program code
from them.  Here are one of them:
\url{https://zoomch02.github.io/LCD-Character-Generator/}

After the symbol is drawn, it must be written into LCD memory so we can use it
later.  From the embedded character table we have access to codes from 0 to 7
inclusive, that allows us to write 8 custom characters into the display memory.
The rest of the symbols in the table cannot be overwritten.

\end{document}
