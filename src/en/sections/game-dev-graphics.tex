\documentclass[../sparc.tex]{subfiles}
\graphicspath{{\subfix{../images/}}}
\begin{document}

\section{Graphics}
\index{Game development!Graphics}
\newglossaryentry{NPC}{name=NPC, description={Non-Playable Character}}

%%%%%%%%%%%%%%%%%%%%%%%%%%%%%%%%%%%%%%%%%%%%%%%%%%%%%%%%%%%%%%%%%%%%%%%%%%%%%%%%
\subsection{Using pre-defined symbols}
\index{Programming!Character table}
\index{Game development!Graphics!Standard symbols}

Computers ``perceive'' symbols as everything else -- that is, as numbers.  For
each symbol there's some code that the computer remembers and processes, when we
ask it to store a symbol into a memory or print it to the screen.

\note{en}{ One of the simplest form of symbol encoding is ASCII (American
  Standard Code for Information Interchange) table that describes symbols with
  codes from 0 to 127.  In modern operating systems Unicode encoding is used
  instead to support different languages.  Unicode table is way bigger than
  ASCII and includes most of the languages of the world.  Nevertheless, Unicode
  keeps the backward compatibility with ASCII, as the first 128 symbols are the
  same as in the ASCII table. }

In text-based LCDs there's an embedded symbol table that controls how the
display prints the symbols.  The set of the symbols can differ depending on the
model of the display in question, that's why it is recommended to refer to a
symbol table for the specific screen.

For example, in many displays the symbol with 255 code (\mintinline{cpp}{0xFF}
in hexadecimal form) represents a filled rectangle.  To use this symbol in a
project, it is convenient to create a constant that represents this game
``object''.

\begin{minted}{cpp}
  const char WALL = 0xFF; // Symbol code.
\end{minted}

It should be noted that the symbol code is specified without quotes, as we are
specifying the number of the symbol in the table instead of its graphical
representation.

After we created the constant for the symbol in the code, we can print it in the
code as follows:

\begin{minted}{cpp}
  void loop() {
    lcd.setCursor(0, 0);
    lcd.print(WALL);
  }
\end{minted}

\end{document}
