\documentclass[../sparc.tex]{subfiles}
\graphicspath{{\subfix{../images/}}}
\begin{document}

%%%%%%%%%%%%%%%%%%%%%%%%%%%%%%%%%%%%%%%%%%%%%%%%%%%%%%%%%%%%%%%%%%%%%%%%%%%%%%%%
\section{Adding sounds}
\index{Game development!Sound}

Rarely we can find a game that is without sound effects.  Now we are going to
add some sound to our game as well.

Let us connect the speaker for outputting sounds to the eighth port -- we will
specify it in the code as a named constants, before \mintinline{cpp}{setup}
procedure.

\begin{minted}{cpp}
  const byte SPEAKER_PIN = 8; // Speaker port.
\end{minted}

Now in in \mintinline{cpp}{setup} procedure we will configure the speaker port
to the output mode.

\begin{minted}{cpp}
  void setup() {
    lcd.init();
    lcd.backlight();

    // Control buttons configuration.
    pinMode(BUTTON_R, INPUT_PULLUP);
    pinMode(BUTTON_L, INPUT_PULLUP);

    // Configuration of the speaker port.
    pinMode(SPEAKER_PIN, OUTPUT);

    // Calling to the procedure of
    // the map generation.
    map_generate();
  }
\end{minted}

Now we will copy \mintinline{cpp}{play_tone} procedure from the chapter
\ref{chapter:music-and-technology-synthesis} to the our game.  Also we will add
constants for notes.

\begin{minted}{cpp}
  // Fourth (in the scientific notation,
  // starting from zero) octave.
  const float c4 = 261.630;
  const float d4 = 293.660;
  const float e4 = 329.630;
  const float f4 = 349.230;
  const float g4 = 392.000;
  const float a4 = 440.000;
  const float b4 = 493.880;

  // Procedure for generating a sound
  // with the specified frequency.
  void play_tone(int port, float f, long t) {
    const long T = 1000000 / f;
    long d = T / 2;
    int count = t / T;
    for (int i = 0; i < count; i++) {
      digitalWrite(port, HIGH);
      delayMicroseconds(d);
      digitalWrite(port, LOW);
      delayMicroseconds(d);
    }
  }
\end{minted}

\end{document}
