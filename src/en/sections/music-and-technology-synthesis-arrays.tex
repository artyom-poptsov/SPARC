\documentclass[../sparc.tex]{subfiles}
\graphicspath{{\subfix{../images/}}}
\begin{document}

%%%%%%%%%%%%%%%%%%%%%%%%%%%%%%%%%%%%%%%%%%%%%%%%%%%%%%%%%%%%%%%%%%%%%%%%%%%%%%%%
\section{Arrays}
\index{Programming!Array}

An \emph{array} is a variable that holds a sequence of values of the same type.
In an array we can store the notes of our melody.

Visually we can imagine an array as the sequence of ``boxes'' (or cells), each
of which has a serial number (that is called an \emph{index}) and can hold one
element.

If we lay down seven notes from our melody into an array, we get something like
is shown on the fig. \ref{table:array-example-1}.

\begin{tabular}{|p{1cm}|p{1cm}|p{1cm}|p{1cm}|p{1cm}|p{1cm}|p{1cm}|}
  \hline
  c4 & c4 & g4 & g4 & a4 & a4 & g4 \\
  \hline
  \multicolumn{1}{c}{0}
  & \multicolumn{1}{c}{1}
  & \multicolumn{1}{c}{2}
  & \multicolumn{1}{c}{3}
  & \multicolumn{1}{c}{4}
  & \multicolumn{1}{c}{5}
  & \multicolumn{1}{c}{6}
  \label{table:array-example-1}
\end{tabular}

In this example the array holds seven elements, while the index of the first
element is equal to zero (as people say, ``real programmers always start
counting from zero''), and the last element of the array is the sixth.  If we
try to get a non-existing element from an array (say, seventh element, or an
element with the index -1) it will lead to an error.

We need to create an array with the right amount of cells to hold our melody.
To be precise, we need an array of 28 cells to hold toe ``Twinkle, Twinkle,
Little Start'' melody in it.  We can put a value for each cell of the array
right in the process of defining the array.

The rest of the array definition is similar to definition of a variable: we must
specify the type of the array, its name and its size (inside a square brackets),
and then, after ``='' sign we can provide the list of values.

\begin{listing}[ht]
  \begin{minted}{cpp}
    float melody[28] = {
      c4, c4, g4, g4,
      a4, a4, g4,
      f4, f4, e4, e4,
      d4, d4, c4,
      g4, g4, f4, f4,
      e4, e4, d4,
      g4, g4, f4, f4,
      e4, e4, d4,
    };
  \end{minted}
  \label{listing:music-array-example-1}
  \caption{An example of definition of an array with notes.}
\end{listing}

To address an element of the array we have to use its name with the number of
the element inside the square brackets.  For example, if we want to change the
zeroth element we can do it as follows:

\begin{listing}[ht]
  \begin{minted}{cpp}
    melody[0] = g4;
  \end{minted}
  \label{listing:music-array-example-2}
  \caption{An example of changing a value inside an array.}
\end{listing}

It is reasonable to define the array \ref{listing:music-array-example-1} just
before our \texttt{loop} procedure.  Inside the \texttt{loop} we can walk across
the array inside a \texttt{for} loop and play a note from each array cell.  To
do that we can use the counter variable from the \texttt{for} loop as the array
index to address the elements.


\begin{listing}[ht]
  \begin{minted}{cpp}
    void loop() {
      const long BPM = 120;
      const long MINUTE = 60000000;
      const long T = (MINUTE / BPM) * 4;

      for (int note_idx = 0; note_idx < 28; note_idx++) {
        play_tone(SPEAKER_PIN, melody[note_idx], T / 4);
        delay(100);
      }
    }
  \end{minted}
  \label{listing:music-array-example-3}
  \caption{Playing a melody from an array.}
\end{listing}

By now we should clearly see how short our melody implementation is became.  But
we temporarily lost the ability to set the length of each individual note in our
melody.

To fix this we can use another array to hold the lengths of our notes -- let's
call this second array \texttt{melody\_t}.  Each element of this array will hold
the length of the pair note from the array \texttt{melody} as the fraction
denominator, while the numerator is the bar length.

For example, the zeroth note of our melody is ``C4'' has the length of
$\frac{1}{4}$ -- so we need to put 4 in the zeroth cell of \texttt{melody\_t}.

\begin{listing}[H]
  \begin{minted}{cpp}
    // Array with notes (frequencies.)
    float melody[28] = {
      c4, c4, g4, g4,
      a4, a4, g4,
      f4, f4, e4, e4,
      d4, d4, c4,
      g4, g4, f4, f4,
      e4, e4, d4,
      g4, g4, f4, f4,
      e4, e4, d4,
    };

    // Array with note lengths.
    float melody_t[28] = {
      4,  4,  4,  4,
      4,  4,  2,
      4,  4,  4,  4,
      4,  4,  2,
      4,  4,  4,  4,
      4,  4,  2,
      4,  4,  4,  4,
      4,  4,  2,
    };
  \end{minted}
  \label{listing:music-array-example-4}
  \caption{An example of the additional array that holds the lengths of the
    notes.}
\end{listing}

After that we have to update our code that plays the melody:

\begin{listing}[H]
  \begin{minted}{cpp}
    // ...

    void loop() {
      const long BPM = 120;
      const long MINUTE = 60000000;
      const long T = (MINUTE / BPM) * 4;

      for (int note_idx = 0; note_idx < 28; note_idx++) {
        play_tone(SPEAKER_PIN,
                  melody[note_idx],
                  T / melody_t[note_idx]);
        delay(100);
      }
    }
  \end{minted}
  \label{listing:music-array-example-5}
  \caption{Updated code for playing the melody.}
\end{listing}

The code \texttt{T / melody\_t[note\_idx]} calculates the length of the current
note -- we take \texttt{T} as the fraction numerator and
\texttt{melody\_t[note\_idx]} as the denominator.

\begin{equation}
  \mbox{Note length (in microseconds)} = \frac{\mbox{T}}{\mbox{melody\_t[note\_idx]}}
\end{equation}

With all this, the code with two arrays looks quite awkward as we need to pay
attention that both arrays have the same sizes.

To solve this problem we can use two-dimensional arrays that we will discuss in
the next section.

But for those who champing at the bit, wanting to hear the melody from the
Arduino, we provided the full program code for the ``Twinkle, Twinkle, Little
Start'' that uses two arrays, in the appendix
\ref{app:twinkle-twinkle-little-star-02}.

\end{document}
