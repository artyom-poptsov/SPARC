\documentclass[../sparc.tex]{subfiles}
\graphicspath{{\subfix{../images/}}}
\begin{document}

%%%%%%%%%%%%%%%%%%%%%%%%%%%%%%%%%%%%%%%%%%%%%%%%%%%%%%%%%%%%%%%%%%%%%%%%%%%%%%%%
\section{Serial port}
\label{section:communication-serial-port}
\index{Electronics!Serial port}

%%%%%%%%%%%%%%%%%%%%%%%%%%%%%%%%%%%%%%%%%%%%%%%%%%%%%%%%%%%%%%%%%%%%%%%%%%%%%%%%
\subsection{General information}

\emph{Serial port} is one of the simplest ways of communication between digital
devices.  As it follows from its name the protocol transfers bits sequentially,
one bit at the time.  There are other hardware interfaces that also use serial
data transfer -- for example, USB and Ethernet.  But under the term ``serial
port'' most people mean the hardware that is compatible with the RS-232 and
other related standards (e.g. RS-485, RS-422 and others.)

A serial port usually implements the \emph{full-duplex} communication as it
allows to transfer data in both directions simultaneously.  For that two
independent lines are used: one for transferring the data (``Transmit'' or
``Tx'') and another for receiving the data (``Receive'' or ``Rx''.)

\end{document}
