\documentclass[../sparc.tex]{subfiles}
\graphicspath{{\subfix{../images/}}}
\begin{document}

%%%%%%%%%%%%%%%%%%%%%%%%%%%%%%%%%%%%%%%%%%%%%%%%%%%%%%%%%%%%%%%%%%%%%%%%%%%%%%%%
\section{Serial port}
\label{section:communication-serial-port}
\index{Electronics!Serial port}

%%%%%%%%%%%%%%%%%%%%%%%%%%%%%%%%%%%%%%%%%%%%%%%%%%%%%%%%%%%%%%%%%%%%%%%%%%%%%%%%
\subsection{General information}

\emph{Serial port} is one of the simplest ways of communication between digital
devices.  As it follows from its name the protocol transfers bits sequentially,
one bit at the time.  There are other hardware interfaces that also use serial
data transfer -- for example, USB and Ethernet.  But under the term ``serial
port'' most people mean the hardware that is compatible with the RS-232 and
other related standards (e.g. RS-485, RS-422 and others.)

A serial port usually implements the \emph{full-duplex} communication as it
allows to transfer data in both directions simultaneously.  For that two
independent lines are used: one for transferring the data (``Transmit'' or
``Tx'') and another for receiving the data (``Receive'' or ``Rx''.)

We may wonder why usually ``Receive'' is shortened to ``Rx'' and ``Transmit'' to
``Tx''. The reason for that can be found in the history: back in the times when
telegraph was in use it was easier to send a letter than to send a dot, thus
operators used ``x'' letter in the place of a dot.

The cost of telegraph usage was fixed and it included the salary of an operator,
the cost of printer usage and the cost of the telegraph line between the
stations.  The more data you could transfer, the more money you could earn.
That lead to the shear number of shorthands for the common words, especially
long ones.  Thus instead of the long word ``Transmission'' telegraph operators
preferred to use just ``T.'' (knowing that on the receiving site they will be
understood.)  But the dot symbol wasn't available in telegraph when the letter
input mode was active.  That forced operators to input ``T'' letter, then switch
telegraph to the numbers input mode (to input a dot symbol) and then switch back
to the letter input mode.  That took a long time.  Thus each time when it was
needed to send a dot, telegraphers used ``X'' symbol instead, that could be
written without switching to the number input mode.  Because very small number
of English words end with ``X'' this symbol turned out to be an ideal
replacement for a dot.

That the origin of such shorthands as ``Rx'' and ``Tx''.\autocite{so:krisw}

One drawback of the serial port is a lower data transfer speed comparing to the
\emph{parallel port}, where you can transfer eight bit simultaneously.  But to
transfer 8 bits once over the parallel port we have to have 8 wires connecting
the transmitter and the receiver, while the serial port only requires one wire
for data transfer.  Thus the serial data transfer bus is more compact.

%%%%%%%%%%%%%%%%%%%%%%%%%%%%%%%%%%%%%%%%%%%%%%%%%%%%%%%%%%%%%%%%%%%%%%%%%%%%%%%%
\subsection{Universal Asynchronous Receiver-Transmitter (UART)}
\label{section:communication-uart}

The universal asynchronous receiver-transmitter, widely known as \emph{UART} --
is a special device or a logical circuit for asynchronous serial data transfer.
In fact UART is one of the implementations of the serial port that is commonly
used for data exchange between devices.

On the transmitting side, UART receives data and sequentially transmits them as
separate bits over the bus, one after another, with equal periods of time.  On
the receiver the second UART assembles received bits into the original bytes.
Both the transmitter and the receiver has a special chip called \emph{shift
register} which provides a fundamental way to convert the data between the
serial and the parallel representation.

\end{document}
