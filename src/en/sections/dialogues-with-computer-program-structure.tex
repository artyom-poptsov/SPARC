\documentclass[../sparc.tex]{subfiles}
\graphicspath{{\subfix{../images/}}}
\begin{document}

%%%%%%%%%%%%%%%%%%%%%%%%%%%%%%%%%%%%%%%%%%%%%%%%%%%%%%%%%%%%%%%%%%%%%%%%%%%%%%%%
\section{The program structure for Arduino}

A program that is written for an Arduino board usually consists of two basic
parts, also called \emph{functions}: \texttt{setup} and \texttt{loop}.  An
example of a program that blinks an LED:

\begin{minted}{cpp}
  void setup() {
    pinMode(2, OUTPUT);
  }

  void loop() {
    digitalWrite(2, HIGH);
    delay(500);
    digitalWrite(2, LOW);
    delay(500);
  }
\end{minted}


The \texttt{setup} function performs the initialization of the micro-controller
after the power-up.  We should put here all the commands that must be executed
once after the system starts.

A digital port (or, as it sometimes called, \emph{digital pin}) of an Arduino
can be in one of the two states.  In the \emph{input} mode it reads the voltage
value form some external source, and in the \emph{output} mode it allows to
output voltage to some external circuit.

Let's look closely on the aforementioned example.  In the \texttt{setup}
function the \texttt{pinMode} function is executed, that configures the
operation mode of the specified port as \texttt{OUTPUT}.  The overall syntax of
\texttt{pinMode} call is as follows:

\begin{minted}{cpp}
  pinMode(port, mode);
\end{minted}

where \texttt{port} is the port number, and \texttt{mode} is the operation mode
for that port (either \texttt{INPUT}, \texttt{INPUT\_PULLUP} or
\texttt{OUTPUT}.)

In \texttt{loop} function we use two functions: \texttt{digitalWrite} and
\texttt{delay}.

The \texttt{digitalWrite} function has the following call syntax:

\begin{minted}{cpp}
  digitalWrite(port, value);
\end{minted}

where \texttt{port} is the port number and \texttt{value} is the output voltage
level: \texttt{HIGH} (usually corresponds to 5V) or \texttt{LOW} (outputs 0V.)

The \texttt{delay} function allows us to pause the program execution for the
specified period of time in milliseconds.  The call syntax for this function is
as follows:

\begin{minted}{cpp}
  delay(value);
\end{minted}

Where \texttt{value} is the time period in milliseconds.

\subsection{Exercises}
\begin{itemize}
\item Build a ``Chasing lights'' project on a breadboard to make the LEDs blink
  in turn, one after another.  Use at least 5 LEDs.
\item Modify the ``Chasing lights'' project to make the LEDs go in one
  direction, and then in another.
\end{itemize}

\end{document}
