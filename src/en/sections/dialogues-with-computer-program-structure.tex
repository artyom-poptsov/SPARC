\documentclass[../sparc.tex]{subfiles}
\graphicspath{{\subfix{../images/}}}
\begin{document}

%%%%%%%%%%%%%%%%%%%%%%%%%%%%%%%%%%%%%%%%%%%%%%%%%%%%%%%%%%%%%%%%%%%%%%%%%%%%%%%%
\section{The program structure for Arduino}

Our language for describing algorithms will be a general-purpose language called
C++.  This is one of the popular programming language (at the time of writing of
this book) and it has a bunch of interesting applications.  One of such
applications is the micro-controller programming.

A programming language provides for a programmer the basis for expressing ides,
while allows us to implement missing parts, thus expanding its toolkit.  In our
case the integrated development environment for developing programs for Arduino
(called Arduino IDE) includes the set of required libraries that makes it easier
to develop programs for the platform.

Aside from that, to make our job easier, Arduino IDE generates a stub for our
code when we are creating an empty project.  An Arduino program usually consists
of two basic parts, that also called \emph{functions}: \texttt{setup} and
\texttt{loop}.  The stub looks like the following:

\begin{minted}{cpp}
  void setup() {
    // put your setup code here, to run once:
  }

  void loop() {
    // put your main code here, to run repeatedly:
  }
\end{minted}

The lines starting with double slash (``//'') are considered by computer as
commentaries and ignored.  Usually we can add or remove those lines without
affecting the program logic, as the comments are for humans.

The \texttt{setup} procedure performs initialization of the micro-controller
when it starts.  Here we should put all the commands that must be done once on
the system start.

The \texttt{loop} procedure executes after the finishing of \texttt{setup}
execution and automatically re-started by the system as soon as it finishes.
Thus, \texttt{loop} is enclosed in some kind of internal loop as the function
title suggest -- until the micro-controller is powered off.

In the previous subsection we formulated an algorithm for blinking one LED.
Let's try to implement the algorithm for Arduino.

To control an LED we have to connect it to a \emph{digital port}.  A
\emph{digital port} (or \emph{pin}, for short) is called this way, because it
operates on a digital signal -- that is, it works only with two logical levels of
voltage, that represent ones and zeroes.

Also a digital port on Arduino can be configured in one of the two main
\emph{modes}:
\begin{enumerate}
\item In the \emph{input} mode (\texttt{INPUT}) a pin can read the voltage from
  some external source.
\item IN the \emph{output} mode (\texttt{OUTPUT}) a pin can output voltage to
  some external circuit.
\end{enumerate}

Thus to implement the blinking effect on an LED we have to configure the pin to
the \texttt{OUTPUT} mode first, for the pin to which we connected our LED.  It
is reasonable to do this in the \texttt{setup} procedure, as this configuration
can be done just once, at the system start.

Then in the \texttt{loop} procedure we can proceed to the implementation of the
LED blinking algorithm that we have formulated in the previous section.  With
that said, we have to replace our natural language description of the algorithm
steps with the commands that our computer can understand:

\begin{enumerate}
\item \texttt{digitalWrite(2, HIGH);} // Turn the LED on.
\item \texttt{delay(500);} // Wait for 500 ms.
\item \texttt{digitalWrite(2, LOW);} // Turn the LED off.
\item \texttt{delay(500);} // Wait for 500 ms.
\item Repeat the algorithm.
\end{enumerate}

One of the possible implementations is shown below:

\begin{minted}{cpp}
  void setup() {
    pinMode(2, OUTPUT);
  }

  void loop() {
    digitalWrite(2, HIGH);
    delay(500);
    digitalWrite(2, LOW);
    delay(500);
  }
\end{minted}

Let's look closely on the aforementioned example.  In the \texttt{setup}
function the \texttt{pinMode} function is executed, that configures the
operation mode of the specified port as \texttt{OUTPUT}.  The overall syntax of
\texttt{pinMode} call is as follows:

\begin{minted}{cpp}
  pinMode(port, mode);
\end{minted}

where \texttt{port} is the port number, and \texttt{mode} is the operation mode
for that port (either \texttt{INPUT}, \texttt{INPUT\_PULLUP} or
\texttt{OUTPUT}.)

In \texttt{loop} function we use two functions: \texttt{digitalWrite} and
\texttt{delay}.

The \texttt{digitalWrite} function has the following call syntax:

\begin{minted}{cpp}
  digitalWrite(port, value);
\end{minted}

where \texttt{port} is the port number and \texttt{value} is the output voltage
level: \texttt{HIGH} (usually corresponds to 5V) or \texttt{LOW} (outputs 0V.)

The \texttt{delay} function allows us to pause the program execution for the
specified period of time in milliseconds.  The call syntax for this function is
as follows:

\begin{minted}{cpp}
  delay(value);
\end{minted}

Where \texttt{value} is the time period in milliseconds.

\subsection{Exercises}
\begin{itemize}
\item Build a ``Chasing lights'' project on a breadboard to make the LEDs blink
  in turn, one after another.  Use at least 5 LEDs.
\item Modify the ``Chasing lights'' project to make the LEDs go in one
  direction, and then in another.
\end{itemize}

\end{document}
