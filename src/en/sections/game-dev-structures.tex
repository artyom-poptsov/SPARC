\documentclass[../sparc.tex]{subfiles}
\graphicspath{{\subfix{../images/}}}
\begin{document}

%%%%%%%%%%%%%%%%%%%%%%%%%%%%%%%%%%%%%%%%%%%%%%%%%%%%%%%%%%%%%%%%%%%%%%%%%%%%%%%%
\section{Structures}
\index{Programming!Structures}

When we are developing complex projects, such as computer games, it is important
to keep the source code structured.  As we saw before, some of the auxiliary
game procedures accept two or even more parameters.  Also now we are operating
with such complex entities as ``Player'', ``Game map'', ``Game object'' and so
on.

In such conditions, passing separate variables that describe some concrete
object to procedures does not seem like a good idea.  It would be better to
group, for example, player coordinates and the health points into a complex
object that we can pass to procedures for processing.

Fortunately we have a way to do it -- by the means of \emph{structures}.  A
structure is a complex type of a variable that allow to store several values
(possible of a different type) inside.  This is akin to a simple array but while
in an array all the elements belong to one type, in a structure we can use
several different types for fields.

%%%%%%%%%%%%%%%%%%%%%%%%%%%%%%%%%%%%%%%%%%%%%%%%%%%%%%%%%%%%%%%%%%%%%%%%%%%%%%%%
\subsection{Declaration of a structure}
\index{Programming!Structures!Declaration}

For example, a structure that describes a player may look like this:

\begin{listing}[H]
  \begin{minted}{cpp}
    struct player {
      int x;
      int y;
      char image;
      byte hp;
    };
  \end{minted}
  \caption{A description of a player as a structure.}
  \label{listing:game-dev-structure}
\end{listing}

Variables \mintinline{cpp}{x}, \mintinline{cpp}{y}, \mintinline{cpp}{image} and
\mintinline{cpp}{hp} are \emph{fields} of the structure -- as they describe the
\emph{properties} of an object.

We can declare a structure, for example, somewhere in the global scope, outside
the procedures of our program.

%%%%%%%%%%%%%%%%%%%%%%%%%%%%%%%%%%%%%%%%%%%%%%%%%%%%%%%%%%%%%%%%%%%%%%%%%%%%%%%%
\subsection{Creation and initialization of structure instances}
\index{Programming!Structures!Creation and initialization}

To create a player we have to make an \emph{instance} of the structure.

\begin{listing}[H]
  \begin{minted}{cpp}
    // Creation of "p" variable,
    // that holds an instance of "player" structure:
    struct player p;
  \end{minted}
  \caption{Creation of a structure instance.}
  \label{listing:game-dev-structure-instance}
\end{listing}

When we are creating variables we have to initialize them.  The same with
structures -- and we can perform initialization right after creating of a
structure instance:

\begin{listing}[H]
  \begin{minted}{cpp}
    // Creation of "p" variable,
    // that holds an instance of "player" structure:
    struct player p = {
      0,   // x
      0,   // y
      '@', // image
      100  // HP
    };
  \end{minted}
  \caption{Initialization of a structure instance.}
  \label{listing:game-dev-structure-instance-init}
\end{listing}

It's important to note that the initialization of the structure fields must be
done in the same order as the fields listed in the structure declaration.  The
comments written to the right side of each field in the listing
\ref{listing:game-dev-structure-instance-init} were added specifically so as not
to confuse the order of values.  That helps us to avoid situations like when the
health points of a player suddenly become the symbol code for the player image.

We can use the following syntax to solve the problem of getting the order of
values right:

\begin{listing}[H]
  \begin{minted}{cpp}
    // Creation of "p" variable,
    // that holds an instance of "player" structure:
    struct player p = {
      .x     = 0,
      .y     = 0,
      .image = '@',
      .hp    = 100
    };
  \end{minted}
  \caption{Structure initialization using field names.}
  \label{listing:game-dev-structure-instance-init-names}
\end{listing}

The way that is shown in the listing
\ref{listing:game-dev-structure-instance-init-names} allows us to set the values
for structure fields in an arbitrary order, by explicitly specifying the names
of the fields.  With that, we don't have to add comments to the values anymore,
as it follows from the field names what the value means.  Unfortunately this
approach is not supported on some platforms, namely on Arduino.

Another way of the initialization is to set the field values after a structure
instance is created.

\begin{listing}[H]
  \begin{minted}{cpp}
    // Creation of "p" variable,
    // that holds an instance of "player" structure:
    struct player p;

    void setup() {
      // Setting values for the structure fields.
      p.x = 0;
      p.y = 0;
      p.hp = 100;
      p.image = '@';
    }
  \end{minted}
  \caption{Setting the structure values after the creation of the structure.}
  \label{listing:game-dev-structure-assignment}
\end{listing}

To access the structure fields we can use syntax where the structure field name
is written after the variable that holds the structure instance, separated with
a dot.  As in the case of regular variables, we can assign values to the
structure fields and substitute the field values in expressions.

\end{document}
