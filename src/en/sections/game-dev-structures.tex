\documentclass[../sparc.tex]{subfiles}
\graphicspath{{\subfix{../images/}}}
\begin{document}

%%%%%%%%%%%%%%%%%%%%%%%%%%%%%%%%%%%%%%%%%%%%%%%%%%%%%%%%%%%%%%%%%%%%%%%%%%%%%%%%
\section{Structures}
\index{Programming!Structures}

When we are developing complex projects, such as computer games, it is important
to keep the source code structured.  As we saw before, some of the auxiliary
game procedures accept two or even more parameters.  Also now we are operating
with such complex entities as ``Player'', ``Game map'', ``Game object'' and so
on.

In such conditions, passing separate variables that describe some concrete
object to procedures does not seem like a good idea.  It would be better to
group, for example, player coordinates and the health points into a complex
object that we can pass to procedures for processing.

Fortunately we have a way to do it -- by the means of \emph{structures}.  A
structure is a complex type of a variable that allow to store several values
(possible of a different type) inside.  This is akin to a simple array but while
in an array all the elements belong to one type, in a structure we can use
several different types for fields.

%%%%%%%%%%%%%%%%%%%%%%%%%%%%%%%%%%%%%%%%%%%%%%%%%%%%%%%%%%%%%%%%%%%%%%%%%%%%%%%%
\subsection{Declaration of a structure}
\index{Programming!Structures!Declaration}

\end{document}
