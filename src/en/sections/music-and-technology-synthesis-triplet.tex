\documentclass[../sparc.tex]{subfiles}
\graphicspath{{\subfix{../images/}}}
\begin{document}

%%%%%%%%%%%%%%%%%%%%%%%%%%%%%%%%%%%%%%%%%%%%%%%%%%%%%%%%%%%%%%%%%%%%%%%%%%%%%%%%
\section{Irrational rhythms}

Sometimes there's a necessity in the music to create special, ``irrational''
division of beats.  The most common form of such rhythms are \emph{triplets}.

A \emph{triplet} is a grouping of three notes that takes the same length as two
of such notes.  Thus, each of the notes inside a triplet is getting squeezed to
free the enough space for the others.  We can say that a triplet makes the
effect that is inverse to the dotted notes that we discussed in the previous
section \ref{section:dotted-notes}.

\figureMusicTripletExample{en}

Let's assume, for example, that we have an excerpt from ``Hey You'' song by Pink
Floyd, as is shown on the fig. \ref{fig:music-triplet-example}.  Here we can see
that three first notes has the length of $\frac{1}{4}$, while they are grouped
in a triplet with special line, marked with number ``3''.  The length of a bar
in the composition is $\frac{4}{4}$ -- which equals to 1.  But if we to calculate
all the note length in the bar without getting the triplet into account, we will
get the value greater than 1:

\begin{equation}
  \mbox{Bar length} = \frac{1}{4} + \frac{1}{4}
  + \frac{1}{4} + \frac{1}{8} + \frac{1}{4} + \frac{1}{8} = \frac{5}{4}
\end{equation}

\end{document}
