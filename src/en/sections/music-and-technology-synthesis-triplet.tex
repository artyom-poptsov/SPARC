\documentclass[../sparc.tex]{subfiles}
\graphicspath{{\subfix{../images/}}}
\begin{document}

%%%%%%%%%%%%%%%%%%%%%%%%%%%%%%%%%%%%%%%%%%%%%%%%%%%%%%%%%%%%%%%%%%%%%%%%%%%%%%%%
\section{Irrational rhythms}

Sometimes there's a necessity in the music to create special, ``irrational''
division of beats.  The most common form of such rhythms are \emph{triplets}.

A \emph{triplet} is a grouping of three notes that takes the same length as two
of such notes.  Thus, each of the notes inside a triplet is getting squeezed to
free the enough space for the others.  We can say that a triplet makes the
effect that is inverse to the dotted notes that we discussed in the previous
section \ref{section:dotted-notes}.

\figureMusicTripletExample{en}

Let's assume, for example, that we have an excerpt from ``Hey You'' song by Pink
Floyd, as is shown on the fig. \ref{fig:music-triplet-example}.  Here we can see
that three first notes has the length of $\frac{1}{4}$, while they are grouped
in a triplet with special line, marked with number ``3''.  The length of a bar
in the composition is $\frac{4}{4}$ -- which equals to 1.  But if we to calculate
all the note length in the bar without getting the triplet into account, we will
get the value greater than 1:

\begin{equation}
  \mbox{Bar length} = \frac{1}{4} + \frac{1}{4}
  + \frac{1}{4} + \frac{1}{8} + \frac{1}{4} + \frac{1}{8} = \frac{5}{4}
\end{equation}

As we can see, we managed to get $\frac{5}{4}$ which is $\frac{1}{4}$ less than
one.

To play a triplet properly we have to shorten the $\frac{3}{4}$ grouped by it so
they will fit into $\frac{2}{4}$ by their lengths (which is equivalent to
$\frac{1}{2}$.)

Mathematically we can achieve it by dividing the total note lengths by the
length of the triplet:

\begin{equation}
  \mbox{Multiplier} = \frac{3}{4} : \frac{2}{4} = \frac{3}{4} * \frac{4}{2}
  = \frac{3}{2} = 1.5
\end{equation}

To divide fractions we can them instead multiply the first fraction by the
second, while inverting the second fraction.  That will allow us to shorten the
fraction, by removing 4 from the nominator and denominator.  As the result, we
end up with just $\frac{3}{2}$.  When we divide 3 by 2 we will get 1.5 -- and
that is our multiplier, that we have to apply to the denominator of the notes
length inside the triplet.

\begin{equation}
  \mbox{The length of a note} = \frac{1}{4 * 1.5} = \frac{1}{6}
\end{equation}

\end{document}
