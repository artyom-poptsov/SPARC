\documentclass[../sparc.tex]{subfiles}
\graphicspath{{\subfix{../images/}}}
\begin{document}

%%%%%%%%%%%%%%%%%%%%%%%%%%%%%%%%%%%%%%%%%%%%%%%%%%%%%%%%%%%%%%%%%%%%%%%%%%%%%%%%
\section{Voltage}
\index{Electronics!Voltage}

Let's imagine that we have some vessel filled with water (see figure
\ref{fig:electronics-current-0}.)

\figureElectronicsVoltage{en}

The water in the vessel has some \emph{potential energy} that can be spent to
achieve some goal.  For example, if we make a hole in the bottom of the vessel
the water will be pouring down (as can be seen on
fig. \ref{fig:electronics-current-1}); if we put a waterwheel under the pouring
water we can use it as an engine for some mechanism.

\figureElectronicsVoltageWithCurrent{en}

Drawing an analogy with the electricity we can say that our vessel has some
\emph{pressure} of water; in case of a charged electric battery we have some
``pressure'' of \emph{electrons}.  In case of the battery the energy is kept in
the form of \emph{chemical energy}.

The pressure of water is measured in \emph{Pascals} (Pa.)  The ``pressure'' of
electrons is measured in \emph{Volts} (V).

Thus we can draw our first conclusion: to make the electric current flow through
a circuit we have to have some current source that has some Voltage.

\end{document}
