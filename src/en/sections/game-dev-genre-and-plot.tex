\documentclass[../sparc.tex]{subfiles}
\graphicspath{{\subfix{../images/}}}
\begin{document}

%%%%%%%%%%%%%%%%%%%%%%%%%%%%%%%%%%%%%%%%%%%%%%%%%%%%%%%%%%%%%%%%%%%%%%%%%%%%%%%%
\section{Choosing the genre and the plot}
\index{Game development!Genre and plot}

Game development, as is the case with other serious projects, usually starts
with design -- in our case, it is the stage where we're choosing the genre of the
game and plotting a plot.

In terms of genre we have very wide range of options: there are role-playing
games (RPG), first-person shooters (FPS), platformers, real-time strategies
(RTS) and so on.

Preference for some genres instead of others is very subjective thing, so let's
make it clear right away that all our decisions made in this book shouldn't be
considered as the only possible way to go.  We can use the skills and knowledge
gained through this practice for developing games of other genres.

Let's choose for our current game the ``platformer'' genre, as it is the genre
of many old games.

One of the old games that we can select as the source of our inspiration is 1991
game ``Dangerous Dave in the Haunted Mansion'' developed by ``id Software''
company.

The plot of our game will be set in an old mansion which is full of aggressive
monsters.  Our task in the game will be to go through a set of levels to a door
that will let us exit the building.

\end{document}
