\documentclass[../sparc.tex]{subfiles}
\graphicspath{{\subfix{../images/}}}
\begin{document}

%%%%%%%%%%%%%%%%%%%%%%%%%%%%%%%%%%%%%%%%%%%%%%%%%%%%%%%%%%%%%%%%%%%%%%%%%%%%%%%%
\section{Arduino Platform}

\newglossaryentry{CNC}{name=CNC, description={Computer Numerical Control}}
\newglossaryentry{MCU}{name=MCU, description={Micro-Controller Unit}}

A micro-controller (or \gls{MCU}) is a simple and often comparatively cheap
embedded computer that usually solves one task.  MCUs control modern domestic
devices, children toys, musical instruments and industrial robots; upon MCUs are
built 3D-printers and many other types of \gls{CNC}.

We will be working with Arduino MCU platform that provides convenient
programming interface.

Most of the Arduino platforms are built on AVR micro-controllers.

Here are some of the popular Arduino platforms:
\begin{itemize}
\item \textbf{Arduino Nano} -- One of the cheapest and most accessible Arduino versions.  It has small size and modest number of input/output ports.
\item \textbf{Arduino Uno} -- Popular version of the platform that allows to
  connect to it extension boards (or so called \emph{shields}) directly (we will
  talk about shields later.)
\item \textbf{Arduino Mega 2560} -- Big board that is based on the MCU ATmega
  2560.  The board has more than 50 digital input/output ports plus 16 analog
  ports.  Arduino Mega 2560 can be used as the basis for CNC (such as
  3D-printers.)
\item \textbf{Arduino Leonardo} -- This board is notable because it can
  ``pretend'' to be a USB-device -- or as programmers say, this board can
  \emph{emulate} it.
\end{itemize}

\section{Extension Boards (Shields)}

Many Arduino versions allow to connect to them not only standalone sensors but
complete extension boards (also known as \emph{shields}.)  Such shields are
connected to the top of an Arduino board.  Moreover often you can connect
another shield on top of the already connected -- it's like a layered pancake
with Arduino on the bottom.  Shields provide different functions starting from
extending Arduino with Wi-Fi connectivity to the controlling full-fledged
\gls{CNC} (such as a 3D-printer or a laser cutter.)

\end{document}
