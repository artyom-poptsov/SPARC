\documentclass[../sparc.tex]{subfiles}
\graphicspath{{\subfix{../images/}}}
\begin{document}

%%%%%%%%%%%%%%%%%%%%%%%%%%%%%%%%%%%%%%%%%%%%%%%%%%%%%%%%%%%%%%%%%%%%%%%%%%%%%%%%
\section{Electric Circuits}
\index{Electronics!Electric Circuits}

\newglossaryentry{LED}{name=LED, description={Light-Emitting Diode}}

Now let's imagine two vessels, one of which holds ten liters of water and the
second one holds zero liters (or, as people usually say, it is \emph{empty}), as
shown on the fig. \ref{fig:electronics-circuits-0}.

\figureElectronicsWaterVessels{en}

If we connect vessel ``A'' with vessel ``B'' the water will flow from ``A'' to
``B'' due to the laws of physics (as shown in fig.
\ref{fig:electronics-circuits-1}.)

\figureElectronicsConnectedWaterVessels{en}

When we connected vessels ``A'' and ``B'' with a pipe we've got a \emph{closed
circuit} that allows the flow of water.  Electrical circuits work in the similar
way, but instead of the water flow, they conduct the flow of elementary charged
particles.

Thus here goes the second rule that we need to learn: electrical current is
possible only in closed circuits.

Electrical current is measured in \emph{Amperes} (A).

Figure \ref{fig:electronics-simple-circuit} shows an electrical circuit that is
analogous to the one pictured on fig. \ref{fig:electronics-circuits-1}.

\figureElectronicsSimpleCircuit{en}

As can be seen on fig. \ref{fig:electronics-simple-circuit} a light-emitting
diode (LED) is used as the functional component.  LEDs are widely used in modern
devices and have variety of different forms and sizes.  But sooner or later you
will likely find LEDs that look like as shown on fig. \ref{fig:electronics-led}.

\figureElectronicsLED{en}

Simple single-color LEDs commonly have two legs: one is anode (which is
shortened to ``A'') and the second one is cathode (``C''.)  New LEDs usually
have legs of different lengths: the long one is anode and the short one is
cathode.

\end{document}
