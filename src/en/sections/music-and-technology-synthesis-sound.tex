\documentclass[../sparc.tex]{subfiles}
\graphicspath{{\subfix{../images/}}}
\begin{document}

\section{Sound}
\index{Music!Sound frequency}
\newglossaryentry{Hz}{name=Hz, description={Hertz}}
\newglossaryentry{kHz}{name=kHz, description={Kilohertz ($10^3$ Hz)}}
\newglossaryentry{MHz}{name=MHz, description={Megahertz ($10^6$ Hz)}}
\newglossaryentry{GHz}{name=GHz, description={Gigahertz ($10^9$ Hz)}}

As it is commonly known, \emph{sound} is a mechanical disturbance from a state
of equilibrium that propagates through some medium (such as the air) in a form
of an acoustic wave. \cite{britannica:sound}

As a wave it one of the main parameters of the sound is its \emph{frequency}.

Frequencies are measured in \emph{Hertz} (\gls{Hz}), and one Hertz (1 Hz) means
one oscillation per second.  10 oscillations per seconds gives us 10 Hz, 100
oscillations per second gives us 100 Hz and so on.  When we are talking about
1000 Hz or more, it is more convenient to use standard multiples of a Hertz:
``Kilo-'' (\gls{kHz}), ``Mega-'' (\gls{MHz}) and ``Giga-'' (\gls{GHz}).

Some of the common multiples of a Hertz are shown in the table below.

\begin{table}[h]
  \centering
  \begin{tabular}{p{3cm}|p{4cm}|p{3.5cm}}
    Measurement unit & Relation to Hertz & Examples \\
    \hline \hline
    Hertz (Hz)
    & $ 1 \mbox{Hz} $ or $ 10^0 \mbox{Hz} $
    & $ 100 * 10^0 \mbox{Hz} = 100 \mbox{Hz} $ \\
    \hline
    Kilohertz (kHz)
    & $ 1000 \mbox{Hz} $ or $ 10^3 \mbox{Hz} $
    & $ 100 * 10^3 \mbox{Hz} = 100 \mbox{kHz} $ \\
    \hline
    Megahertz (MHz)
    & $ 1000000 \mbox{Hz} $ или $ 10^6 \mbox{Hz} $
    & $ 100 * 10^6 \mbox{Hz} = 100 \mbox{MHz} $ \\
    \hline
    Gigahertz (GHz)
    & $ 1000000000 \mbox{Hz} $ или $ 10^9 \mbox{Hz} $
    & $ 100 * 10^9 \mbox{Hz} = 100 \mbox{GHz} $ \\
  \end{tabular}
  \caption{Some units of the frequency measurement units.}
  \label{table:sound-hertz-scale}
\end{table}

Thus we need to know the sound frequency to an acoustic wave with proper wave
length to make that sound.  And vice versa, when we know the wave length, we can
calculate the sound frequency from it.  This can be visualized as is shown on
fig. \ref{fig:sound-fig-1}.

\begin{figure}[h]
  \centering
  \includegraphics[width=10cm]{sound-fig-1}
  \caption{A visual representation of an acoustic wave with the frequency of 5
    Hz.}
  \label{fig:sound-fig-1}
\end{figure}

\end{document}
