\documentclass[../sparc.tex]{subfiles}
\graphicspath{{\subfix{../images/}}}
\begin{document}

%%%%%%%%%%%%%%%%%%%%%%%%%%%%%%%%%%%%%%%%%%%%%%%%%%%%%%%%%%%%%%%%%%%%%%%%%%%%%%%%
\newpage
\section{Rhythm theory}
\index{Music!Rhythm}

Our way to the music starts from the discussion of the \emph{rhythm theory}.
Most of us know that the rhythm is -- in many people it produces the reflex of
head nodding to the rhythm, or mechanical finger tapping on a tabletop to the
rhythm beats.

To understand how a rhythm is created we need to know some basic things about
the rhythm theory which will discuss below.

%%%%%%%%%%%%%%%%%%%%%%%%%%%%%%%%%%%%%%%%%%%%%%%%%%%%%%%%%%%%%%%%%%%%%%%%%%%%%%%%
\subsection{The \emph{musical bar}}
\index{Music!Rhythm!Scale}

Let's start with a musical joke: \footnote{A variation of the joke from
\url{https://www.classicfm.com/discover-music/humour/long-bar-notes-joke/}}

\begin{quotation}
C, E-flat and G walk into a bar.  The bartender says, ``Sorry, but we don’t
serve minors.''
\end{quotation}

Any musical composition is divided into time measures also known as \emph{bars}
-- usually of the same length. \footnote{Music has great variety and compositors
come up with the new tricks all the time that allow them to make the desired
impression on the listeners.  So here we are discussing the topic with some
assumptions.}

On the fig. \ref{fig:music-six-bar} we can see how could a musical composition
consisting of six bars look.

\begin{tikzpicture}
  \draw[thick, ->] (0, 0.5) -- (12, 0.5) node[anchor=north west] {t};
  \foreach \x/\n in {0/1, 2/2, 4/3, 6/4, 8/5, 10/6} {
    \draw (\x, 0) -- (\x, 1) -- (\x, 1) node[midway, above] {\n};
  };
  \label{fig:music-six-bar}
\end{tikzpicture}

The length of one bar in time is determined by the tempo of the rhythm, and we
will discuss this later.  For now, we can imagine that one one bar always takes
one abstract unit of time.  We can substitute this unit of time with any
convenient time duration -- such as one second.

\end{document}
