\documentclass[../sparc.tex]{subfiles}
\graphicspath{{\subfix{../images/}}}
\begin{document}

%%%%%%%%%%%%%%%%%%%%%%%%%%%%%%%%%%%%%%%%%%%%%%%%%%%%%%%%%%%%%%%%%%%%%%%%%%%%%%%%
\section{Implementing a game map}
\label{section:game-map}
\index{Game development!Game map}
\index{Programming!Array!Two-dimensional array}

Usually some action takes place around the player in a game -- non-playable
characters (NPC) jealously patrol the map territory; some objects, that player
can interact with, appear and disappear; and finally everywhere lie various
\emph{static} objects that can block the path for the player or can be just a
element of the decoration.

A simple way to place an object on the map is to specify its coordinates on the
map and draw the object there -- the same way as we do it with the game
character.  But if we would create two separate variables for each object (for
storing its coordinates on X and Y axis), the number of the variables would grow
very fast.  We can imagine that a map of 20x4 in size can potentially hold up to
80 objects and it gives us 160 variables for addressing each of them!  To solve
this problem we have to resort to already familiar for us arrays, with just one
caveat -- we have to use \emph{two-dimensional arrays}.

A schematic representation of a two-dimensional array is shown on the
fig. \ref{fig:2d-array-example}.

\figureArray{en}

We can also imagine a two-dimensional array of 4x20 as an one-dimensional array
consisting of 4 elements, where each of which holds a reference to another
one-dimensional array of 20 cells in length, as is shown on the
fig. \ref{fig:2d-array-example-with-references}.

\figureArrayOfPointers{en}

\end{document}
