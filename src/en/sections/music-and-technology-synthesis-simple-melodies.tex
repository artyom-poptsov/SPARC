\documentclass[../sparc.tex]{subfiles}
\graphicspath{{\subfix{../images/}}}
\begin{document}

%%%%%%%%%%%%%%%%%%%%%%%%%%%%%%%%%%%%%%%%%%%%%%%%%%%%%%%%%%%%%%%%%%%%%%%%%%%%%%%%
\section{Programming of simple melodies}

To program a melody, we have to figure out the notes that compose the melody,
and their order.  Usually this information is written down in the form of a
musical sheet -- but if we don't know how to read this notation yet, very often
we can find a melody written in a simplified form that is using the scientific
notation.  Let's take a melody ``Twinkle, Twinkle, Little
Star'' \footnote{\url{https://ru.wikipedia.org/wiki/Twinkle,_Twinkle,_Little_Star}}, the English lullaby, as the example.

\figureMusicTwinkleTwinkleLittleStar{en}

We already know from the table \ref{table:music-notes-legths} how to distinguish
musical ``squiggles'' to understand their lengths.  But we are yet to discuss
how to read the frequencies form this notation.  Thus the full list of notes
from the melody is shown in the table
\ref{table:twinkle-twinkle-little-star-notes} in a more readable form.  Each row
of the table represents one bar of the melody.

\tableMusicTwinkleTwinkleLittleStarNotes{en}

Using our knowledge of the musical ``squiggles'', let's try to set the length
for each note by adding it into the brackets next to the note.

\tableMusicTwinkleTwinkleLittleStar{en}

To program this melody it is convenient to define all the required notes at the
beginning of the program as constants.  Each constant will store the frequency
of a sound for that note.

The names of the constants are usually written in capital letters, so it would
be more appropriate to write those constants as ``C4'', ``D4'', ``E4'' and so
on.  But we are using regular letters here to avoid naming conflicts with the
constants that are already defined in the Arduino (e.g. ``A4''.)

Currently we need only the notes from the fourth octave.  It is reasonable to
define those constants before \texttt{setup} procedure.

\begin{minted}{cpp}
  const float c4 = 261.630;
  const float d4 = 293.660;
  const float e4 = 329.630;
  const float f4 = 349.230;
  const float g4 = 392.000;
  const float a4 = 440.000;
  const float b4 = 493.880;
\end{minted}

We should not forget that we have to define a constant for the digital port to
which our speaker is connected.

\begin{minted}{cpp}
  // The digital port number to which
  // the speaker is connected.
  const int SPEAKER_PIN = 8;
\end{minted}

Also we have to know the melody tempo -- that is, \gls{BPM}.  For ``Twinkle,
Twinkle, Little Star'' this parameter is equal to 120 BPM.

As soon as we defined all the required constants and determined the BPM value,
programming our melody will be easy.  Let's write down the program code for our
\texttt{loop} procedure:

\begin{minted}{cpp}
  void loop() {
    const long BPM = 120;
    const long MINUTE = 60 * 1000000;
    const long T = (MINUTE / BPM) * 4;

    // Zeroth bar
    play_tone(SPEAKER_PIN, c4, T / 4);
    delay(100);
    play_tone(SPEAKER_PIN, c4, T / 4);
    delay(100);
    play_tone(SPEAKER_PIN, g4, T / 4);
    delay(100);
    play_tone(SPEAKER_PIN, g4, T / 4);
    delay(100);

    // First bar
    play_tone(SPEAKER_PIN, a4, T / 4);
    delay(100);
    play_tone(SPEAKER_PIN, a4, T / 4);
    delay(100);
    play_tone(SPEAKER_PIN, g4, T / 2);
    delay(100);

    // And so on.
  }
\end{minted}

Now we have add missing parts of the code (e.g. \texttt{setup},
\texttt{play\_tone} and so on) and program the rest of the bars.  After we
finish we can upload our program to the Arduino and test the results.  If we did
everything right, we should hear the melody from the speaker.

For those who can't wait to test the melody, we have provided the full code for
the program in the appendix \ref{app:twinkle-twinkle-little-star-01}.

Note that this solution is sub-optimal in terms of the amount of code we had to
write.  To fix this we will have to use \emph{arrays}.

\end{document}
