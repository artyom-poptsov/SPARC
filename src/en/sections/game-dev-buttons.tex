\documentclass[../sparc.tex]{subfiles}
\graphicspath{{\subfix{../images/}}}
\begin{document}

%%%%%%%%%%%%%%%%%%%%%%%%%%%%%%%%%%%%%%%%%%%%%%%%%%%%%%%%%%%%%%%%%%%%%%%%%%%%%%%%
\section{Implementing the controls}
\index{Game development!Controls}
\label{section:game-dev-controls}

To control the game character, we must have some device to input information
into a computer (or in our case, a micro-controller.)  We will implement such
device using several buttons connected through a breadboard to Arduino.

%%%%%%%%%%%%%%%%%%%%%%%%%%%%%%%%%%%%%%%%%%%%%%%%%%%%%%%%%%%%%%%%%%%%%%%%%%%%%%%%
\subsection{Button connection}
\index{Game development!Button connection}

Let's try to connect a button to an Arduino board.  A button is but a contractor
between two contacts.  The simplest button replacement is two disconnected wires
(an open electric circuit) that we can connect together (close the circuit), or
open the circuit again.  Of course this method of control is inconvenient (and
even dangerous, when high voltages and currents are used in an electric
circuit), and that's why buttons usually implemented as some enclosed devices
that provide a way to connect two wires without the need to grab and connect the
wire ends with bare hands.

Buttons differ in terms of working principles.  The simplest type of buttons are
\emph{push-buttons}.

They are called \emph{push-buttons} because they don't ``remember'' their state,
and after they pushed they return the default (usually ``open'') state.

\end{document}
