\documentclass[../sparc.tex]{subfiles}
\graphicspath{{\subfix{../images/}}}
\begin{document}

%%%%%%%%%%%%%%%%%%%%%%%%%%%%%%%%%%%%%%%%%%%%%%%%%%%%%%%%%%%%%%%%%%%%%%%%%%%%%%%%
\newpage
\subsection{Two-dimensional arrays}
\index{Programming!Array!Two-dimensional array}

Wouldn't it be cool to store the notes of our composition along with their
lengths in one array?  Luckily we have an option to do that: using a
\emph{two-dimensional array}.

Schematic representation of a two-dimensional array is shown as the table
\ref{table:array-example-2}.

\tableTwoDimensionalArray{en}

Each row of the array stores a description of one note.  The zeroth column holds
the note frequency and the first column holds its length as the fraction
denominator, where the numerator is the length of the bar.  For example, the
zeroth note (``C4'') has the length of $\frac{1}{4}$, so the in the array the
length is specified as 4.

In the program code we can write down this table as the following array:

\begin{minted}{cpp}
  float melody[28][2] = {
    {c4, 4}, {c4, 4}, {g4, 4}, {g4, 4},
    {a4, 4}, {a4, 4}, {g4, 2},
    {f4, 4}, {f4, 4}, {e4, 4}, {e4, 4},
    {d4, 4}, {d4, 4}, {c4, 2},
    {g4, 4}, {g4, 4}, {f4, 4}, {f4, 4},
    {e4, 4}, {e4, 4}, {d4, 2},
    {g4, 4}, {g4, 4}, {f4, 4}, {f4, 4},
    {e4, 4}, {e4, 4}, {d4, 2},
  };
\end{minted}

As can be seen from this example, each element of our array in fact just an
one-dimensional array, surrounded with curly brackets.  For example, the element
with the index 0 of our array stores an one-dimensional array \texttt{\{c4,
 4\}}, that in turn holds the note (``c4'') and its length (4.)

Now we can adapt the code that plays the melody for using the two-dimensional
array:

\begin{minted}{cpp}
  // ...

  void loop() {
    const long BPM = 120;
    const long MINUTE = 60000000;
    const long T = (MINUTE / BPM) * 4;

    for (int note_idx = 0; note_idx < 28; note_idx++) {
      play_tone(SPEAKER_PIN,
                melody[note_idx][0],
                T / melody[note_idx][1]);
      delay(100);
    }
  }
\end{minted}

Using a two-dimensional array, we can briefly and succinctly describe even more
complicated melodies than ``Twinkle, Twinkle, Little Star''.

At this stage we have to discuss how the \emph{musical staff}, that holds the
music notes, works.  We need this to be able to read the notes on our own.

%%%%%%%%%%%%%%%%%%%%%%%%%%%%%%%%%%%%%%%%%%%%%%%%%%%%%%%%%%%%%%%%%%%%%%%%%%%%%%%%
\subsection{Getting the size of an array}
\index{Programming!Array!Getting the size of an array}

Our implementation of the melody has a very important drawback: we are
calculating the number of the notes in the melody ourselves and setting the size
of an array at the time of its declaration.  For example, the melody we was
discussing in the previous section, has 28 notes.  The more notes the melody
has, the more tedious becomes the task of the array size calculation.

In fact, we can skip the number in the first pair of square brackets in the
array declaration -- as we are setting the array elements right at the time of
its initialization, the c computer can calculate the amount of memory it
requires on its own.  Thus we can declare the array as follows:

\begin{minted}{cpp}
  float melody[][2] = {
    {c4, 4}, {c4, 4}, {g4, 4}, {g4, 4},
    {a4, 4}, {a4, 4}, {g4, 2},
    {f4, 4}, {f4, 4}, {e4, 4}, {e4, 4},
    {d4, 4}, {d4, 4}, {c4, 2},
    {g4, 4}, {g4, 4}, {f4, 4}, {f4, 4},
    {e4, 4}, {e4, 4}, {d4, 2},
    {g4, 4}, {g4, 4}, {f4, 4}, {f4, 4},
    {e4, 4}, {e4, 4}, {d4, 2},
  };
\end{minted}

As we can see, the number of rows in the array is not specified.

Unfortunately that does not help us to avoid the task of notes calculation as we
have to use this value to enumerate the notes in the array:

\begin{minted}{cpp}
  void loop() {
    const long BPM = 120;
    const long MINUTE = 60000000;
    const long T = (MINUTE / BPM) * 4;

    for (int note_idx = 0; note_idx < 28; note_idx++) {
      play_tone(SPEAKER_PIN,
                melody[note_idx][0],
                T / melody[note_idx][1]);
      delay(100);
    }
  }
\end{minted}

But we can put this task on the shoulders of a computer as well -- we can do this
by using the \texttt{sizeof} procedure.  The procedure returns the number of
bytes that an object, passed as an argument, occupies in the memory.

To calculate the number of the elements in an array, we have to get its size (in
bytes) using \texttt{sizeof} and then divide this value by the size of the first
element of the array.  We will be using the zeroth element of an array for this
purpose:

\begin{minted}{cpp}
  int n = sizeof(melody) / sizeof(melody[0]);
\end{minted}

Then we can use the resulting value in a loop:

\begin{minted}{cpp}
  void loop() {
    const long BPM = 120;
    const long MINUTE = 60000000;
    const long T = (MINUTE / BPM) * 4;

    int n = sizeof(melody) / sizeof(melody[0]);
    for (int note_idx = 0; note_idx < n; note_idx++) {
      play_tone(SPEAKER_PIN,
                melody[note_idx][0],
                T / melody[note_idx][1]);
      delay(100);
    }
  }
\end{minted}

Now the computer will calculate the number of notes automatically and that frees
us from manual calculations, so we can focus on the melodies programming.

\end{document}
