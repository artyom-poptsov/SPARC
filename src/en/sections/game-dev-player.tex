\documentclass[../sparc.tex]{subfiles}
\graphicspath{{\subfix{../images/}}}
\begin{document}

%%%%%%%%%%%%%%%%%%%%%%%%%%%%%%%%%%%%%%%%%%%%%%%%%%%%%%%%%%%%%%%%%%%%%%%%%%%%%%%%
\section{Displaying the player}
\label{section:player}
\index{Game development!Player}

In most of the computer games there is some character which represents us in the
game as some form of an avatar, which can be controlled by us.

First of all we can choose some symbol to show our player -- for example, we can
use ``@'' for now.

Let's set the ``image'' of our player as some character constant in our code,
somewhere at the beginning of our program:

\begin{listing}[ht]
  \begin{minted}{cpp}
    const char PLAYER = '@';
  \end{minted}
  \caption{Declaring a constant that stores the image of our player.}
  \label{listing:game-dev-player-image}
\end{listing}

Then we have to set the position of our player in the two-dimensional space of
our game ``map'':

\begin{listing}[ht]
  \begin{minted}{cpp}
    int player_x = 0;
    int player_y = 0;
  \end{minted}
  \caption{Declaring variables that store the position of our player on a game
    map.}
  \label{listing:game-dev-player-position}
\end{listing}

After that we can display our player on the screen.  The full source code of our
program may look as is shown below.

\begin{listing}[ht]
  \begin{minted}{cpp}
    #include <LiquidCrystal_I2C.h>

    LiquidCrystal_I2C lcd(0x27,  16, 2);

    const char PLAYER = '@';

    int player_x = 0;
    int player_y = 0;

    void setup() {
      lcd.init();
      lcd.backlight();
    }

    void loop() {
      lcd.setCursor(player_x, player_y);
      lcd.print(PLAYER);
    }
  \end{minted}
  \caption{A code example that shows a player on the LCD.}
  \label{listing:game-dev-player-example}
\end{listing}

After uploading this program to an Arduino, on the zeroth line and zeroth column
the player image (``@'') should appear.

As we declared the player position as variables, we can move the player across
the screen by changing those variables.  To move a player in games usually
buttons and/or a joystick is used.  In the next section we will discuss how to
connect buttons to an Arduino and how to handle button presses.

\end{document}
