\documentclass[../sparc.tex]{subfiles}
\graphicspath{{\subfix{../images/}}}
\begin{document}

%%%%%%%%%%%%%%%%%%%%%%%%%%%%%%%%%%%%%%%%%%%%%%%%%%%%%%%%%%%%%%%%%%%%%%%%%%%%%%%%
\newpage
\chapter{Melody "Twinkle Twinkle Little Star" (version 2.0)}
\label{app:twinkle-twinkle-little-star-02}

In this example we provided the full example of ``Twinkle Twinkle, Little Star''
that is written with one-dimensional arrays.  If we compare this code with the
code from the appendix \ref{app:twinkle-twinkle-little-star-01} we can see that
the code became shorter; but we can make it even shorter yet, and even make it
simpler -- by using a two-dimensional array.

\begin{minted}{cpp}
  // Notes:
  const float c4 = 261.630;
  const float d4 = 293.660;
  const float e4 = 329.630;
  const float f4 = 349.230;
  const float g4 = 392.000;
  const float a4 = 440.000;
  const float b4 = 493.880;

  // The port number where the speaker
  // is connected:
  const int SPEAKER_PIN = 8;

  // The procedure to play a sound T microsecond
  // in length, with the frequency F on the PORT:
  void play_tone(int port, float f, long t) {
    const long T = 1000000 / f;
    long d = T / 2;
    int count = t / T;
    for (int i = 0; i < count; i++) {
      digitalWrite(port, HIGH);
      delayMicroseconds(d);
      digitalWrite(port, LOW);
      delayMicroseconds(d);
    }
  }

  void setup() {
    pinMode(SPEAKER_PIN, OUTPUT);
  }

  // The array with notes.
  float melody[28] = {
    c4, c4, g4, g4,
    a4, a4, g4,
    f4, f4, e4, e4,
    d4, d4, c4,
    g4, g4, f4, f4,
    e4, e4, d4,
    g4, g4, f4, f4,
    e4, e4, d4,
  };

  // The array with note lengths.
  float melody_t[28] = {
    4,  4,  4,  4,
    4,  4,  2,
    4,  4,  4,  4,
    4,  4,  2,
    4,  4,  4,  4,
    4,  4,  2,
    4,  4,  4,  4,
    4,  4,  2,
  };

  void loop() {
    const long BPM = 120;
    const long MINUTE = 60000000; // 1 minute in microseconds
    const long T = (MINUTE / BPM) * 4; // The bar length
    const byte D = 100; // Delay between notes.

    // The loop that plays the melody.
    for (int note_idx = 0; note_idx < 28; note_idx++) {
      play_tone(SPEAKER_PIN,
                melody[note_idx],
                T / melody_t[note_idx]);
      delay(D);
    }
  }
\end{minted}

\end{document}
