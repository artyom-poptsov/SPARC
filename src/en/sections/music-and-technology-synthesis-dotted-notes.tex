\documentclass[../sparc.tex]{subfiles}
\graphicspath{{\subfix{../images/}}}
\begin{document}

%%%%%%%%%%%%%%%%%%%%%%%%%%%%%%%%%%%%%%%%%%%%%%%%%%%%%%%%%%%%%%%%%%%%%%%%%%%%%%%%
\section{Dotted notes}
\label{section:dotted-notes}

In the musical notation a dot that is written to the right of a note, increases
the length of its note on the half of its length.

For example, if a dot is on the right of an eighth note (``\eighthNoteDotted'')
it means that the half of its length is added to the length.  The half of an
eighth note is sixteenth note.  To add up simple fractions we need to convert
them to the common denominator:

\begin{equation}
  \mbox{\eighthNoteDotted} = \mbox{\eighthNote} + \mbox{\sixteenthNote}
  = \frac{1}{8} + \frac{1}{16} = \frac{2}{16} + \frac{1}{16} = \frac{3}{16}
\end{equation}

The final number $\frac{3}{16}$ is not very convenient for us as we use 1 bar as
the nominator of the fraction.  But in case of 3 in the nominator we would have
to use 3 bars instead.  To solve this problem with the 3 in the nominator, we
can divide both nominator and denominator by 3:

\begin{equation}
  \frac{3 / 3}{16 / 3} = \frac{1}{16 / 3}
\end{equation}

We end up with $1$ the nominator and $16 / 3$ in the denominator.  So we have to
use $16 / 3$ as the length of the note.  $16 / 3$ is roughly equals to $5.3$ but
we don't have to calculate it ourselves as the microcontroller can do it for us.
All we need is to write $16.0 / 3.0$ to make sure that the fractional part is
not lost in the process of the division.

In the music sheet for ``If only there will be no winter'' on
fig. \ref{fig:lilypond-melody-prostokvashino} the third note is exactly the
dotted eighths note we just discussed.  So according to the aforementioned logic
we must specify its length as $16.0 / 3.0$, like this:

\begin{minted}{cpp}
  {b4, 16.0 / 3.0}
\end{minted}

In the array it should look like is shown in the listing
\ref{listing:music-dotted-notes}.

\begin{listing}[ht]
  \begin{minted}{cpp}
    float prostokvashino[28][2] = {
      // 1st bar.
      {b3, 8},  {b3, 8}, {b4, 16.0 / 3.0},
      {f4, 16}, {a4, 8}, {g4,          8}, {e4, 4},
      // 2nd bar.
      {d4, 8},  {d4, 8}, {d5, 16.0 / 3.0},
      {c5, 16}, {c5, 8}, {b4,          8}, {R,  4},
      // 3rd bar.
      {d5, 8},  {c5, 8}, {a4,          8},
      {f4, 8},  {c5, 8}, {b4,          8}, {b4, 4},
      // 4th bar.
      {b3, 8},  {b3, 8}, {b4, 16.0 / 3.0},
      {a4, 16}, {a4, 8}, {g4,          8}, {R,  4},
    };
  \end{minted}
  \label{listing:music-dotted-notes}
  \caption{An example of specifying of dotted notes in the melody.}
\end{listing}

Now our melody became a bit closer to the original.

Here we have to note that sometimes in the music notation we can find notes with
two dots on the right -- which means that this note have to be elongated by its
half and half of the half.  But those double-dotted notes are rarely used as it
complicates the reading of the score for a musician.  It's a different matter
for us, programmers -- we can read the score in slow, relaxed pace, program it
and then let the computer to the all complicated stuff!

\end{document}
