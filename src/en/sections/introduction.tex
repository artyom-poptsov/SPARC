\documentclass[../sparc.tex]{subfiles}
\graphicspath{{\subfix{../images/}}}
\begin{document}

%%%%%%%%%%%%%%%%%%%%%%%%%%%%%%%%%%%%%%%%%%%%%%%%%%%%%%%%%%%%%%%%%%%%%%%%%%%%%%%%
\section*{For whom is the book written?}
\addcontentsline{toc}{section}{For whom is the book written?}

Dear reader, welcome to our cozy club of tinkerers and makers.  Here we are
learning how to work with sound, light, electricity by applying our knowledge
and skills to create unexpected, interesting and practical projects.  We will
try to make your journey to the world of electronics and programming as
interesting as we can and will try to lead you around the bumps on the road
where it's possible.  But you'll have certain responsibilities too.  First of
all, our efforts will not give you the desired results without your involvement
in the process.  Second, we are learning with you as well -- and you, dear
reader, can become an active contributor to our work.  If you find any typos and
errors (or, as programmers say, ``bugs'') in the book, please don't hesitate to
report them to us.  We'll try our best to fix all bugs in the next version of
the book.

We are hoping that this book will become your fellow traveler for a while, that
will help you to grasp the art of programming and give you new insights about
the physical world around you.

\section*{The Book Guide}
\addcontentsline{toc}{section}{The Book Guide}

The book is split into chapters, each of which is based on previous ones (except
the first one, obviously.)  But the book can be read in a non-linear manner.
For example if you have some experience with programming and electronics, you
can take a look at ``Music and Technology Synthesis'' chapter where we discuss
music programming on Arduino -- this topic leaves a few people indifferent.

We are using the following conventions throughout the book:

\begin{table}[H]
  \centering
  \def\arraystretch{3.0}%
  \begin{tabular}{|m{4em}|m{15em}|}
    \hline
    \includesvg[width=1.25cm]{the-noun-project/request-mirrored}
    & A practical application of the topic discussed earlier. \\
    \hline
    \includesvg[width=1cm]{the-noun-project/flask}
    & An experiment you can try yourself. \\
    \hline
    \includesvg[width=1cm]{the-noun-project/note}
    & An important note. \\
    \hline
  \end{tabular}
\end{table}

At the end of some sections you can find a list of tasks that you can solve to
improve your understanding of the discussed topics.  If you want to get a good
grasp on the material discussed in the book you definitely must try to solve
those tasks as it is a good way to improve your understanding.

Another recommendation is that you shouldn't stop on the first solution of a
task -- try to solve it in different ways and compare the solutions and the
results.

And finally, try to come up with your own challenges and try to solve them using
your knowledge and skills.  Figuring out new tasks is one of the key skills that
allow you to gain new skills and experience.

\section*{Acknowledgments}
\addcontentsline{toc}{section}{Acknowledgments}

I've spent years in a \textbf{Nizhniy Novgorod Technical College}
(\url{https://nntc.nnov.ru/}) teaching students how to program micro-controllers
(MCUs) in C++ and work with electronics.  Also I did some open classes on the
same topic in \textbf{``CADR'' hackerspace} (\url{https://cadrspace.ru/}) that I
co-founded in the Nizhniy Novgorod city, Russia -- this place has become almost
like a second home for me and here I was able to share my knowledge and learn
from the best myself.  I'd like to thank Nizhniy Novgorod Technical College and
CADR for the provided opportunities to foster my own growth as a specialist and
as a person.  That experience was laid down as the foundation for the book.

Also I want to thank the following people who contributed to the book
development:
\begin{itemize}
\item Denis Kiselyov (рус. \emph{Денис Киселёв}) -- contributions to some
  chapters of the book.  Proofreading of the book.  Participating in the
  development and testing of the examples that used in the book.
\item Sergey Ermeykin (рус. \emph{Сергей Ермейкин}) -- proofreading, fixing
  errors.
\item Ilya Mashtakov (рус. \emph{Илья Маштаков}) – proofreading and improvement
  of the book content.
\item Pyotr Tretyakov (рус. \emph{Пётр Третьяков}) -- in-depth proofreading of
  the whole book.  Many valuable tips on the topic of the electronics, low-level
  MCU programming, protocols and data buses, overall structure and the content
  of the book.
\item Alina Taraeva (рус. \emph{Алина Тараева}) -- proofreading of the book.
\item \href{https://github.com/V4n0M4sk}{Van0Mask} -- proofreading and correcting some of the punctuation and syntax errors.
\item Anton Sheffer (рус. \emph{Антон Шеффер}) (Agaffer) -- fixing errors in
  chapter 5 of the book.
\end{itemize}

It would be very hard to start, let alone to finish, the book without your help.
Thank you!
\vspace{5mm}

\hspace{1cm}Sincerely,

\hspace{1cm}Artyom V. Poptsov

\section*{Source Code}
\addcontentsline{toc}{section}{Source Code}

Source code of the book can be found here: \\
\url{https://github.com/artyom-poptsov/SPARC}.

\section*{License}
\addcontentsline{toc}{section}{License}

Copyright © 2016-2024 Artyom ``avp'' Poptsov
<\href{mailto:poptsov.artyom@gmail.com}{poptsov.artyom@gmail.com}>.

The rights to the third party materials that are used in this book are belong to
their owners.

This book is published under the terms of \\ ``Attribution--ShareAlike'' 4.0
International (CC BY-SA 4.0)
\\ \url{https://creativecommons.org/licenses/by-sa/4.0/deed.en}

\end{document}
