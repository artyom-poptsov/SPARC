\documentclass[../sparc.tex]{subfiles}
\graphicspath{{\subfix{../images/}}}
\begin{document}

%%%%%%%%%%%%%%%%%%%%%%%%%%%%%%%%%%%%%%%%%%%%%%%%%%%%%%%%%%%%%%%%%%%%%%%%%%%%%%%%
\section{Control Flow}

Statements (or so called \emph{operators}) of a programming language that are
responsible for the controlling the flow of program execution are called
\emph{control flow statements}.  They include \emph{branching} statements,
\emph{loop} statements and others alike.  Generally we can say that if a
statement changes the sequential flow of execution of a program then this
statement is one of the control flow statements group.

In this section we will discuss the main types of the control flow statements
that are present in C++ and many other popular programming languages.

%%%%%%%%%%%%%%%%%%%%%%%%%%%%%%%%%%%%%%%%%%%%%%%%%%%%%%%%%%%%%%%%%%%%%%%%%%%%%%%%
\subsection{Conditionals}
\index{Programming!Conditionals}

%%%%%%%%%%%%%%%%%%%%%%%%%%%%%%%%%%%%%%%%%%%%%%%%%%%%%%%%%%%%%%%%%%%%%%%%%%%%%%%%
\subsection{Loops}
\index{Programming!Loops}

\end{document}
