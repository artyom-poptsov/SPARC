\documentclass[../sparc.tex]{subfiles}
\graphicspath{{\subfix{../images/}}}
\begin{document}

%%%%%%%%%%%%%%%%%%%%%%%%%%%%%%%%%%%%%%%%%%%%%%%%%%%%%%%%%%%%%%%%%%%%%%%%%%%%%%%%
\section{Control Flow}

Statements (or so called \emph{operators}) of a programming language that are
responsible for the controlling the flow of program execution are called
\emph{control flow statements}.  They include \emph{branching} statements,
\emph{loop} statements and others alike.  Generally we can say that if a
statement changes the sequential flow of execution of a program then this
statement is one of the control flow statements group.

In this section we will discuss the main types of the control flow statements
that are present in C++ and many other popular programming languages.

%%%%%%%%%%%%%%%%%%%%%%%%%%%%%%%%%%%%%%%%%%%%%%%%%%%%%%%%%%%%%%%%%%%%%%%%%%%%%%%%
\subsection{Conditionals}
\index{Programming!Conditionals}

Sometimes a program needs to decide what to do next in the runtime.  For a
computer to choose the right way to go we (the programmers) need to describe a
conditional statement in the program code: if a logical expression is true then
we do one thing, if it is not -- we do something other.

Conditional statements are described using special control flow statements.  In
C++ we have two main statements.  The first one is \texttt{if} operator.

Here's an example:

\begin{listing}[ht]
  \begin{minted}{cpp}
    int a = 42;
    if (a > 10) {
      // A code that is executed only when the value of "a'' variable
      // is greater than 10.  In this case this code will always
      // be executing because the variable is always equals to 42.
    }
  \end{minted}
  \label{listing:dialogues-with-computer-if-more-than}
  \caption{Example of (\texttt{if}) usage to check if the variable value is
    greater than 10.}
\end{listing}

When we need to check if a variable value is equal to some other value we use
``=='' (strict equal) as shown on the listing
\ref{listing:dialogues-with-computer-if-equals}.

\begin{listing}[ht]
  \begin{minted}{cpp}
    int a = 42;
    if (a == 10) {
      // A sequence of commands that will be executed if
      // the 'a' value is equal to 10.  In this example this
      // piece of code will never be executed as the variable
      // value is always 42 and will never be 10.
    }
  \end{minted}
  \label{listing:dialogues-with-computer-if-equals}
  \caption{An example of usage of \texttt{if} to check if the variable value is
    equal to 10.}
\end{listing}

\note{ Don't confuse \emph{equality operator} ``=='' with \emph{assignment
  operator} ``='' -- this is important!  The point is that in many cases C/C++
  programming languages allow us to to use either of the operators in one place
  interchangeably but the results will be completely different depending on the
  used operator.  In C++ ``='' symbol has different roles but if we are talking
  about working with variables then this operator is always changes the variable
  that is on the left of assignment operator. }

The easiest way to remember the difference between those two operations is that
single ``='' sign means assignment and ``=='' is equality operator.

\begin{listing}[ht]
  \begin{minted}{cpp}
    int a = 42;
    if (a = 10) {
      // This sequence of commands will always be executed
      // as the ``a = 10'' stores 10 in ``a'' variable so
      // the value of the overall expression is 10 which means
      // ``true'' in C/C++.
    }
  \end{minted}
  \label{listing:dialogues-with-computer-if-equals-assignment}
  \caption{An example of the error that is introduced into the code by using
    ``='' instead of ``==''.}
\end{listing}

In the listing \ref{listing:dialogues-with-computer-if-equals-assignment} the
\texttt{if} body will always be executed a despite the fact that \texttt{a} is
equal to 42 before \texttt{if} -- the reason for that is that the assignment
operator inside the parenthesis \emph{changes} the value of \texttt{a} to 10.
Then \texttt{a} variable is substituted with its value inside the parenthesis.
In C/C++ Boolean values are just numbers: zero is ``false'', any other value is
considered as ``true''.  Which means that the number 10 means ``true'' as well
when is used as a value in logical expressions.

Often you need to provide not only the code to be executed if the condition
inside \texttt{if} parenthesis is true, but also some code that needs to be
executed when the condition is false.  In such cases one can use
\texttt{if..else} statement:

\begin{listing}[ht]
  \begin{minted}{cpp}
    int a = 42;
    if (a > 10) {
      // Some code that needs to be executed when
      // the value of 'a' is greater than 10.
    } else {
      // Code that is executed when 'a' value is less
      // or equal to 10.
    }
  \end{minted}
  \label{listing:dialogues-with-computer-if-with-else}
  \caption{An example of (\texttt{if}) usage with \texttt{else} block.}
\end{listing}

Another statement that is often used is \texttt{switch..case} also known as
\emph{multiple choice operator}.  This statement is convenient when it comes to
choosing from multiple cases of a variable value.

\begin{listing}[ht]
  \begin{minted}{cpp}
    int a = 42;
    switch (a) {
      case 10:
      // A set of instructions to be executed
      // when 'a' value equals to 10.
      break;
      case 42:
      // A set of instructions to be executed
      // when 'a' value equals to 42.
      break;
      case 20:
      // A set of instructions to be executed
      // when 'a' value equals to 20.
      case 30:
      // A set of instructions to be executed
      // when 'a' value equals to either 20 or 30.
      break;
      case 1:
      case 2:
      // A set of instructions to be executed
      // when 'a' value equals to 1 or 2.
      break;
      default:
      // A set of instructions to be executed
      // when 'a' does not match with any of the
      // cases described above.
    }
  \end{minted}
  \label{listing:dialogues-with-computer-switch-case}
  \caption{An example of \texttt{switch..case} usage.}
\end{listing}

In the listing \ref{listing:dialogues-with-computer-switch-case} an example of
\texttt{switch..case} statement is shown. As can be seen in this example the
value of \texttt{a} variable is sequentially compared with the values described
in \texttt{case} clauses.  The first \texttt{case} that will match the value of
the variable is executed and on the first \texttt{break} instruction the
\texttt{switch..case} exits.

Note that the \texttt{break} keyword shows the exit point of the
\texttt{switch..case}.  We can see in the example
\ref{listing:dialogues-with-computer-switch-case} that there are two
\texttt{case} instructions that go one after another -- the first one is for
number 20 and the second one is for number 30.  But the \texttt{break}
instruction is used only after the \texttt{case 30}.  It means that if
\texttt{a} variable is equal to 20 then not only the sequence of instructions
after \texttt{case 20} will be executed but the instructions after \texttt{case
  30} as well, without any further checks.  The same situation can be seen in
\texttt{case 1} and \texttt{case 2}, the difference is only that \texttt{case 1}
does not have any instructions to execute on its own, and the code \texttt{case
  2} will be executed instead.

Another very important idea to consider is that when we are using conditionals
in the code (also known as \emph{branching operators} for a good reason) they
are executed without any looping.  Nevertheless sometimes there is a temptation
to compare \emph{loop statements} with condition statements as they are both
using some condition that controls their execution.  It shouldn't be a surprise
as loops that we will discuss in \ref{Programming!Loops} are based on the
condition statements and can be implemented with the \texttt{if} statement.  In
the listing \ref{listing:dialogues-with-computer-if-based-loop} we provide an
example of loop implementation based on the plain \texttt{if} statement and a
global variable.

\begin{listing}[ht]
  \begin{minted}{cpp}
    void setup() {
      // Here we must configure ports 2..6 in OUTPUT
      // mode through 'pinMode' procedure.
    }

    // Create a global 'p' variable
    // that sets the number of LED.
    int p = 2;

    void loop() {
      // Do a single LED blinking.
      digitalWrite(p, HIGH);
      delay(100);
      digitalWrite(p, LOW);
      delay(100);

      p++; // Increment the variable.

      // If 'p' is equal to 7 then...
      if (p == 7) {
        // ...set the variable value to 2.
        p = 2;
      }
    }
  \end{minted}
  \label{listing:dialogues-with-computer-if-based-loop}
  \caption{An example of loop implementation through \texttt{if} operator.}
\end{listing}

Usually we don't use this approach as there are standalone loop operators that
we will discuss in the next section.

%%%%%%%%%%%%%%%%%%%%%%%%%%%%%%%%%%%%%%%%%%%%%%%%%%%%%%%%%%%%%%%%%%%%%%%%%%%%%%%%
\subsection{Loops}
\index{Programming!Loops}

Simple programs like ``Chasing lights'' that we did before can be written just
by simple copying and pasting the LED blinking algorithm (probably with some
modifications.)

Now imagine that we need to program ``Chasing lights'' using 100 LEDs.  That's a
tedious task, isn't it?  To avoid the dumb work of copying the same code many
times programmers came up with special control statements that are called
\emph{loops}.

There are several types of loops.  The main ones that we will need in the most
of the programming languages:
\begin{itemize}
\item Loop with a counter.
\item Pre-test loop.
\item Post-test loop.
\end{itemize}

Each of the loop type has its own implementation in the programming language
that we are using here (C++.)

%%%%%%%%%%%%%%%%%%%%%%%%%%%%%%%%%%%%%%%%%%%%%%%%%%%%%%%%%%%%%%%%%%%%%%%%%%%%%%%%
\subsubsection{Pre-Test Loop}
\index{Programming!Loops!Pre-test loop (while)}

The easiest type of a loop is the \emph{pre-test loop} that is implemented in
C++ as \texttt{while} statement.  This type of a loop comes handy when we don't
know the exact number of executions of a loop body that must be performed (in
other words when we don't know the exact number of \emph{iterations}.)

The overall structure of a \texttt{while} loop is shown in the listing
\ref{listing:dialogues-with-computer-while}.

\begin{listing}[ht]
  \begin{minted}{cpp}
    int t = 0;

    while (t < 12) {
      // Loop body.
    }
  \end{minted}
  \label{listing:dialogues-with-computer-while}
  \caption{Pre-test loop (\texttt{while}) example.}
\end{listing}

Often visualization of information helps us better understand the essence of the
matter.  One of the ways to visualize a loop is to draw a timeline that contains
the sequence of events as shown on fig. \ref{fig:control-flow-while-loop}.

\begin{figure}[ht]
  \centering
  \begin{tikzpicture}
    \draw[-{Stealth[gray]}] (0, 0) -- (11, 0);
    \foreach \x in {1, 2, ..., 10} {
      \draw (\x, 0.1) -- (\x, -0.1) node[below] {$\x$};
      \draw[dotted, red] (\x, 0.1) -- (\x, 0.5);
    };
    \draw (1.25, 0.5)  node[above, red] {\rotatebox[origin=l]{45}{$\mbox{int t = 0}$}};
    \draw (2.25, 0.5)  node[above, red] {\rotatebox[origin=l]{45}{$\mbox{t < 12}$}};
    \draw (3.50, 0.5)  node[above, red] {\rotatebox[origin=l]{45}{$\mbox{Loop body}$}};
    \draw (4.25, 0.5)  node[above, red] {\rotatebox[origin=l]{45}{$\mbox{t < 12}$}};
    \draw (5.50, 0.5)  node[above, red] {\rotatebox[origin=l]{45}{$\mbox{Loop body}$}};
    \draw (6.25, 0.5)  node[above, red] {\rotatebox[origin=l]{45}{$\mbox{t < 12}$}};
    \draw (7.50, 0.5)  node[above, red] {\rotatebox[origin=l]{45}{$\mbox{Loop body}$}};
    \draw (8.00, 0.5)  node[above, red] {$\mbox{...}$};
    \draw (11, 0) node[below] {$t$};
  \end{tikzpicture}
  \label{fig:control-flow-while-loop}
  \caption{A visualization of \texttt{while} loop.}
\end{figure}

As we can see here a loop is a process that spans over some period of time.
This process can be divided in the following steps:

\begin{enumerate}
\item Declare a variable and set its value to 0.
\item Check the condition of a loop.
\item If the condition is true, go to the loop body.
\item Checking the condition again (as on the step 2) -- if the condition is
  still evaluating to true, go to the loop body etc.
\end{enumerate}

\end{document}
