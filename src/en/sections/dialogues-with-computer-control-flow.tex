\documentclass[../sparc.tex]{subfiles}
\graphicspath{{\subfix{../images/}}}
\begin{document}

%%%%%%%%%%%%%%%%%%%%%%%%%%%%%%%%%%%%%%%%%%%%%%%%%%%%%%%%%%%%%%%%%%%%%%%%%%%%%%%%
\section{Control Flow}

Statements (or so called \emph{operators}) of a programming language that are
responsible for the controlling the flow of program execution are called
\emph{control flow statements}.  They include \emph{branching} statements,
\emph{loop} statements and others alike.  Generally we can say that if a
statement changes the sequential flow of execution of a program then this
statement is one of the control flow statements group.

In this section we will discuss the main types of the control flow statements
that are present in C++ and many other popular programming languages.

%%%%%%%%%%%%%%%%%%%%%%%%%%%%%%%%%%%%%%%%%%%%%%%%%%%%%%%%%%%%%%%%%%%%%%%%%%%%%%%%
\subsection{Conditionals}
\index{Programming!Conditionals}

Sometimes a program needs to decide what to do next in the runtime.  For a
computer to choose the right way to go we (the programmers) need to describe a
conditional statement in the program code: if a logical expression is true then
we do one thing, if it is not -- we do something other.

Conditional statements are described using special control flow statements.  In
C++ we have two main statements.  The first one is \texttt{if} operator.

Here's an example:

\begin{listing}[ht]
  \begin{minted}{cpp}
    int a = 42;
    if (a > 10) {
      // A code that is executed only when the value of "a'' variable
      // is greater than 10.  In this case this code will always
      // be executing because the variable is always equals to 42.
    }
  \end{minted}
  \label{listing:dialogues-with-computer-if-more-than}
  \caption{Example of (\texttt{if}) usage to check if the variable value is
    greater than 10.}
\end{listing}

When we need to check if a variable value is equal to some other value we use
``=='' (strict equal) as shown on the listing
\ref{listing:dialogues-with-computer-if-equals}.

\begin{listing}[ht]
  \begin{minted}{cpp}
    int a = 42;
    if (a == 10) {
      // A sequence of commands that will be executed if
      // the 'a' value is equal to 10.  In this example this
      // piece of code will never be executed as the variable
      // value is always 42 and will never be 10.
    }
  \end{minted}
  \label{listing:dialogues-with-computer-if-equals}
  \caption{An example of usage of \texttt{if} to check if the variable value is
    equal to 10.}
\end{listing}


%%%%%%%%%%%%%%%%%%%%%%%%%%%%%%%%%%%%%%%%%%%%%%%%%%%%%%%%%%%%%%%%%%%%%%%%%%%%%%%%
\subsection{Loops}
\index{Programming!Loops}

\end{document}
