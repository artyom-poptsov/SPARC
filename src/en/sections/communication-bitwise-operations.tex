\documentclass[../sparc.tex]{subfiles}
\graphicspath{{\subfix{../images/}}}
\begin{document}

%%%%%%%%%%%%%%%%%%%%%%%%%%%%%%%%%%%%%%%%%%%%%%%%%%%%%%%%%%%%%%%%%%%%%%%%%%%%%%%%
\section{Bitwise operators}
\index{Programming!Bitwise operators}

There are special \emph{bitwise operators} in C language that allow us to
conveniently handle individual bits.  They include: bitwise ``AND'' (denoted by
``\&'' symbol), bitwise ``OR'' (denoted by ``|'' symbol), bitwise ``NOT''
(denoted by ``$\sim$'' symbol), left shift (``$<<$'') and right shift (``$>>$''.)

We will use an I2C device to clearly see those operators in action.

For example, to set the fifth from the right bit in a byte to one, we can use
the following code:

\begin{minted}{cpp}
  int a = 0b10000000;
  a = a | 0b00010000;
  // Result:
  //      0b10010000;
\end{minted}

To make it easier for us to set ``the fifth from the right bit'' we can use the
left shift operator, and the result will be the same -- only the notation will
become shorter:

\begin{minted}{cpp}
  int a = 0b10000000;
  a = a | (1 << 4);
  // Result:
  //      0b10010000;
\end{minted}

To check if a particular bit is set to one, we can use bitwise ``AND'' -- this
operation is called ``applying a bitmask'':

\begin{minted}{cpp}
  int a = 0b10010000;

  int mask = 0b00010000;
  // We can also create a bit mask using a bitshift:
  //   mask = (1 << 4);

  if (a & mask) {
    // Some action that must be performed if the fifth
    // from the right bit is set to 1.
  }
\end{minted}

To ``switch off'' a bit in a value we can use bitwise ``AND'' as well, but this
time with the inversion of the bitmask:

\begin{minted}{cpp}
  int a = 0b10010000;
  a = a & ~(1 << 4);
  // Result:
  //      0b10000000;
\end{minted}

%%%%%%%%%%%%%%%%%%%%%%%%%%%%%%%%%%%%%%%%%%%%%%%%%%%%%%%%%%%%%%%%%%%%%%%%%%%%%%%%
\subsection{Exercises}

\begin{enumerate}
\item Implement the ``Chasing lights'' effect using the I2C-adapter for an LCD
  with connected LEDs, using \ref{listing:communication-pcf8574} example.  For
  achieving the effect we can use bitwise operators.
\end{enumerate}
