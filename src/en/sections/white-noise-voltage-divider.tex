\documentclass[../sparc.tex]{subfiles}
\graphicspath{{\subfix{../images/}}}
\begin{document}

\newpage
%%%%%%%%%%%%%%%%%%%%%%%%%%%%%%%%%%%%%%%%%%%%%%%%%%%%%%%%%%%%%%%%%%%%%%%%%%%%%%%%
\section{Voltage divider}

\subsection{Implementation of a simple voltage divider based on resistors}

Now that we have learned how to read a signal from analog ports we have to
connect something substantial to an Arduino analog port -- some device which
signal we can physically change, thus affect the value on the analog port.

\note{en}{ When implementing voltage dividers we have to take into account which
  maximal voltage value we can supply to a port of a microcontroller platform.
  For example, Arduino Mega 2560 and Arduino UNO allows us to use voltages up to
  5V, while other platforms maximum voltage value may be constrained to 3.3V.
  When we building electronic circuits we always have to refer to the
  documentation for the chosen MCU platform! }

Let's begin with a simple experiment.  Let's take two 5kOhm resistors and
connect them as is shown in the fig.  \ref{fig:electronics-arduino-circuit-01}.

\figureVoltageDivider{en}

As can be seen in the figure, the ``A'' mark points to a branching in the
circuit where one of the wires goes to ``A0'' port on Arduino.  As the electric
current has two ways where it can go (either ``A0'' or ``GND'') it is divided
into two parts.  If resistors $R_1$ and $R_2$ have equal resistance value
(e.g. 5kOhm) then the current is divided by 2.

The circuit we just built is called a \emph{voltage divider}.  Such circuit
allows us to divide input voltage by some coefficient, which value depends on
the relation between $R_1$ and $R_2$ resistance.  The smaller the value of $R_1$
resistance and the larger the value of $R_2$ resistance, the closer the output
voltave of the voltage divider to the input voltage (in our case, to 5V.)  And
vice versa, the larger the value of $R_1$ and smaller the value of $R_2$, the
closer the output value to 0V.

%%%%%%%%%%%%%%%%%%%%%%%%%%%%%%%%%%%%%%%%%%%%%%%%%%%%%%%%%%%%%%%%%%%%%%%%%%%%%%%%
\subsection{Implementation of a voltage divider based on a potentiometer}

Thus, we can get the required voltage in ``A'' point by choosing the right
values of $R_1$ и $R_2$, and therefore on the analog input ``A0''.  However,
this is not very convenient if we want to adjust voltage (and the value on an
analog port) in real time.  For such tasks, a \emph{variable resistor} (also
often called \emph{potentiometer}) is more suitable.

A circuit of a voltage divider based on a potentiometer is shown on
fig. \ref{fig:electronics-potentiometer}.

\figureVoltageDividerPotentiometer{en}

A potentiometer usually has three pins: let's call them left, middle and right
one -- as is shown on schematic depiction of a potentiometer in fig.
\ref{fig:electronics-potentiometer}.  Resistance between the left and the right
pin is fixed (as in a regular resistor), and depends only on the potentiometer
value -- for example it could be equal to 10kOhm.  However, the middle pin is
adjustable (usually it can be moved using a special handle) -- and resistance
between the middle and other two pins depends on the position of the
potentiometer handle.  Thus a potentiometer combines two resistors (sort of),
where the sum of resistor values is set by the potentiometer value, while the
relation between their resistance value can be changed.

\end{document}
