\documentclass[../sparc.tex]{subfiles}
\graphicspath{{\subfix{../images/}}}
\begin{document}

\newpage
%%%%%%%%%%%%%%%%%%%%%%%%%%%%%%%%%%%%%%%%%%%%%%%%%%%%%%%%%%%%%%%%%%%%%%%%%%%%%%%%
\section{Voltage divider}

\subsection{Implementation of a simple voltage divider based on resistors}

Now that we have learned how to read a signal from analog ports we have to
connect something substantial to an Arduino analog port -- some device which
signal we can physically change, thus affect the value on the analog port.

\note{en}{ When implementing voltage dividers we have to take into account which
  maximal voltage value we can supply to a port of a microcontroller platform.
  For example, Arduino Mega 2560 and Arduino UNO allows us to use voltages up to
  5V, while other platforms maximum voltage value may be constrained to 3.3V.
  When we building electronic circuits we always have to refer to the
  documentation for the chosen MCU platform! }

Let's begin with a simple experiment.  Let's take two 5kOhm resistors and
connect them as is shown in the fig.  \ref{fig:electronics-arduino-circuit-01}.

\figureVoltageDivider{en}

As can be seen in the figure, the ``A'' mark points to a branching in the
circuit where one of the wires goes to ``A0'' port on Arduino.  As the electric
current has two ways where it can go (either ``A0'' or ``GND'') it is divided
into two parts.  If resistors $R_1$ and $R_2$ have equal resistance value
(e.g. 5kOhm) then the current is divided by 2.

The circuit we just built is called a \emph{voltage divider}.

%%%%%%%%%%%%%%%%%%%%%%%%%%%%%%%%%%%%%%%%%%%%%%%%%%%%%%%%%%%%%%%%%%%%%%%%%%%%%%%%
\subsection{Implementation of a voltage divider based on a potentiometer}



\end{document}
