\documentclass[../sparc.tex]{subfiles}
\graphicspath{{\subfix{../images/}}}
\begin{document}

%%%%%%%%%%%%%%%%%%%%%%%%%%%%%%%%%%%%%%%%%%%%%%%%%%%%%%%%%%%%%%%%%%%%%%%%%%%%%%%%
\newpage
\section{Creating a music band}

Now as we understand how rhythm works and can read the notes, we can start
creating a music ``band'' from two or more Arduinos.

Many musical compositions as we might notice, include two or more musical
tracks.  Often the second track is for the left hand when playing a piano, or
several tracks for other musical instruments (e.g. guitar and vocals.)

In this situation the easiest way to play this composition as the whole is to
take several Arduino micro-controllers and ``teach'' them to play in tune.

%%%%%%%%%%%%%%%%%%%%%%%%%%%%%%%%%%%%%%%%%%%%%%%%%%%%%%%%%%%%%%%%%%%%%%%%%%%%%%%%
\subsection{Choosing the name for the band}

First of all to make a musical band we have to come up with a good name for it.
Let's use ``Second Hand Of Darkness'' (which can be shortened to ``SHOD'') for
our Arduino band.  This name sounds appropriate for a rock-band, and simply
sounds very cool.

Then we have to stylize the name, so the other professional musicians accept us
as one of their own. \footnote{For the title the ``Guttural'' font is used,
which is distributed under Creative Commons Attribution license:
\url{https://www.fontspace.com/guttural-font-f8257}.}

\begin{figure}[ht]
  \centering
  \includesvg[width=10.00cm]{shod}
  \caption{The logo for our music band.}
  \label{fig:shod-band-logo}
\end{figure}

Preparatory stages have been completed.  Now we have to choose a musical
composition and play it.

%%%%%%%%%%%%%%%%%%%%%%%%%%%%%%%%%%%%%%%%%%%%%%%%%%%%%%%%%%%%%%%%%%%%%%%%%%%%%%%%
\subsection{Selecting the debut composition}

Many musical bands star their way on the big musical scene by performing covers
on the famous songs.  Let out first composition will be an Arduino-cover of the
song ``Sweet Dreams'' by Marilyn Manson (see
fig. \ref{fig:lilypond-melody-sweet-dreams}.)\footnote{As the basis, we took the
musical score \url{https://musescore.com/user/13484026/scores/5329623} by the
user ``Instrumental Rock'' and licensed under CC-BY-NC
(\url{https://creativecommons.org/licenses/by-nc/4.0/}.)}

This composition has two main tracks -- the 1st one is the vocals, and the 2nd
one is the guitar part.

\newpage
\begin{figure}[H]
  \begin{lilypond}
    <<
    \new Staff \with { instrumentName = "Vocals" } {
      \relative c' {
        \tempo 4 = 93
        \key es \major
        \numericTimeSignature
        \time 4/4
        %% 01
        \bar ".|:" r1 |
        %% 02
        r1 \bar ":|."
        %% 03
        r4 es'4 es4 c8 es8~ |
        %% 04
        es8 es4. d2 |
        %% 05
        es4 c8 es4. c4 |
        %% 06
        es8 f4. d4. c8 |
        %% 07
        es8 es8 c8 es4. c8 c8 |
        %% 08
        es8 es4. d2 |
        %% 09
        es4 c8 es4 c4. |
        %% 10
        es8 es8 f8 es8 d8 d4. |
        %% 11
        es8 es8 c4 es4 c8 es8~ |
        %% 12
        es8 c4. r2 |
      }
    }
    \new Staff  \with { instrumentName = "Guitar" } {
      \relative c' {
        \key es \major
        \numericTimeSignature
        \time 4/4
        %% 01
        \bar ".|:" c8 c g'8 c,8 as'8 c,8 g'8 c,8 |
        %% 02
        as8 as es'8 f8 g,8 g8 d'8 es8 \bar ":|."
        %% 03
        c8 c g'8 c,8 as'8 c,8 g'8 c,8 |
        %% 04
        as8 as es'8 f8 g,8 g8 d'8 es8 |
        %% 05
        c8 c g'8 c,8 as'8 c,8 g'8 c,8 |
        %% 06
        as8 as es'8 f8 g,8 g8 d'8 es8 |
        %% 07
        c8 c g'8 c,8 as'8 c,8 g'8 c,8 |
        %% 08
        as8 as es'8 f8 g,8 g8 d'8 es8 |
        %% 09
        c8 c g'8 c,8 as'8 c,8 g'8 c,8 |
        %% 10
        as8 as es'8 f8 g,8 g8 d'8 es8 |
        %% 11
        c8 c g'8 c,8 as'8 c,8 g'8 c,8 |
        %% 12
        as8 as es'8 f8 g,8 g8 d'8 es8 |
      }
    }
    >>
    \layout {
      indent = 0\mm
      line-width = 120\mm
      ragged-last = ##t
    }
  \end{lilypond}
  \caption{Marilyn Manson, Trent Reznor, ``Sweet Dreams''.}
  \label{fig:lilypond-melody-sweet-dreams}
\end{figure}

%%%%%%%%%%%%%%%%%%%%%%%%%%%%%%%%%%%%%%%%%%%%%%%%%%%%%%%%%%%%%%%%%%%%%%%%%%%%%%%%
\newpage
\subsection{Implementation of the melody}

To create the 1st track of our melody, we have to create a new project for the
Arduino, as we did before.  Likewise, the playback is organized through our
procedure \texttt{play\_tone}.

We define both parts of the melody as two two-dimensional arrays.  The 1st array
will store the guitar part.  As is seen from the listing
\ref{listing:music-band-sweet-dreams-1}, the main theme of the melody is
repeated several times.

\begin{listing}[H]
  \begin{minted}{cpp}
    float sweet_dreams_guitar[28][2] = {
      /* 01 */
      {c4,  8}, {c4,  8}, {g4,  8}, {c4,  8},
      {a4b, 8}, {c4,  8}, {g4,  8}, {c4,  8},
      /* 02 */
      {a3b, 8}, {a3b, 8}, {e4b, 8}, {f4,  8},
      {g3,  8}, {g3,  8}, {d4,  8}, {e4b, 8}
      /* 03 */
      {c4,  8}, {c4,  8}, {g4,  8}, {c4,  8},
      {a4b, 8}, {c4,  8}, {g4,  8}, {c4,  8},
      /* 04 */
      {a3b, 8}, {a3b, 8}, {e4b, 8}, {f4,  8},
      {g3,  8}, {g3,  8}, {d4,  8}, {e4b, 8}
      /* 05 */
      {c4,  8}, {c4,  8}, {g4,  8}, {c4,  8},
      {a4b, 8}, {c4,  8}, {g4,  8}, {c4,  8},
      /* 06 */
      {a3b, 8}, {a3b, 8}, {e4b, 8}, {f4,  8},
      {g3,  8}, {g3,  8}, {d4,  8}, {e4b, 8}
      /* 07 */
      {c4,  8}, {c4,  8}, {g4,  8}, {c4,  8},
      {a4b, 8}, {c4,  8}, {g4,  8}, {c4,  8},
      /* 08 */
      {a3b, 8}, {a3b, 8}, {e4b, 8}, {f4,  8},
      {g3,  8}, {g3,  8}, {d4,  8}, {e4b, 8}
      /* 09 */
      {c4,  8}, {c4,  8}, {g4,  8}, {c4,  8},
      {a4b, 8}, {c4,  8}, {g4,  8}, {c4,  8},
      /* 10 */
      {a3b, 8}, {a3b, 8}, {e4b, 8}, {f4,  8},
      {g3,  8}, {g3,  8}, {d4,  8}, {e4b, 8}
      /* 11 */
      {c4,  8}, {c4,  8}, {g4,  8}, {c4,  8},
      {a4b, 8}, {c4,  8}, {g4,  8}, {c4,  8},
      /* 12 */
      {a3b, 8}, {a3b, 8}, {e4b, 8}, {f4,  8},
      {g3,  8}, {g3,  8}, {d4,  8}, {e4b, 8}
    };
  \end{minted}
  \caption{Guitar part of ``Sweet Dreams''.}
  \label{listing:music-band-sweet-dreams-1}
\end{listing}

The 2nd part of the melody for the vocals will be implemented the same way.

\begin{listing}[!h]
  \begin{minted}{cpp}
    float sweet_dreams_vocals[28][2] = {
      /* 01 */
      {R, 1},
      /* 02 */
      {R, 1},
      /* 03 */
      {R, 4}, {e5b, 4}, {e5b, 4}, {c5, 8}, {e5b, 4},
      /* 04 */
      {e5b, 8.0 / 3.0}, {d5, 2},
      /* 05 */
      {e5b, 4}, {c5, 8}, {e5b, 8.0 / 3.0}, {c5, 4},
      /* 06 */
      {e5b, 8}, {f5, 8.0 / 3.0}, {d5, 8.0 / 3.0}, {c5, 8},
      /* 07 */
      {e5b, 8}, {e5b, 8}, {c5, 8}, {e5b, 8.0 / 3.0},
      /* 08 */
      {e5b, 8}, {e5b, 8.0 / 3.0}, {d5, 2},
      /* 09 */
      {e5b, 4}, {c5, 8}, {e5b, 4}, {c5, 8.0 / 3.0},
      /* 10 */
      {e5b, 8}, {e5b, 8}, {f5, 8}, {e5b, 8}, {d5, 8}, {d5, 8.0 / 3.0},
      /* 11 */
      {e5b, 8}, {e5b, 8}, {c5, 4}, {e5b, 4}, {c5, 8}, {e5b, 4},
      /* 12 */
      {c5, 8.0 / 3.0}, {R, 2}
    };
  \end{minted}
  \caption{Vocal part of ``Sweet Dreams''.}
  \label{listing:music-band-sweet-dreams-2}
\end{listing}

While working with two Arduino boards a new problem occurs -- how to upload
different melodies to each of them?  The first naive (but not very convenient)
approach is to create a different project for each Arduino and upload them each
time we change something in the code to each Arduino.  The inconvenience here is
that we have to make sure that we are uploading the right project into each
Arduino.

We can offer a better approach -- to use the project for both Arduino boards,
using a value from a digital port to determine in the code which melody should
be played.  Using a regular Arduino wire we can connect the
\texttt{SWITCH\_PORT} to the ground, and the Arduino will play the guitar part.
If the port is not connected to the ground, the Arduino will play the vocal
part.

\begin{listing}[!h]
  \begin{minted}{cpp}
    // The digital port where a speaker
    // is connected.
    const int SPEAKER     = 2;
    // The port which allows us to select
    // the melody part.
    const int SWITCH_PORT = 10;

    void setup() {
      pinMode(SPEAKER, OUTPUT);
      pinMode(SWITCH_PORT, INPUT_PULLUP);
    }

    void loop() {
      const int BPM = 93;
      const long MINUTE = 60000000;
      const long T = (MINUTE / BPM) * 4;

      // Judging from the SWITCH_PORT state
      // we decide which part of the melody
      // have to be played.
      if (digitalRead(SWITCH_PORT) == LOW) {
        int count = sizeof(sweet_dreams_guitar)
        / sizeof(sweet_dreams_guitar[0]);
        for (int i = 0; i < count; i++) {
          play_tone(SPEAKER,
                    sweet_dreams_guitar[i][0],
                    T / sweet_dreams_guitar[i][1]);
          delay(10);
        }
      } else {
        int count = sizeof(sweet_dreams_vocals)
            / sizeof(sweet_dreams_vocals[0]);
        for (int i = 0; i < count; i++) {
          play_tone(SPEAKER,
                    sweet_dreams_vocals[i][0],
                    T / sweet_dreams_vocals[i][1]);
          delay(10);
        }
      }
    }
  \end{minted}
  \caption{The code to play ``Sweet Dreams'' on two Arduino boards.}
  \label{listing:music-band-sweet-dreams-3}
\end{listing}

%%%%%%%%%%%%%%%%%%%%%%%%%%%%%%%%%%%%%%%%%%%%%%%%%%%%%%%%%%%%%%%%%%%%%%%%%%%%%%%%
\newpage
\subsection{Synchronization of two melodies}

To start both parts of the melody simultaneously, we have to press ``Reset''
button on both Arduino boards and release them at the same time.

Despite the fact that we now managed to play both parts of the melody, over time
we may notice that our ``instruments'' are getting out of sync.

There are several causes of this unpleasant effect.  We discuss them according
to their degree of influence on the melody.

The first cause of desyncronization is hidden in those lines:

\begin{listing}[!h]
  \begin{minted}{cpp}
    for (int i = 0; i < count; i++) {
      play_tone(SPEAKER,
                sweet_dreams_vocals[i][0],
                T / sweet_dreams_vocals[i][1]);

      delay(10); // <----- Here's the problem.
    }
  \end{minted}
  \label{listing:music-band-problem-with-delay}
  \caption{One of the cause of the accumulating error in the composition.}
\end{listing}

Let's imagine that we play the first two bars of the melody (see
fig. \ref{fig:lilypond-melody-sweet-dreams-part}.)

\begin{figure}[!h]
  \begin{lilypond}
    <<
    \new Staff \with { instrumentName = "Vocals" } {
      \relative c' {
        \tempo 4 = 93
        \key es \major
        \numericTimeSignature
        \time 4/4
        %% 01
        \bar ".|:" r1 |
        %% 02
        r1 \bar ":|."
      }
    }
    \new Staff  \with { instrumentName = "Guitar" } {
      \relative c' {
        \key es \major
        \numericTimeSignature
        \time 4/4
        %% 01
        \bar ".|:" c8 c g'8 c,8 as'8 c,8 g'8 c,8 |
        %% 02
        as8 as es'8 f8 g,8 g8 d'8 es8 \bar ":|."
      }
    }
    >>
    \layout {
      indent = 0\mm
      line-width = 120\mm
      ragged-last = ##t
    }
  \end{lilypond}
  \caption{Marilyn Manson, Trent Reznor, ``Sweet Dreams'' (part.)}
  \label{fig:lilypond-melody-sweet-dreams-part}
\end{figure}

Here we can see that in the two bars of the vocal part we have two whole rests --
that's two silent notes.  Meanwhile in the guitar part in two bars we play $8 +
8 = 16$ notes in total.  Currently after each note (that is, after each
\texttt{play\_tone} execution) we have clearly defined a delay in fixed number
of milliseconds.  Thus, if we add up all the delays, in the 1st bar the total
delay will be:

\begin{equation}
  \mbox{Total delay}_{(\mu\mbox{s})} = 2 * 10 \mbox{ms} = 20 \mbox{ms}
\end{equation}

In the guitar part the total delay will be:

\begin{equation}
  \mbox{Total delay}_{(\mu\mbox{s})} = 16 * 10 \mbox{ms} = 160 \mbox{ms}
\end{equation}

That gives us 140ms of difference between two parts of the melody, and this
error will accumulate over the time.  The longer the melody, the bigger the
difference between the number of notes in each part of the melody, and the
greater the error value will be.

To solve this problem we have to get rid of the fixed \texttt{delay} in favor of
adding an adaptive delay after each note, according the note length.  The most
convenient way to do that is to implement the logic inside \texttt{play\_tone}
procedure, as is shown in the listing
\ref{listing:music-band-play-tone-with-delay}.

\begin{listing}[H]
  \begin{minted}{cpp}
    // The procedure of generating a sound
    // with the specified parameters.
    void play_tone(int port, float f, long t) {
      if (f > 0) {
        const int T = 1000000 / f;
        int d = T / 2;
        int count = t / T;
        for (int i = 0; i < count; i++) {
          digitalWrite(port, HIGH);
          delayMicroseconds(d);
          digitalWrite(port, LOW);
          delayMicroseconds(d);
        }
      } else {
        delay(t / 1000); // Pause
      }
      // The delay after a note equals to
      // 10% percent of its length:
      delayMicroseconds(t * 0.10);
    }
  \end{minted}
  \caption{The modification of \texttt{play\_tone} with added adaptive delay
    after notes.}
  \label{listing:music-band-play-tone-with-delay}
\end{listing}

\newpage
After those changes in the \texttt{play\_tone} procedure we have to change the
\texttt{loop} procedure as well, and remove \texttt{delay} after
\texttt{play\_tone} calls, as is shown in the listing
\ref{listing:music-band-using-new-play-tone}.

\begin{listing}[H]
  \begin{minted}{cpp}
    void loop() {
      const int BPM = 93;
      const long MINUTE = 60000000;
      const long T = (MINUTE / BPM) * 4;

      if (digitalRead(SWITCH_PORT) == LOW) {
        int count = sizeof(sweet_dreams_guitar)
        / sizeof(sweet_dreams_guitar[0]);
        for (int i = 0; i < count; i++) {
          play_tone(SPEAKER,
                    sweet_dreams_guitar[i][0],
                    T / sweet_dreams_guitar[i][1]);
        }
      } else {
        int count = sizeof(sweet_dreams_vocals)
        / sizeof(sweet_dreams_vocals[0]);
        for (int i = 0; i < count; i++) {
          play_tone(SPEAKER,
                    sweet_dreams_vocals[i][0],
                    T / sweet_dreams_vocals[i][1]);
        }
      }
    }
  \end{minted}
  \caption{The modification of \texttt{play\_tone} without the fixed delays.}
  \label{listing:music-band-using-new-play-tone}
\end{listing}

The second cause of desynchronization is the error that occurs during the
calculation of acoustic wave period from the frequency in the
\texttt{play\_tone} procedure.  Let's take C4 as an example -- its frequency
equals to 261.63 Hz.  The code that calculates the period uses the formula
\ref{equation:frequency-to-period}.

\begin{equation}
  T_{(\mu\mbox{s})} = 1_{(s)} / F_{(\mbox{Hz})}
  \label{equation:frequency-to-period}
\end{equation}

In the program code it looks like this:

\begin{minted}{cpp}
  const int T = 1000000 / f;
\end{minted}

If we substitute the frequency of C4 in the formula
\ref{equation:frequency-to-period} we will get the formula
\ref{equation:frequency-to-period-c4}.

\begin{equation}
  T_{(\mu\mbox{s})} = 1000000_{(\mu\mbox{s})} / 261.63 \mbox{Hz} = 3822.191644689 \mu\mbox{s}
  \label{equation:frequency-to-period-c4}
\end{equation}

As can be seen from the formula \ref{equation:frequency-to-period-c4} the length
of one wave period for C4 is not an integer number.  The fractional part will be
lost during the calculations (as we store the result as the \texttt{int} type),
and we end up with the value $3822 \mu\mbox{s}$.

To get the number of iterations for the loop that creates the speaker membrane
oscillations, we divide the length of the sound \texttt{t} by the period
\texttt{T}, according to the formula
\ref{equation:music-band-period-count-calculation}.

\begin{equation}
  \mbox{count} = t_{(\mu\mbox{s})} / T_{(\mu\mbox{s})}
  \label{equation:music-band-period-count-calculation}
\end{equation}

Where \texttt{count} is the variable that stores the number of oscillations.
The program code for the formula look as follows:

\begin{minted}{cpp}
  int count = t / T;
\end{minted}

Let the length of our note be 2s ($2000000 \mu\mbox{s}$.)  When we substitute the
length of the wave period for C4 ($3822 \mu\mbox{s}$) we will get the following
result:

\begin{equation}
  \mbox{Count} = 2000000 \mu\mbox{s} / 3822 \mbox{Hz} = 523.286237572
\end{equation}

Again, the fractional part will be lost -- so we end up with the value 523.  This
is how many iterations the loop will have.  Each loop iteration has the length
of around $3822 \mu\mbox{s}$, if we don't consider the overhead from the procedure
calls.

Thus if we repeat the loop 523 times, we will get the following time of loop
execution:

\begin{equation}
  \mbox{Time}_{(\mu\mbox{s})} = 523 \mu\mbox{s} * 3822 \mu\mbox{s} = 1998906 \mu\mbox{s}
\end{equation}

So the error is:

\begin{equation}
  \mbox{Error}_{(\mu\mbox{s})} = 2000000 \mu\mbox{s} - 1998906 \mu\mbox{s} = 1094 \mu\mbox{s}
\end{equation}

Or around 1.1 ms.  This value is relatively small, and its role in the
desynchronization can be insignificant.  Also the calls to \texttt{digitalWrite}
and to \texttt{delayMicroseconds} also take some time.  Nevertheless, if we
still want to compensate this error, we can add after the loop inside the
\texttt{play\_tone} procedure additional delay calculated from the error as per
the formula \ref{equation:music-band-error-calculation}.

\begin{equation}
  \mbox{Error}_{(\mu\mbox{s})} = \mbox{t}_{(\mu\mbox{s})} -
  (\mbox{T}_{(\mu\mbox{s})} * \mbox{count})
  \label{equation:music-band-error-calculation}
\end{equation}

The code of the \texttt{play\_tone} procedure with this error compensation can
look as follows:

\begin{listing}[!h]
  \begin{minted}{cpp}
    void play_tone(int port, float f, long t) {
      if (f > 0) {
        const int T = 1000000 / f;
        int d = T / 2;
        int count = t / T;
        for (int i = 0; i < count; i++) {
          digitalWrite(port, HIGH);
          delayMicroseconds(d);
          digitalWrite(port, LOW);
          delayMicroseconds(d);
        }

        // Error compensation:
        delayMicroseconds(t - (T * count));

      } else {
        delay(t / 1000); // Rest
      }
      delayMicroseconds(t * 0.10);
    }
  \end{minted}
  \label{listing:music-band-play-tone-with-error-compensation}
  \caption{The modification of \texttt{play\_tone} procedure with the additional
    error compensation.}
\end{listing}

\end{document}
