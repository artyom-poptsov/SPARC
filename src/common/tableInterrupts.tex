\newcommand{\tableInterrupts}[1]{
  \def\lang{\detokenize{#1}}
  \def\langRu{\detokenize{ru}}
  \def\langEn{\detokenize{en}}
  \def\tableCaption{XXX: No translation.}
  \def\tableArduinoI{XXX: No translation.}
  \def\tableArduinoII{XXX: No translation.}
  \def\tableFirstColumn{XXX: No translation.}
  \ifx \lang\langRu
  \def\tableFirstColumn{Плата}
  \def\tableArduinoI{
    Основанные на чипе 32u4 (такие, как Leonardo, Micro)
  }
  \def\tableArduinoII{
    Uno WiFi Rev2, Due, Zero, семейство MKR и 101-е платы
  }
  \def\tableCaption{
    Номера прерываний и связанных с ними цифровых портов.
  }
  \fi
  \ifx \lang\langEn
  \def\tableFirstColumn{Board}
  \def\tableArduinoI{
    Based on 32u4 chip (e.g. Leonardo, Micro)
  }
  \def\tableArduinoII{
    Uno WiFi Rev2, Due, Zero, MKR family and 101 boards
  }
  \def\tableCaption{
    Interrupt numbers and digital ports related to them.
  }
  \fi
  \begin{table}[ht]
    \centering
    \def\arraystretch{1.5}%
    \begin{tabular}{|m{2.5cm}|m{1cm}|m{1cm}|m{1cm}|m{1cm}|m{1cm}|m{1cm}|}
      \hline
      \textbf{\tableFirstColumn}
      & \textbf{\texttt{int.0}}
      & \textbf{\texttt{int.1}}
      & \textbf{\texttt{int.2}}
      & \textbf{\texttt{int.3}}
      & \textbf{\texttt{int.4}}
      & \textbf{\texttt{int.5}}\\
      \hline
      Arduino Uno, Ethernet
      & 2 & 3 &    &    &    &   \\[1.5ex]
      \hline
      Arduino Mega
      & 2 & 3 & 21 & 20 & 19 & 18\\[1.5ex]
      \hline
      \tableArduinoI
      & 3 & 2 & 0  & 1  & 19 &   \\[1.5ex]
      \hline
      \tableArduinoII
      & 0 & 1 & 2  & 3  & 4  & 5 \\[1.5ex]
      \hline
    \end{tabular}
    \caption{\tableCaption}
    \label{table:interrupts-table}
  \end{table}
}
