\newcommand{\figureInterruptTypes}[1]{
  \def\lang{\detokenize{#1}}
  \def\langRu{\detokenize{ru}}
  \def\langEn{\detokenize{en}}
  \def\figureCaption{XXX: No translation.}
  \def\figureDataTransfer{XXX: No translation.}
  \def\figureParallel{XXX: No translation.}
  \def\figureSerial{XXX: No translation.}
  \ifx \lang\langRu
  \def\figureCaption{
    Классификация видов прерываний.
  }
  \def\figureInterrupt{Прерывания}
  \def\figureHardware{Аппаратные}
  \def\figureSoftware{Программные}
  \def\figureInternal{Внутренние}
  \def\figureExternal{Внешние}
  \fi
  \ifx \lang\langEn
  \def\figureCaption{
    Interrupt types classification.
  }
  \def\figureInterrupt{Interrupts}
  \def\figureHardware{Hardware}
  \def\figureSoftware{Software}
  \def\figureInternal{Internal}
  \def\figureExternal{External}
  \fi
  \begin{figure}[H]
    \centering
    \begin{tikzpicture}[
        level distance=15mm,
        level 1/.style={sibling distance=30mm},
        level 2/.style={sibling distance=20mm},
        level 3/.style={sibling distance=30mm},
        every node/.style={
          minimum height=0.75cm,
          rectangle,
          draw
        },
      ]
      \node {\figureInterrupt}
      child {node {\figureSoftware}
      }
      child {node {\figureHardware}
        child {node {\figureInternal}}
        child {node {\figureExternal}}
      };
    \end{tikzpicture}
    \caption{\figureCaption}
    \label{fig:interrupt-types}
  \end{figure}
}
